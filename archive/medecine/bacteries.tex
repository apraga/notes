\section{Bactéries}%
\label{sec:bacteries}

Aérobies

\begin{longtable}[]{@{}lll@{}}
\toprule
\begin{minipage}[b]{0.08\columnwidth}\raggedright
\strut
\end{minipage} & \begin{minipage}[b]{0.41\columnwidth}\raggedright
Gram +\strut
\end{minipage} & \begin{minipage}[b]{0.42\columnwidth}\raggedright
Gram -\strut
\end{minipage}\tabularnewline
\midrule
\endhead
\begin{minipage}[t]{0.08\columnwidth}\raggedright
Cocci\strut
\end{minipage} & \begin{minipage}[t]{0.41\columnwidth}\raggedright
En amas :\\

\begin{itemize}
\tightlist
\item
  \emph{Staphylococcus aureus}\\
\item
  Staphylocoques coagulase negative\\
\end{itemize}

En chaînettes :\\

\begin{itemize}
\tightlist
\item
  streptocoques bêta-hémolytiques :\\
  S. \emph{pyogenes}, \emph{agalactiae}, \emph{dysgalactiae}\\
\item
  S. \emph{pneumoniae}\\
\item
  autres : S. \emph{salivarius}, \emph{sanguis}, \emph{oralis},\\
  \emph{mutans},complexe ``milleri''\\
  (S. \emph{constellatus}, \emph{intermedius}, \emph{anginosus}),\\
  S. \emph{gallolyticus}\\
\end{itemize}

Entérocoques : E. \emph{faecalis}, \emph{faecium}\strut
\end{minipage} & \begin{minipage}[t]{0.42\columnwidth}\raggedright
\emph{Neisseria}

\begin{itemize}
\tightlist
\item
  N. \emph{meningitidis}
\item
  N. \emph{gonorrhoeae}
\end{itemize}

Cocco-baccilles

\begin{itemize}
\tightlist
\item
  \emph{Moraxella} spp
\end{itemize}\strut
\end{minipage}\tabularnewline
\begin{minipage}[t]{0.08\columnwidth}\raggedright
Bacilles\strut
\end{minipage} & \begin{minipage}[t]{0.41\columnwidth}\raggedright
\emph{Listeria} spp.\\
\emph{Corynebacterium} spp.\\
\emph{Bacillus} spp.\\
\emph{Erysipelothrix} spp.\\
\emph{Nocardia} spp.\strut
\end{minipage} & \begin{minipage}[t]{0.42\columnwidth}\raggedright
Entérobactéries :

\begin{itemize}
\tightlist
\item
  \emph{Escherichia coli}, \emph{Klebsiella} spp.,\\
  \emph{Enterobacter} spp., \emph{Serratia}, spp.,\\
  \emph{Proteus} spp., \emph{Salmonella} spp.\\
  \emph{Shigella} spp., \emph{Yersinia} spp., \emph{Citrobacter} spp.
\end{itemize}

Autres :

\begin{itemize}
\tightlist
\item
  \emph{Pseudomonas} spp.\\
\item
  \emph{Stenotrophomonas} spp.\\
\item
  \emph{Acinetobacter} spp.\\
\item
  \emph{Campylobacter} spp.\\
\item
  \emph{Helicobacter} spp.\\
\item
  \emph{Vibrio} spp.\\
\item
  \emph{Bordetella} spp.\\
\item
  \emph{Haemophilus} spp.\\
\item
  \emph{Brucella} spp. (cocco-bacille)\\
\item
  \emph{Pasteurella} spp.\\
\item
  \emph{Legionella} spp.\\
\item
  \emph{Aeromonas} spp.\\
\item
  \emph{Burkholderia} spp.\\
\item
  \emph{Kingella} spp.\\
\item
  \emph{Francisella} spp\\
\end{itemize}\strut
\end{minipage}\tabularnewline
\bottomrule
\end{longtable}

\subsection{Anaérobies}

\begin{longtable}[]{@{}ll@{}}
\toprule
\begin{minipage}[b]{0.71\columnwidth}\raggedright
Gram +\strut
\end{minipage} & \begin{minipage}[b]{0.23\columnwidth}\raggedright
Gram -\strut
\end{minipage}\tabularnewline
\midrule
\endhead
\begin{minipage}[t]{0.71\columnwidth}\raggedright
\emph{Clostridium}: \emph{tetani}, \emph{botulinum}, \emph{perfringens},
\emph{difficile}\\
\emph{Peptococcus} spp.\\
\emph{Propionibacterium acnes}\\
\emph{Actinomyces} spp.\strut
\end{minipage} & \begin{minipage}[t]{0.23\columnwidth}\raggedright
\emph{Bacteroides} spp. \emph{Fusobacterium} \emph{Prevotella} spp.
\emph{Porphyromonas} spp.\strut
\end{minipage}\tabularnewline
\bottomrule
\end{longtable}

\subsection{Autres}

\begin{longtable}[]{@{}ll@{}}
\toprule
\begin{minipage}[b]{0.71\columnwidth}\raggedright
Atypiques\strut
\end{minipage} & \begin{minipage}[b]{0.23\columnwidth}\raggedright
Spirochètes\strut
\end{minipage}\tabularnewline
\midrule
\endhead
\begin{minipage}[t]{0.71\columnwidth}\raggedright
Intracellulaires :\\

\begin{itemize}
\tightlist
\item
  \emph{Chlamydia} spp.\\
\item
  \emph{Rickettsiales} :~\emph{Rickettsia} spp, \emph{Bartonella} spp.,
  ~\\
\item
  \emph{Anaplasma} spp., \emph{Coxiella} spp\\
\end{itemize}

Sans paroi (mollicutes) : \emph{Mycoplasma} spp., \emph{Ureaplasma}
spp.\\
\strut
\end{minipage} & \begin{minipage}[t]{0.23\columnwidth}\raggedright
\emph{Treponema} spp.\\
\emph{Borrelia} spp.\\
\emph{Leptospira} spp.\\
\strut
\end{minipage}\tabularnewline
\bottomrule
\end{longtable}

\begin{longtable}[]{@{}ll@{}}
\toprule
\begin{minipage}[b]{0.66\columnwidth}\raggedright
Mycobactéries\strut
\end{minipage} & \begin{minipage}[b]{0.28\columnwidth}\raggedright
Autres\strut
\end{minipage}\tabularnewline
\midrule
\endhead
\begin{minipage}[t]{0.66\columnwidth}\raggedright
\emph{Mycobacterium tuberculosis}\\
\emph{M. leprae}\\
Mycobactéries atypiques (\emph{M. avium} intracellulaire)\\
~\\
\strut
\end{minipage} & \begin{minipage}[t]{0.28\columnwidth}\raggedright
\emph{Tropheryma whipplei}\strut
\end{minipage}\tabularnewline
\bottomrule
\end{longtable}

\subsection{Bonus : Etymologie}

\emph{Acinetobacter} : \emph{acineto} (qui ne bouge pas) + \emph{bacter}
(bâton)\\
\emph{Actinomyces} : \emph{actis} (rayon) + \emph{myces} (champignon)\\
\emph{Aeromonas} : \emph{aeros} (air) + \emph{monas} (unité) = une unité
qui produit du gaz\\
\emph{Anaplasma} : \emph{an} (sans) + \emph{plasma} (forme)\\
\emph{Bacteroides} : \emph{bacter} (bâton) + \emph{oides} (qui
ressemble)\\
\emph{Bartonella} : d'après Alberto Leonard Barton Thompson\\
\hspace*{0.333em} - \emph{henselae} : d'après D. M. Hensel\\
\emph{Bordetella} : d'après Jules Bordet\\
\emph{Borellia} : d'après A. Borrel\\
\emph{Brucella} : d'après Sir David Bruce\\
\emph{Burkholderia} : d'après W. H. Burkholder\\
\emph{Campylobacter} : \emph{campylo} (courbé) + \emph{bacter} (bâton)\\
\emph{Chlamydia} : \emph{chlamus} (manteau)\\
\emph{Clostridium} : \emph{kloster} (fuseau)\\
\hspace*{0.333em} - \emph{botulinum} : \emph{botulus} (saucisse) =
d'après les premiers aliments mis en cause\\
\hspace*{0.333em} - \emph{difficile} : difficile à étudier\\
\hspace*{0.333em} - \emph{perfringens} (qui casse en morceaux)\\
\hspace*{0.333em} - \emph{tetani} : \emph{tetanus} (tension)\\
\emph{Corynebacterium} : \emph{korune} (gourdin) + \emph{bakteria}
(bâton) = bactérie en forme de gourdin\\
\emph{Coxiella} : d'après Harold R. Cox\\
\emph{Citrobacter} : \emph{citrus} (citron) + \emph{bacter} (bâton) =
bactérie utilisant du citrate\\
\emph{Enterobacter} : \emph{enteron} (intestin) + \emph{bacter}
(bâton)\\
\emph{Erysipelothrix} : erysipelas (erysipèle) + \emph{thrix} (cheveux)
= fil d'erysipèle\\
\emph{Escherichia} : d'après Theodor Escherich\\
\emph{Francisella} : d'après Edward Francis\\
\emph{Fusobacterium} : \emph{fusus} (fuseau) + \emph{bacterium}
(bâton)\\
\emph{Haemophilus} : \emph{haema} (sang) + \emph{philus} (qui aime)\\
\emph{Helicobacter} : \emph{helix} (tordu) + \emph{bacter} (bâton)\\
\emph{Kingella} : d'après Elizabeth King\\
\emph{Klebsiella} : d'après Edwin Klebs\\
\emph{Legionella} (une petite légion/armée)\\
\emph{Leptospira} : \emph{leptos} (fin) + \emph{spira} (spire, hélice) =
en forme d'hélice fine\\
\emph{Listeria} : d'après Joseph Lister\\
\emph{Moraxella} : d'après Victor Morax\\
\emph{Mycobacterium} : \emph{mycos} (champignon) + \emph{bacter}
(bâton)\\
\hspace*{0.333em} - \emph{leprae} : \emph{lepra} (lèpre)petite
protubérance)\\
\hspace*{0.333em} - \emph{tuberculosis} : \emph{tuberculum} (petite
protubérance)\\
\emph{Mycoplasma} : \emph{mycos} (champignon) + \emph{plasma} (forme)\\
\emph{Neisseria} : d'après Albert Neisser\\
\hspace*{0.333em} - \emph{gonorrhea} : \emph{gonos} (semence) +
\emph{rhein} (flux) = écoulement de pus\\
\hspace*{0.333em} - \emph{meningitidis} : \emph{meninx} (membrane) +
\emph{itis} (inflammation)\\
\emph{Nocardia} : d'après Endmon Nocard\\
\emph{Pasteurella} : d'après Louis Pasteur\\
\emph{Peptococcus} : \emph{pepto} (digérer) + \emph{coccus} (baie) = le
coccus qui digère\\
\emph{Prophyromonas} : \emph{porphyro} (violet) + \emph{monas} (unité)\\
\emph{Prevotella} : d'après A. R. Prévot\\
\emph{Pseudomonas} : \emph{pseudo} (faux) + \emph{monas} (unité)\\
\emph{Propionibacterium} : de l'acide propionique\\
\emph{Proteus} : cf le dieu grec Proteus des rivières et oceans\\
\emph{Rickettsia} : d'après H. T. Ricketts\\
\emph{Salmonella} : d'après D.E Salmon\\
\emph{Serratia} : d'après Serafino Serrati\\
\emph{Shigella} : d'après K. Shiga\\
\emph{Staphylococcus} : \emph{staphyle} (grappe) + \emph{coccus}
(baie)\\
\hspace*{0.333em} - \emph{aureus} (doré)\\
\emph{Stenotrophomonas} : \emph{stenos} (étroit) + \emph{trophos}
(nourisseur) + \emph{monas} (``unité'') = une unité qui se nourrit de
peu\\
\emph{Streptococcus} : \emph{streptos} (chaîne) + \emph{kokkos} (baie)\\
\hspace*{0.333em} - \emph{agalactiae} (sans lait)\\
\hspace*{0.333em} - \emph{dygalactiae} (mauvais lait)\\
\hspace*{0.333em} - \emph{pneumoniae} (poumon)\\
\hspace*{0.333em} - \emph{pyogenes} : \emph{pyon} (pus) + \emph{genes}
(egendré)\\
\emph{Treponema} : \emph{trepo} (tourner) + \emph{nema} (fil) = un fil
qui tourne\\
\emph{Tropheryma} : \emph{trophe} (nourriture) + \emph{eruma} (barrière)
= car cause malabsorption\\
\hspace*{0.333em} - \emph{Whipplei} : d'après George Whipple\\
\emph{Ureaplasma} : \emph{urea} (urée) + \emph{plasma} (forme)\\
\emph{Vibrio} (qui vibre)\\
\emph{Yersinia} : d'après A.J.E Yersin

