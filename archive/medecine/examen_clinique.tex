\documentclass[a5paper,12pt]{article}
\usepackage[utf8]{inputenc}
\usepackage[francais]{babel}
\usepackage{fouriernc,parskip,booktabs,array}
\usepackage[margin=7mm,portrait]{geometry}
\usepackage{enumitem}

\usepackage{pgfpages}                                 % <— load the package
  \pgfpagesuselayout{4 on 1}[a4paper,border shrink=3mm] % <— set options

\pagestyle{empty} % No page number

\setlist{nolistsep}
\renewcommand\labelitemi{--}

% Redefine section commands to use less space
\makeatletter
\renewcommand{\section}{\@startsection{section}{1}{0mm}%
                                {-1ex plus -.5ex minus -.2ex}%
                                {0.5ex plus .2ex}%x
                                {\normalfont\large\bfseries}}
\renewcommand{\subsection}{\@startsection{subsection}{2}{0mm}%
                                {-1explus -.5ex minus -.2ex}%
                                {0.5ex plus .2ex}%
                                {\normalfont\normalsize\bfseries}}
\renewcommand{\subsubsection}{\@startsection{subsubsection}{3}{0mm}%
                                {-1ex plus -.5ex minus -.2ex}%
                                {1ex plus .2ex}%
                                {\normalfont\small\bfseries}}
\makeatother


% Don't print section numbers
\setcounter{secnumdepth}{0}


\setlength{\parindent}{0pt}
\setlength{\parskip}{0pt plus 0.5ex}

\begin{document}

\begin{center}
     \Large{\textbf{Examen clinique}} \\
\end{center}

\subsection{ATCD}
\label{sec:org32b2aff}
chir, med, gynéco ($G_{X}P_{Y}$), familial
\subsection{Ttt}
\label{sec:orge6567b5}
habituel, allergies (dernier repas)
\subsection{MDV}
 profession, main dominante, sport\\
 toxiques, alcool, tabac, viral\\
 conditions de vie, autonome
\subsection{HDLM}
\label{sec:org13306b7}
 anamnèse: déclenché par ? favorisé par ?\\
 clinique : quantifier, signes\\
 complications, retentissement, fièvre, AEG
\subsection{Signes généraux}
 SF: asthénie, anorexie, amaigrissement, sueurs, frissons, prurit, soif\\
 Fièvre, hypothermie\\
 Hydratation
\subsection{Urgences}
\underline{Choc} : Confusion, Hypertension, Oligurie, Cutané (TRC), TAchycardie,
POlypnée, Marbrures, Pouls\\
+ extrémités froides/cyanosées, collapsus, pouls paradoxal\\
 \underline{SDRA} : \textit{hypoxémie} (cyanose, tachycardie, trouble CS), \textit{hypercapnie}
 (Sueurs, Asterixis, Cephalées, HTA, Agitation)
 \textit{épuisement ventilatoire} (respi > 30 ou < 10, tirage, respi paradoxale, encombrement bronchique), IC droite\\
 Confusion, coma (glasgow, sd méningé, glycémie)
\subsection{Vasculaire}
Terrain : âge, ATCD familiaux IDM, diabète, HTA, tabac, dyslipidémie, obésité\\
 SF: douleur thoracique (coronale, pleural, péricardite, dissection aortique), dyspnée, lipothymie/syncope, palpitation\\
 Reflux abdomino-jugulaire
 Turgescence jug, oedème périph\\
 Palpation
 Auscultation
\subsection{Vasculaire artériel}
 SF: AIT (troubles paroles transitoire, déficit M/S, cécité transitoire), HTA
 (céphalées, acouphènes, vertiges, épistaxis), claudication d'effort, ischémie
 (douleur, froideur, déficit M, paresthésies)\\
 PA, FC, IPS\\
 TRC, pouls, auscult vasc.
\subsection{Veino-lymphatique}
 SF: douleur, membre rouge chaud tuméfié ($\rightarrow$ score Wells, tour
 mollet, cordon veineux)\\
 Troubles trophiques, \oe{}dème, pincement dorsal 2eme orteil (lymph\oe{}dème)
\subsection{Pulmonaire}
 SF: douleur thoracique (pleurale, EP), dyspnée, expectoration, toux, hémoptysie, hoquet, ronflement\\
 FR, régularité\\
 Distention thoracique, sd cave supérieure (CVC, \oe{}dème en pelerine, TJ),, cyanose, hippocratisme digital\\
 Palpation, auscultation
\subsection{Digestif}
 SF: douleur (ulcère, bilaiaire, pancrétatique, intestinale), transit (nausée,
 vomissement, occlusif, diarrhée, constipation), hémorragie digestive
 (haute/basse), sd \oe{}sophagien, sd rectal (épreinte, faux-besoins, tenesme)\\
 Cicatrice, éventration, hernie, ictère, IHC (angiome stellaire, ongles,
 astérixis, érythrose palmaire, \oe{}dème périph)\\
 Palpation (point Mc Burney, sd péritonéal, HMG, SMG, masse )\\
 Percussion, auscultation\\
 TR
\subsection{Uro}
SF : douleur (rénale, vésicale, prostate, bourse), sd
irritatif/obstructif/brûlures, couleurs, écoulement, érection\\
Palpation (abdo, flanc, globe, bourse)\\
Percussion (fosse, globe)\\
TR/TV
\newpage
\subsection{Locomoteur}
\emph{SF}: douleur (articulation, méca/infla), raideur, gonflement, craquement,
blocage, marche (boîterie, claudication)\\
\emph{Rachis} : courbures, épineuses (palpation, percussion), paravértébraux
(sonnette) [NCB = sonnette, étirement bras, sciatique/cruralgie = sonnette,
lasègue, Léri]\\
\emph{Sacroiliaque} : interligne, écartement/rapportement, cisaillement H, V, trépied\\
\emph{Articulation périph} : inflammation, palpation, déformation, mobilité passiv./act.\\
Genou :
\begin{itemize}
\item épanchement (choc rotulien), palpé rotulien, rabot, Zohlen, ressaut rotulien
\item méniscal : palpation, grinding test, Thessaly (monopoday + RI-RE)
\item ligaments : LCA (Lachman , tiroir 60$^{\circ}$), LCP (60$^{\circ}$),
    capsule (valg/valr ext.), coque (valg/valr 30$^{\circ}$)
\end{itemize}
Épaule : conflit sous-acromial (Neer, Hawkins, Yocum), coiffe (Jobe = supra,
Patte = infra, belly-press= sous-scap)\\
Poignet : Finkelsten (tendinop. de De Quervain)

Pied : interligne Chopart, Lisfranc, éversion/inversion

Cheville : Ottawa (5eme métatarsien/naviculaire ; malléole +/6cm; appui)
\subsection{Neuro}
SF: céphalée, douleur neuro, confusion/mémoire, 5 sens, équilibre, vertiges,
déficit S/M, mouvements anormaux,, dificultés marche\\
Droitier/gaucher\\
Tremblements, dyskinésie, myoclonie, dystonie\\
Testing musculaire (Barré, Migazzini, court abd. pouce, interosse, ext. doigts,
relevurs pied, fléchisseurs orteils)\\
RCP\\
Tonus musculaire, ROT\\
Sensibilité (tact, thermique + talon-genou, doigt-nez)\\
Nerf crâniens (II [AV, RPM], oculomotricité, V [trijumeau], VII [occlusion palpébrale, grimace], VII [Romberg, nystagmus], IX et X [déglutition], XI [sterno-cléido-mastoïdien] XII [langue])\\
Signe méningé (nuque, Brudzinksi, Kernig)\\
Doigts-nez, marionnettes\\
Romberg, marche, ordres\\
Fonction sup : langage, articulation, mémoire


\subsection{Gynéco}
SF: douleur (dyspareunie, cyclique), écoulement, aménorrhée, SFU, prurit, , mammaire\\
Abdo (inspection, palpation)\\
Sein (gangliona axillaire, sus-claviculaire)\\
Périnée (vulve, glandes, prolapsus)\\
Spéculum, TV (col, vagin, utérius)

\subsection{Hémato}
Sd anémique : asthénie, dyspnée, pâleur, tachycardie, acouphènes, céphalées,
palpitation
+ pâleur conjonctives, cut-muqueuse, souffle cardiaque\\
Sd tumoral : fièvre, sueurs, -10\% poids, aire ganglionnaire (tête, cou,
axillaier, coude, inguinales, poplitées), SMG, HMG\\
Sd hémorragique : épistaxis, gingivorragie, méno-MTR, hémorragie digestive,
épistaxis, hémarthrose, IC, purpuras\\
Sd infectieux

\subsection{Endocrino}
Thyroïde\\
Diabéte : SF (angor, AIT, vision), pied : cutané, pouls, TRC, réflexe, monofilament
\subsection{ORL}
Audition:
hypoacousie, cophose, acouphène, otodynie/algie, otorrhée
otoscopie, Rinne, Weber
Équilibre\\
Larynx: dysphonie, dyspnée laryngée (rauque expi), dysphagie, toux\\
Oropharynx: hernie/colique, odynophagie, trismus, xérostomie + palpation\\
Fosses nasales: rhinorrhée, épistaxis, obstruction rhinolalie, anosmie, douleur,
céphalie\\

\subsection{Ophtalmo}
SF: BAV, phosphène, métamorphopsie, ptosis,  diplopie, rougeur, prurit, larmoiement,\\
Paupière, glande, cornée (cercle PK, hémorragie, conjonctivite), conjonctives (pâleur, ictère), globe (tonus,
exo/endo), pupille (mydriasie, myosis, leucocorie)\\
AV

\end{document}
