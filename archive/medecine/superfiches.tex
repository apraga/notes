% Created 2021-04-22 Thu 11:27
% Intended LaTeX compiler: pdflatex
\documentclass[11pt]{article}
\usepackage[utf8]{inputenc}
\usepackage[T1]{fontenc}
\usepackage{graphicx}
\usepackage{grffile}
\usepackage{longtable}
\usepackage{wrapfig}
\usepackage{rotating}
\usepackage[normalem]{ulem}
\usepackage{amsmath}
\usepackage{textcomp}
\usepackage{amssymb}
\usepackage{capt-of}
\usepackage{hyperref}

\usepackage{longtable}
\usepackage{booktabs}
\usepackage{biocon}
%\usepackage[draft]{graphicx}
\usepackage{graphicx}
\usepackage{fancyhdr}
% French
\usepackage[T1]{fontenc}
\usepackage[francais]{babel}
\usepackage{caption}
\usepackage[nointegrals]{wasysym} % Male-female symbol
% Smaller marign
\usepackage[margin=2.5cm, marginparwidth=2cm]{geometry}
\usepackage{latexsym}
\usepackage{subcaption}
\usepackage[table]{xcolor} % Color in tables, must be called before tikz !
% -------------------------------------------------------------------------------
% For graphs
\usepackage{tikz}
\usepackage{tikzscale}
\usetikzlibrary{graphs}
\usetikzlibrary{graphdrawing}
\usetikzlibrary{arrows,positioning,decorations.pathreplacing}
\usetikzlibrary{calc}

% Trees package not found after update
\usepackage{luacode}
\begin{luacode}
  function pgf_lookup_and_require(name)
    local sep = '/'
    if string.find(os.getenv('PATH'),';') then
      sep = '\string\\'
    end
    local function lookup(name)
      local sub = name:gsub('%.',sep)
      local find_func = function (name, suffix)
        if resolvers then
          local n = resolvers.findfile (name.."."..suffix, suffix) -- changed
          return (not (n == '')) and n or nil
        else
          return kpse.find_file(name,suffix)
        end
      end
      if find_func(sub, 'lua') then
        require(name)
      elseif find_func(sub, 'clua') then
        collectgarbage('stop')
        require(name)
        collectgarbage('restart')
      else
        return false
      end
      return true
    end
    return
      lookup('pgf.gd.' .. name .. '.library') or
      lookup('pgf.gd.' .. name) or
      lookup(name .. '.library') or
      lookup(name)
  end
\end{luacode}


\usegdlibrary{trees, layered}
\usetikzlibrary{quotes}
\usetikzlibrary{shapes.misc} % for rounded rectangle

% Custom graph style, for most of our graphs
\tikzgraphsset{decision/.style={
   % Labels at the middle 
    %edge quotes mid,
    % Needed for multi-lines
    nodes={align=center},
    %sibling distance=6cm,
    layer distance=2cm,
    %edges={nodes={fill=white}}, 
    tree layout
    }
  }
\tikzset{
  organ/.style = {rounded rectangle, draw=black}
}

\usepackage{booktabs}
\usepackage[shortcuts,nonumberlist]{glossaries} % shorcuts = \acs{} command
\makeglossaries

% -------------------------------------------------------------------------------
% No spacing in itemize
\usepackage{enumitem}
\setlist{nolistsep}
% tightlist from pandoc
\providecommand{\tightlist}{%
  \setlength{\itemsep}{0pt}\setlength{\parskip}{0pt}}
 % Danger symbol (need fourier package)
\newcommand*{\TakeFourierOrnament}[1]{{%
\fontencoding{U}\fontfamily{futs}\selectfont\char#1}}
\newcommand*{\danger}{\TakeFourierOrnament{66}}
% Hospital sign, bomb sign
\usepackage{fontspec} % For fontawesome
\usepackage{fontawesome}

% Itemize in tabular
\newcommand{\tabitem}{~~\llap{\textbullet}~~}
% No numbering
\setcounter{secnumdepth}{0}
% In TOC, only section
\setcounter{tocdepth}{1}
% Set header
\pagestyle{fancy}
\fancyhf{}
\fancyhead[L]{\leftmark}
\fancyhead[R]{\thepage}
%\renewcommand{\headrulewidth}{0.6pt}
% Custom header : no uper case
\renewcommand{\sectionmark}[1]{%
  \markboth{\textit{#1}}{}}
% Footnote in section
\usepackage[stable]{footmisc}
% Chemical compound
\usepackage{chemformula}

% Negate \implies
\usepackage{centernot}
% No paragraph indent
\setlength\parindent{0pt}

% Nice box but small for our purpose
\usepackage{tcolorbox}
\tcbset{left=1pt, right=1pt, top=1pt, bottom=1pt, boxrule=0.2mm}

% Footnote in table
\usepackage{tablefootnote}

% hyperref setup
\hypersetup{
  colorlinks = true,
  linkcolor=teal
}
%-------------------------------------------------------------------------------
% Custom commands
%-------------------------------------------------------------------------------
% Logical and, or
\def\land{$\wedge$}
\def\lor{$\vee$}
\def\dec{$\searrow{}$}
\def\inc{$\nearrow{}$}
\def\thus{$\implies$}

% Hightlight cell
\def\hlc{\cellcolor{blue!10}}
\def\hl{\colorbox{blue!10}}

\input bacteries-header

\usepackage{adjustbox}
\usepackage{tabularx}
\author{Alexis Praga}
\date{\today}
\title{Superfiches}
\hypersetup{
 pdfauthor={Alexis Praga},
 pdftitle={Superfiches},
 pdfkeywords={},
 pdfsubject={},
 pdfcreator={Emacs 28.0.50 (Org mode 9.4.4)}, 
 pdflang={English}}
\begin{document}

\maketitle
\tableofcontents

\section{Cardiologie}
\label{sec:org13b2d0c}
\subsection{218 Athérome}
\label{sec:org5a6b9bb}
1ere cause mortalité dans monde.

Prévention : aspirine, statines, antihypertenseurs

Si rupture de plaque : aspirine, clopidogrel, héparine 

Localisations : coronaires, aorte, carotides, artères MI

Prévention secondaire : aspirine/clopidogrel, statines, IEC/ARA II

\subsection{219 Facteur de risques cardio-vasculaires}
\label{sec:orgda2c634}
FR : LDL > 1.6G/L, HDL < 0.4g/L, tabac\footnote{Arrêt < 3 ans}, âge\footnote{> 50 ans \male{}, > 60 ans \female{}}, tension \(\ge\) 140/90, diabète, ATCD
coronaires familiaux\footnote{Père < 55 ans, mère < 65 ans}

Objectifs : tension < 140/90, LDL < 1g/L
\subsection{220 Dyslipidémies}
\label{sec:orgb72c816}
Haut risque : maladie vasculaire/coronaire, diabète 2 et atteinte rénale,
\(p(\text{evt coronaire à 10 ans}) > 0.2\)

Ttt : hypercholestorémie, hyperlipidémies mixtes : statines. Hypertriglycéridémie : fibrate si TG > 4g/L
\subsection{334 Syndromes coronariens aigüs}
\label{sec:org3472b26}
\begin{table}
\caption{SCA}
\centering
\adjustbox{max width=\linewidth}{
\begin{tabular}{llll}
Type & douleur & PEC & Ttt\\
\hline
Angor stable & angineuse & ECG d'effort & Dérivés nitrés (crise)\\
 & à l'effort &  & Aspirine, \(\beta\)-bloquants\\
 & trinitro-sensible &  & \\
SCA sans ST & angineuse & \faHospital{} urgente & Aspirine, inhib P2Y12\\
 & spontanée > 20min & ECG, troponine & , héparine, \(\beta\)-bloquants, statines\\
 &  &  & Angioplastie percut.\\
SCA ST+ & angineuse & ECG & Angioplastie si < 120min\\
 & repos > 30min/jour &  & Sinon FIV\\
 & trinitro-résistante &  & Morphine, \(O_2\), aspirine\\
 &  &  & , inhib P2Y12, héparine (crise)\\
\end{tabular}
}
\end{table}
ST- : sous-ST pendant la douleur \(\ge\) 1 mm dans 2 dérivations contigües

ST+ : sus-ST \(\ge\) 2mm (\(V_1\) à \(V_6\)) ou \(\ge\) 1mm (autres) dans 2 dérivations contigües

\subsection{228 Douleur thoracique aigüe}
\label{sec:org2ccf7fd}
Examens sytématiques : ECG 12 dérivation, RX pulmonaire, troponines

\begin{table}[htbp]
\caption{4 urgences = PIED}
\centering
\begin{tabular}{lll}
Urgence & Caractéristiques & Examens\\
\hline
Péricardite (tamponnade) & signes droits\tablefootnote{Turgescence jugulaire, reflux hépatojugulaire}, choc avec tachy, PAs < 90mmHg & échocardio\\
IDM & douleur spontanée de repos > 20min & ECG, troponine x2\\
Embolie pulmonaire & terrain, douleur basithoracique + dyspnée & D-dimère \thus angioscan/scinti\\
Dissection aortique & douleur déchirement, irradie dans dos & echo cardio et ETO\\
 &  & \\
\end{tabular}
\end{table}
Non CV : 4P = pneumonie, pleurésie, pneumothorax, pancréatite
\subsection{223 Artériopathie oblitérante (aorte, MI)}
\label{sec:org31627a2}
AOMI : pronostic grave !
\begin{itemize}
\item asymptom. - claudication intermittente - douleur ischémique de repos -
ulcération/gangrène
\item diag: IPS < 0.70 ( + écho-doppler MI en pratique)
\item ttt : FR, antiagrégant, statines, IEC + arrêt tabac, marche \textpm{}
revascularisation (stent/chir)
\end{itemize}
Anévrisme aorte abdominale : âge, tabac, ATCD familiaux
\begin{itemize}
\item scanner abdopelvien/IRM = référence
\item asymptomatique : chir si \diameter \(\ge\) 50-55mm
\item symptomatique : urgence
\end{itemize}
Ischémie aigüe des membres inférieurs = urgence  (embolie/thrombose)
\begin{itemize}
\item diag : douleur brutale, intense, broiement avec impotence fonctionnelle et membre froid, livide, douleur palpation musc
\item ttt : anticoagulant, antalgique niv. 3, \(O_2\) et revasc. chir
\end{itemize}

\subsection{231 Rétrécissement aortique}
\label{sec:org3fcf757}
Pronostic vital mis en jeu si symptômes !

Auscult : souffle systolique éjectionnel au foyer aortique

Diag : ETT = \(V_{max} > 4m/s\), gradient moyen > 40mmHg, surface aortique <
1\(cm^2\)

Ttt : chir si symptomatique = valve mécanique ou bio
\subsection{231 Insuffisance mitrale}
\label{sec:org0a43992}
Classif. de Carpentier (I à III)

Auscult : souffle apexo-axillaire holosytolique de régurgitation dès B1

Examen : ETT \textpm{} ETO

Ttt : chir = reconstruction valve ou prothèse (bio/méca)\footnote{Si asymptomatique, chir si dilatation VG/fraction d'éjection < 60\%}
\subsection{231 Insuffisance aortique}
\label{sec:org8850875}
Formes : dystrophiques (valves normales mais non jointive), bicuspidie aortique

Auscult : souffle diastolique. ETT = diagnostic

Opérer :
\begin{itemize}
\item urgence si chronique volumineuse symptomatique ou aigüe volumineuse
\item si dystrophie et aorte dilatée\footnote{Même critère que précédemment : \(\ge\) 55mm}, si chronique volumineuse asymptomatique et (aorte
dilatée, FEVG < 50\%\ldots{})
\end{itemize}
\subsection{150 Surveillance des porteurs de valves, prothèses vasculaires}
\label{sec:org81695d4}
Complications : embolie, thromboses\footnote{Prothèse mécaniques}, désinsertion, EI ,
dégénérescence\footnote{Bioprothèses}, ttt anticoagulant

INR \(\in\) [2.5, 4] à vie pour prothèse mécanique
\subsection{149 Endocardite infectieuse}
\label{sec:org59b0c2e}
Diag = ETO. Critères de Duke : 2 majeur/1 majeur et 3 mineurs/5mineurs
\begin{itemize}
\item majeurs : 
\begin{itemize}
\item hémoc = typique sur 2HC ou positive > 12h ou \emph{C. burnetii}
\item végétation/abcès/désinsertion (écho) ou nv souffle régurgitation valvulaire
\end{itemize}
\item mineurs : cardiopathie à risque, > 38\(^\circ\), complication vasc,
immunologique
\end{itemize}
Complications : insuf cardiaque , embolie septiques, neuro , rénale
Ttt :
\begin{itemize}
\item streptocoques (oraux/groupe D) : amoxicilline + gentamicine
\item staphylocoques : cloxacilline si sensible (sinon vancomycine)
\end{itemize}
\subsection{236 Souffle cardiaque chez l'enfant}
\label{sec:org8582cd5}
Néonatale : coarctation aortique préductale (chir urgente ), transposition des gros vaissaux,
malformations complexes

Nourisson (2M - marche) : shunt gauche-droite (communication intraventricul), tétralogie de Fallot

2A-16A : souffles fonctionnels\footnote{Asymptomatique, systolique (!), proto/mésosystoliques, éjectionnels,
faible intensité, doux}
\subsection{337 Malaise, perte de connaissance}
\label{sec:org1e40ef4}
\begin{itemize}
\item Éliminer épilepsie
\item Éliminer urgence (SCA, EP\ldots{}.)
\item ECG, clinique : cause évidente = mécanique\footnote{Rétrécissement aortique, EP, cardiomyopathies obstructives, tamponnades, thromboses de valves}, électrique\footnote{TV, bradycardies par bloc AV/dysfonction sinusale}, hypotensive\footnote{Orthostatiques iatrogène/sujet âgé ou hypotension réflexe} ?
\item Sinon chercher cardiopathie  (échocardio)
\end{itemize}
\subsection{230 Fibrillation atriale}
\label{sec:org588e7d7}
ECG indispensable : pas de P, QRS irréguliers

Risques = insuf cardiaque, thromboembolique artériel systémique

Étiologie : HTA, valvulopathies

Ttt 
\begin{itemize}
\item 1er épisode : HNF seule
\item FA persistante\footnote{Retour sinal > 7j/cardioversion} : 3 semaines HNF IV puis choc électrique puis 4 semaine anticoagulants oraux
\item FA permanente\footnote{Échec cardioversion} : bradycardisants (\(\beta\)-bloquants)
\item FA paroxystique : anti-arythmique et bradycardisants
\end{itemize}

\subsection{234 Troubles de la conduction intracardiaque}
\label{sec:org7893052}
Bradychardie grave = urgence 

Ttt : tachycardisants (atropine, catécholamine)

\begin{table}[htbp]
\caption{EEP  = étude électrophysiologique endocavitaire}
\centering
\begin{tabular}{llll}
Type & Causes & Diagnostic & Pacemaker\\
\hline
Dysfonction & dégénérative & ECG + Holter & Si symptomes\\
sinusale & médicaments, vagal &  & \\
BAV & hyperkaliémie & ECG \textpm{} Holter/EEP & BAV II symptom.\\
 &  & si paroxystique & BAV III non curable\\
BdB & dégénérative & ECG + EEP & Si symptômes\\
\end{tabular}
\end{table}
\subsection{229 ECG}
\label{sec:org7826340}
\begin{center}
\begin{tabular}{lll}
Hypertrophie atriale & gauche : P > 0.12s (largeur) & droite : P > 2.5mm (hauteur)\\
Hypertrophie ventriculaire & gauche : \(SV_1 + rV_5 > 35\) mm & droite +110\(^\circ\)\\
\hline
BdB & gauche: QRS > 0.12s ) & droite : QRS > 0.12s\\
 & et (rS ou QS en V\textsubscript{1}) & et RsR' en V\textsubscript{1}\\
Hémibloc & antérieur : \(S_3 > S_2\) & postérieur : \(S_1Q_3\)\\
BAV & I : PR > 0.2s , II : \inc PR & III : aucun P\\
Angor & sous-ST & \\
IDM & sus-ST convexe (vers le haut) & \\
Péricardite aigüe & sus-ST concave & \\
\end{tabular}
\end{center}
Troubles supraventriculaire : FA (100-200/min), flutter atrial (300/min), tachy. atriale,
tachy. jonctionnelle, extrasystole

Troubles ventriculaires : tachy. ventricale, fibrillation ventricalure (
absolue), torsade de pointe
\subsection{235 Palpitations}
\label{sec:orgae01964}
Éliminer urgences immédiates (tachy à QRS large !!)

Fréquents : fibrillation/flutters atriaux, tachy jonctionnelle

Tachy sinusale (grossesse, hyperthyroïde, SAS, alcoolisme) 
\subsection{232 Insuffisance cardiaque}
\label{sec:org09d0a0a}
Grave : 50\% de DC à 5 ans

EC : dyspnée d'effort, signes d'IC droite\footnote{Turgescence jugulaire, reflux hépatojugulaire, HMG, \oe{}dème périph, ascite}

Examens : ECG, RX thorax, dosage BNP, \emph{ETT}

Cause : ischémie, HTA, cardiomypathie

IC aigüe \(\approx\) OAP : polypnée, sueurs, cyanose, expectorations "mousseuses
saumon", crépitants \thus PEC immédiate 

\begin{table}[htbp]
\caption{Ttt insuffisance cardiaque}
\centering
\begin{tabular}{ll}
IC à FE < 40\% & diurétiques + IEC + \(\beta\)-bloquants\footnotemark.\\
 & Si échec : ivabradine puis défibrillateur auto\\
IC à FE conservée & mal définie (idem ?)\\
OAP & assis, furosémide, dérivés nitrés, HBPM\\
\end{tabular}
\end{table}
\footnotetext{Pas si crise aigüe ! \danger}

\subsection{221 Hypertension artérielle}
\label{sec:orgd64fbff}
3 catégories. Étiologies :
\begin{itemize}
\item Essentielles (90\%)
\item Secondaire : néphropathie parenchym., HTA rénovasculaire, phéochromocytome, sd Conn,
coarctation aortiques, SAS, médicaments, HTA gravidique
\end{itemize}

Examens : glycémie, cholestérol, kaliémie, créat, BU, ECG

Complications : rein, neuro, CV

Ttt : objectif < 140/90 \thus hygiéno-diét et \{IEC, diurétiques thiazidiques, \(\beta\)-bloquants, ARA II
\subsection{225 Insuffisance veineuse chronique}
\label{sec:orgad66ae8}
FR : varices = âge, ATC, obésité, grossesse; MTEV = immobilisation, K, anomalies hémostase

Clinique : jambes lourdes\footnote{Calmé par froid, surélévation, marche}, dermite ocre, varices

Examen : echo-doppler veineux des MI   

Ttt (non curatif) : contention, hygiène de vie, invasif
\subsection{233 Péricardite aigüe}
\label{sec:orgd0cd9f6}
Douleur : thoracique, trinitro-résistant, \inc inspiration, \dec assis en avant

Examens : ECG (4 phases\footnote{sus-ST concave, T plates, T négatives, normales}), marqueurs de nécrose cardio, sd inflammatoire, \emph{échocardio}

Étiologie : péricardite aigüe virale, néoplasiques, tuberculeuse surtout

Ttt : AINS, colchicine

Complications : tamponnade = urgence  (drainage/ponction)
\subsection{327 Arrêt cardiocirculatoire}
\label{sec:org6e7212a}
ABCD : maintien voies Aéoriennes, assistance respi (B), massage Cardiaque,
Défibrillation et Drogues

Ttt : adrénaline 1mg/4min et après 2eme choc amiodarone (300mg dans 30mL)

Ventilation, hypothermie

Étio : FA, bradyc, asystolie
\subsection{264 Diurétiques}
\label{sec:orge340436}
\begin{table}[htbp]
\caption{Diurétiques}
\centering
\begin{tabular}{llll}
Type & Molécule & Indication & ES\\
\hline
de l'anse & furosémide & insuf cardiaque, IR & hypoK, déshydratation\\
 &  & \oe{}dème/ascite cirrhose & \\
thiazidique & hydrochlorothiazidique & HTA & hypoK\\
épargnant le potassium & spironolactone & Insuf cardiaque, HTA & hyperK, IR\\
\end{tabular}
\end{table}

\subsection{326 Antithrombotiques}
\label{sec:org1a7af3e}
\begin{table}
\caption{Antithrombotiques}
\centering
\adjustbox{max width=\linewidth}{
\begin{tabular}{llll}
Type & Action & Indication & Dangers\\
\hline
Antiagrégants & aspirine & AVC, coronaropathie & \\
 & \emph{clopidogrel, prasugrel, ticagrelor}\tablefootnote{Inhibiteurs P2Y12} & SCA, post-angio coronaires & hémorr. cérébrale (prasugrel)\\
Héparines & HNF, HBPM, \emph{fondaparinux} & urgence : TVP, EP, SCA & \danger hémorragie, thrompopénie\\
Anti-vit. K & Oral, relais héparine & FA, TVP, EP & \danger hémorragie\\
 &  & valve cardiaque & \\
Nv anticoag oraux & \emph{dabigatran, rivaroxaban, apixaban} &  & Surv. rein\\
 & rapide, &  & 0 antidote\\
Thrombolytique & \emph{-kinase}, \emph{-téplase} & EP grave, AVC < 6h30 & \\
 &  & IDM < 6-12h & \\
\end{tabular}
}
\end{table}

\section{Pneumologie}
\label{sec:org60a8ad8}
\subsection{73 Addiction au tabac}
\label{sec:org987c7a2}
16 millions de fumeurs en France. 1 fumeur sur 2 décède d'une malaide liée au
tabac.
Cause 90\% K bronchopulomaire (25\% si passif)

Sevrage : TCC, éducation thérapeutique, substituts nicotiniques.
\subsection{108 Troubles du sommeil de l'adulte}
\label{sec:org9bfdbee}
Sd d'apnée obstructive du sommeil : 
\begin{itemize}
\item diagnostic : \{ronflement, pause respi, nycturie et somnolence diurne\} et
polysomnographie\footnote{Ou enregistrement polygraphie ventilation}
\item ttt : pression positive continue
\end{itemize}

Autres : sd d'apnée centrale\footnote{Pas d'effort respi, contrairement au SAOS} (IC cardiaque sévère !), insuf respi chronique avec
hypoventilation alvéolaire, sd obésite hypoventilation (ttt = VNI)

\subsection{151 Infections broncho-pulmonaires communautaires}
\label{sec:org100754b}
Bronchite aigüe : virale++, diag clinique (épidémie, toux, expectoration, pas de
crépitants), ttt symptomatique

Exacerbation BPCO (cf item 205)

Pneumonie aigüe communautaires : grave. Cf table \ref{tab:orgb6d9cd9}
\begin{itemize}
\item Clinique : signes auscult en foyer, crépitants. RX thorax !
\item Ttt : ATB urgence (pneumocoque !), réévaluation 48-72h
\end{itemize}
\begin{table}[htbp]
\caption{\label{tab:orgb6d9cd9}PAC}
\centering
\begin{tabular}{lll}
Pneumocoque\tablefootnote{Pas de transmission interhumaine} & Légionellose\tablefootnote{Pas d'isolement} & Atypiques\\
\hline
fréquent++ & progressif & virale \thus inhib neuramidases\\
début brutal, fièvre & myalgies & \emph{Mycoplasmia pneumoniae}\\
RX: condensation systématisée & macrolides & \emph{Chlamydia pneumoniae}, \emph{psitacci}\\
amoxicilline &  & \thus macrolides\\
 &  & RX: opacités multifocales\\
\end{tabular}
\end{table}

\subsection{151 Tuberculose}
\label{sec:orgfffce74}
Diag = IDR/IFN \(\gamma\) si primo infection. Sinon bacille de Koch sur prélèvement
Cf table \ref{tab:org5dc68ce}.

Diag = ED ou culture Löwenstein \thus granulome épithélioïde gigantocellulaire
avec nécrose caséeuse

\begin{table}[htbp]
\caption{\label{tab:org5dc68ce}Formes pulmonaires}
\centering
\begin{tabular}{lll}
Forme & pulmonaire & miliaire\\
\hline
Clinique & AEG, fièvre, sueurs nocturnes & idem, VIH !\\
RX & nodules, infiltrats, cavernes & sd interstitiel\\
Prélèvement & crachat/tubage/fibro & biopsie\\
Diag & sécrétions/tubages & LCS, hémoc, biopsie\\
\end{tabular}
\end{table}


Ttt (PERI) : isoniazide (6M), rifampicine (6M), ethambutol (2M), pyranizamide (2M).
Vaccination !
\subsection{180 Accidents du travail}
\label{sec:org3948441}
Asthme (10-15\%), BPCO (10-20\%), K (mésothéliome ou bronchique primitif), PID
(hypersensibilité, silicose, bérylliose, sidérose, asbestose)

Amiante : K ou pleural

Idemnité : assurance maladie ou FIVA
\subsection{182 Hypersensibilités et allergies respiratoires}
\label{sec:orga61bcb6}
Allergie : le plus souvent médiée par IgE. Asthme et rhinite (cf item 184)

Diag = prick test (\diameter \(\ge\) 3 mm/témoin)

Ttt : éviction, \{antihistaminique (rhinite seulement), corticoïdes\footnote{Systémique en cas d'urgence.},
(adrénaline si choc)\}, immunothérapie spécifique
\subsection{184 Asthme, rhinite}
\label{sec:org98b5459}
Asthme :
\begin{itemize}
\item suspicion sur clinique : dyspnée, gêne, respi, siflement \emph{variable et réversible}
\item diag\footnote{Obstruction des voies aérienne variable} : TVO (VEMS/CVF < 0.7) réversible aux BDCA (\(\ge\) +200mL et \(\ge\) +12\%)
\item ttt de fond \footnote{Jusqu'au contrôle}= corticostéroïdes inhalés (voire anti-leucotriènes, voire cortico
oraux) et symptomatique = \(\beta\)2-mimétique courte durée
\item \{\} exacerbation grave : SAMU + \(\beta\)\textsubscript{2}-mimétique \(\rightarrow\) \faHospital : \(\beta\)\textsubscript{2}-mimétique nébulisé, corticoïdes
oraux, \(O_2\)
\end{itemize}
Rhinite : 
\begin{itemize}
\item PAREO : Prurit, Anosmie, Rhinorrhée, Éternuement, Obstruction nasale
\item ttt : antihistaminiques/corticoïdes nasaux
\end{itemize}
\subsection{188, 189 : Pathologies auto-immunes}
\label{sec:org19b2346}
Si ttt pour connectivite, vascularite,
penser 1. infenction 2. médicament 3. manifestation maladie 4. manifestation
indépendante

Cf Table \ref{tab:org541a558}

\begin{table}[htbp]
\caption{\label{tab:org541a558}Manifestations respiratoires}
\centering
\begin{tabular}{ll}
Sclérodermie systémique & PID, HTAP\\
Lupus érythémateux & Pleurésie lupique, sd hémorragie alvéolaire\\
Polymyosite & PID chronique/aigùe\\
Sd Gougerot-Sjögren & Toux chronique , PID\\
Granulomatose avec polyangéite & Nodules\tablefootnote{Évoluant vers excavation, infiltrats diffus bilatéraux}\\
Granulomatose éosinophile avec polyangéite & Asthme, pneumopathie à éosinophiles\\
Polyangéite microscopique & Sd hémorragique alvéolaire\\
\end{tabular}
\end{table}

\subsection{199 Dyspnée aigüe et chronique}
\label{sec:org38ecaf7}
Examens : ECG, RX thorax, gaz du sang, D-dimère, BNP, NFS a minima

Étiologies : cf Tab \ref{tab:org902aa45}, \ref{tab:org89d059b}
\begin{table}[htbp]
\caption{\label{tab:org902aa45}Étiologies de dyspnée aigüe (E = enfant)}
\centering
\begin{tabular}{lll}
Inspiratoire & Expiratoire & Sinon\\
\hline
corps étranger (E) & asthme (sibilants) & EP (ascult normale)\\
épiglottite (E) & BPCO (ATCD, sibilants) & pneumothorax, épanchement pleural\\
laryngite (E) & OAP (crépitants, expector mousseuse) & pneumopathie infectieuse\\
\oe{}dème de Quincke &  & OAP (crépitants, expector mousseuse)\\
 &  & \\
\end{tabular}
\end{table}
\begin{table}[htbp]
\caption{\label{tab:org89d059b}Étiologies de dyspnée chroniques}
\centering
\begin{tabular}{lll}
Sibilants & Crépitants & Auscult normale\\
\hline
BPCO & PID & EP\\
asthme & Insuf cardiaque gauche) & Neuromusc\\
Insuf cardiaque gauche &  & Pariétale, hyperventilation\\
\end{tabular}
\end{table}

\subsection{200 Toux chronique}
\label{sec:orga69aa40}
Diagnostic principaux : rhinorrée chronique, RGO, asthme, tabac, médicaments (IEC)\footnote{Ou ARA II, \(\beta\)-bloquants}, coqueluche

Pas d'orientation (dans l'odre) : test rhinorrée post chronique, EFR (asthme),
RGO, sinon antitussif\footnote{Toux sèche, invalidande}/kiné respi

Penser bronchectasies si toux productive quotidienne, hémoptysies RX normale
\thus diag = TDM , ttt = drainage \(\pm\) ATB si exacerbation
\subsection{201 Hémoptysie}
\label{sec:orgf26ec33}
Étiologies : tumeurs bronchopulomaires, bronchectasies, tuberculose,
idiopathique, (EP, ICG)

PEC : confirmer hémoptysie, angio-scanner (ou RX thorax)

Ttt 1ere intention : \faHospital : \(O_2\), terlipressine, protection voie aériennes\footnote{Risque = asphyxie}
\subsection{202 Épanchement pleural}
\label{sec:org2de3370}
Clinique : douleur thoracique, toux sèche + sd pleural\footnote{Silence auscult, matité, 0 transmission cordes vocales, souffle pleurétique} 

RX (opacité dense, homogène, concave) et ponction (sauf ICG) !

Selon liquide, voir Tab. \ref{tab:orge05b891}
\begin{table}[htbp]
\caption{\label{tab:orge05b891}Types d'épanchement pleural}
\centering
\begin{tabular}{ll}
Transsudat & Exsudat\\
(prot < 25g/L, liquide clair: mécanique) & (prot > 35g/L, aspect variable : inflammatoire, )\\
\hline
\emph{ICG} & \emph{Tumorale} : K bronchique, mésothéliome pleural\\
cirrhose & \emph{Tuberculose}\\
sd néphrotique & bactérienne, virale\\
 & EP\\
\end{tabular}
\end{table}
Si abondant/purulent/à germe : ttt anti-infectieux + évacuation du liquide
pleural 
\subsection{203 Opacités et masses thoraciques : tab. \ref{tab:org2a34048}}
\label{sec:org7fd193d}
\begin{table}[htbp]
\caption{\label{tab:org2a34048}Masses thoraciques}
\centering
\begin{tabularx}{\textwidth}{XX}
Nodules (\diameter < 3 cm) & Masses (\diameter > 3cm) médiastinales\\
\hline
- malin si terrain, > 1cm, morphologie\footnotemark, fixe TEP, \inc récente taille & - antérieur : goître, K thyroïdes, thymome, lymphome\\
- chez 1 gros fumeur sur 2 et cancéreux 1 sur 10 & - moyen : lymphome\\
- scanner, TEP-FDG. Diag = ponction transpariétale & - postérieur : neurogène\\
- malignes : métastates (pulmon), primitif (bronchopulmon) & \\
\end{tabularx}
\end{table}\footnotetext[26]{\label{orgc2000f4}spiculés, polylobé, irrégulier}
\subsection{204 Insuffisance respiratoire chronique}
\label{sec:orgbbfb660}
Clinique : dyspnée, hypoxémie (PaO\textsubscript{2} < 70mmHg)

Examens : EFR, RX thorax, écho cardiaque.

Étiologies : table \ref{tab:org499d095}

\begin{table}[htbp]
\caption{\label{tab:org499d095}Étiologies IRC selon TV}
\centering
\begin{tabular}{llll}
Obstructive & Restrictive & Mixte & Pas de TV\\
\hline
BPCO & PID & DDB & HTP\\
asthme & Obésité & Mucoviscidose & \\
\end{tabular}
\end{table}

TTT : arrêt tabac, \(O_2\) longue durée (si \emph{obstructif et PaO\textsubscript{2} < 55 mmHG ou
   restrictif et < 60mmHg}) ou ventilation long cours (restrictif)
\subsection{205 BPCO}
\label{sec:orgda128b0}

Clinique : tabac++, dyspnée, toux, expectoration \textpm{} distension thoracique 

Diag = VEMS/CV < 0.70 \emph{non réversible}

Ttt : 
\begin{itemize}
\item broncho-dilatateur courte durée/longue durée si exacerbation. + corticoïdes
inhalés si besoin.
\item Arrêt tabac, kiné respi, vaccins grippe, pneumocoque
\end{itemize}

Exacerbation BPCO : déclenché par infection
\begin{itemize}
\item diag : BPCO connu, \inc dyspnée, toux ou expectoration
\item ttt : bronchodilatateur courte durée (nébulisé) + amox-acide clav \footnote{Ou macrolides, pristinamycine}si
aggravation/\inc purulence
\end{itemize}
\subsection{206 Pneumopathies infiltrantes diffuses}
\label{sec:orgf083101}
Clinique : dyspnée d'effort (apparition progressive)
   Orientation = scanner thoracique++

PID aigüe : éliminer \oe{}dème cardiogénique puis LBA. Si fièvre, ATB
probabilist (sur pneumocoque \textpm{} pneumocystose, tuberculose)

PID subaigüe/chronique : 
\begin{itemize}
\item interrogatoire, LBA (fibro), scanner thoracique 1ere intention
\item sarcoïdose , insuf cardiaque gauche , médicaments , fibrose pulmonaire idiopathique lymphangite carcinomateuse
\end{itemize}
\subsection{207 Sarcoïdose}
\label{sec:org4ff6054}
Manifestations: 
\begin{itemize}
\item respi : RX thorax 4 stades\footnote{I = ADP médiastin brchoniques hilaires bilat, symétriques non
compressives. II = idem + parenchyme. III = parenchyme seulement. IV = fibrose}
\item oculaire (uvéite), peau (sarcoïdes), ADNP (superficielles), foie (cholestase non ictérique)
\end{itemize}

Diag : clinique + histologie (granulomes sans nécrose caséuse) + élimination
autres granulomatose (ou sd Löfgren typique\footnote{Fièvre, arthalgie, érythmèe noueux MI, ADP médiastinale})

Pronostic favorable 80\%

Ttt : corticoïdes > 1 an 
\subsection{222 HTA pulmonaire artérielle}
\label{sec:org47e8d8f}
Classification : HTAP, HTP cardiopathies gauches (\emph{post-capillaire}), HTP respiratoires chroniques, HTP
post-embolique chroniques, HTP multi-factorielle

HTAP = PAP moyenne \(\ge\) 25mmHg (=HTP) et PAPO \(\le\) 15mmHg
(pré-capillaire). Pronostic sombre

Clinique : dyspnée à l'effort

Complémentaire : ECG, RX thorax, EFR, gaz du sang normaux

Diagnostic : ETT et cathétérisme cardiaque droit

\subsection{224 Thrombose veineuse profonde et embolie pulmonaire}
\label{sec:org9f948f0}
\subsubsection{TVP}
\label{sec:org3d931eb}
Probabilité : score de Wells (unilatéral : OMI, douleur trajet veineux\ldots{})

Diagnostic : (D-dimère si proba faible puis) Écho-doppler veineux MI

Étiologies : congénitale ("thrombophilie") ou acquise (K, alitement,
contraception, chir\ldots{})

Prévention : HBPM, contention, lever
\subsubsection{EP}
\label{sec:org8b39091}
Suspicion : proba clinique, RX thorax, ECG, gaz du sang \footnote{Hypoxie-hypocapnie}

Diagnostic : (D-dimère si proba faible puis) scanner (ETT en attendant). Si
scanner toujours non disponible et patient à haut risque, traiter !

\subsubsection{Ttt (TVP + EP)}
\label{sec:org27334e7}
Ttt : 
\begin{itemize}
\item HBPM/fondaparinux + relais AVK ou nv anticoag 3 mois (1ere EP, TVP
proximale provoquée) ou 6 mois
\item thrombolyse si EP grave
\item contention si TVP ou EP + TVP
\end{itemize}
\subsection{228 Douleur thoracique aigüe et chronique}
\label{sec:orgc4080b2}
Examens : ECG, \(SpO_2\), RX thorax

\begin{enumerate}
\item Urgences vitales : SCA++, EP, tamponnade, dissection aortique, pneumothorax oppressif
\item Sinon : 
\begin{itemize}
\item rythmée par la respiration : pneumothorax, infectieuses, péricardite,
pariétales, EP, trachéobronchites
\item non rythmée par la respiration : SCA, dissection aortique, RGO++, psychogène
\end{itemize}
\end{enumerate}
Chronique : paroi thoracique, plèvre
\subsection{306 Tumeurs du poumon}
\label{sec:orgc09ec78}
K bronchique : 1ere cause de DC par K en France (17\% survie 5 ans). Cause = tabac !

Y penser si SF respi chez tabagique > 40 ans, AEG chez tabagique. Toux souvent révélatrice

2 types :
\begin{itemize}
\item non à petites cellules (80\%) : chir/radio/chimio
\item à petite cellule (15\%) : mauvais pronostic, chimio \textpm{} radio (pas de chir)
\end{itemize}

Diagnostique = histologique (fibroscopie). Extension = TDM
\subsection{333 Oedème de Quincke et anaphylaxie}
\label{sec:org979560c}
Réaction anaphylactique médiée par IgE ou non
Diag = clinique + contexte. Toujours doser tryptase sérique
\begin{itemize}
\item fortement probable : gêne respi "haute" ou asthme aigü ou choc impliquant PV,
 \{rash, urticaire, angioedème\}, début brutal et progression rapide
Urgence (atteinte multiviscérale menaçant la vie) : adrénaline \uline{IM}
\end{itemize}
0.01mg/kg toutes 5min, arrêt agent, allongé/PLS 

Sinon : antihistaminique + corticoïdes

\subsection{354 Corps étranger des voies aériennes}
\label{sec:org82448b3}
Pics : enfant, âgé

Y penser si symptôme respiratoire chronique/récidivant dans même territoire

Clinique : mobile (sd pénétration, suffocation), expulsé (pétéchies), enclavement

CAT : Toux/Heimlich/réa. Extraction bronchoscopie/centré spé (enfant)
\subsection{354 Détresse respiratoire aigüe}
\label{sec:org6724cf6}
PEC : \(O_2\), VNI/VI et investigation

Diagnostic 
\begin{itemize}
\item RX thorax anormale : urgence = pneumonie infectieuse, \oe{}dème pulmonaire
aigu, pneumothorax. Puis SDR, exacerbation path. infiltrative
\item Sinon clinique + gaz du sang : asthme aigü grave, EP, BPCO, anomalie paroi,
neuromusc, pneumothorax
\end{itemize}

SDRA : détresse respi < 7j + anomalie RX (opacités alvéolaire bilat diffuse)
sans défaillance cardiaque
\subsection{356 Pneumothorax}
\label{sec:org4be9689}
Primaire (jeune, longiligne, 0 patho, fumeur) VS secondaire (âgé, patho connue)

Clinique : douleur pleurale (\inc inspiration et otux)

Diag = RX face.

Ttt : sevrage tabac
\begin{itemize}
\item urgence si compressif (aiguille simple).
\item Sinon (mal toléré/grande taille) : exsufflation ou drain.
\end{itemize}

Prévention récidive : pleurodèse 

\section{Endocrinologie - Nutrition}
\label{sec:org90fea6e}
\subsection{32 Allaitement maternel}
\label{sec:org38a3110}
\subsection{35 Contraception (gynéco)}
\label{sec:org2f54e3e}
\begin{center}
\begin{tabular}{lll}
 & \oe{}stroprogestatif & Micro/macroprogestatifs\\
\hline
 & 1ère intention (Minidril), le plus efficace & 2eme intention\\
CI & K sein, HTA non contrôlée, thrombose & K sein, accident TEV récent\\
 & hépatopathie sévère, diabète & \\
Surveiller & chostérole, TG, glycémie & spotting\\
\end{tabular}
\end{center}

Urgence : lévonorgestrel < 72h
\subsection{37 Stérilité du couple (gynéco)}
\label{sec:orga8a0b33}
Infertilité : 0 conception à 1 an. Stérilité = définitif

\begin{itemize}
\item \female : anovulation (courbe de température), bilan hormonal, écho ovarienne, hystérographie.
\item \male : volume testiculaire, testostérone, spermogramme. Hormonal si oligo/azoospermie
\end{itemize}

\subsection{40 Aménorrhée (gynéco)}
\label{sec:org75b10ed}
Primaire
\begin{itemize}
\item pas dév. pubertaire :
\begin{itemize}
\item FSH, LH \dec : tumeur H-H\footnote{Hypothalamo-hypophysaire}, sd de Kallmann
\item FSH, LH \inc : sd de Turner (45X)
\item retard pubertaire simple (diag élimination)
\end{itemize}
\item examen gynéco + écho : hyperplasie congénitale des surrénales, anomalie
utéro-vaginale, anomalie sensibilité androne
\end{itemize}
Secondaire :
\begin{enumerate}
\item hCG pour éliminer grossesse
\item prolactine \inc : médicaments, IRM H-H (adénome prolactine, tumeur/infiltration)
\item LH \inc : écho ovaires (sd ovaires polykystiques)
\item estradiol, LH, FSH \dec (déficit gonadotrope) : tumeur/infiltration H-H(IRM) ou nutrition
\item \inc testostérone (insuffisance ovarienne) : scanner surrénales, écho ovarienne
\end{enumerate}
\subsection{47 Puberté (pédia)}
\label{sec:orgb2a28fe}
Retard : pas de seins > 13 ans ou pas de règle > 15 (\female), volume testicule
< 4mL après 14 ans(\male)
\begin{itemize}
\item hormonal (hypogonadotrope)
\begin{itemize}
\item FSH, LH \dec \footnote{"Cassure" courbe croissance} : IRM H-H (infiltratif, tumoral), nutrition, sport, sd Kallmann
\item FSH, LH \inc : caryotype (sd Turner si \female, Klinefelter \male)
\end{itemize}
\item retard pubertaire simple
\end{itemize}
Précoce < 8 ans \female{}, < 9.5 ans \male. Table \ref{tab:orge08904b}
\begin{table}[htbp]
\caption{\label{tab:orge08904b}Puberté précoce}
\centering
\begin{tabularx}{\textwidth}{ccXX}
Type & Diagnostic & Étiologie & Examens\\
\hline
centrale & FSH, LH \inc & ovaire/testicule, surrénale & écho pelvien/testic., test stimulation GnRH\\
périphérique & FSH, LH \dec & idiopathique, tumeurs SNC & imagerie cérébrale\\
\end{tabularx}
\end{table}
\subsection{48 Cryptorchidie (pédia)}
\label{sec:org46489bd}
Exploration : c. de Leydig, c. de Sertoli, FSH, LH + 17-hydroxyprogestérone si
bilatérale

Ttt chir (sinon = infertilité, hypogonadisme, K testiculaire)
\subsection{51 Retard de croissance (pédia)}
\label{sec:orgf337e29}
Définition : taille < -2DS ou ralentissement croissance ou \(<<\) parents

Cf table \ref{tab:org3117b18} + bilainVS, NFS, foie, rein.
\begin{table}[htbp]
\caption{\label{tab:org3117b18}Retards de croissance}
\centering
\begin{tabular}{ll}
Cause & Exploration\\
\hline
Constitutionnelle++ & \\
RCIU & \\
Déficit en GH, hypothyroïdie & GH, \{TSH, T4L\}\\
Maladie coeliaque, mucoviscidose & \{IgA, IgA anti-transglutamase\},test sueur\\
Os & radio\\
Retard pubertaire simple & \\
\end{tabular}
\end{table}
\subsection{69 Troubles des conduites alimentaires (à compléter)}
\label{sec:org15eb36f}
\subsection{78 Dopage}
\label{sec:orgb619925}
\emph{Stéroïdes anabolisant}, testostérone (\inc masse musculaire, puissance)

Autres : GH (\inc masse musculaire), IGF-1. Glucocorticoïdes, ACTH (antalgique, psychostimulant)
\subsection{120 Ménopause et andropause (gynéco, uro)}
\label{sec:org1e3ca70}
Ménopause :
\begin{itemize}
\item diag clinique : sd climatérique, aménorrhée \(\ge\) 1 an (bio si doute FSH \inc, oestradiol \inc)
\item ttt hormonale = \oe{}strogène et progestatif. Surveiller et réévaluation annuelle
\begin{itemize}
\item bénéfice : ttt sd climatérique, prévention ostéoporose
\item risque : \inc incidence K sein, \inc accidents TE veineux \thus \uline{CI}
\end{itemize}
\end{itemize}
Andropause : si testostérone < 2.3ng/mL\footnote{Ou si testostérone \(\in\) [2.3, 3.2] et SHBG et index T libre bas} :
\begin{itemize}
\item FSH, LH \inc = insuf testiculaire primitive (sd Klinefelter)
\item sinon hypogonadisme hypogonadotrope \thus IRM H-H pour adénome hypophysaire
\end{itemize}

\subsection{122 Troubles de l'érection  (uro)}
\label{sec:org3416c89}
Étiologies factorielles !
\begin{itemize}
\item psychogène (érections nocturnes) : rassurer, psychothérapie
\item diabète++, hypogonadisme++, hyperprolactinémie
\item vasculaire (HTA), chir pelvienne, anti-hypertenseur, neuro dégénératif, trauma médullaire
\end{itemize}
Bilan bio : glycémie jeun, testostéronémie \textpm{} prolactine, hormones thyroïdiennes

Ttt 
\begin{itemize}
\item hypogonadisme confirmé : androgènes (CI nodule prostatique, PSA > 3ng/mL)
\item 1ere intention : inhibiteurs phosphodiéstérase type 5 (Viagra) + stimulation
\end{itemize}
\subsection{124 Ostéopathies}
\label{sec:org0328191}
Ostéoporose secondaires de \female{} : endocrino++ (Table
\ref{tab:orgdbf215f}). \male{} : y penser si hypogonadisem, hypercortisolisme
\begin{table}[htbp]
\caption{\label{tab:orgdbf215f}Causes endocriniennes d'ostéoporose chez \female}
\centering
\begin{tabular}{lll}
Cause & Sous-cause & Ttt\\
\hline
Hypogonadisme & Anorexie mentale & Pilule \oe{}stroprogestative\\
 & Activité physique intense & Si aménorrhée, \dec activité ou \oe{}stroprogestatif\\
 & Patho. hypophysaire & \OE{}strogènes\\
 & Iatrogène & Bisphosphonates\\
 & Sd Turner & \OE{}Strogène + GH ou (adulte) \oe{}stroprogestatif\\
Hyperthyroïdie &  & Surveillance (ttt supressif) \textpm{} bisphosphonate\\
Hypercortisolisme/corticoïdes &  & Vitaminocalcique, bisphosphonates\\
\end{tabular}
\end{table}

\subsection{207 Sarcoidose}
\label{sec:org06474ca}
Penser à sarcoïdose hypothalamo-hypophysaire si diabète insipide.
Diag = défici endocrinien + infiltration HH à l'imagerie si sarcoïdose
connue. Sinon arguments de sarcoïdose (cf \hyperref[sec:org4ff6054]{item de pneumo})
\subsection{215 Hémochromatose}
\label{sec:orgd63eed5}
Suspicion clinique : Asthénie, Arthalgie, \inc ALAT (3 A)

Atteinte : foie (cirrhose++), diabète sucré++, hypogonadisme, chondrocalcilnose
articulaire, c\oe{}ur

Diag : CST \inc et ferritine \inc : chercher mutation C282Y sur HFE

Ttt dès CST \inc : saignées jusque ferritine < 50g/L. Dépistage parents 1er degré
\subsection{221 HTA : hyperaldostorénonisme primaire, sd de Cushing}
\label{sec:org06f889b}
\label{org4c5b3d3}
Iatrogène : \oe{}stroprogestatifs, corticoïdes, réglisse

Hyperminéralocorticisme primaire : 
\begin{itemize}
\item suspicion : HTA (\textpm{} résistante) + hypokaliémie
\item diag : aldostérone \inc et rénine \dec \footnote{Si aldostérone \inc et rérine \inc, hyperaldostéronisme secondaire
(sténose des artères rénale++)}
\item TDM ou IRM : 
\begin{itemize}
\item adénome de Conn\footnote{Nodule de la cortico-surrénale} \thus chir
\item idiopathique \thus spironolactone, antihypertenseurs
\end{itemize}
\end{itemize}

Phéochromocytomes\footnote{Tumeur de la médullo-surrénale}
\begin{itemize}
\item dépistage : HTA paroxystique, "triade de Ménard" = céphalées + sueurs +
palpitations, NEM2, NF1
\item diag : métanéphrine \inc
\item Imagerie puis chir
\end{itemize}
Sd de Cushing
\begin{itemize}
\item clinique : obésité androïde et graisses facio-tronculaire + vergétures,
ostéoporose + hyperandrogénie\footnote{Hirsutisme, acné}
\item diag = cortisolurie 24h \inc, test freinage minute négatif à DXM
\item étiologie :
\begin{itemize}
\item ACTH \dec \thus adénome surrénalien, cortico-surrénalome malin
\item sinon : test de freinage fort à DXM, test de stimulation ACTH \thus maladie
de Cushing (positif) ou sd paranéoplasique (négatif)
\end{itemize}
\end{itemize}
\subsection{238 Hypoglycémie}
\label{sec:org4f1453b}
Diag : neuroglucopénie\footnote{Faim brutale, trouble concentration, moteurs, sensitifs, visuels,
convulsion, confusion} et glycémie < 0.50g/L et correction à normalisation\footnote{Coma hypoglycémique possible}

Cause : 
\begin{itemize}
\item surdosag insuline++ chez diabétique \thus sucre si CS, sinon glucagon IM/SC
(ou perf glucose)
\item insulinome : diag par épreuve de jeune \thus chir
\end{itemize}
\subsection{239 Goitre, nodules thyroïdiens, cancers thyroïdiens}
\label{sec:org5653899}
Goître (hypertrophie thyroïdie) : évolue en multinodulaire (complications ?)
\begin{itemize}
\item étiologie : tabac, déficience iode
\item diag : TSH  et T4  (si TSH \inc: Ac anti-TPO et anti-TG  (auto-immunité))
\item ttt : ado = correction par levothyroxine , adulte = surveillance. Chir chez l'adulte si symptomatique, hyperfonctionnel,
morphologie suspecte (ou iode 131 si âgé)
\end{itemize}
Nodules (hypertrophie localisée thyroïdie) : doser TSH 
\begin{itemize}
\item si signes : hématocèle (brutal + douleur), thyroïdite subaigüe (douleur +
fièvre), cancer (compressif + ADP), toxique (hyperthyroïdie), thyroïdite
lymphocytaire (hypothyroïdie)
\item nodule isolé cf \ref{tab:org3318f03}. Selon cytologie : surveillance si bénin, chir
\end{itemize}
\begin{table}[htbp]
\caption{\label{tab:org3318f03}Orientation pour nodule isolé}
\centering
\begin{tabular}{lll}
TSH \dec & TSH N & TSH \inc\\
\hline
nodule hyperfonct ? & tumeurs & thyroiidite ?\\
scinti & écho, cytologie & Ac anti-TPO\\
\end{tabular}
\end{table}

Cancers thyroïdiens : 

\begin{itemize}
\item types : 
\begin{itemize}
\item différencié d'origine vésiculaire = Papillaire (85\%, excellent
pronostic), Vésiculaire (5\%, très/moins bon pronostic), Anaplasique (1\%, 15\%
survie 1 an)
\item Medullaire (5\%, 80\% survie 5 an)
\end{itemize}
\item ttt = thyroïdectomie totale \textpm{} curage ganglionnaire
\item si origine vésiculaire : iode 131 (après thyroïdectomie totale si haut risque),
L-T4 (si récidive)
\item si médullaire : chercher autres lésions NEM2\footnote{Phéochromocytome, hyperparathyroïdie}
\end{itemize}

\subsection{240 Hyperthyroïdie}
\label{sec:org0d62149}
Clinique : sd thyrotoxicose = asthénie, amaigrissement, sueurs, CV (tachycardie)

Diagnosic : TSH \dec puis T4L \inc

\begin{table}[htbp]
\caption{Étiologies}
\centering
\begin{tabular}{lll}
Étiologie & Clinique & Diagnostic\\
\hline
\emph{maladie de Basedow} & goitre, oculaire (exophtalmie) & oculaire\\
 &  & ou écho + Ac anti récepteur TSH \tablefootnote{Scinti: fixation homogène diffuse}\\
\emph{goître multinodulaire toxique} &  & Scinti: en damier\\
\emph{adénome toxique} &  & Scinti: hyperfixation (reste = "froid")\\
De quervain & virale, goitre dur, douloureux & clinique\\
\end{tabular}
\end{table}

Ttt :
\begin{itemize}
\item \(\beta\)-bloquants, contraception, \emph{anti-thyroidiens de synthèse} [Neomercazole] (\danger
agranulocytose\footnote{Arrêt si neutrophiles < 1G/L})
\item spécifique : 
\begin{itemize}
\item cardiothyréose = propranolol, anticoag, ATS, chir/radio-iode
\item Crise aigüe thyrotoxique  ATS, pronanolol, corticoïdes puis iode
\item Orbitopathie : pas d'ATS, ni d'iode !
\item enceinte : surveillance/ATS/thyroïdectomie selon
\end{itemize}
\end{itemize}
\subsection{241 Hypothyroïdie}
\label{sec:org2125b9e}
Clinique : sd myxoedemateux\footnote{Faciès "lunaire", voix rauque, macroglossie, hypoacousie}, hypométabolisme\footnote{Asthénie, somnolence, hypothermie}

Diag = TSH \inc. Puis doser T4. Étio : Ac anti-TPO, échographie

Étiologies :
\begin{itemize}
\item atteinte thyroïde
\begin{itemize}
\item thyroïdite d'Hashimoto : Ac anti-TPO. Écho = hypoéchogène, hétérogène
\item thyroïdite atrophique (pas de goitre), du post-partum
\item carence iode (endémie), iatrogène (interféron)
\end{itemize}
\item atteinte hypothalamo-hypophysaire \thus IRM
\end{itemize}

Complications : insuf cardiaque, coma myxoedemateux  

Ttt : lévothyroxine \footnote{Obj: TSH \(\in\) [0.5, 2.5]mUI/L.}. Surveiller TSH (primaire) ou T4L (atteinte H-H) !

\subsection{242 Adénome hypophysaire}
\label{sec:org673e880}
Découverte : sd tumoral = céphalée, hémianopsie bi-temporale, apoplexie
hypophysaire (rare mais urgence !)

IRM (référence) : microadénome (hypointense, non rehaussé à l'injection) ou
macroadénome (> 10mm, rehaussé injection)

Hypersécrétion par l'hypophyse :
\begin{itemize}
\item prolactine : galactorrhée, spanioménorrhée. Diagnostic = 
\begin{itemize}
\item vérifier hyperprolactinémie, éliminer grossesse,  médicaments, hypothyroïdie
périph, insuf rénale
\item microadénome : positif. Sinon, test agoniste dopaminergique
\end{itemize}
\item GH (acromégalie) : sd dysmorphique\footnote{Élargissement extrémité, visage (prognathisme)}, HTA
\begin{itemize}
\item complication : insuf cardiaque !, diabète
\item diagnostic : pas de freinage de GH à l'HGO et \inc{} IGF-1
\end{itemize}
\item glucocorticoïdes (indirectement) : cf \hyperref[org4c5b3d3]{HTA et Cushing}
\end{itemize}

Insuffisance hypophysaire : table \ref{tab:orgd0747bc}
\begin{table}[htbp]
\caption{\label{tab:orgd0747bc}Insuffisance hypophysaire}
\centering
\begin{tabular}{ll}
gonadotrope & oestradiol/testostérone \dec ou FSH, LH \dec\\
corticotrope & cortisolémie \dec, synacthène (aldostérone normale !)\\
thyréotrope & T4l \dec mais TSH normale\ldots{}\\
somatotrope & (GH) stimulation GH négative\\
\end{tabular}
\end{table}
\subsection{243 Insuffisance surrénale}
\label{sec:org8824ecc}
Chronique : 
\begin{itemize}
\item clinique = asthénie, anorexie, hypotension. Hyperk, hypoNa\footnote{Aldostérone : réabsorption Na+ et sécrétion K+} mélanodermie
(surrénale), pâleur (hypophyse)
\item ttt sans attendre diag : hydrocortisone, fludrocortisone \footnote{Glucocorticoïdes et minéralocorticoïdes respectivement.} + cause
\item diag : cortisol \dec, ACTH \inc si primaire
\item étiologies : 
\begin{itemize}
\item primaire (surrénale) = \emph{autoimmune} 80\%\footnote{Doser Ac anti-21 hydroxylase}, \emph{tuberculose} des surrénale 
10\%\footnote{Calcifications aux scanner}), VIH, iatrogène, métastases bilatérales
\item secondaire (hypophyse) = interruption corticothérapie prolongé
\end{itemize}
\end{itemize}
IS aigüe = urgence 
\begin{itemize}
\item déshydratation extracellulaire, confusion, fièvre. Bio : hémoconcentration,
hypoNa, hyperK, hypoglycémie
\item hydrocortisone 100mg puis \faHospital : NaCL et facteur déclenchant (IS chronique++)
\end{itemize}
\subsection{}
\label{sec:org1660814}
\subsection{244 Gynécomastie}
\label{sec:org507a917}
Palpation : ferme, mobile, centré mamelon. Si doute, mammographie pour élimiter
K et adipomastie

Physiologique : NN, ado (< 20 ans), > 65A (palper testicules !)

Étiologies : médicaments, idiopathique (25\%), cirrhose, insuf
testiculaire/gonadotrope (8\%), tumoral (rare).
\begin{itemize}
\item bilan hormonal si non évident
\end{itemize}

Ttt : cause. Androgène si idiopathique voire chir plastique
\subsection{245 Diabète sucré}
\label{sec:org7f87069}
Glycémie jeun \(\ge\) 1.26 g/L \time 2 ou (\(\ge\) 2g/L + signes d'hyperglycémie)
\subsubsection{Diabète 1}
\label{sec:org666b67d}
insulinopénie (destruction c. \(\beta\) pancréas) auto-immun++ ou idiopathique

Début brutal, sujet jeune, sd cardinal (polyuro-polydipsie, amaigrissement,
polyphagie\}, acidocétose (cétonurie)

Diagnostic : hyperglycémie + triade \{cétonurie, < 35 ans\} ou auto-Ac

PEC : 
\begin{itemize}
\item insuline à vie : lent et rapide (3-4)
\item objectif HbA1c < 7\%
\item surveillance : glycémie 4/jour
\item CS : ophtalmo 1/an, dentiste 1/an, diabétologique 3/an \textpm{} cardio 1/an
si sympto/âgé/compliqué
\end{itemize}

Dépistage pendant grossesse !

\subsubsection{Diabète 2 (90 \%)}
\label{sec:orga9ed0bf}
Insulinorésistance \uline{et} déficit insulinosécrétoire

Découverte fortuite (asymptomatique longtemps). 
Dépistage : clinique d'hyperglycémie, sd métabolique\footnote{IMC > 28 kg/m\textsuperscript{2}, HTA, HDL < 0.35g/L ou TG > 2g/L ou dyslipidémie, ATCD
diabète (familiale, gestationnel)}

PEC :
\begin{itemize}
\item objectif HbA1c < 7\%
\item activité physique \footnote{Sauf insuf. coronarien, rétinopathie proliférante non stabilisée}, alimentation équilibrée sans sucres rapides
\item metformine (sinon sulfamide, inhibiteurs DPP-4, inhibiteurs
\(\alpha\)-glucosidase)
\item + insuline et HGO si insulinorequérance
\end{itemize}

\subsubsection{Complications}
\label{sec:orgfe90cd1}
\OE{}il : voir \hyperref[orgc28cb24]{partie ophtalmo}

Rein : diabète = 1ere cause d'IR terminale. 
\begin{itemize}
\item dépistage 1/an chez D2 : BU (protéinure), albuminure/créatinurie
\item ttt : prévention : diabète, HTA. Puis cf tab \ref{tab:org940e1d4}
\end{itemize}
\begin{table}[htbp]
\caption{\label{tab:org940e1d4}2 premiers stade de néphro diabètes = asymptomatique}
\centering
\begin{tabular}{rll}
Stade & type & Ttt\\
\hline
3 & microalbuminurie & Obj: HBA1c < 7\%, PA < 140/85 mmHg\\
4 & macroalbuminurie & IEC/sart + diurétique thiazidique\\
5 & IR avec DFG < 30/mL/min/1.73m\textsuperscript{2} & insuline, répaglinide, inhib. \(\alpha\)-glucosidase\\
\end{tabular}
\end{table}

Neuropathies
\begin{itemize}
\item sensorimotrice : polynévrite symétrique distale++ (hypoesthéie, 0 ROT achilléen, bilat , douleur
neurogène)
\item autonome : digestiv (gastroparésie), dysfonction érectile, parésie vésicale
\item dépistage : examen neuro (pied !!), ECG annuel
\item ttt : préventif++ (glycémie, alcool, tabac\ldots{})
\end{itemize}
Macroangiopathie : 2/3 DC pour cause CV
\begin{itemize}
\item ischémie myocardite silencieuse !!, AOMI
\item prévention : glycémie (metformine), activité physique, LDL (statines),
aspirine, anti-hypertenseurs, poids, 0 tabac
\end{itemize}
Mal perforant plantaire : creux autour d'hyperkératose
\begin{itemize}
\item examen réguliers des pieds et chaussures quotidien
\item ttt : décharge, excision kératose / parage et drainage + ATB si infection \textpm{}
revascularisation si nécrose
\item risque d'ostéite
\end{itemize}
Autres : 
\begin{itemize}
\item infections \thus examen cutané++, stomato, uro-génital, respi
\item dentiste tous 6 mois
\end{itemize}
Complications métabolique :
\begin{itemize}
\item coma cétoacidosique 
\begin{itemize}
\item diag= cétonémie, cétonurie
\item ttt insuline rapide IV, recharge volumique, K+ \textpm{} glucose, cause
\end{itemize}
\item coma hyperosmolaire : 
\begin{itemize}
\item diag = glycosurie, cétonurie (BU) et hyperglycémie (dextro)
\item ttt : réhydratation lente, insuline IV, héparine, cause
\end{itemize}
\item hypoglycémie : inévitable, non mortelle. Cf \hyperref[sec:org4f1453b]{item 238}
\end{itemize}
\subsection{246 Prévention par la nutrition}
\label{sec:orga976228}
\subsection{247 Modifications thérapeutiques du mode de vie}
\label{sec:org5a76e5d}
\subsection{248 Dénutrition (à compléter)}
\label{sec:org0f59f67}
\subsection{249 Amaigrissement (à compléter)}
\label{sec:org07c46a4}
Vérifier l'amaigrissement !

Causes endocrino : insuf. surrénale (primaire/secondaire)diabète, hyperthyroïdie, hypercalcémie
\subsection{250 Troubles nutritionnels chez sujet âgé (à compléter)}
\label{sec:orgb39cad6}
\subsection{251 Obésité (à compléter)}
\label{sec:org55955ff}
IMC \(\ge\) 30 kg/m\textsuperscript{2} (grade 1 si < 35, grade 3 si \(\ge\) 40)

Étiologie : génétique, communes++ (déséquilibre apport-dépense).

Complications : \inc RR mortablité, métabolique, CV, respi, ostéoarticulaire, digestive, rénale,
gynéco, cutanée, néoplasique, psychosociale

Interrogatoire. Ttt = diététique, activité physique, psychologique + orlistat si
IMC \(\ge\) 30 (ou 27 et comobridité). Chir bariatrique en 2eme intention.

\subsubsection{Enfant/ado}
\label{sec:org57c5557}
IMC > 25 kg/m\textsuperscript{2}

Étiologie : commune++ (facteur env., prédisposition génétiuqe), génétique,
secondaire

Complication : HTA, insulinorésistance, stéatose hépatique non alcoolique,
articulaire, psycho

Interrogatoire + EC. Ttt = prévention
\subsection{252 Diabète gestationnel + nutrition et grossesse (à compléter)}
\label{sec:org0365044}
Pré-gestationnel
\begin{itemize}
\item risque f\oe{}tus : fausses couches, malformations congénitales, mort f\oe{}tales,
\item risque mère : HTA ++, rétinopathie, néphropathie
\item avant grossesse : HbA1c < 7\%, glycémie \(\in\) [0.7, 1.20] préprandial et [1,
1.14] en post (\inc insuline si DT1, +insuline si DT2
\item pendant : 6 glycémie/jour pour équilibre
\end{itemize}
Dépister diabète gestationnel ssi FR : \{\(\ge\) 35 ans, IMC \(\ge\) 25 kg/m\textsuperscript{2}, ATCD DG,
DT parents 1er degr\}
\begin{itemize}
\item PEC si glycémie jeun \(\ge\) 0.92g/L (début), sinon teste 24-28SA avec HGO
\item ttt : diét., \textbf{pas} d'antidiabétique, surveillance
\end{itemize}

\subsection{253 Nutrition chez le sportif}
\label{sec:org5629b1b}
Examen d'aptitude : ATCD familiaux (CV),  EC complet, ECG d'effort

Bénéficices du sport : 150min/semaine (ou 75min si intense) chez
l'adulte. 60min/jour chez l'enfant + renforcement
\begin{itemize}
\item adulte : 
\begin{itemize}
\item prévention = K (colon, sein), CV, métabolique, ostéoporose \female,
\item maintien = \dec mortalité prématuré, \inc autonomie
\item traite = cardiomyopathie, ischém, BPCO, obésité, diabète 2 neuro,
rhumatismal, dégénératif
\end{itemize}
\item enfant : dev psychosocial, dev psychomotoeur, prévient sd métabolique,
surpoids, CV, \inc masse maigre et densité osseuse
\end{itemize}

Bseoins :
\begin{itemize}
\item glucides (50\%) : IG faible à distance, élevé juste avant. reconstituer stocks après
\item lipides (30\%) à limiter si compétition
\item protéines (20\%)
\item calcium 1g/j, Fer 10-15mg/j, vit D 5 \(\mu\)g/j \thus surveiller chez l'enfant
\end{itemize}
\subsection{265 Hypocalcémie, dyskaliémie, hyponatrémie}
\label{sec:orgaa220ce}
Hypocalcémie :
\begin{itemize}
\item clinique : troubles du rythme, parésthésie, tétanie
\item causes hypoparathyroïdie, anomalie vitamine D (carence, malabsorption\ldots{})
\item ttt : aigü  calcium IV lente. Chronique : vitamine D et calcium
\end{itemize}
Hyperkaliémie (risque cardiaque )
\begin{itemize}
\item étiologies : hypoaldostéronisme (IS) , acidose  métabolique \footnote{par \dec élimination rénale, par sortie du K+ de la cellule respectivement.}
\end{itemize}
Hypokaliémie (risque cardiaque )
\begin{itemize}
\item étiologie : excès d'insuline, hyperaldostéronisme, dénutrition sévère
\end{itemize}
Hyponatrémie endocrinienne (hospitalisé++)
\begin{itemize}
\item étiologie selon osmolalité : si \inc, hyperglycémie, si N, hyperTG,
hyperprotidémie. Si \dec, "vraie" hyponatrémie : table \ref{tab:orga8ac5b0}
\end{itemize}
\begin{table}[htbp]
\caption{\label{tab:orga8ac5b0}hyponatrémie "vraie"}
\centering
\begin{tabular}{lll}
volémie \inc & IC, cirrhose, sd néphrotique & Sérum isotonique\\
volémie R & hypothyroïdie, insuf corticotrope, SIADH & Sérum hypertonique\\
volémie \dec & perte digestive, rénale, insuf corticosurrénale aigüe & Restriction hydrosodée\\
\end{tabular}
\end{table}


\subsection{266 Hypercalcémie}
\label{sec:org2935cf0}
Étiologie :
\begin{itemize}
\item PTH \inc ou N
\begin{itemize}
\item hyperparathyroïdie primaire (55\%) : ttt = chir conventionnelle
\item hypercalcémie-hypocalciurie familiale bénigne
\end{itemize}
\item PTH \dec
\begin{itemize}
\item maligne (30\%)
\item iatrogène, granulomatose
\end{itemize}
\end{itemize}
Ttt (hors hyperparathyroïdie primaire) : bsiphosphonates, calcimimétique
\subsection{303 Tumeurs de l'ovaire (hormono-sécrétante)}
\label{sec:org797547a}
Sécrète des \oe{}strogènes : tumeurs de la granulosa++ : 
\begin{itemize}
\item sympto : pseudo-puberté précoce ou aménorrhée/ménométrorragie ou saignement vaginal
\item ttt = ovariectomie unilatérale
\end{itemize}
Sécrète des androgènes : tumeurs à c. de sertoli-Leydig (Hirsutisme, ttt
conservateur), à c. de Leydig (virilisante, ttt = ovariectomie bilat),
germinales sécrétantes \footnote{Si sécrète hCG : ttt = conservateur/chimie, si sd
Turner : ttt = gonadectomie préventive}
\subsection{305 Tumeurs du pancréas (endocrine)}
\label{sec:org3942bb4}
Clinique selon sécrétion (insuline, gastrine, ACTH, glucagon, VIP, GHRH)
\subsection{310 Tumeurs du testicule (aspects endocriniens)}
\label{sec:org3511abe}
Clinique : pseudo-puberté précoce/gynécomastie

Sécrète testostérone/\oe{}stradiol ou gonadotrophine chorionique/hCGA (tumeurs
germinales)

Diag = palpation testiculaire + écho

Ttt = chir (glucocorticoïdes si inclusions surrénaliennes)

\section{Ophtalmo}
\label{sec:org4c0f288}
\subsection{1 Sémiologie oculaire}
\label{sec:orgfb47abe}
Anatomie 
\begin{itemize}
\item Membrane externe (cornée, conjonctive, sclère), uvée (iris, corps ciliaire,
choroïde), rétine
\item Contenu : humeur aqueuse (chambre antérieure), cristallin, corps vitré
(postérieur)
\item Voie optique : nerf optique - chiasma - bandelettes - corps genouillés
externes - radiations optiques - cortex occipital
\item Nerfs oculomoteurs : IV = oblique sup, VI = droit externe et les autres = III
\end{itemize}
Examens :
\begin{itemize}
\item AV (Parinaud = près, Monoyer = loin)
\item lampe à fente (segment antérieur), gonioscopie (angle iridocornéen)
\item pression intraoculaire par tonomètre à pair pulsé (hypertonie si \(\ge\) 22mHg)
\item FO
\end{itemize}
Complémentaires
\begin{itemize}
\item CV (périmétrie statique = dépistage glaucome), couleurs
\item angiographie (DMLA)
\item électrorétinogramme, potentiels évoqués visuels (SEP), électro-oculogramme
\item écho (mode A = longeur, mode B = décollement rétine)
\item OCT (glaucome chronique, macula : DLMA)
\end{itemize}
\subsection{2 Réfraction}
\label{sec:org517a374}
\OE{}il = 60 dioptries (42 cornée, 20 cristallin).

Myope = trop convergent \thus verre concave/chir. Hypermétrope = pas assez convergent
\thus verre convexe/chir. Astigmate \thus verres cylindriques/chir
\subsection{3 Suivi d'un nourisson}
\label{sec:org377e670}
Dépister enter 9 et 12 mois : réfraction, strabisme + réfraction après
cycloplégie (amblyopie).
\subsection{4 Strabisme de l'enfant}
\label{sec:org9cad7ec}
= symptôme !

Examen motilité (paralytique vs motilité normale)
Ttt : correction optique \textpm{} occlusion \oe{}il si amblyopie (chir si persistance)
\subsection{5 Diplopie (binoculaire)}
\label{sec:org9c72580}
Clinique : dédoublement même direction, disparaît à l'occlusion d'un \oe{}il

Exament : motilité, cover-test, verre rouge, Lancaster

Étiologie :
\begin{itemize}
\item anévrisme intracrânien++ (jeune, céphalée, 0 FR vasc),
\item tumeur (25\%), SEP (10\%) révélee, myasthénie
\item autres : accidents vasculaires ischémique/hémorragiques
\end{itemize}
\subsection{6 \OE{}il rouge/douloureux}
\label{sec:orga00098a}
BAV, douloureux : cf \ref{org3776bac}.

Sans BAV, douloureux : 
\begin{itemize}
\item épisclérite (ttt = corticoïdes)
\item sclérite (étio = polyarthrite rhumatoïde)
\item conjonctivite : allergique (collyre), virale++ (ttt =0)
\item sd sec oculaire++ (ttt = substituts lacrymaux)
\end{itemize}

Sans BAV, sans douleur : 
\begin{itemize}
\item conjonctivite (bactérienne : collyre antiseptiqu)
\item hémorragie sous-conjonctivale (HTA, trouble coag ?)
\end{itemize}

\begin{table}
\caption{Avec BAV, douloureux: étiologies}
\centering
\adjustbox{max width=\linewidth}{
\begin{tabular}{llll}
Type & Symptômes & Étiologie & Ttt\\
\hline
Glaucome aigu par FA & Douleur++, "bille de verre" &  & inhib. anhydrase carbonique, solutés hyperosmolaires\\
 &  &  & hypotoniques, myotique (collyre)\\
Uvéite antérieure aigǜe & Synéchies iridocristaliniennes & inconnue (50\%) & \\
 & Tyndall\tablefootnote{Protéines, c. inflammatoires dans l'humeur aqueuse} & Spondylarthrie ankylosante & \\
 & précipités rétro-cornéens & herpès, sarcoïdose & mydriatiques, corticoïdes (collyre)\\
Kératite aigüe & Ulcération (à la fluorescine) & Herpès, bactéries & \\
Glaucome NV & Néovaisseaux sur l'iris &  & Hypotonisants, photocoag/anti-VEGF\\
Endophtalmie post-op & \oe{}dème palpébral, hyalite\tablefootnote{Inflammation du corps vitré} &  & \\
\end{tabular}
}
\label{org3776bac}
\end{table}

\subsection{7 Altération de la fonction visuelle}
\label{sec:org21384b8}
BAV progressive
\begin{itemize}
\item transparence anormale : cataracte (ttt = chir), cornée, vitré
\item sinon : 
\begin{itemize}
\item atteinte nerf optique : glaucome chronique à AO (ttt = hypotonisants,
chir), neuropathies toxiques
\item atteinte rétine = rétinopathies pigmentaires, ou macula = DMLA,
maculopathies héréditaires, \oe{}dème maculaire, antipaludéens, séparation vitré/macula
\end{itemize}
\end{itemize}
Altération CV
\begin{itemize}
\item ateinte rétine (scotomes centrau, déficits périph)
\item atteinte nerf optique : SEP, NOIA, toxique
\item atteinte chiasma optique : adénome hypophysaire
\item atteinte rétrochiasmatique : vasc, tumoral, trauma
\end{itemize}
\subsection{8 Anomalies de la vision d'apparition brutale}
\label{sec:orgdf221ea}
BAV, \oe{}il rouge douloureux : cf \hyperref[sec:orga00098a]{chap 6}

BAV, \oe{}il blanc indolore :

\begin{center}
\begin{tabular}{lll}
FO non visible & hémorragie intra-vitréenne & Écho B\\
 & uvéite du vitré & \\
FO visible anormale & occlusion artère centrale rétine & mydriase aréflexique\\
 &  & macula rouge cerise\\
 &  & (urgence si Horton)\\
 & occlusion veine centrale rétine & \oe{}dème papillaire\\
 &  & hémorragie rétiniennes dissémines\\
 &  & nodules cotonneux\\
 & DMLA & métamorphopsies brutale\\
 & décollement rétine rhegmatogène & myodésopsies, phosphène\\
 &  & (semi-urgence)\\
 & neuropathie optique ischémique antérieure & \dec RPM direct\\
 &  & \oe{}dème papillaire\\
FO visible normale & névrite optique rétrobulbaire & scotome central\\
Cécité monoculaire transitoire &  & \\
\end{tabular}
\end{center}
\subsection{9 Prélèvement de cornée}
\label{sec:orgea39042}
Faire sérologies HIV, HTLV, VHB, VHC, syphilis

CI : locale, infectieuses (sida, rage, Creutzfeld-Jakob, hépatite), neuro
inexpliqué, démence
\subsection{10 Greffe de cornée}
\label{sec:org291d0d2}
Transfixiantes ou juste endothelium. 

Indications : trauma, kératocône, kératite herpétiques/infectieuse
\subsection{11 Traumatismes oculaires}
\label{sec:org77febaf}
Globe ouvert, corps étranger : si doute TDM (\danger pas d'IRM si corps
étranger)

Chir en urgence : recherche/suture plaie du globe, extraction CE intraoculaire,
plaie du cristallin
\subsection{12 Brûlures oculaires}
\label{sec:org9c538b5}
Brûlures basiques = grave \danger

Ttt urgence = lavage 20min au sérum phys puis collyre corticoïdes 
\subsection{13 Cataracte}
\label{sec:org28d6cf2}
Opacification du cristallin.

Diag clinique : BAV progressive en vision de loin, lampe à fente

Étiologies : âge, diabète, crorticoïdes, traumatique

Ttt = chir (extraction par phacoémulsification puis implant en chambre
postérieure)

Complications : endophtalmie (ATB !), décollement rétine, \oe{}dème maculaire
\subsection{14 Glaucome chronique}
\label{sec:org4aa2b2b}

Glaucome primitif à pression élevée (\(\approx\) occidentaux), ou pression \uline{normale}
(\(\approx\) asiatique)

FR: âge, hypertonie oculaire. Physio : perte accélée des fibres optiques.

Caractéristiques : \inc excavation papille, altération CV

Ttt à vie : prostaglandine ou \(\beta\)-bloquants\footnote{\inc élimination humeur aqueuse et \dec sécrétion respectivement.} (collyre), laser
(trabéculoplastie) ou chir (trabéculectomie) possible
\subsection{15 Dégénérescence maculaire liée à l'âge}
\label{sec:orga35d626}
1ere cause de malvoyance après 50 ans.

Diag = AV, FO, OCT (cf tab \ref{org6dea95e})
\begin{table}
\caption{Formes de DMLA}
\centering
\adjustbox{max width=\linewidth}{
\begin{tabular}{llll}
Forme & Clinique & FO & Ttt\\
\hline
Débutante &  & \emph{drusen}\tablefootnote{Petites lésions profondes jaunes} & vit E, C, zinc, lutéine, zéaxantine\\
Atrophique & BAV sévère, scotome central & atrophie épith. pigment. & \(\emptyset\)\\
Exsudative & BAV, métamorphopsies brutales & \oe{}dème intrarétienne & anti-VEGF\\
(néovaisseux ss rétine) &  & décollement maculaire exsudatif & \\
\end{tabular}
}
\label{org6dea95e}
\end{table}
\subsection{16 Occlusions artérielles rétiniennes}
\label{sec:org9b429b0}
Artère centrale de la rétine 
\begin{itemize}
\item Diag = 
\begin{itemize}
\item BAV brutale, \oe{}il blanc indolore, mydriase aréflective
\item FO : macula "rouge cerise"
\end{itemize}
\item Étiologie : embolies (athémore carotidien), maladie de horton = urgence
\item CAT = urgence . Très mauvais pronostic fonctionnel
\begin{itemize}
\item ttt de l'étio (antiagrégant/anti vit-K)
\end{itemize}
\end{itemize}
Branche de l'ACR : pronostic visuel bon (amputation), pas d'Horton
\subsection{17 Occlusions veineuses rétiniennes}
\label{sec:org7474813}
Veine centrale de la rétine
\begin{itemize}
\item Diag (facile) : BAV brutale, FO = dilatation veineuse, nodules cotonneux,
hémorragies disséminées, \oe{}dème papillaire
\item FR : > 50 ans avec FR vasc, hypertonie oculaire
\item 2 formes (angiographie fluorescine)
\end{itemize}

\begin{table}
\caption{Formes d'OVR}
\centering
\adjustbox{max width=\linewidth}{
\begin{tabular}{lll}
Forme & Évolution & TTT\\
\hline
non ischémique & normalisation & anti-VEGF, surveillance mensuelle\\
 & ou ischémique & \\
ischémiques & pas de récupération fonct. & PPR\tablefootnote{Photocoagulation panrétinienne}\\
 & glaucome NV \danger & PPR en urgence\\
 &  & \\
\end{tabular}
}
\label{org48cfef9}
\end{table}

Branche veineuse rétinienne : évolution favorable / hémorragie du vitré (pas
  de GNV !)
\subsection{18 Pathologies des paupières}
\label{sec:org17e0b53}
Cf Table \ref{tab:org4452f44}

Autres : entropion (sénile/paralysie VII), ectropion, ptosis (penser anévrisme
IC ), lagophtalmie

\begin{table}[htbp]
\caption{\label{tab:org4452f44}Pathologies des paupières}
\centering
\begin{tabular}{llll}
Orgelet & follicule du cil & infection bactérienne & ATB 8 j\\
Chalazion & glande Meibomius & inflammation & corticoïde\\
\end{tabular}
\end{table}
Tumeurs malignes : carcinome basocellulaire/épidermoide, mélanique, carcinomes sébacés
\subsection{19 SEP}
\label{sec:orgc9fffe7}
20\% inaugurant SEP. 50\% de SEP à 15 ans. Formes :
\begin{itemize}
\item neuropathie optique : 
\begin{itemize}
\item BAV variable, douleurs rétro-oculaire, pupille Marcus-gunn
\item scotome (caeco) central, dyschromatopsie d'aexe rouge-vert
\item IRM
\item Ttt = corticoïdes + (SEP) interféron
\end{itemize}
\item paralysie du VI, nystagmus, périphlébite rétinienne
\end{itemize}
\subsection{20 Neuropathie optique ischémique antérieure}
\label{sec:orgdfcb5dd}
Diag = BAV brutale indolore
\begin{itemize}
\item FO : \oe{}dème papillaire. Examen CV ++ (déficit altidudinal)
\end{itemize}

Étiologies
\begin{itemize}
\item artériosclérose (freq) : FR. Pas de ttt efficaces
\item maladie de Horton = urgence . Bio = VS \inc, CRP \inc \thus
corticothérapie générale forte dose
\end{itemize}
\subsection{21 Rétinopathie diabétique}
\label{sec:org67bc131}
\label{orgc28cb24}
 1ère cause de cécité chez < 50 ans. 30\% des diabétique
\begin{itemize}
\item maculopathie diabétique
\item rétinopathie diabétique : non proliférante (hémorragies rétiniennes dans 4
quadrans/dilatations veineuses 2 quadrants/AMIR 1 quadrant) puis proliférante (néovaisseaux)
\item dépistage annuel FO, OCT
\item ttt : équilibre glycémie et TA++. Photocoagulation panrétinienne, anti-VEGF
\end{itemize}
\subsection{22 Orbitopathie dysthyroïdienne}
\label{sec:orgf52d2b2}
Maladie de Basedow++, thyroïdite d'Hashimoto

Manif = exophtalmie (bilatérale, axile, non pulsatile, indolore), rétraction
paupière, diplopie

TSH effondrée, scanner, IRM, \{CV, couleurs, PEV\}
Ttt 
\begin{itemize}
\item médical : hyperthyroïdes, anti-inflammatoire si score CAS \(\ge\) 3
\item si grave : corticothérapie générale, chir
\end{itemize}
\subsection{23 Rétinopathie et choroïdopathie hypertensive}
\label{sec:org136b1b2}
Rétinopathie hypertensive : si HTA sévère.
\begin{itemize}
\item hémorragie en flammèches superficielles, \oe{}dème maculaire
\item hémorragies profondes, rondes, nodules cotonneuxs
\item pas BAV
\end{itemize}
Choroïdopathie hypertensive : nécrose de l'épithelium pigmentaire, cicatrices
(décollement de rétine exsudatif si sévère)

Artériosclérose : asymptomatique, irréversibles. Signe du croisement, \inc reflet artériolaire
\end{document}
