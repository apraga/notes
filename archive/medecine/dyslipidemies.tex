% Created 2021-02-23 Tue 23:37
% Intended LaTeX compiler: pdflatex
\documentclass[11pt]{article}
\usepackage[utf8]{inputenc}
\usepackage[T1]{fontenc}
\usepackage{graphicx}
\usepackage{grffile}
\usepackage{longtable}
\usepackage{wrapfig}
\usepackage{rotating}
\usepackage[normalem]{ulem}
\usepackage{amsmath}
\usepackage{textcomp}
\usepackage{amssymb}
\usepackage{capt-of}
\usepackage{hyperref}
\author{Alexis}
\date{\today}
\title{Dyslipidémies}
\hypersetup{
 pdfauthor={Alexis},
 pdftitle={Dyslipidémies},
 pdfkeywords={},
 pdfsubject={},
 pdfcreator={Emacs 27.1 (Org mode 9.5)}, 
 pdflang={English}}
\begin{document}

\maketitle
\tableofcontents

Risque CV athéromateux : LDL haut, HDL bas et selon hyperTG (facteurs associés: surpois, diabète, HDL bas)
Bila lipidique (EAL) selon formule de Friedwal : LDL = CT - HDL - TG/5 (en g/L) ou TG/2.2 si mmol/L
Normal = \textbf{LDL < 1.6g/L, HDL < 0.4g/L, TG > 1.5g/L}

Hyperlipidémies secondaires.
\begin{center}
\begin{tabular}{lll}
Étiologies & Diagnostique & Type d'hyperlipidémie\\
\hline
Hypothyroïdie & TSH & HCH/HLM\\
Cholestase & Bilirubine, phosphatase alcaline & HCH\\
Syndrome néphrotique & Protéinurie, œdèmes & HLM\\
Insuffisance rénale chronique & Créatinine & HTG/HLM\\
Alcoolisme & Interrogatoire & HTG\\
Diabète & Glycémie, HbA1c & HTG\\
Hyperlipidémie iatrogène & Interrogatoire & \\
Œstrogènes & `` & HTG\\
Corticoïdes & `` & HLM/HTG\\
Rétinoïdes & `` & HTG\\
Antirétroviraux & `` & HTG\\
Ciclosporine & `` & HCH/HLM\\
Diurétiques, bêtabloquants & `` & HTG modérée\\
\end{tabular}
\end{center}

\begin{center}
\begin{tabular}{ll}
Hypercholestérolémies familiales & \emph{mutation du gène du LDL-récepteur} (fréquent++)\\
monogéniques & -  hétérozygote : fréquente (1/500), LDL-C : 2 - 4 g/L,\\
 & \textbf{xanthomes tendineux, arc cornéen prématuré}, risque CV élevé\\
 & -  homozygote : \textbf{exceptionnelle} (1/1 000 000),  LDL-C : >  5 g/L\\
 & , xanthomes dès l'enfance, \textbf{grave} (RA\ldots{} dès la 1re décennie)\\
 & \emph{Mutation gène de l'apolipoprotéine B} (apoB), gène proétine \emph{PCSK9} (rare)\\
\hline
Hypercholestérolémies polygéniques & \textbf{Très fréquente}, LDL augmenté +/- hyperTG. Risque CV selon LDL et autres fR\\
\hline
Hyperlipidémie familiale combinée & fréquente (1 à 2 \% )\\
 & hypercholestérolémie/hyperTG/mixte, risque CV variable\\
\hline
Dysbêtalipoprotéinémie ( III) & rare,  prédisposition génétique  +  surpoids, diabète, hypothyroïdie, ttt\\
 & LDL et TG élevé\\
 & \textbf{xanthomes plans palmaires,  xanthomes tubéreux jaune orangé} = caractéristiques\\
 & risque CV élevé\\
\hline
Hypertriglycéridémie familiale & rare, risque athérogène incertain.\\
Hyperchylomicronémies primitives & très rares. TG >  10 g/L voire 100 Risque de pancréatite aiguë++\\
\end{tabular}
\end{center}

\begin{center}
\begin{tabular}{llll}
Risque & Critère & Objectif LDL & TTT\\
\hline
Faible & SCORE < 1\%< & < 1.15g/L & Diététique\\
\hline
Modéré & 1\% ≤ SCORE < 5\%< & < 1.15g/L (3 mmol/L) & Diététique\\
 & Diabète de type 1/2 avant 40 ans sans atteinte d’organe cible &  & \\
\hline
Elevé & 5\% ≤ SCORE < 10\%< & < 1g/L (2.6 mmol/L) & + médic\\
 & Diabète de type 1 ou 2: &  & \\
 & Avant 40 ans avec au moins un FdRCVou atteinte d’organe &  & \\
 & Après 40 ans sans atteinte d’organe cible ou FdRCV &  & \\
 & IRC modérée &  & \\
 & PA ≥ 180/110mmHg &  & \\
\hline
Très élevé & SCORE ≥ 10\% &  & \\
 & Diabète de type 1 ou 2 ≥ 40 ans avec au moins un FdRCV & < 0.7g/L (1.8 mmol/L) & + médic\\
 & IRC sévère &  & \\
 & Maladie CV documentée &  & \\
\end{tabular}
\end{center}

\section{Diététique :}
\label{sec:orgfc6fdd5}
\begin{itemize}
\item diminuer acides gras saturés, augmenter mono/poly-insaturés
\item augmenter fruits, légumes céréales
\item diminuer cholestérol alimentaire
\item huile d'olive, fruits à coque
\item + limiter alcool, controle poids, sédentarité
\end{itemize}

Pour hyperTG modérée : diminuer poids, alcool et sucres simples

\section{C Traitement médicamenteux}
\label{sec:orgdd4964a}
\begin{itemize}
\item Risque faible/modéré : diététique 3 mois puis médic si échec
\item Sinon médic d'emblée
\item statines contre-indiquées en cas de grossesse.
\end{itemize}

\begin{center}
\begin{tabular}{ll}
Hypercholestérolémies pures & \textbf{statine} -> - echec: augmenter dose -> echec : ajout ézétimibe\\
et hyperlipidémies mixtes & si intolérance statine: ézétimibe\\
\hline
Hypertriglycéridémies pures & TG < 5 g/L:  diététique seul. sinon ajout fibrate\\
hypercholestérolémie familiale hétérozygote & dépistage (1er degré). Cf hypercholestérolémie pure\\
\end{tabular}
\end{center}


Surveillance:
\begin{itemize}
\item EAL à 12 semaine puis à 8-12 après chaque changement.
\item \textbf{myalgies} (statines++) => surveillance musculaire clinique (dosage CPK si risque)
\item hépatique: avant puis à 8-12 semaines. Si ALAT >  3 N, arrêter/diminuer statine
\end{itemize}

Nouveautés thérapeutiques: immunoglobuline monoclonales humaines = Ac anti-PCSK9, alirocumab, évolocumab
\end{document}
