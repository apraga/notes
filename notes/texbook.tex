\documentclass{article}

% Nice backslash
\newcommand\bSLASH{\char`\\}
\renewcommand{\labelitemi}{$\bullet$}

\newcommand\BSLASH{\textbackslash}
% Tat
\newcommand{\tab}{\hspace*{2cm}}

% Bourbaki symbol
\usepackage{manfnt}
\makeatletter
\def\hang{\hangindent\parindent}
\def\d@nger{\medbreak\begingroup\clubpenalty=10000
  \def\par{\endgraf\endgroup\medbreak} \noindent\hang\hangafter=-2
  \hbox to0pt{\hskip-\hangindent\dbend\hfill}}
\outer\def\danger{\d@nger}
\makeatother

% Smaller margins
\usepackage{fullpage}

\begin{document}

\title{Notes on Knuth's \TeX{} book}
\date{}
\maketitle

\danger Sometimes, the difference between the `good' and `bad' typography
can be quite subtle to see. Do not hesitate to zoom in !
\danger Knuth's book is about plain \TeX, but it is not always compatible 
with  \LaTeX\footnote{Text resizing and tables are managed differently for 
example}. I have put here only samples which will work with both. 

\section{Symbols}
\paragraph{Special symbols}
Here is a list of special symbols and how to type them. You will only have to
escape most of them with a \bSLASH.
\begin{center}
\begin{tabular}{c|c c c c c c c c c} 
Symbole & \BSLASH  & \{ & \} & \$ & \& & \# & \_ & \% & \textasciitilde \\
\hline
Code   & \BSLASH textbackslash&\BSLASH\{ &\BSLASH\} &
\BSLASH\$ & \BSLASH\& & \BSLASH\#  & \BSLASH\_  & 
\BSLASH\% & \BSLASH textasciitilde \\
\end{tabular}
\end{center}

A backslash can also be obtained with \BSLASH backslash in math mode or
in Typewriter interface with \BSLASH char~`\BSLASH\BSLASH.
\paragraph{Dashes}
There are several types :
\begin{itemize}
\item the hyphen -, used for compounds like `kick-ass',
\item the en-dash --, used in number ranges like `18-20',
\item the em-dash ---, used in sentences --- for separation.
\end{itemize}

\section{Fonts}
\paragraph{Alternatives} Instead of using {\tt \bSLASH textit\{italics\}} for
typing \textit{italics}, you can use the command {\tt \{\bSLASH it italics\}}.
The difference between the two is quite simple : {\tt \bSLASH textit} will put in
italics only the text between curly brackets, while {\tt \bSLASH it} will typeset
everything after it into italics, thus the need for exterior curly brackets.

Of course, the situtaion is the same with Typewriter fonts. The comparison is then 
between {\tt \bSLASH textit} and {\tt \bSLASH tt}.

\paragraph{Italics}
Did you know the space between an italic letter and a roman letter can be
inadequate ? It appears to be too small due to the inclinaison. \\
\centerline{{\it As an example, compare that} space {\it with the correction,
that\/} space.}\\
You can change it with {\tt \bSLASH /}, like\\
\centerline{\tt \{\bSLASH it See that \bSLASH /\} space}

\section{Glue}
In \TeX{}, everything are represented by boxes. Each boxes has a "glue", which can
stretch or shrink according to its definition. For somes cases, the glue
features are different. After a comma, the stretch rate is 1.25 the normal rate.
After a period or a "!, the rate is 3 times the normal one.

This can be problematic if the period is in the middle of a sentence. For
example, abbreviations like `Mr. Brightside' contains a larger space than it should. A
solution is to place a space `\BSLASH ' but it is prerable to use a
\textasciitilde{} like that :\\
\centerline{\tt Mr.\textasciitilde{}Brightside}

If you prefer to adopt the so-called `French spacing', which has only one sort
of space, the command is {\tt \bSLASH frenchspacing} (and {\tt \bSLASH
nonfrenchspacing} for deactivating it).

\section{Lines breaking}
\TeX{} is trained to be good at breaking lines. However in some cases, you do
not want a set of words to be breaked. A tilde put at the right place will avoid
these troubles. Examples :\\
\tab{\tt Theorem\textasciitilde{}1.2} \\
\tab{\tt function\textasciitilde{}\$f(x)\$} \\
\tab{\tt Diderik van\textasciitilde{}der\textasciitilde{}Waals}

\section{Math}
\paragraph{Fractions}
Did you know there is an alternative to \texttt{\bSLASH frac} for typing fractions ?
{\tt \bSLASH over} will put everything before in the numerator and everything
after in the denominator. So {\$\$\tt x+1 \bSLASH over y-1\$\$} will output
:\\
$$ x+1 \over y-1$$
Likewise a binomial coefficient can be written with {\tt \bSLASH choose}. 
{\$\$\tt x+1 \bSLASH over y-1\$\$} gives :
$$ n+1 \choose k-1$$

\paragraph{Styles}
According to the context, a certain size, or `style', will be chosen. 
As an example, a sum will not be displayed the same if put between \$ \$ or between \$\$ \$\$. 
The {\tt \bSLASH scriptstyle} command will reduce the input size like this :
$$ a+\scriptstyle b+\scriptstyle c$$
On the contrary, {\tt \bSLASH displaystyle} will make it larger. It can be useful
for more readeable sums like $\displaystyle \sum_{i=0}^{n}$ instead of 
$ \sum_{i=0}^{n}$, or continued fractions :
$$ a_0 + {1 \over \displaystyle a_1+ {1 \over \displaystyle a_2 + {a_3 \over  a_4}}}
\mbox{ is nicer than }
a_0 + {1 \over  a_1+ {1 \over  a_2 + {a_3 \over  a_4}}}$$
The previous continued fraction was obtained with :\\
\centerline{\tt \$\$ a\_0+\{1 \bSLASH over \bSLASH displaystyle a\_1+
\{1 \bSLASH over \bSLASH displaystyle a\_2+\{a\_3 \bSLASH over
a\_4\}\} \$\$}\\
Finally, the {\tt \bSLASH atop} command will output vertically-aligned variables.
We can use it for adding several indices in a sum. With the help of {\tt \bSLASH
displaystyle}, we can have
$$ \sum_{\displaystyle i \le n \atop \displaystyle j \le n} a_{ij} $$
by typing \\
\centerline{ \tt 
\$\$ \bSLASH sum\_\{\bSLASH displaystyle i \bSLASH le n \bSLASH atop 
 \bSLASH displaystyle j \bSLASH le n\} a\bSLASH \_\{ij\}\bSLASH \$\$
}

\paragraph{Accents} A wide range of accents is available in math mode :

\begin{table}[h]
\centering
\begin{tabular}{c|c c c c c} 
Symbole & $\acute a$ & $\bar a$ & $\breve a$ & $\check a$ & $\dot a$ \\
\hline
Code  & \$\BSLASH acute a\$ & \$\BSLASH bar a\$ & \$\BSLASH breve a\$ & 
\$\BSLASH check a\$ & \$\BSLASH dot a\$
\end{tabular}

\vspace{10pt}
\begin{tabular}{c|c c c c c} 
Symbole & $\ddot a$ & $\grave a$ & $\hat a$ & $\tilde a$ & $\vec a$ \\
\hline
Code  &\$\BSLASH ddot a\$ & \$\BSLASH grave a\$ & \$\BSLASH hat a\$ 
& \$\BSLASH tilde a\$ & \$\BSLASH vec a\$
\end{tabular}
\end{table}

\section{Math revisited}
\paragraph{Roman} In math mode, text is in italics by default. If you need to
type an expression in Roman, and if it is undefined, just use {\tt \bSLASH rm} or
{\tt \bSLASH hbox}. For example :
$$ \sqrt{{\rm Euler}(x)}$$
is obtained by\\ 
\centerline{\tt \$\$\bSLASH sqrt\{\bSLASH hbox\{Euler\}(x)\}\$\$ {\rm or} 
\$\$ \bSLASH sqrt\{\{\bSLASH rm Euler\}(x)\}\$\$}

\paragraph{Spacing} In equations, you may want to add some spaces between
different formulas  :
$$ u_{n+1} = 3 n + 2, \qquad \forall n \ge 0 $$
\texttt{\bSLASH qquad} will serve this purpose. \texttt{\bSLASH quad} is 
also available for a lesser space. Example :\\
\centerline{\tt \$\$ u\_\{n+1\} = 3 n + 2, \bSLASH qquad \bSLASH forall n
\bSLASH ge 0 \$\$}

The default spacing in formulas is quite good but you can always add more space.
with the following commands
\begin{center}
\begin{tabular}{c c c c} 
Thin & Medium & Thick & Negative \\
\BSLASH, & \BSLASH $\ge$ & \BSLASH ;&  \BSLASH !
\end{tabular}
\end{center}

\paragraph{Dots} There are two kinds of dots :
$$ f(x_1,\ldots,x_n) \quad \mbox{and} \quad x_1 +\cdots+x_n$$
The {\tt \bSLASH ldots} gives `lower' dots, whereas 
 {\tt \bSLASH cdots} gives `higher' dots. Examples :\\
\tab \tab \tab{\tt 
\$\$ f(x\_1,\bSLASH ldots,x\_n) \$\$}\\
\tab \tab \tab {\tt 
\$\$ x\_1 +\bSLASH cdots+x\_n \$\$
}



\end{document}
