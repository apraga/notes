\documentclass[12pt]{report}

\usepackage[francais]{babel}
\usepackage{amsmath} 

\begin{document}

\chapter{M\'ethodes num\'eriques}
\section{Syst\`emes lin\'eaires}
$$Ax = b$$
Pour des syst\`emes pleins, m\'ethodes directes.
\subsection{D\'ecomposition LU}
Si A est inversible, on peut appliquer le pivot de Gauss pour avoir une
d\'ecomposition \\
$A = LU$ , L triangulaire inf\'erieure
U triangulaire sup\'erieure.\\
Cout total en $O(\frac{2}{3}n^3)$.\\
\textit{Attention: } Si le pivot est trop petit, les approximations num\'eriques
peuvent donner de r\'esultats faux. Pivot partiel : on choisit la ligne de
coefficient maximal.

\subsection{M\'ethode de Cholesky}
Si la matrice A est sym\'etrique d\'efinie positive, elle peut se mettre sous la
forme $T T^t$ avec T triangulaire inf\'erieure.\\
Cout total en $O(\frac{1}{3}n^3)$.\\

Si le syst\`eme est creux, m\'ethodes it\'eratives. On d\'ecompose $A=M-N$ :
$$M x_{k+1} = N x_k + b$$

\subsection{M\'ethode de Jacobi}
$M=D$ et $N=A-D$ o\`u D est la diagonale de A.\\
Converge ssi \`a diagonale strictement dominante ($a_{ii} > 0$)\\
Peu utilis\'ee.

\subsection{M\'ethode de Gauss-Seidel}
$M=D+L$ et $N=-U$ o\`u D est la diagonale de A, L triangulaire inf\'erieure, U
triangulaire sup\'erieure.\\
Peu utilis\'ee.

\subsection{M\'ethode SOR}
Gauss-Seidel avec un param\`etre de relaxation $\omega$.\\
$M=\frac{1}{\omega}D+L$ et $N=\frac{1-\omega}{\omega}D-U$\\
Si A est sym\'etrique, d\'efinie positive, convergence ssi $\omega \in [0,2]$


\section{Fonctions non lin\'eaires}
\subsection{M\'ethode de Newton}
Permet de trouver les racines de $f(x)=0$
$$x_{k+1} = x_k-Df^{-1}(x_k) f(x_k)$$
Convergence quadratique mais calcul de la jacobienne couteux.
\subsection{M\'ethode de Broyden}
Quasi newton : on approche la jacobienne par
$$J_n(x_n-x_{n-1}) = F(x_n)-F(x_{n-1}) $$
Peut \^etre sous d\'etermin\'e. D'autres m\'ethodes de calcul de $J_n$ existent.

\section{M\'ethodes d'optimisations}
Les deux premi\`eres m\'ethodes avec le gradient s'appliquent pour des fonctions
$C^1$. Le gradient conjugu\'e s'utilise pour des fonctions $C^2$.
\subsection{Gradient \`a pas constant}
$x_{k+1} = x_k - \rho \nabla f(x_k)$ avec $\rho > 0$ fix\'e.\\
Peu utilis\'ee \`a cause d'instabilit\'es num\'eriques.

\subsection{M\'ethode de plus grande descente}
$$x_{k+1} = x_k - \rho_k \nabla f(x_k)$$ 
$\rho_k$ est adapt\'e \`a chaque \'etape en minimisant 
$$\phi(\rho)=f(x_k - \rho_k \nabla f(x_k))$$
Si f est convexe, on peut utiliser la m\'ethode de Newton pour r\'esoudre
$\phi'(\rho)=0$. 
On peut aussi proc\'ed\'er par dichotomie : sur $[a,b]$, f doit \^etre
unimodale\footnote{Il existe $c \in [a,b]$ tel que
$\phi'(x) < 0$ sur $]a,c[$ et 
$\phi'(x) > 0$ sur $]c,b[$}
Convergence lin\'eaire.

\subsection{Gradient conjugu\'e}
Pour une fonction quadratique $f(x)=\frac{1}{2} x^t A x - x^t b$.\\
Soit $r k=Ax_k-b$.
It\'eration : $x_{k+1}=x_k-\rho_k \omega_k$ avec 
\begin{equation*}
\left\{
\begin{array}{l}
\omega_k =r_k+\theta_k \omega_{k-1} \text{ et }
\theta_k=\frac{||r_k||_2^2}{||r_{k-1}||_2^2}\\
\rho_k =\frac{||r_k||_2^2}{w_k^t A w^k}
\end{array}
\right.
\end{equation*}

\chapter{EDP}
\section{EDP lin\'eaires d'ordre 1}
On veut r\'esoudre
$$\nabla \phi(X)\cdot F(x)+g(X)\phi(X) = h(X)$$
Les caract\'eristiques sont donn\'ees par 
$$ \frac{dX}{ds}=F(X)$$
Sur les caract\'eristiques, cela revient \`a r\'esoudre une EDP d'ordre 1
$$ \frac{d \phi(X(s))}{ds}=-g(X(s)) \phi(X(s)) + h(X(s))$$

\section{EDP lin\'eaires d'ordre 2}
$$ a(x,y) \frac{\partial^2 u}{\partial x^2}+
b(x,y) \frac{\partial^2 u}{\partial x \partial y}+
c(x,y) \frac{\partial^2 u}{\partial y^2}+
f(x,y,\frac{\partial u}{\partial x}, \frac{\partial u}{\partial y})=0$$
On fait une discussion sur le type d'EDP sur un domaine D.
\subsection{Cas hyperbolique}
Si $b^2-4ac > 0$, on fait un changement de coordonn\'ees pour avoir 2 \'equations de
transports :
\begin{equation*}
\left\{
\begin{array}{l}
2a \xi_x+(b-\sqrt{b^2-4ac}) \xi_x = 0\\
2a \eta_x+(b-\sqrt{b^2-4ac}) \eta_x = 0\\
\end{array}
\right.
\end{equation*}

Forme canonique :
$$ \frac{\partial^2 v}{\partial \xi \partial \eta}=G(\xi,\eta,v,
\frac{\partial v}{\partial \xi},\frac{\partial v}{\partial \eta})$$

\subsection{Cas parabolique}
Si $b^2-4ac = 0$, on fait un changement de coordonn\'ees pour avoir 2 \'equations de
transports :
\begin{equation*}
\left\{
\begin{array}{l}
2a \xi_x+ b \xi_x=0\\
\eta \text{ est choisi tel que } \phi \text{ est un } C^2 \text{-diff\'eomorphisme}
\end{array}
\right.
\end{equation*}

Forme canonique :
$$ \frac{\partial^2 v}{\partial \eta^2}=G(\xi,\eta,v,
\frac{\partial v}{\partial \xi},\frac{\partial v}{\partial \eta})$$

\subsection{Cas elliptique}
Forme canonique :
$$ \frac{\partial^2 v}{\partial \xi^2}
+\frac{\partial^2 v}{\partial \eta^2}+G(\xi,\eta,v,
\frac{\partial v}{\partial \xi},\frac{\partial v}{\partial \eta})=0$$

A COMPLETER

\section{Sch\'emas num\'eriques}
\subsection{Consistance, convergence, stabilit\'e}
Discr\'etisation : $u_{n+1}=C(\Delta t,\Delta x) u_n$\\
Consistance d'un sch\'ema : caract\'erise l'erreur de troncature.
$$\lim_{(\Delta t,\Delta x)\rightarrow 0} 
\sup_{t_n} ||\frac{u(t_n+\Delta t)-C(\Delta t,\Delta x)u(t_n)}{\Delta t}||=0 $$
Sch\'ema d'ordre r en espace et q en temps
$$\sup_{t_n} ||u(t_n+\Delta t)-C(\Delta t,\Delta x)u(t_n)||=
O(|\Delta t|^{q+1}+
|\Delta t| |\Delta x |^r) $$
Stabilit\'e d'un sch\'ema :
Il existe une constante $K_T$ telle que 
$\forall \Delta x, \Delta t, n \Delta t \le T$
$$|| C(\Delta t,\Delta x)^n|| \le K_T$$
Convergence d'un sch\'ema :
L'\'ecart entre la solution approch\'ee et la solution exacte tend vers 0 quand le
pas de discr\'etisation tend vers 0.\\
Th\'eor\^eme de Lax : soit un sch\'ema consistant. Alors il est convergent ssi il est
stable.\\
Analyse de von Neumann : prendre la transform\'ee de Fourier 
$$\hat{U}^{n+1}(\lambda)=G(\Delta x, \Delta t, 2\pi \lambda) \hat{U}^n $$
Stabilit\'e :
Il existe une constante $K_T$ telle que 
$\forall \Delta x, k, \Delta t, n \Delta t \le T$
$$|| G(\Delta t,\Delta x,k)^n|| \le K_T$$
Soit $\lambda_i$ les valeurs propres de G.\\
Th\'eoreme de von Neumann : \\
CN de stabilit\'e 
$$\sup_k \max_i |\lambda_i| \le 1+O(|\Delta t|) $$
Si G est normale, c'est une CNS.

\subsection{Types de schémas}
Exemple de l'équation de transport.
\subsubsection{Sch\'ema euler explicite décentré}
Aval :
$$\frac{u_j^{n+1}- u_j^n}{\Delta t} +c \frac{u_j^n- u_{j-1}^n}{\Delta x}=0$$
Ordre 1 en espace et 1 en temps pour l'équation de transport.\\
Toujours instable\\
Amont :
Stable si $c \frac{\Delta x}{\Delta t} \le 1$

\subsubsection{Sch\'ema euler explicite centré}
$$\frac{u_j^{n+1}- u_j^n}{\Delta t} +
c \frac{u_{j+1}^n- u_{j-1}^n}{2 \Delta x}=0$$
Ordre 2 en espace et 1 en temps pour l'équation de transport.\\
Instable si $\frac{\Delta x}{\Delta t}$ reste constant.

\subsubsection{Sch\'ema de Lax Wendrof}
$$\frac{u_j^{n+1}- u_j^n}{\Delta t} +
c \frac{u_{j+1}^n- u_{j-1}^n}{2 \Delta x}-
\frac{c^2}{2}\frac{\Delta t}{\Delta x^2} 
\frac{u_{j+1}^n+2 u_j- u_{j-1}^n}{2 \Delta x}=0$$
Ordre 2 en temps et en espace.\\
Stable si $|c| \frac{\Delta x}{\Delta t} \le 1$

\chapter{El\'ements finis}
\section{}

\chapter{Volume finis}
\section{}

\end{document}
