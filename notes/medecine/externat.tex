% Created 2021-04-22 Thu 11:24
% Intended LaTeX compiler: pdflatex
\documentclass[11pt]{article}
\usepackage[utf8]{inputenc}
\usepackage[T1]{fontenc}
\usepackage{graphicx}
\usepackage{grffile}
\usepackage{longtable}
\usepackage{wrapfig}
\usepackage{rotating}
\usepackage[normalem]{ulem}
\usepackage{amsmath}
\usepackage{textcomp}
\usepackage{amssymb}
\usepackage{capt-of}
\usepackage{hyperref}

\usepackage{longtable}
\usepackage{booktabs}
\usepackage{biocon}
%\usepackage[draft]{graphicx}
\usepackage{graphicx}
\usepackage{fancyhdr}
% French
\usepackage[T1]{fontenc}
\usepackage[francais]{babel}
\usepackage{caption}
\usepackage[nointegrals]{wasysym} % Male-female symbol
% Smaller marign
\usepackage[margin=2.5cm, marginparwidth=2cm]{geometry}
\usepackage{latexsym}
\usepackage{subcaption}
\usepackage[table]{xcolor} % Color in tables, must be called before tikz !
% -------------------------------------------------------------------------------
% For graphs
\usepackage{tikz}
\usepackage{tikzscale}
\usetikzlibrary{graphs}
\usetikzlibrary{graphdrawing}
\usetikzlibrary{arrows,positioning,decorations.pathreplacing}
\usetikzlibrary{calc}

% Trees package not found after update
\usepackage{luacode}
\begin{luacode}
  function pgf_lookup_and_require(name)
    local sep = '/'
    if string.find(os.getenv('PATH'),';') then
      sep = '\string\\'
    end
    local function lookup(name)
      local sub = name:gsub('%.',sep)
      local find_func = function (name, suffix)
        if resolvers then
          local n = resolvers.findfile (name.."."..suffix, suffix) -- changed
          return (not (n == '')) and n or nil
        else
          return kpse.find_file(name,suffix)
        end
      end
      if find_func(sub, 'lua') then
        require(name)
      elseif find_func(sub, 'clua') then
        collectgarbage('stop')
        require(name)
        collectgarbage('restart')
      else
        return false
      end
      return true
    end
    return
      lookup('pgf.gd.' .. name .. '.library') or
      lookup('pgf.gd.' .. name) or
      lookup(name .. '.library') or
      lookup(name)
  end
\end{luacode}


\usegdlibrary{trees, layered}
\usetikzlibrary{quotes}
\usetikzlibrary{shapes.misc} % for rounded rectangle

% Custom graph style, for most of our graphs
\tikzgraphsset{decision/.style={
   % Labels at the middle 
    %edge quotes mid,
    % Needed for multi-lines
    nodes={align=center},
    %sibling distance=6cm,
    layer distance=2cm,
    %edges={nodes={fill=white}}, 
    tree layout
    }
  }
\tikzset{
  organ/.style = {rounded rectangle, draw=black}
}

\usepackage{booktabs}
\usepackage[shortcuts,nonumberlist]{glossaries} % shorcuts = \acs{} command
\makeglossaries

% -------------------------------------------------------------------------------
% No spacing in itemize
\usepackage{enumitem}
\setlist{nolistsep}
% tightlist from pandoc
\providecommand{\tightlist}{%
  \setlength{\itemsep}{0pt}\setlength{\parskip}{0pt}}
 % Danger symbol (need fourier package)
\newcommand*{\TakeFourierOrnament}[1]{{%
\fontencoding{U}\fontfamily{futs}\selectfont\char#1}}
\newcommand*{\danger}{\TakeFourierOrnament{66}}
% Hospital sign, bomb sign
\usepackage{fontspec} % For fontawesome
\usepackage{fontawesome}

% Itemize in tabular
\newcommand{\tabitem}{~~\llap{\textbullet}~~}
% No numbering
\setcounter{secnumdepth}{0}
% In TOC, only section
\setcounter{tocdepth}{1}
% Set header
\pagestyle{fancy}
\fancyhf{}
\fancyhead[L]{\leftmark}
\fancyhead[R]{\thepage}
%\renewcommand{\headrulewidth}{0.6pt}
% Custom header : no uper case
\renewcommand{\sectionmark}[1]{%
  \markboth{\textit{#1}}{}}
% Footnote in section
\usepackage[stable]{footmisc}
% Chemical compound
\usepackage{chemformula}

% Negate \implies
\usepackage{centernot}
% No paragraph indent
\setlength\parindent{0pt}

% Nice box but small for our purpose
\usepackage{tcolorbox}
\tcbset{left=1pt, right=1pt, top=1pt, bottom=1pt, boxrule=0.2mm}

% Footnote in table
\usepackage{tablefootnote}

% hyperref setup
\hypersetup{
  colorlinks = true,
  linkcolor=teal
}
%-------------------------------------------------------------------------------
% Custom commands
%-------------------------------------------------------------------------------
% Logical and, or
\def\land{$\wedge$}
\def\lor{$\vee$}
\def\dec{$\searrow{}$}
\def\inc{$\nearrow{}$}
\def\thus{$\implies$}

% Hightlight cell
\def\hlc{\cellcolor{blue!10}}
\def\hl{\colorbox{blue!10}}

\input bacteries-header

\usepackage{adjustbox}
\usepackage{multirow, makecell}
\usepackage[linesnumbered,ruled,vlined]{algorithm2e}
\usepackage{enumitem}
\def\ttt{\hspace*{1cm}Ttt: }
\usepackage{adjustbox}
\usepackage{titlesec}
\graphicspath{{../../pictures/medecine/}}
\usetikzlibrary{quotes}
\titlespacing{\paragraph}{%
0pt}{%              left margin
0.5\baselineskip}{% space before (vertical)
1em}%               space after (horizontal)
\newacronym{ADP}{ADP}{Adénopathie}
\newacronym{ARA II}{ARA II}{Antagonistes des récepteurs de l'angiotensine}
\newacronym{ATS}{ATS}{Antithyroïdiens de synthèse}
\newacronym{AT}{AT}{Antithrombine}
\newacronym{BAV}{BAV}{Bloc auriculoventriculaire}
\newacronym{BBG}{BBG}{Bloc de branche gauche}
\newacronym{BD}{BD}{Bronchodilatateur}
\newacronym{BGT}{BGT}{bilirubine glucoronide-transférase}
\newacronym{BSA}{BSA}{Bloc sinuso-atrial}
\newacronym{CAIS}{CAIS}{Complete Androgen Insensitivity Syndrome}
\newacronym{CHC}{CHC}{Carcinome hépato-cellulaire}
\newacronym{CLU}{CLU}{Cortisol libre urinaire}
\newacronym{CPRE}{CPRE}{Cholangio-pancréatographie rétrograde endoscopique}
\newacronym{CST}{CST}{Coefficient de saturation de la transferrine}
\newacronym{DAI}{DAI}{Défibrillateur automatique implantable}
\newacronym{DIP}{DIP}{Pneumopathie Interstitielle Desquamante}
\newacronym{DMLA}{DMLA}{Dégénérescence maculaire liée à l'âge}
\newacronym{DO}{DO}{Déclaration obligatoire}
\newacronym{DT}{DT}{Delirium tremens}
\newacronym{ECPA}{ECPA}{Echelle comportementale d'évaluation de la douleur chez la personne âgée}
\newacronym{EI}{EI}{Endocardite infectieuse}
\newacronym{EN}{EN}{Echelle numérique}
\newacronym{EP}{EP}{Embolie pulmonaire}
\newacronym{EVA}{EVA}{Echelle visuelle analogiqu}
\newacronym{EVS}{EVS}{Echelle verbale simple}
\newacronym{FA}{FA}{Fibrillation atriale}
\newacronym{FE}{FE}{Fraction d'ejection}
\newacronym{FIVA}{FIVA}{Fonds d'indemnisation des victimes de l'amiante}
\newacronym{FIV}{FIV}{Fibrinolyse intra-veineuse}
\newacronym{FO}{FO}{Fond d'oeil}
\newacronym{FPI}{FPI}{Fibrose Pulmonaire Idiopathique}
\newacronym{GH}{GH}{Hormone de croissance (Growth hormone)}
\newacronym{GNV}{GNV}{Glaucome néovasculaire}
\newacronym{GPAO}{GPAO}{Glaucome primitif à angle ouvert}
\newacronym{HAD}{HAD}{Hospital Anxiety and Depression Scale}
\newacronym{HMG}{HMG}{Hépatomégalie}
\newacronym{HSH}{HSH}{Hommes ayant des relations sexuelles avec des hommes}
\newacronym{HVG}{HVG}{Hypertrophie ventriculaire gauche}
\newacronym{IC}{IC}{Insuffisance cardiaque}
\newacronym{ID}{ID}{Immunodéprimé}
\newacronym{IEC}{IEC}{Inhibiteurs de l'enzyme de conversion}
\newacronym{IOP}{IOP}{Insuffisance ovarienne primitive}
\newacronym{IPC}{IPC}{Intervention coronaire percutanée}
\newacronym{IS}{IS}{Insuffisance surrénale}
\newacronym{ITK}{ITK}{Inhibiteur de tyrosine kinase}
\newacronym{IVA}{IVA}{Artère intraventriculaire antérieure}
\newacronym{JPDC}{JPDC}{Jusqu'à preuve du contraire}
\newacronym{LBA}{LBA}{Lavage Broncho-Alvéolaire}
\newacronym{MTEV}{MTEV}{Maladie Thrombo-Embolique Veineuse}
\newacronym{NOIA}{NOIA}{Neuropathie optique ischémique antérieure}
\newacronym{OCA}{OCA}{Occlusion coronaire aigüe}
\newacronym{OGD}{OGD}{Oestro-gastro-duodénale}
\newacronym{OGE}{OGE}{Organes génitaux externes}
\newacronym{OG}{OG}{Oreillette gauche}
\newacronym{OMI}{OMI}{Oedème des membres inférieurs}
\newacronym{PAD}{PAD}{Pression artérielle diastolique}
\newacronym{PAPm}{PAPm}{Pression de l'artère pulmonaire moyenne}
\newacronym{PAS}{PAS}{Pression artérielle systolique}
\newacronym{PA}{PA}{Pression artérielle}
\newacronym{PEV}{PEV}{Potentiels évoqués visuels}
\newacronym{PGG}{PGG}{Paragangliomes}
\newacronym{PINS}{PINS}{Pneumonpathie Interstitielle Non Spécifique}
\newacronym{POC}{POC}{Pneuompathie organisée cryptogénique}
\newacronym{QCD}{QCD}{Questionnaire Concis de la Douleur}
\newacronym{QDSA}{QDSA}{Questionnaire Douleur Saint-Antonne}
\newacronym{RCT}{RCT}{Rapport cardiothoracique}
\newacronym{RGO}{RGO}{Reflux gastro-oesophagien}
\newacronym{RPM}{RPM}{Réflexe photomoteur}
\newacronym{SAS}{SAS}{Syndrome d'apnée du sommeil}
\newacronym{SMG}{SMG}{Splénomégalie}
\newacronym{SOPK}{SOPK}{Syndrome des ovaires polymicrokystiques}
\newacronym{SPT}{SPT}{Syndrome post-thrombotique}
\newacronym{TAVI}{TAVI}{Transcatheter Aortic Valve Implantation.}
\newacronym{TG}{TG}{Tryglycérides}
\newacronym{TIH}{TIH}{Thrombopénie induite par l'héparine}
\newacronym{TVO}{TVO}{Troubles Ventilatoires Obstructifs}
\newacronym{TVP}{TVP}{Thrombose veineuse profonde}
\newacronym{TVR}{TVR}{Troubles Ventilatoires Restrictif}
\newacronym{TVS}{TVS}{Thrombose veineuse superficielle}
\newacronym{VAS}{VAS}{Voies Aériennes Supérieures}
\newacronym{VBP}{VBP}{Voie biliaire principale}
\newacronym{VG}{VG}{Ventricule gauche}
\newglossaryentry{NEM1}{name={NEM1},
description={Néoplasie endocrinienne multiple 1. 3 "P" : hyperParathyroïdie primaire, adénome hyPophysaire, tumeur neuro-endocrine du Pancreas. Voir aussi NEM2}}
\newglossaryentry{NEM2}{name={NEM2},
description={Néoplasie endocrinienne multiple 2. Cancer médullaire de la thyroïde et phéochromocytome. Voir aussi \gls{NEM1}}}
\newglossaryentry{Leydigcell}{name={cellule de Leydig},description={Produit de la testostérone. Localisé près des tubules séminifères (testicules)}. Activé par LH}
\newglossaryentry{NF1}{name=NF1, description={Neurofibromatose 1. Tâches café au lait, neurofibromes (cutanées, nodulaires [le long d'un trajet d'un nerf] ou plexiformes [K possible]), nodules de Lisch sur l'iris.}}
\newglossaryentry{PCC}{name={Phéochromocytomes}, description={Tumeur de la médullo-surrénale}}
\newglossaryentry{PTH}{name={Parathyroide Hormone (PTH)},description={Stimule la résorbtion osseuse (ostéoclastes) pour libérer plus de calcium}}
\newglossaryentry{SHBG}{name=SHBG,description={Sex Hormone-Binding Globulin. Diminue avec des androgènes, augmente avec les oestrogènes}}
\newglossaryentry{Sertolicell}{name={cellule de Sertoli},description={Participe à la production du sperme. Localisé dans un tubule séminifère. Activé par FSH}}
\newglossaryentry{TPO}{name={Thyroid peroxydase (TPO)},description={Enzyme de la thyroïde servant à générer la thyroxine (T4) et triiodothyroine (T3)}}
\newglossaryentry{VHL}{name={von Hippel-Lindau}, description={Hémangioblastome du cervelet/moelle épinière, de la rétine, phéochromocytome}}
\newglossaryentry{trophozoïtes}{name={Trophozoïtes},description={Formes végétatives mobiles}}
\newglossaryentry{sdMetabolique}{name={Syndrome métabolique},
description={IMC > 28 kg/$m^2$, HTA,
(HDL < 0.35g/L ou TG > 2g/L ou dyslipidémie traitée),
ATCD diabète familial/gestionnel, temporairement induit.
Autre définition (NCEP III) : (\diameter abdo > 100cm \male ou 88cm \female),
hyperglycémie (glycémie à jeun > 1g/L),
dyslipidémie (TG > 1.5g/L et (HDL < 0.4g/L \male ou 0.5g/L \female)),
HTA (> 130mmHg systole ou > 85mmHg diastole)}}
\newglossaryentry{sdmetabolique}{name={Syndrome métabolique},
description={$\diameter \ge 94$ cm $\male{}$, 80 cm $\female{}$,
TG $\ge$ 1.7mmol/L, HDL < 1 mmol/L $\male{}$ ou 1.3mmol/L $\female{}$,
PAs $\ge 130$ mmHg ou PAd $\ge 85$ mmHg, glycémie jeun $\ge$ 1 g/L
}}
\newglossaryentry{VEMS}
{
name={VEMS},
description={volume expiratoire maximal en 1s (après inspiration maximale)}
}
\newglossaryentry{CV}
{ name = Capacité Vitale,
description = volume total mobilisable maximal = VC + VRI + VRE
}
\newglossaryentry{VC}
{ name=Volume courant,
description={volume mobilisé pendant une respiration normale}
}
\newglossaryentry{VRI}
{ name = Volume de réserve inspiratoire,
description = volume supplémentaire (par rapport au VC) avec
une inspiration forcé
}
\newglossaryentry{VRE}
{ name = Volume de réserve expiratoire,
description = idem VRI mais en expiration forcée
}
\newglossaryentry{VR}
{ name = Volume résiduel,
description = volume restant (impossible à expirer)
}
\newglossaryentry{CVF}
{ name = Capacité Vitale Forcée,
description = volume expulsé avec force (CPT - VR)
}
\newglossaryentry{CVL}
{ name = Capacité Vitale Lente,
description = idem CVF mais lentement
}
\newglossaryentry{CPT}
{ name = {Capacité Pulmonaire Totale},
description = {Capacité Vitale + volume résiduel}
}
\newglossaryentry{PAPO}{
name = PAPO,
description = Pression artérielle pulmonaire occluse $\approx$ pression
capillaire pulmonaire
}
\newacronym{MDPH}{MDPH}{Maison département des personnes handicapées}
\newacronym{CNSA}{CNSA}{Caisse nationale de solidarité pour l'autonomie}
\newacronym{AAH}{AAH}{Allocation aux Adultes Handicapés}
\newacronym{CDAPH}{CDAPH}{Commission des droits et de l'autonomie des personnes handicapées (départemental)}
\newacronym{PAF}{PAF}{Polypose adénomateuse familiale}
\newacronym{BPCO}{BPCO}{Bronchopneumopathie chronique obstructive}
\newacronym{VNI}{VNI}{Ventilation non invasive}
\newacronym{TIPMP}{TIPMP}{Tumeurs intracanalaires papillaire mucineuses pancréatiques}
\newacronym{FID}{FID}{Fossie Iliaque droite}
\newacronym{FIG}{FIG}{Fossie illiaque gauche}
\newacronym{TFI}{TFI}{Troubles fonctionnels intestinaux}
\newglossaryentry{PET1}{name={Polyendocrinopathie auto-immune de type 1},description={Hypoparathyroïdie, candidose, insuf. surrénale}}
\newglossaryentry{PET2}{name={Polyendocrinopathie auto-immune de type 2},description={insuf. surrénale + 1 maladie autoimmune parmi thyroïdite d'Hashimoto, maladie de Basedow, diabète type 1}}
\newacronym{SPUPD}{SPUPD}{Syndrome polyuro-polydipsique}
\newglossaryentry{IPS}{name={Index de pression systolique},description={Pression systolique cheville/bras}}
\author{Alexis Praga}
\date{\today}
\title{}
\hypersetup{
 pdfauthor={Alexis Praga},
 pdftitle={},
 pdfkeywords={},
 pdfsubject={},
 pdfcreator={Emacs 28.0.50 (Org mode 9.4.4)}, 
 pdflang={English}}
\begin{document}

\tableofcontents

\section{Cardiologie}
\label{sec:orgd428186}
\def\arrow{$\rightarrow$}
\subsection{220 Dyslipidémies}
\label{sec:orgd36ca82}
\subparagraph{Diagnostic}
\label{sec:org5922280}
Clinique :
\begin{itemize}
\item hypercholestérolémie : arc cornéen (< 50A), xanthélasmas, xanthomes
tendineux/fesses ou mains/coudes
\item hyperTG : HMG stéatosique, SMG, xanthomes cutanés éruptifs
\end{itemize}
Bio 
\begin{itemize}
\item lipides = hypercholestérolémie : CT/TG > 2.5, hyperTG : TG/CT > 2.5, mixte sinon
\item sérum clair/limpide si hypercholestérolémie, lactescent sinon
\end{itemize}

Bilan normal : LDL  < 1.6g/L, HDL  > 0.4g/L, TG  < 1.5g/L

\subparagraph{Classification}
\label{sec:org95771c2}
Tab \ref{tab:org8d824a0}, \ref{tab:org4855194}
\begin{table}[htbp]
\caption{\label{tab:org8d824a0}Hyperlipidémies primaires}
\centering
\begin{tabular}{llllll}
Impact & Classif & Type & Fréquence & Dépôts & Risque\\
\hline
Cholestérol & IIa & familale monogénique &  & oui & CV\\
 &  & - hétérozygote & freq & parfois & \\
 &  & - homozygote & très rare & fréquents & \\
 &  & - mutation apoB & freq &  & \\
 &  & polygénique & freq++ & rare & \\
TG & IV & familiale & rare & rare & \\
 & I, V & hypercholymicronémie primitive & très rare & inconstant & \danger TG > 10g/L !\\
 &  &  &  &  & pancréatite aigüe\\
Mixte & IIb & familiale combinée & freq & rares & \\
 & III & dys\(\beta\)lipoprotéinémie & rare & inconstant & \\
\end{tabular}
\end{table}

\begin{table}[htbp]
\caption{\label{tab:org4855194}Hyperlipidémies secondaire}
\centering
\begin{tabular}{lll}
Cause & Diagnostic & Impact\\
\hline
Hypothyroïdie & TSH & cholestérol, mixte\\
Sd néphrotique, grossesse & protéinurie, \oe{}dème & cholestérol\\
Cholestase & Bilirubine, phosphatase alcaline & cholestérol\\
IR chronique, & créatinine & TG, mixte\\
Alcoolisme & interrogatoire & TG\\
Diabète & glycémie & TG\\
\hline
Iatrogène : ciclosporine, corticoïdes & oestrogènes oraux, rétinoïdes, & IFN-\(\alpha\), antétroviraux, neuroleptiques\\
diurétiques thiazidiques, \(\beta\)bloquant &  & \\
\end{tabular}
\end{table}

\paragraph{Risque faible (0 FR), intermédiaire (\(\ge\) 1 FR), haut (ATCD)}
\label{sec:org74f9106}

FR semblables au\textasciitilde{}\hyperref[subsec:fr]{score précédent} : tabac \(\le\) 3 ans, HTA, diabète, HDL < 0.40g/L, âge > 50
(\male) ou 60 (\female), ATCD familiaux IDM ou mort subite

\subparagraph{Traitement}
\label{sec:org15a8837}
Hypercholestérolémies : primaire si LDL élèvé à ttt+6 mois, secondaire si complication ischémique
\begin{itemize}
\item Objectifs LDL < 0.7g/L si risque très élevé, < 1g/L si élevé, < 1.15g/L sinon
\item hypercholestérolémies : \uline{statines}\footnote{ES : myalgies, \inc CPK, \inc transaminases, \inc risque diabète 2
CI : HS, \emph{grossesse}, allaitement} (sinon ézétimibe)
\item hypertriglycéridémies : \uline{diététique} \textpm{} statines si TG > 2g/L (fibrates si échec)
\end{itemize}

Diététique
\begin{itemize}
\item lipides < 40\%, graisses : mono- et polyinsaturées, cholestérol alimentaire < 300mg/j
\item 5 fruits ou légumes/j, sodium < 6g/j, \dec poids
\item HyperTG : 
\begin{itemize}
\item modérées : -20\% calories ++, \inc activité physique
\item majeur : arrêt alcool, régime hypocalorique avec < 30g lipides (obèse)
\end{itemize}
\end{itemize}

\section{Endocrinologie}
\label{sec:orge3831a8}
\subsection{32 \textdagger{} Allaitement maternel}
\label{sec:org65c64a0}
\subsection{35 \textdagger{} Contraception}
\label{sec:orge7bb55a}
\subsubsection{Contraception hormonale}
\label{sec:org91f8a92}
Oestroprogestatifs
\begin{itemize}
\item Contient \{oestrogène, progestatif (gen 1,2 ou 3), autres progestatif\}
\item Administration orale, transdermique ou vaginale
\item Action : \{pas d'ovulation, endomètre peu à apte à la nidation, glaire
cervicale imperméable aux spermatozoïdes\}
\end{itemize}
Progestatifs seuls 
\begin{itemize}
\item Microprogestatifs : action sur glaire cervicale, endomètre
\item Macroprogestatifs : si CI oestroprogestatifs
\item Administration : orale, injection, implant, intra-utérin (stérilet)
\end{itemize}
\subsubsection{Pratique}
\label{sec:org30d2f10}
Oestroprogestatifs : le premier jour des règles pendant 21j puis 7 j
d'arrêt. \emph{Tjrs au même moment}. Si oubli < 12h, ASAP sinon contraception mécanique \(\ge 7\) jours

Microprogestatifs : toujours à la même heure. Si oubli < 3h, ASAP, sinon
contraception mécanique \(\ge 7\) jours

Macroprogestatifs : commencer le 5eme jour du cycle

\subsubsection{Contre-indications}
\label{sec:orgf3c36c7}
Oestroprogestatifs : absolues =
\begin{itemize}
\item thromboemboliques veineux/artériels, prédisposition thromboses
\item lupus évolutif, connectivites, porphyries
\item vasc, cardiaque, cérébrales, oculaires
\item valvulopathie, troubles rythmes thrombogènes
\item HTA non contrôlée
\item diabète et micro/macroangiopathie
\item tumeur hormono-dépendantes (sein, utérus\ldots{})
\item hépatiques sévères
\item hémorragies génitales non diagnostiquées
\item (tumeurs hypophysaires)
\end{itemize}
Macro/microprogestatifs : cancers \{sein, endomètre\}, insuf hépatique, accident
TEV récents

\subsubsection{Recommandation}
\label{sec:orgea9844a}
Sans CI, oestroprogestatif minidosé et progestatif 2eme génération monophasique
(Minidril)

\subsubsection{Efficacité}
\label{sec:org1f9afa6}
Indice de Pearl\footnote{\(\frac{N}{N_e/10}\times 100\) avec N = nb grossesses
accidentelles, \(N_e\) nombres de mois d'exposition} < 0.07\% pour oestroprogestatif
(< 2\% pour les microprogestatifs)

Attention : certains inducteurs enzymatiques réduisens l'efficacité (ou
millepertuis).

Ado : sous- ou mal utilisée

\subsubsection{Tolérance}
\label{sec:org570eec7}
Oestroprogestatifs :
\begin{itemize}
\item bien tolérée, pas de perte de poids
\item surveiller métabolisme
\item active coagulation mais \inc fibrinolyse. Légère augmentation du risque
d'accident TEV
\item vasc : faible \inc PA
\item cancer : ovaire = risque -50\%, idem pour l'endomètre, faible \inc pour sein
\end{itemize}
Microprogestatifs : troubles des règles (spotting, aménorrhées), grossesse
extra-utérine

Macroprogestatifs : hypoestrogénie, aménorrhées, spotting

\subsubsection{Surveillance}
\label{sec:org9a0e24e}
Consulter si céphalée, déficit sensitivomoteur, (douleur ou oedème) MI, dyspnée,
douleur thoracique

Examen clinique : 
\begin{itemize}
\item préthérapeutique : gynéco, frottis cervico-vaginal dès 25 ans si
asymptomatique.
\item PA à +3mois puis tous 6 mois
\item hyperoestrogénie (tension mammaire), hypoestrogénie (sécheresse vaginale)
\end{itemize}
Biologie : cholestérol total, triglycérides, glycémie à jeun à +3mois. Si FR, le
faire avant (!) prescription

Gynéco : métrorragies, spottings. 

Frottis cervico-utérin dès 25 ans (+1 an puis tous 3 ans) indépendamment contraception

\subsubsection{Femmes à risques}
\label{sec:orga6d3929}
Diabétique :
\begin{itemize}
\item non hormonale : si diabète 1 > 15 ans ou micro/macroangiopathie \thus locale
(nullipare, peu de rapports) ou intra-utérin (multipare, diabète équilibré)
\item hormonale :pas d'oestropregestatifs si \{tabac, non équilibré, HTA, surpoids,
diabète compliqué\} \thus progestatif
\end{itemize}
Dyslipidémie : oestroprogestatif si < 3g/L cholestérol total, triglycérides <
2g/L

Thrombose veineuse
\begin{itemize}
\item prédisposant : anomalies de l'hémostase (génétique, acquises), ATCD familiaux
\item dépistage : thrombose, multiples fausses couches, ATCD thrombose < 45 ans
\item CI oestrogène, acétate de chlormadinone à la place
\end{itemize}

Autres :
\begin{itemize}
\item HTA : oestroprogestatifs si 0 FR
\item tabac = CI
\item si migraine et vascularite, voir spécialiste
\end{itemize}

\subsubsection{Contraception d'urgence}
\label{sec:org3ff8664}
\begin{itemize}
\item lévonorgestrel ASAP < 72h
\item ulipristal acétate ASAP < 120h mais 3x plus cher
\end{itemize}

\subsection{37 \textdagger{} Stérilité du couple}
\label{sec:org2de2ca5}
Infertile : 0 grossesse après 1 an de rapports non protégés. Stérilité si
définitif.

Stérilité = partagée !!

\subsubsection{Interrogatoire}
\label{sec:org4c43b0a}
\begin{itemize}
\item Couple
\item Femme : âge++ (détérioration après 35 ans), \{grossesses antérieure,
avortements\}, infections/curetages++, ATCD chir/infectieux, douleurs
pelviennes (rapports, règles), conditions de vie, radio/chimio
\item Homme : trouble libido/érection, ATCD cryptorchidie/trauma testiculaire, ATCD
chir pelvienne/scrotale, ATCD médicaux (orchite ourlienne++), tabac/anabolisants\ldots{}
\end{itemize}

\subsubsection{Examen clinique}
\label{sec:org347b41d}
\begin{itemize}
\item \female : âge++, obésité/maigreur, tour taille et hanche, pilosité, PA,
galactorrhé provoqué, gynéco.
\begin{itemize}
\item Si anovulation (a/oligo-ménorrhée) : hyperprolactinémie, hyperandrogénie,
troubles comportement alimentaire, bouffées chaleur
\end{itemize}
\item \male : IMC, pilosité, hypoandrisme, cicatrice chir, varicocèle\footnote{Dilatation variqueuse des veines du cordon spermatique},
gynécomastie, gynoïde/enuchoïde
\begin{itemize}
\item volume testiculaire++, palpation cordospermatiques
\end{itemize}
\end{itemize}

\subsubsection{Examens complémentaires}
\label{sec:orgd2f44e1}
Premiere intention, femme
\begin{itemize}
\item Hormonale++ : oestradiol, LH, FSH, prolactine plasmatique. Puis progestérone plasmatique (si cycle réguliers)
\item Écho ovarienne++
\item Hystérographie++
\end{itemize}
Première intention, homme :
\begin{itemize}
\item spermogramme++ (concentration, mobilité, morphologie). Attention aux variabilités !
\item hormonale++ si oligo-/azoo-spermie : testostérone, LH, FSH puis gls:SHBG
\end{itemize}
Test poist-coïtal (discuté)

\subsubsection{Étiologie}
\label{sec:org0b0e201}
Femme :
\begin{itemize}
\item \emph{anovulation} : très fréquent ! Souvent aménorrhées ou irrégularités. Causes :
gls:SOPK, hyperprolactinémie, insuf ovarienne primitive, déficit gonadotrope, psycho-nutritionnel
\item \emph{obstacle mécanique} :
\begin{itemize}
\item anomalie du col utérin et insuf glaire cervicale : post-conisation/curetage
\item obstacle, anomalie utérine : manoeuvres post-partum, polypes muqueux\ldots{} \thus
echographie
\item obstacle tubaire : cause majeure++. Souvent salpingite (Chlamydia++)
\end{itemize}
\item \emph{endométriose} : rarement en cause si modérée. hystérosalpingographie puis coelioscopie
\end{itemize}
Homme :
\begin{itemize}
\item \emph{azoospermie}
\begin{itemize}
\item \emph{sécrétoire} : diagnostic = volume testiculaire < 10ml, concentration FSH
faible
\end{itemize}
\thus caryotype, analyse bras long Y, écho testiculaire (élimine K), déficit gonadotrope (rare)
\begin{itemize}
\item \emph{obstructive} : volume et concentration ormale, volume séminal \dec \thus
examen clinique
\begin{itemize}
\item cause congénitale : agénésie bilat des canaux déférent++ (soit anomalie
biallélique gène CFTR, soit isolée)
\item acquis : infectieux  (gonocoque, tuberculose, Chlamydia) \thus échographie
\end{itemize}
\item exploration chir testiculaire et des voies excrétrices : si azoospermie
confirmée par plusieurs spermogrammes, bilan génétique
\end{itemize}
\item \emph{oligo-asthéno-térato-spermie} : \dec nombre et mobilité, \inc formes anormales
\thus caryotype, brang long Y. Traitement = assistance médicale procréation
\end{itemize}
\subsection{40 \textdagger{} Aménorrhée}
\label{sec:org528f07c}
Déf: absence de cycle menstruel après 16 ans (primaire) ou interruption chez
femme réglée (secondaire). Physiologique : grossesse, lactation, ménopause

Tout arrêt > 1 mois \thus enquête étiologique \danger

Atteinte de l'axe hypothalamo-hypophysio-ovarien ou anomalie tractus utérin

\begin{tcolorbox}
Pas de traitement oestrogénique sans enquête étiologique
\end{tcolorbox}

\subsubsection{Conduite}
\label{sec:orgcfc0f91}
\paragraph{Primaire}
\label{sec:orgec73915}
Forte proba de cause génétique/chromosomique. Chercher carences nutritionnelle

\begin{itemize}
\item Si absence de dév. pubertaire : doser FSH, LH
\begin{itemize}
\item Si basses, tumeur hypothalamo-hypophysaire, dénutrition ou génétique : \{sd
de Kalmann (anosmie), mutation récepteur GnRH (rare), atteinte
gonadotrophines (exceptionnels), mutation LH\}
\item Si hautes : sd de Turner (caryotype 45, X0),
\end{itemize}
\item Examen gynéco, écho pelvienne
\begin{itemize}
\item Pas d'utérus : sd de Rokitanski, tissu testiculaire dans les canaux
inguinaux (ex: acrshort:CAIS)
\item ambiguité acrshort:OGE : dysgénésie gonadique, hyperplasie congénitale surrénales,
anomalies sensibilité/biosynthèse androgènes
\end{itemize}
\end{itemize}
\paragraph{Secondaire}
\label{sec:org7e3bd87}
Souvent acquises. 

Interrogatoire : médic, maladie endoc/chronique,
gynoc/obstétriques, insuf ovarienne (bouffées de chaleur). Douleurs pelviennes
cycliques : cause utérine

Examen clinique : 
\begin{itemize}
\item poids et taille (carence nutritionnelle)
\item hyperandrogénie : gls:SOPK, déficit 21-hydroxylase, (sd Cushing)
\item carence oestrogénique : pas de glaire +2 semaines après saignement \thus
anovulation
\item pas de signe d'appel : enquête nutritionnelle
\end{itemize}


Dosages hormonaux : cf Table \ref{tab:amenorrhe_second}

\begin{table}
\begin{tabular}{llllll}
\toprule
hCG & prolactine \inc & FSH \inc & estradiol& testostérone > 1.5ng/mL & sinon\\
& & & LH, FSH \dec & & \\
\midrule
grossesse & médicaments & \acrshort{IOP} & tumeur H-H & tumeur surrénales & \gls{SOPK}\\
 & adénome à prolactine &  & nutrition & tumeur ovarienne sécrét. & \\
 & tumeur H-H &  &  &  & \\
\bottomrule
\end{tabular}
\caption{Évaluation hormonale d'une aménorrhée secondaire. H-H = hypothalamo-hypophysaire}
\label{tab:amenorrhe_second}
\end{table}

\subsubsection{Causes}
\label{sec:org8a0f8db}

\paragraph{1. Déficit gonadotrope organique/fonctionnel}
\label{sec:orgaffcea9}

\subparagraph{Prolactine normale \footnote{Hypothalamus n'arrive pas à libérer la GnRH au bon rythme.}}
\label{sec:org0309a56}
\begin{itemize}
\item Atteintes organiques : tumeur/infiltration \thus IRM
\begin{itemize}
\item macroadénomes hypophysaires, craniopharyngiomes
\item chercher hyperprolactinémie, insuf antéhypophysaire associé
\end{itemize}
\item Atteintes fonctionnelles : apports nutritionnels insufisants par rapport à l'activité physique intense+++
\end{itemize}

\subparagraph{Hyperprolactinémie}
\label{sec:org7402127}
Atteinte hypothalamo-hypophysaire (majeure++)

Médicaments ou tumeurs \thus pas de traitement dopaminergique sans imagerie \danger

\subparagraph{Autres}
\label{sec:org2f3ce71}
\begin{itemize}
\item Endocrinopathies : sd de Cushing, dysthyroïdes déficits 21-hydroxylase
\item Hypophysaire (rare) : auto-immune (majorité), sd de Sheehan (très rare, nécrose hypophysaire post-partum)
\end{itemize}

\paragraph{2. Anovulation non hypothalamique}
\label{sec:org9691d2e}
\subparagraph{SOPK (majorité)}
\label{sec:orgd508f6a}
Pas de pic de LH, ni de progestérone. Oestradiol mais non cycliques

Irrégularité menstruelles, puis aménorrhées avec acné, hirsutisme

Diagnostic :
\begin{itemize}
\item 2 parmi : \{hyperandrogénie clinique\footnote{Séborrhéee, acné, hirsutisme, \inc testostérone}, oligo-/a-novulation, hypertrophie
ovarienne/folliculaire\footnote{V > 10mL, \(\ge\) 19 follicule par ovaires} (écho)\}
\item exclure bloc 21-hydroxylase, tumeur de l'ovaire, sd Cushing
\item exclure hyperprolactinémie
\end{itemize}

Diagnostic parfois difficile :
\begin{itemize}
\item sans hyperandrogénie \thus écho
\item \{atteinte partielle axe gonadotrope, macroprolactinémie\} peuvent y ressembler
\end{itemize}

Acné : cherche hyperandrogénie, régularité cycle menstruel \thus éliminer
hyperplasie congénitale des surrénales

2 causes :
\begin{itemize}
\item tumeur ovarienne ou résistance insuline
\begin{itemize}
\item virilisation si tumeur
\item imagerie si testostérone > 1.5ng/mL. Si normale, cherche hypothécose
(obésité morbide androïde, acanthosis nigricans, insulino-résistance)
\end{itemize}
\item pathologie surrénale :
\begin{itemize}
\item sd de Cushing si signes hypercortisolisme \thus cortisol libre urinaire et
freinage minute
\item tumeur surrénale \thus scanner des surrénales
\item déficit enzymatique en 21-hydroxylase (\danger formes tardives qui peuvent
mimer SOPK)
\end{itemize}
\end{itemize}

\paragraph{3. Insuf ovarienne primitive}
\label{sec:orgaaabeb6}
\inc FSH

Causes :
\begin{itemize}
\item chir, chimio, radiothérapie
\item anomalie caryotype (sd Turner)
\item anomalie gènes \emph{FMR1} (sd X fragile)\footnote{Transmission mère-fils. Expansion instable des triplets CGG jusqu'à
l'absence de transcript de FMR1 (Fragile X Mental Retardation 1). \\
\danger Risque d'IOP pour la pré-mutation seulement \thus dépister chez
\female + IOP < 40 ans par PCR et Southern Blot}
\item auto-immune
\end{itemize}

\paragraph{4. Anomalie utérine}
\label{sec:orgc6d03d2}
\subparagraph{Congénitales}
\label{sec:orgbf75a60}
Si dév pubertaire normal :
\begin{itemize}
\item et douleurs pelviennes cycliques :  imperforation hyménéale/malformation vaginale \thus examen gynéco.
\item ou sans douleurs \thus agénésie utérus ?
\end{itemize}

Difficulté : différence agénésie mullérienne isolée (46,XX)- anomalies androgènes
(46,XY) \thus testostérone

\subparagraph{Secondaires}
\label{sec:org22c9a8f}
Synéchies utérines (trauma de l'utérus), tuberculose utérine
\subsection{47 \textdagger{} Puberté}
\label{sec:org601d2cc}
\subsubsection{Normale}
\label{sec:orge70a372}
\textasciitilde{}4 ans, acquisition de la taille définitive, fonction de
reproduction. Classification de Tanner (5 stades)

\begin{table}[htbp]
\caption{Puberté normale}
\centering
\begin{tabular}{llll}
\female &  & \male & \\
\hline
seins & 11 ans [8,13] & volume testiculaire & 11.5 ans [9.5,14]\\
règles & 13 ans [10,15] & \inc taille verge & 12.5 ans\\
croissance & 5 \(\rightarrow\) 8cm/an & croissance & 5 \(\rightarrow\) 10cm/an\\
taille & 163cm & taille & 175cm\\
\end{tabular}
\end{table}

\subsubsection{Retards}
\label{sec:org9028eba}
\begin{tcolorbox}
\male :  volume testiculaire < 4mL après 14 ans \footnotemark\\
\female : pas de seins à 13 ans, pas de règles à 15 ans
\end{tcolorbox}
\footnotetext{ou longueur < 25mm}

\begin{tcolorbox}
Hypogonadisme\footnotemark central ou périphérique ?
\begin{itemize}
\item FSH, LH \inc : pour compenser le manque des gonades (hypergonadotrope = primaire) 
\item FSH, LH N ou \dec : problème hypothalamo-hypophysaire\footnotemark (hypogonadotrope = secondaire)
\end{itemize}
\end{tcolorbox}
\footnotetext{Chez \male, manque de testostérone}
\footnotetext{Rappel : LH entraîne la production de testostérone}

\begin{itemize}
\item centrale : congénital (pas de cassure de croissance, ni micropénis, ni
cryptorchidie), acquis (tumeur ?), "fonctionnel" (maladie générale, trouble
comportement alimentaire), isolé
\item périphérique : sd de Turner chez \female, sd Klinefelter \male
\item retard simple (élimination)
\end{itemize}

Clinique : 
\begin{itemize}
\item parents, grossesse, courbe de croissance. Chercher trbles digestifs, polyuro-polydispsie, céphalée, anomalies champ visuel
\item pathologie acquise, OGE, testicules, anosmie (Kallmann)
\end{itemize}

Âge osseux : 13 ans \male, 11 ans \female

Biologie : stéroïdes sexuels
\begin{itemize}
\item FSH, LH basses \thus hypothalamo-hypophysaire
\item testostérone chez \male, oestradial/écho chez \female
\end{itemize}

IRM indispensable si déficit gonadotrope (tumeur) \danger

Caryotype si :
\begin{itemize}
\item FSH élevé
\item toujours chez \female{} de taille < -2DS avec retard pubertaire/gonadotrophine \inc
\end{itemize}

\paragraph{Étiologies}
\label{sec:org5432bae}
Hypogonadotropes
\begin{itemize}
\item congénitaux : isolés, sd de Kallman, autres déficits hypophysaires, sd
polymalformatifs
\item acquis : tumeurs hypophysaires, post-radiothérapie
\end{itemize}
Hypogonadotropes fonctionnels
\begin{itemize}
\item maladies chroniques digestives/cardiaques/respi
\item sport intense
\item maladies endocriniennes
\end{itemize}
Hypergonadotropes
\begin{itemize}
\item congénitaux : sd Turner, sd Klinefelter, autres atteintes primitives
\item acquis : castration, trauma, oreillons, chimio/radio
\end{itemize}

\paragraph{Traitement}
\label{sec:orgdb3ab8e}
Cause si possible. Sinon doses \inc de testostérone (\male) ou oestrogènes puis
oestroprogestatif (\female)
\subsubsection{Précoces}
\label{sec:org253b15e}
Avant 8ans \female ou 9.5 ans \male
\paragraph{Centrales}
\label{sec:orgc9e6e1e}
8x plus fréquent chez \female{} que \male{}. Chez \female{}, causes
idiopathiques. Chez \male{}, causes tumorales à 50\%

Clinique : 
\begin{itemize}
\item dév prématuré harmonieux (pas de règles chez \female)
\item crises de rires (harmatome hypothalamique), tâches cutanées (neurofibromatose
I ou sd McCune-Albright)
\end{itemize}
Biologie :
\begin{itemize}
\item testostérone élevée chez \male{} mais variabilité d'oestradiol chez \female{}
\end{itemize}

IRM hypothalamo-hypophysaire indispensable \danger (petite taille
définitive). Écho pelvienne pour \female{}

Traitement si risque de petite taille adulte : analogues GnRH jusque âge normal
de puberté
\paragraph{Périphériques}
\label{sec:org44f1084}
Clinique : \inc vitesse de croissance, avance maturation osseusse

Stéroïdes élevées, LH et FSH bas. Écho pelvienne chez fille

Étiologie :
\begin{itemize}
\item tumeurs ovarienne (rares) : écho puis histologies
\item kystes folliculaires : bénins, régression spontanée possible
\item sd McCune-Albright : 
\begin{itemize}
\item \{puberté précoce ovarienne, taches cutanées "café-au-lait", dysplasie fibreuses os\}. \danger tableau pas toujours complet !
\item oestradiol élevé, gonadotrophines basses, écho = utérus stimulé, kystes ovariens. Dominance \female
\end{itemize}
\item médicaments
\item testotoxicose (rare, cellule de Leydig activé et LH basses), adénome leydigien
(très rare)
\item tumeurs à hCG (\male)
\end{itemize}
\paragraph{Avances dissociées}
\label{sec:org7b3ba32}
\begin{itemize}
\item Isolé des seins : beaucoup de filles ( de 3 mois à 3 ans)
\item Métrorragies isolées : chercher vulvite, vulvovaginite, prolapsus urétrale,
corps étranger. Éliminter kyste ovarien, sd McCune-Albright par l'absence des
sein
\end{itemize}
\thus écho pelvienne
\begin{itemize}
\item Pilosité pubienne isolée : chercher forme d'hyperplasie congénitale des
surrénales (\inc 17-hydroxyprogestérone, stimulation ACTH), prémature pubarche
(élimination !)
\end{itemize}

\subsection{48 \textdagger{} Cryptorchidie}
\label{sec:orge109198}
\subsubsection{Enfant}
\label{sec:org6586e5e}
Localisation anormale et inaboutie du testicule. Très fréquente : 3\%
nouveaux-nés, 20\% préma. 2/3 descendent spontanément à 1 an de vie

Clinique : chercher micropénis (< 2cm, hypospadias, autres)

Explorations : endocrinienne pour toute cryptorchidie \danger
\begin{itemize}
\item bilatérale : doser 17-hydroxyprogestérone chez \female{} virilisée pour éliminer hyperplasie
congénitale des surrénales
\item testostérone, gls:Leydigcell (INSL3), gls:Sertolicell (AMH, inhibine B sérique), FSH, LH mesurée jusque 4-6mois\footnote{\danger Testostérone, FSH, LH interprétables [6mois, puberté]}
\item si bilatéral, écho (vérifier l'absence de dérivés mülleriens)
\end{itemize}

Étiologie
\begin{itemize}
\item hypogonadisme hypogonadotrope congénital
\item anorchidie rare
\item si hypospade en plus, chercher dysgénésie testiculaire
\item sd polymalformatif
\end{itemize}

Suivre l'âge de l'apparition de la puberté !

Traitement : chir dès 2 ans, indispensable ! (risque de cancer)
\subsubsection{Adulte}
\label{sec:orgefdc659}
\begin{itemize}
\item Risque : hypogonadisme, infertilité, cancer testicule
\item Examen clinique : scrotum, gynécomastie, signes d'hypogonadisme
\item Complémentaire : \{FSH, LH, testostérone\}, hCG si tumeur à la palpation, écho
scrotale, spermogramme
\end{itemize}
\subsection{51 \textdagger{} Retard de croissance}
\label{sec:org67e6f06}
\danger Ne pas passer à côté de pathologies sévères

Phases : 
\begin{itemize}
\item foetale (rapide, \{nutrition, insuline, IGF-2\})
\item précoce 0-3ans (rapide, \{insuline, IGF, hormones thyroïdiennes\})
\item prépubertaire (plus lente, décroît, \{génétique, GH/IGF, hormones thyroïdiennes\})
\item pubertaire (\{stéroïdes sexuels, GH, nutrition\})
\end{itemize}

Retard statural = \{taille < -2DS, ralentissements croissance, croissance \(\le\) parents\}

Prise de poids, obésité, ralentissement croissance \thus chercher
hypercorticisme, tumeux craniopharyngiome sur l'hypothalamus, hypopituitarisme

Examen :
\begin{itemize}
\item ATCD : taille, parents, néonatale, médicaux/chir, contexte social
\item morphotype, dév. pubertaire, tous les système, psychoaffectif
\end{itemize}

\subsubsection{Principales causes}
\label{sec:org6ee9e07}
\begin{itemize}
\item Si poids < poids idéal : cf table \ref{tab:orgcd962b1}
\item Si poids \(\ge\) poids idéal : cf table \ref{tab:orgb58b7a0}. Précisions :
\begin{itemize}
\item Test de stimulation de l'hormone de croissance (\danger si doute, IRM)
\item Ralentissement sévère \thus bilan en urgence (craniopharyngiome, thyroïdite de
Hashimoto)
\item\relax [0, 3] ans : digestives pédiatrique (coeliaque, mucoviscidose), [3,puberté] :
endoc constitutionnelle, à la puberté : déficit hormone, patho osseuse
\item Savoir différencier retard pubertaire simple d'un vrai retard
\end{itemize}
\end{itemize}

\begin{table}[htbp]
\caption{\label{tab:orgcd962b1}Causes de retard pondéral}
\centering
\begin{tabular}{ll}
Maladie coeliaque & IgA totales, IgA anti-transglutamase, fibro\\
Crohn & VS, écho anse grêle\\
Mucoviscidose & Test sueur\\
Anorexie mentale & Courbe de poids\\
Insuf rénale chroniques & Créat, iono, explo fonctionnelles\\
Anémie chroniques & NFS\\
Rachitisme hypophosphatémique & Bilan phosphocalcique\\
Patho mitochondriales & lactate/pyruvate, génétique, biopsie musc, fond d'oeil\\
Nanisme psychosocial & \\
\end{tabular}
\end{table}

\begin{table}[htbp]
\caption{\label{tab:orgb58b7a0}Causes de retard statural}
\centering
\begin{tabular}{lll}
Endocrino & Déficit GH (congénital, acquis [tumeur]) & IRM\\
 & Hypothyroïdie & T4L, TSH, Ac anti-TPO\\
 & Hypercorticisme (iatrogène) & Cortisol libre urinaire/à 23h, ACTH\\
 & Déficit hormones sex. & Testostérone, GnRH, IRM\\
\hline
Constitutionelles & Sd Turner & Caryotype\\
 & Sd Noonan & Gène PTPN11\\
\hline
Autres & Osseuses (a-/hypo-chondroplasie) & Radio\\
 & RCIU & Taille naissance\\
 & Petite taille idiopathique & Élimination\\
\end{tabular}
\end{table}


\subsubsection{Exploration :}
\label{sec:orgc783281}
\begin{itemize}
\item Caryotype : fille taille < -2DS ou < -1.5DS sous taille parentale moyenne
\item NFS, VS, foie, rein
\item IgA totales, anti-transglutaminase
\item GF-1, T4L, TSH
\item Radio
\end{itemize}

\subsection{69 \textdagger{} Troubles des conduites alimentaires (à compléter)}
\label{sec:orgbb3c4a0}
\subsection{78 \textdagger{} Dopage}
\label{sec:orgd33e482}
\subsubsection{Substances augmentant la testostérone}
\label{sec:org6cad35c}
\begin{itemize}
\item Stéroïdes anabolisant, testostérone : \inc masse musc, puissance
\item Risque : thrombotique, rupture musculo-tendineuse, trouble personnalité, foie, trouble libido, adénome/cancer de la prostate
\item Femmes : masculinisation, hirsutisme, acné, aménorrhée, anovulation, hypertrophie clitoridienne, libido exacerbée
\end{itemize}

\vspace*{0.5cm}
\begin{itemize}
\item \emph{Testostérone} : test chromatographique + spectrométrie de masse (très sensible
et spécifique)
\item \emph{Dihydrotestostérone} (DHT) : traitement gynécomastie
\item \emph{Anabolisants} : \inc tissu cellulaire (muscle).
\end{itemize}
ES : rétention hydrosodée, HTA, IDM, hépatite
\begin{itemize}
\item \emph{hCG} : diminuer épitestostérone/testostérone après dopage (IM, SC). Testée dans
le sang ou urine.
\item \emph{Anti-oestrogène} : stimule production testiculaire de stéroïdes
\end{itemize}

\subsubsection{Hormone de croissance (GH), IGF-1}
\label{sec:org0581239}
\begin{itemize}
\item GH \inc masse musculaire, modifie architecture sequelette, acromégalie \emph{mais}
pas d'effet sur volume d'activité physique. Détection difficile : approche
indirecte (cascade biologique) et mesure des forme circulante et comparaison à r-hGH
\item IGF-1 mime certains effet GH
\end{itemize}

\subsubsection{Glucocorticoïdes, ACTH}
\label{sec:orgd9172f8}
\begin{itemize}
\item Glucocorticoïdes : antalgiques, psychostimulants, combativité. ES : HTA,
oedème, rupture ligament/tendon
\end{itemize}
\danger arrêt brutal = dangereux 

\subsection{120 \textdagger{} Ménopause et andropause}
\label{sec:org19ac9c3}
\label{sec:120}
\subsubsection{Ménopause}
\label{sec:org60252ed}
Déf: plus de règle > 1 an \textpm{} sd climatérique, lié à une carence
oestrogénique. Vers 51 ans.

Pré-ménopause : irrégularités cycles puis dysovulation puis anovulation \textasciitilde{}5 ans
avants.

\paragraph{Diagnostic}
\label{sec:orge36f298}
Clinique seulement !! : plus de règles \(\ge\) 1 an, sd climatérique (bouffées de chaleur, troubles du
sommeil et humeur, sueurs nocturnes, sécheresse vaginal), \female > 50 ans. 

Bio seulement si hystérectomie \thus \dec oestradiol et \inc FSH

En pratique : progestatif seul 10j/mois x3 \thus pas de saignement à l'arrêt =
diagnostic

Aménorrhée < 40 ans = pathologique !

\paragraph{Conséquences}
\label{sec:org48e86f2}
Court terme : sd climatérique

Moyen terme : douleurs ostéoarticulaires, \inc perte osseuse (selon ATCD d'insuf
ovarienne prématurée, fractures non traumatiques, médicaments, calcium/vit D)

Long terme : \inc risque CV. Incertitude sur SNC

\paragraph{Traitement}
\label{sec:orgcd3e2ca}
Bénéfices
\begin{itemize}
\item court terme : qualité de vie à +5-10 ans
\item long terme :
\begin{itemize}
\item prévention ostéoporose
\item cardiovasculaire et neuro = incertain
\item cancer du côlon
\end{itemize}
\end{itemize}
Risques :
\begin{itemize}
\item \inc cancer du sein, accident veineux thromboemboliques (mais chiffres absolus
faibles)
\item \inc AVC ischémique, lithiase bilaires
\end{itemize}

\subparagraph{Thérapeutique}
\label{sec:org66a30dd}
\begin{itemize}
\item oestrogène (17\(\beta\)-oestradiol) oral/percutané/transdermique\footnote{Percutané, transdermique : limite \inc facteur de coagulation.} 25 jours/mois
\item \textbf{et} progestatif (au moins les 12 derniers jours) per os/transdermique
\end{itemize}
\danger hémorragie de privation possible. Si pendant le traitement, faire écho
pelvienne, hystéroscopie

\subparagraph{CI}
\label{sec:org0c9f44d}
Cancer du sein, endomètre, ATCD thromboembolique artériel (ischémique,
cardiopathie embolinogène) ou veineux, hémorragie génitale sans diagnostic, hépatique

\subparagraph{Mise en route}
\label{sec:org3e290d9}
\begin{itemize}
\item Interrogatoire : ATCD \{cancer, métabolique, vasculaire\}, carence oestrogénique
\item Examen physique : poids, PA, palpation seins, gynéco, frottis cervico-vaginal
\item Mammographie !
\item Cholestérol, triglycérides, glycémie
\end{itemize}

\subparagraph{En pratique}
\label{sec:orgc8a72ca}
1ere intention si trouble fonctionnels importants. 2eme si risque
d'ostéoporose. Sinon au cas par cas.

\subparagraph{Surveillance}
\label{sec:org80512ac}
3-6mois (surdosage = douleur, tension mammaire). Puis tous les 6-12 mois,
mammographie tous les 2 ans, frottis CV tous les 3 ans.

Traitement \(\ge\) 5 ans !!

\subparagraph{Alternatives}
\label{sec:orgef144d3}
\begin{itemize}
\item Modulateurs spécifiques du récepteur des oestrogènes : raloxifène
\item tibolone
\item traitement local préserve tractus urogénital
\end{itemize}

NB : Dépister FR CV. Promouvoir exercice, calcium, vit D

\subsubsection{Andropause}
\label{sec:orge615047}
Chez majorité des hommes mûrs/âgés en bonne santé non obèse, baisse de
testostérone inconstante (2\%).

\paragraph{Démarche}
\label{sec:org88c94cd}
\begin{itemize}
\item Interrogatoire : libido, érection, énergie vitale, mobilité/activité physique
\item Examen clinique : IMC, virilisation, gynécomastie, palper testicules
\item Mesure de testostérone totale :
\begin{itemize}
\item > 3.2ng/mL = normale \thus étiologies non endocrino
\item \(\in\) [2.3, 3.2] : doser SHBG\footnote{Se fixe à la testostérone, estradiol}, calculer index de T libre, si bas, chercher cause
\item < 2.3 ng/mL : chercher cause
\end{itemize}
\end{itemize}
\paragraph{Étiologie}
\label{sec:orgdfa6748}
Si FSH, LH élevée, \emph{insuf testiculaire primitive} 
\begin{itemize}
\item lésionnelle : chimio, radiation, alcoolisme surtout. Autres : castration,
torsion, orchite ourlienne
\item cryptorchidie bilatérale
\item chromosomique : sd Klinefelter++
\item lié à sénescene, cause génétique (rare !)
\end{itemize}
Sinon \emph{hypogonadisme hypogonadotrope}
\begin{itemize}
\item tumeur région hypothalamo-hypophysaire : craniopharyngiome, adénome
hypophysaire++, autres
\item infiltratif : sarcoïdose, hémochromatose
\item chir, radiothérapie, trauma
\item hyperprolactinémie, carence nutritionnelle, Cushing, tumeur testiculaire
\end{itemize}

\subsection{122 \textdagger{} Troubles de l'érection}
\label{sec:org33260db}
Nécessite : réseau vasculaire, appareil musculaire lisse, retour veineux, signal  nerveux,
appareil hormonal et psychisme fonctionnels

Déf : incapacité persistante à obtenir/maintenir érection pour rapport sexuel satisfaisant

Âge = FR (car déficit neurosensoriel, \inc testostérone, comorbidités)

\subsubsection{Conduite  diagnostique}
\label{sec:org621ff08}
\paragraph{Interrogatoire}
\label{sec:org8b9e8ad}
\begin{itemize}
\item DD avec perte désir, trouble éjaculation, douleurs pendant, anomalies morphologiques
\item Caractérisation : primaire/secondaire, brutal/progressif,
permanent/situationnel, sévérité (délai trouble-consult, capacité résiduelle,
masturbation)
\item Pathologies, facteur :
\begin{itemize}
\item primaire : trouble psychogène perso, complexe identitaire, trouble
relationnel, conflit socioprof, anomalie génitale
\item secondaire : ATCD abdo-pelvien, diabète, FR CV, patho CV, neuro, trouble
miction, endocrinopathie, troubles sommeil, traitement, déficit
androgénique, sd dépressif, troubles addictifs
\end{itemize}
\end{itemize}
\paragraph{Clinique}
\label{sec:org2d7456e}
\begin{itemize}
\item Gynécomastie, hypoandrisme, petits testicules, anomalies du pénis (La Peyronie)
\item CV : HTA, pouls, souffle
\item neuro : sensibilités périnée, MI
\item endoc : anomalie CV
\end{itemize}
\paragraph{Bio}
\label{sec:orgcd751f4}
Glycémie, lipidique (si > 1 an), \{NFS, iono, créat\}, foie (si > 5 ans), déficit
androgénique

Doser prolactine, hormones thyroïdiennes

\subsubsection{Bilan secondaire et approfondi}
\label{sec:org012a58c}
Secondaire : sexo/psychologique, épreuve pharmacologique (prostaglandine,
inhibiteur de la phospohdiestérase 5)

\subsubsection{Étiologies}
\label{sec:orgefbf669}
Plus fréquentes :
\begin{itemize}
\item vasculaire : FR = HTA
\item endocrino++ : diabète
\item génito-pelvien : chir pelvienne
\item trauma médullaire
\item neuro dégénératif
\item iatrogène : antihypertenseur
\end{itemize}

\subsubsection{Aspects endocriniens}
\label{sec:org28260d6}
\paragraph{Androgènes circulants}
\label{sec:org3a6f32e}
Influe libido, intérêt sexuel, érection (seulement spontanée!)

Hypogonadisme (diag difficile) : 
\begin{itemize}
\item asthénie, gynécomastie, dépilation, perte force musculaire, adiposité androïde
\item doser testostérone totale \textpm{} SHBG, prolactine. FSH, LH pour l'origin
\end{itemize}

\paragraph{Hyperprolactinémie}
\label{sec:orga8966a5}
Tumeur hypophysaire (IRM), champ visuel si tumeur
supra-sellaire, \{T4L, cortisol, IGF-1, testostérone\}
\thus correction par agoniste dopaminergique

\paragraph{Diabète}
\label{sec:orgcb26834}
sucré = 1ere cause de trouble érectile (TE). TE fréquents chez diabètique. 

Facteurs : mal équilibré, complications, âge, ancienneté diabète

Physiopatho : neuropathie autonomie, microangiopathie \thus défaut relaxation
musculaire. Macroangiopathie \thus ischémie organes érectiles

\danger facteurs psychogènes prédominent !

Diabète et TE \thus mesure testostérone systématique (hypogonadisme ?)

Clinique : 
\begin{itemize}
\item TE peut révéler diabète.
\item diabète et TE : cherche trouble endoc, vasc, neuro, médicament, dépression
\item TE = FR d'ischémie myocardite silencieuse \danger
\end{itemize}

\subsubsection{PEC}
\label{sec:org254c9fc}
Ttt étiologie seulement pour : trouble psychogène pur, chir possible, endocrino

\paragraph{Trouble endocrinien}
\label{sec:org0d15d84}
\begin{itemize}
\item Si hypogonadisme confirmé par bio\footnote{Baisse libido, testostérone totale < 3 ng/mL} : androgène oraux/intramusc/transderm
\item CI : nodule prostatique palpable, PSA > 3ng/mL
\item Surveiller prostate, foie, hématocrite
\end{itemize}

\paragraph{Pharmacologique}
\label{sec:org999ccc7}
\begin{itemize}
\item FR, Hb glyquée < 7\%, psycho/sexologique
\item 1ere intention 
\begin{itemize}
\item inhibiteurs des phosphodiésterases type 5\footnote{Efficace si stimulation sexuelle (même chez diabétique)}
\item Sinon apomorphine, yohimbine = peu efficace
\item "Pompe" = efficace mais résistance psycho
\end{itemize}
\item 2eme intention : drogue vasoactive = efficace mais douleurs peniennes, priapisme
\item Prothèses péniennes = dernier recours, par chirurgien spécialisé
\end{itemize}

\subsection{124 Ostéopathies secondaires endocrines \footnote{Secondaires autres : digestives, générale, génétique, médicaments,
autres. Il existe aussi des ostéoporoses primaires}}
\label{sec:org146c240}
Ostéoporose : -score < -2.5 DS de la valeur moyenne par DXA

Marqueurs : résorption (Tx, NTx), formation osseuse : ostéocalcine, phosphatases alcalines osseuses

\subparagraph{Hypogonadisme}
\label{sec:org9e4cd73}
Carence oestrogénique \inc ostéoclastogénèse. Tab \ref{tab:org3fcf9b9}

\begin{table}[htbp]
\caption{\label{tab:org3fcf9b9}Types d'ostéoporose dûes à un hypogonadisme}
\centering
\begin{tabular}{llll}
Type &  & Conséquences & Ttt\\
\hline
Anorexie mentale & Formation os \dec, résorption N & Risque de fractures \(\times 7\) & Multidisciplinaire\\
 &  &  & + oestroprégestatif\\
AP intensive & Origine hypothalamique & \inc résorption os, & \dec activité ou oestroprogestatifs\\
 &  & \inc fractures de fatigue & \\
Lésion hypophysaire & Doit impacter l'axe gonadotrope & Perte osseuse rapide & \oe{}strogènes\\
Iatrogène & Agonistes GnRH\tablefootnote{patho utérines}, &  & Arrêt\\
 & inhibiteurs aromatase\tablefootnote{cancer sein} &  & \textpm{} isphosphonates, denosumab\\
 &  &  & \\
Sd Turner & \dec ontinue à l'adolescence & \inc fracture (adulte) & Estrogénisation + GH\\
 &  &  & Oestroprogestatif\\
\end{tabular}
\end{table}

\subparagraph{Non hypogonadique}
\label{sec:org739c9c6}
Tab \ref{tab:orgd52fe76}
\begin{table}[htbp]
\caption{\label{tab:orgd52fe76}Autres ostéoporoses secondaires endocrino}
\centering
\begin{tabular}{lllll}
Type & Systématiquement & Physio & Atteinte & PEC\\
\hline
Hyperthyroïdie & \emph{doser TSH} & \inc remodelage & Cortical & densitométrie \textpm{} bisphosphonates (âgé)\\
hormones thyroïdiennes &  &  &  & surveillance (ttt suppressif)\\
Hypercortisolisme & \emph{Prévention} & \dec formation osseuse, & Vertébres & vitamine + calcium\\
corticothérapie &  & \inc résorbtion. &  & \textpm{} bisphosphonates \tablefootnote{si prednison > 7.5mg/j et T-score \le -1.5)}\\
Hyperparathyroïdie & \emph{Dépistage}  DXA & PTH \inc résorption & Cortical & chir si T-score < -2.5.\\
primitive & (ménopausée++) &  &  & Sinon anti-ostéoclastiques\tablefootnote{oestrogènes,aloxifène, bisphosphonates},\\
 &  &  &  & calcimimétique\tablefootnote{cinacalcet}\\
\end{tabular}
\end{table}

\subparagraph{Chez l'homme}
\label{sec:orgf784900}
Pas de T-score reconnu. 

Surtout : hypercorticisme, hypogonadisme congénital/acquis/iatrogène, alcoolisme, hypercalciurie
idiopathiques, génétique

\begin{tcolorbox}
\begin{itemize}
\item bisphosphonates (prévention ostéoporose cortisonique++) sinon dénosumbab 
\item raloxifène (si faible risque fractures périphériques)
\item tériparatide (si \ge 2 fractures vertébrales)
\end{itemize}
\end{tcolorbox}

\subsection{207 \textdagger{} Sarcoidose}
\label{sec:orgea7d671}
Atteinte hypothalamo-hypophysaire exceptionnelle. Conséquences : diabète
inspide central, insufisance gonadotrope

Radio : IRM centrée sur hypothalamo-hypophyse = référence (T1,T2 injecté) \thus
infiltration plancher 3eme venticule, infundibulum, tige hypophysaire épaissie
\textpm{} hypophyse augmente de volume

DD : tuberculose, histiocytose, lymphome, autres tumeurs de la région 

Si patient avec sarcoïdose connue : diagnostic = déficit endocrinien et imagerie\footnote{Faire bilan hormonal : natrémie, testostérone totale et libre/ostradiol,
FSH, LH, T4L, TSH, prolactine}

Sinon : atteinte rare\footnote{adénome hypophysaire 90\%, méningiome, craniopharyngiome, patho
inflammatoires infiltratives}, diag = radio et arguments sarcoïdose\footnote{Atteinte poumon évocatrice, \inc{} ACE, pas de tuberculose, granulome
non caséeux (histologie)}.

Traitement : sarcoïdose et déficits hormonaux

\subsection{215 \textdagger{} Hémochromatose}
\label{sec:org52d10da}
Hémochromatose primitive : génétique, surcharge en fer. 5 pour 1 000 !

Physiopatho : 
\begin{itemize}
\item Absorption intestinale régule stockage de fer
\item Fer entre dans l'entérocyte (DMT1), puis stocké via ferritine ou relargé par ferroportine
\item Hepcidine \dec quand besoins fer \inc (!)
\item Hémochromatose : hepcidine effondrée, DMT1 et ferroportine \inc
\end{itemize}

Génétique : gène HFE à 95\% et mutation C282Y/C282Y ou C282Y/H63D

\subsubsection{Clinique}
\label{sec:orga6b87bb}
En pratique, suspicion aux "3 A" : asthénie, arthralgies, \inc ALAT

\paragraph{Atteintes :}
\label{sec:org8dfa2c0}
\begin{itemize}
\item foie : \inc ALAT ou hépatomégalie. Cirrhose \(\approx\) 90\% décès
\item coeur : cardiopathie dilatée, troubles rythme
\item endocrino :
\begin{itemize}
\item diabète++ (accumulation pancréatique de fer) insulino-pénie/-résistance
\item hypogonadisme+ : impuissance \male, aménorrhée \female, \dec libdio,
ostéoporose
\item insuf thyriodienne exceptionnelle
\end{itemize}
\item articulaire : arthrite chronique ("poignée de main douloureuse"), chrondocalcinose
\item cutané : mélanodermie (tardive)
\end{itemize}


\subsubsection{Diagnostic}
\label{sec:org37b6e3e}
\begin{itemize}
\item Si CS-Tf\footnote{Coefficient de saturation de la transferrine} < 45\% : si ferritine \inc, cherche hépatosidérose dysmétabolique,
acéruléoplasminémie, mutation gène de la ferroportine 1
\item Sinon, CS-Tf > 45\% : 
\begin{itemize}
\item si C282Y/C282Y ou C282Y/H63D : diagnostic
\item sinon, si ferritine \inc, test génétique de 2eme intention, biopsie
hépatique
\end{itemize}
\end{itemize}

Examen complémentaires : pancréas (glycémie),  foie (transaminases, écho abdo), ECG \textpm{} écho
cardiaque, radio articulation, bilan testostérone

Dépistage chez parents (1er degré) : bilan martial \textpm{} dépistage génétique. \danger mutation \(\neq\) maladie

\subsubsection{Stades}
\label{sec:org94be75c}
\begin{enumerate}
\item Asymptomatique, CS-Tf, ferritinémie normaux
\item CS-Tf \inc
\item CS-Tf \inc et ferritine \inc
\item Idem et expression clinique affectant qualité de vie
\item Idem et expression clinique affectant pronostic vital
\end{enumerate}

\subsubsection{Traitement}
\label{sec:org6667087}
À partir du stade 2

\paragraph{Saignées = référence}
\label{sec:orga065ad9}
Objectif : ferritine < 50 g/L (hebdomadaire) puis entretien tous les
  2-4 mois. Ne pas dépasser 550mL !

CI : anémie sidéroblastique, thalassémie majeure, cardiopathies sévères

\paragraph{Autres}
\label{sec:org665d31f}
\begin{itemize}
\item Érythraphérèse : coûteuse, plus difficile
\item Chelation du fer : 2eme intention (coût, effets indésirable)
\item diététique : pas d'alcool, éviter vitamine C mais \textbf{conserver} apports en fer !
\item Symptomatique
\end{itemize}

\subsubsection{Suivi}
\label{sec:org35ed9be}
Résultats en 3-6 mois sur état générale. 

Bilan ferrique (stade 0,1) ferritinémie, hémoglobine (stade 2 à 4)

\subsection{221 HTA, causes endocriniennes}
\label{sec:orgbced6ba}
Déf: \(\ge\) 140/90 mmHg.
Enquête :
\begin{itemize}
\item initiale : ATCD familiaux HTA, souffle para-ombilical, rein/masse abdo à la
palpation, signe d'hypercortisolisme/acromégalie, bio \thus
protéinurie/hématurie, imagerie, hormonale (selon signes)
\item si résistance malgré 3 antihypertenseurs (dont 1 diurétique), chercher toutes
les cause d'HTA
\end{itemize}


\subparagraph{Épidémiologie}
\label{sec:org196fe2a}
10\% des HTA sont secondaires et 5\% sont guéries \thus hyperaldostéronisme
primaire, phéochromocytomes, sd Cushing

\subsubsection{Hyperminéralocorticisme primaire (HAP)}
\label{sec:orgf1a68dd}
Physiopatho : aldostérone, cortisol, désoxycorticostérone \thus rétention sodée
\thus HTA et inhibe sécrétion de rénine\footnote{Si la surrénale produit plus d'aldostérone : régulation négative par la rénine (en théorie)}.

\subparagraph{Diagnostic}
\label{sec:orgb9eb178}
Aldostérone \inc (plasma/urine) et urine basse
\begin{itemize}
\item suspicion : hypokaliémie (< 3.5mmol/L) ou HTA résistante
\item confirmation : \ref{algo:HAP}  
\begin{algorithm}
  \caption{Explorations des HAP}
  \label{algo:HAP}
  Arrêt diurétiques\;
  Vérifier natriurèse+, kaliurèse > 20mmol/j\;
  \If{aldostérine/rénine \times 2}{
  aldo \inc et rénine \dec : HAP\;
  aldo \inc et rénine \inc : hyperaldo. secondaire\;
  aldo \dec et rénine \dec : autre minéralocorticisme\;
  }
\end{algorithm}
\end{itemize}

\subparagraph{Ttt selon étiologie}
\label{sec:org5efcec6}
Adénome de Conn : ndule unilatéral hypodense \(\in\) [10, 20] mm au scanner
\begin{itemize}
\item prouver sécrétion aldostérone par cathétérisme si scanner douteux/patient jeune/HTA résistante
\item chir possible (mais tumeur bénigne, risque récidive)
\end{itemize}
Hyperplasies idiopathique: spironolactone à vie (hypoK) + contrôle PA

\subparagraph{Hyperminéralocorticismes familiaux}
\label{sec:orga65a623}
Lié à l'aldostérone, désoxycorticostérone, cortisol

\subsubsection{HTA endocrines iatrogènes}
\label{sec:org73c7c4f}
Contraception oestroprogestative, corticostéroides, réglisse

\subsubsection{Phéochromocytomes, paragangliomes fonctionnels}
\label{sec:org1606065}
\subparagraph{Physiopatho}
\label{sec:orgb63880f}
gls:PCC : médullosurrénale. gls:PGG fonctionnels : autres    ganglions sympathiques
 \{\} PCC : spontanément mortel. 

\subparagraph{Dépistage :}
\label{sec:org5f0c42c}
\begin{itemize}
\item HTA avec céphalées, sueurs, palpitations (triade de Ménard), HTA paroxystiques/diabète sans
surpoids
\item sd familial : gls:NF1, gls:VHL, gls:NEM2, sd phéochromocytomes-paragangliomes familiaux
\end{itemize}

\subparagraph{Diagnostic}
\label{sec:org9129ab4}
\inc métanéphrines

\subparagraph{PEC}
\label{sec:orgd07afbd}
\begin{itemize}
\item Scanner/IRM/écho : PCC = uniques, \textasciitilde{}5cm, PCC siègent dans l'organe de Zuckerkandl, vessie\ldots{}
\item Médicine nucléaire
\item Traitement chir mais surveillance long terme
\end{itemize}

\subsubsection{Sd de Cushing (hypersécrétion de cortisol)}
\label{sec:orgcd71543}

\subparagraph{Diagnostic}
\label{sec:org7511450}
Suspicion clinique + cortisol plasmatique \footnote{Physiologique = minimal à minuit, donc mesure à minuit. Mesure salivaire possible.}, cortisolurie 24h, test de freinage rapide\footnote{1mg de dexaméthasone à minuit Action de rétrocontrôle négative du
cortisol donc on vérifie que le cortisol plasmatique le lendemain à 8h a bien diminué}
\begin{itemize}
\item clinique: acné, ecchymoses, faiblesse musc, hirsutisme, oedèmes, ostéoporose, PAd
> 105mmHg, vergetures pourpres
\end{itemize}

\subparagraph{Étiologie}
\label{sec:orgeac6acf}
\begin{itemize}
\item ATCH diminuée \thus adénome, corticosurrénalome, hyperplasie bilatérale
\item ATCH normale ou \inc \thus test CRH, test freinage fort\footnote{8mg au lieu de 2mg pour DXM et pendant 2 jours au lieu d'un}. si positif : tumeur ectopique ou
maladie de Cushing  (adénome hypophysaire)
\end{itemize}

\subsubsection{Causes rare}
\label{sec:org3e2cd85}
Tumeurs à rénine, acromégalie
\subsection{238 \textdagger{} Hypoglycémie}
\label{sec:org3f0b84d}
Diagnostic : neuroglucopénie et glycémie < 0.50g/L (0.60 chez diabétique) et correction symptômes
à normalisation (triade de Whipple)

Causes :
\begin{itemize}
\item sécrétion inappropriée d'insuline (hypoglycémiante)
\item (rare) : défaut de sécrétion d'hormones hyperglycémiantes (GH, glucagon,
catécholamine, cortisol), déficit néoglucogénèse, défaut substrat
\end{itemize}

\subsubsection{Symptômes}
\label{sec:org58466c6}
Neuroglucopénie : faim brutale, troubles concentration, troubles moteurs,
troubles sensitifs, troubles visuels, convulsions focales/généralisése,
confusion

Coma hypoglycémique : début brutal, agité (sueurs), irritation pyramidale, hypothermie

\begin{itemize}
\item souvent signes adrénergiques : anxiété, tremblements, nausées, sueurs,
pâleur, tachycardie
\end{itemize}

\subsubsection{Causes}
\label{sec:org1774761}
\paragraph{Diabétique}
\label{sec:org4ebc963}
Si traité par insulines, hypoglycémiants oraux

Ttt : sucre (3 morceaux) si CS, sinon glucagon 1mg par IM/SC (CI si
sulfonylurée : glucose en perfusion)

\paragraph{Insulinome}
\label{sec:org18e1a55}
1ere cause tumorale (mais rare). Maligne dans 10\%, < 2cm (90\%)

Clinique : manif. adrénergiques surtout

Diagnostic : épreuve de jeûne, cf table \ref{tab:org36a6bfc}

\begin{table}[htbp]
\caption{\label{tab:org36a6bfc}Diagnostic d'hypoglycémie (jeûne) avec DD}
\centering
\begin{tabular}{llll}
 & Insulinome & Insuline cachée & Sulfonylurée cachée\\
\hline
Glycémie & basse & basse & basse\\
Signes & neuroglucopénie &  & \\
Insulinémie & normale mais inadaptée & dosable & dosable\\
Peptide C & augmenté & \emph{basse} & augmenté\\
Sulfamides & 0 & 0 & \emph{oui}\\
pro-insuline & augmenté & basse & \\
\end{tabular}
\end{table}

Scanner en coupe fine du pancréas et écho-endoscopie si médecin habitué

Traitement : chir

\begin{tcolorbox}
Hypoglycémie par sécrétion inaproppriée d'insuline : triade de Whipple, glycémie \le 0.45g/L\footnotemark avec
insulinémie \ge 3 mUL/L, peptide C \ge 0.6ng/mL
\end{tcolorbox}
\footnotetext{Spontanément/jeûne}

\subsection{239 \textdagger{} Goitre, nodules thyroïdiens, cancers thyroïdiens}
\label{sec:org01c05c5}
Besoins en iode quotidiens (synthèse hormones thyroïdiennes) :  \(\approx\) 150 \(\mu\)g/jour (ado,
  adulte, \texttimes{} 2 chez enceinte)

Goitre = hypertrophie de la thyroïde :
\begin{itemize}
\item palpation > dernière phalange du pouce
\item écho : volume > 20 \(cm^3\) (18 femme adulte, 16 ado)
\end{itemize}

\subsubsection{Évaluation}
\label{sec:orgd843321}
Clinique : mobile déglutition/visible cou en extension/visible à
distance. Chercher : gene fonctionelle, signes de compression, signes de
dysfonction thyroïdienne, acrshort:ADP

Bio : 
\begin{itemize}
\item TSH++ : \inc, déficit production, si \dec, imprégnation excessive en hormones thyroïdiennes.
\item compléter par T4, et si TSH \inc : Ac anti-TPO, anti-Tg
\end{itemize}

Échographie

\subsubsection{Goitre simple}
\label{sec:orgb1484af}
Hypertrophies normo-fonctionnelles non inflammatoires non cancéreuses

Facteurs : \female, tabac, déficience iodée

\paragraph{Évolution}
\label{sec:orgb135ff8}
Constitution à l'adolescence (cliniquement latente) puis plurinodulaire : gêne
cervicale \thus TSH, écho, ponction, scintigraphie
\danger cherche caractère plongeant sur radio !

À ce stade, complications : hématocèle, strumite, hyperthyroïdie, compression
organes de voisinages, cancerisation (5\%)

\paragraph{PEC}
\label{sec:orgb68c6e9}
\begin{itemize}
\item Ado : levothyroxine (1 à 1.5 \(\mu\)g/kg/j) jusque V normal. Vérifier TSH
\item Adulte/agé : si multinodulaire non malin, surveillance. Si symptomatique,
thyroïdectomie totale
\item Goitre ancien, négligé : iode 131
\end{itemize}
Dans tous les cas, \inc iode (grossesse)

\paragraph{Autres pathologies responsables}
\label{sec:org0ea4ece}
\begin{itemize}
\item Maladie de Basedow
\item Thyroïdites : 
\begin{itemize}
\item Hashimoto = hypertrophique. Goitre très ferme, expose à l'hypothyroïdie. Ac Ant-TPO\inc\inc{}, écho : goitre diffus, hypoéchogène
\item autres thyroïdites
\end{itemize}
\item Troubles de l'hormonosynthèse
\end{itemize}

\subsubsection{Nodules thyroïdes}
\label{sec:org9111882}
Déf : toute hypertrophie localisée de la gande thyroïde. Majorité = bénin (5\%
cancers, de très bon pronostic)

Prévalence \(\approx\) décennie du sujet. \texttimes{} 2 chez \female. \inc si grossesse,
déficience iode, irradiation cervicale

\paragraph{Évaluation :}
\label{sec:org69dd6d1}
Si signe d'accompagnement :
\begin{itemize}
\item nodule douloureux brutal : hématocèle
\item nodule douloureux + fièvre : thyroïdite subaigüe
\item nodule compressif + ADP : cancer
\item nodule + hyperthyroïdie : nodule toxique
\item nodule + hypothyroïdie : thyroïdite lymphocytaire
\end{itemize}
Si isolé : 
\begin{itemize}
\item TSH \dec : nodule hyperfonctionnel ? \thus scintigraphie
\item TSH N : tumeur \thus écho, cytologie
\item TSH \inc : thyroïdite lymphocytaire ? \thus Ac anti-TPO
\end{itemize}

Pronostic plutôt suspect : 
\begin{itemize}
\item homme, enfant/âgé, ATCD irradiation cervicale, > 3cm, ovalaire, dur, irrégulier, > 20\% en un an
\item écho : hypoéchogène, contour irrégulier, microcalcifications, ADP
\end{itemize}

Bio : TSH surtout. 
\begin{itemize}
\item si nodule, calcitonine > 100pg/mL = argument solide pour cancer médullaire thyroïde.
\item calcitonine \(\in\) [20,50]pg/mL : idem ou hyperplasise des cellules C ou insuffisant rénal
\end{itemize}

Examens : 
\begin{itemize}
\item Échographie (classification TI-RAD de 1 à 6)
\item Cytologie si nodule suspect (classification Bethesda de 1 à 6)
\item Scinti si cytologie ininterprétable 2 fois ou indéterminée
\end{itemize}

\paragraph{Thérapeutique}
\label{sec:org9ead6fa}
\begin{itemize}
\item Chir si suspect clinique/écho/cyto/calcitonine \inc\inc{} : thyroïdectomie si dystrophie controlatérale
\item Surveillance sinon
\item Hormonal si bénin dans familles avec goitres plurinodulaire, < 50 ans.
\end{itemize}

Kystes, hématocèles : anéchogène \thus ponction \textpm{} hormonothérapie , alcoolisation.

Grossesse : chir possible 2e trimestre ou après accouchement 

Nodule oculte : < 1cm. Risque de cancer 5\%, faible pouvoir agressif
\begin{itemize}
\item \danger si ADP, hérédité cancer médullaire thyroïde, fixation au TEP
\item ponction seulement si hypoéchogène et > 8mm
\end{itemize}
\subsubsection{Cancers thyroïdiens}
\label{sec:orgfa7c68c}
1.5\% cancers, 4eme chez la femme

Découverte : fortuite++, ADP cervicale, signes de compression, flushes/diarrhée,
localisation métastatique

Anatomie :
\begin{itemize}
\item carcinomes différenciés d'origine vésiculaire : papillaire (85\%, excellent
pronostic), vésiculaires (5\%), peu différenciés (2\%)
\item carcinomes anaplasiques (1\%)
\item carcinomes médullaires au dépens des cellules C
\item autres
\end{itemize}

Risque de rechute/décès : 
\begin{itemize}
\item taille tumeur, effraction capsule thyroïdienne, métastase (clasif TNM de I à IV)
\item mortalité \(\propto\) âge, dépend de l'histologie, exérèse
\end{itemize}

\paragraph{Thérapeutique}
\label{sec:org618e165}
\begin{itemize}
\item Plan cancer
\item Chir en 1ere intention (anatomopatho pendant = certitude) : thyroïdectomie
totale. Curage ganglionnaire si besoin (systémique si carcinome médullaire,
si enfant/ado). \\
Complications : hémorragie postopératoire ,  hypoparathyroïdie (calcium + vit D), paralysie transitoire/définitive nerfs récurrents
\end{itemize}
\vspace*{10pt}

\emph{Cancers différenciés d'origine vésiculaire}
\begin{itemize}
\item iode 131 : seulement post-thyroïdectomie totale (haut risque). Nécéssite
stimulation par L-T4 ou injection TSH. Puis hospit après en chambre 2-5 j
et contraception 6-12 mois. \\
ES : \{nausées, oedèmes\}, \{agueusie, sialadénite\}. \\
Scinti  obligatoire à +2-8j : fixation extracervicale à  distance = métastases
\item hormonal : L-T4 si haut risque ou échec traitement initial. Puis mesurer TSH à
+6sem-2mois (pas avant !)
\item surveillance : 80\% des récidives à 5 ans \thus écho cervicale, rhTSH,
Tg\footnote{Thyroglobuline} à 6-12mois : cytoponction puis imagerie si Tg > seuil. Sinon \dec LT4
\item traitement récidives : chir si cervicale. Plus compliqué si métastases
(iode131 si fixant sinon ttt local ou molécules ciblées). Maintenir LT4
\end{itemize}

\emph{Cancers anaplasiques}\\
Tuméfaction cervicale rapidement progressive, dure, adhérente, sujet âgé \thus radio-chimio. Pronostic très péjoratif

\emph{Cancers médullaires}
\begin{itemize}
\item TTT : chir \textpm{} curage ganglionnaire
\item Surveillance : calcitonine > 150\(\mu\)g/L \thus bilan de localisation.
\item Temps doublement : 6 mois = pronostic très mauvais.
\item Traitement métastases = local.
\end{itemize}

Étude génétique dans tous les cas : positif \thus chercher phéochromocytome,
hyperparathyroïdie + enquêtes apparentés
\subsection{240 \textdagger{} Hyperthyroïdie}
\label{sec:org168b72c}
\begin{tcolorbox}
Examen en 1ere intention : TSHus (puis T4L !)
\end{tcolorbox}

Déf : hyperfonctionnement de la glande thyroïdienne. Sd de thyrotoxicose =
conséquence sur les tissus.

Prévalence élevée, 7\texttimes{} femme

Physiopatho :
\begin{itemize}
\item TSH, gls:TPO et Tg peuvent être des auto-antigènes
\item thyroïde produit surtout thyroxine (T4\footnote{[T4] n'est à l'équilibre que +5 semaines après modification de T4}), convertie en T3 par foie, muscle
squelette.
\item effet : 
\begin{itemize}
\item \inc production chaleur, \inc production énergie, \inc consommation \(O_2\)
\item \inc débit cardiaque, système nerveux, \inc ostéclasie, \inc lipolyse, \inc
glycémie, rétrocontrole négatif hypophysaire
\end{itemize}
\end{itemize}

\subsubsection{Sd de thyrotoxicose}
\label{sec:orgda5217e}
Clinique (par fréquence \dec) :
\begin{itemize}
\item CV : tachycardie (régulière, repos, \inc effort), \inc intensité bruits
coeurs, \inc PAs
\item neuropsy : nervosité, tremblement fin régulier des extrémités, fatigue
générale, troubles sommeil
\item thermophobie, hypersudation,
\item amaigrissement rapide, important, avec appétit conservé
\item autre : polydipsie, amyotrophie, \inc frequence selles, rétraction paupière
supérieure (gynécomastie, troubles règle)
\end{itemize}

Examen complémentaire : TSH effondrée. T4 ou T3 libre pour l'importance

Complications : 
\begin{itemize}
\item cardiaque (surtout personnes fragiles) : troubles rythme supraV (FA), insuf
cardiaque (droite, avec débit N ou \inc), aggravation insuf coronaire
\item crise aigüe thyrotoxique (exceptionnelle)
\item musculaire (âgé)
\item ostéoporose (\female ménopausée) : rachis
\end{itemize}

\subsubsection{Étiologies (fréquence \dec)}
\label{sec:orgbb48a0a}
\paragraph{Auto-immunes}
\label{sec:org0b1b329}
\emph{Maladie de Basedow}\\
1\% population. Auto-immune, sur terrain génétique. Poussées puis rémissions

Clinique : 
\begin{itemize}
\item goitre diffus homogène, élastique, souffle
\item oculaire (spécifique, inconstant) : rétraction et asynérgie palpébrale,
inflammation, exophtalmie, oedème paupières, inflammation conjonctive,
limitation mouvement regard
\thus examen ophtalmo ! (acuité visuel, cornée, papille, oculomotricité, tonus
intraoculaire)\\
Mauvais pronostic : exophtalmie importante, paralysie complète, neuropathie
optique, hypertonie oculaire avec souffrance papillaire
\item dermopathie (exceptionnelle) placard rouge, surélevé, induré, face ant jambes
\end{itemize}

Diagnostic : manif oculaire suffit. sinon : écho (hypoéchogène, vascularisé),
(scinti), Ac anti-récepteur TSH\\

\emph{Autres auto-immune}
\begin{itemize}
\item Thyroïdite post-partum (5\%) : hyperthyroïdie transitoire puis hypothyroïdie. Ac
anti-TPO mais pas Ac anti-récepteur TSH
\item Thyroïdite d'Hashimoto : goitre irrégulier, très ferme. Écho :
hypoéchogène. Ac anti-TPO mais pas anti-récepteur TSH
\end{itemize}

\paragraph{Nodules thyroïdiens hypersécrétans}
\label{sec:org2618a98}
Âge plus avancé, sd de thyrotoxicose pur (pas de manif oculaire) 
\begin{itemize}
\item Goitre multinodulaire toxique : à la clinique, puis écho. Scinti : "en damier"
\item Adénome toxique : palpation nodule unique, écho : tissulaire/partiellement
kystique. Scinti nécessaire : reste du parenchyme "froid"
\end{itemize}

\paragraph{Iatrogènes}
\label{sec:org54ce878}
\begin{itemize}
\item Iode : produits contraste, amiodarone. 2 formes : fonctionnelle ou lésionnelle
(lyse des cellules)
\end{itemize}
\danger sous amiodarone : T4L \inc mais T3L, TSH N 
\begin{itemize}
\item Hormones thyroïdiennes : pour maigrir. Diag : scinti (pas de fixation), Tg
effondrée
\item Interféron (fréq++)
\end{itemize}

\paragraph{Thyroïdite subaigüe de De Quervain}
\label{sec:orgfccbae2}
Affection banale virale. Diagnostic clinique (goitre dur et douleureux). Hyper-
puis hypo-thyroïdie. Echo = hypoéchogène

\paragraph{Thyrotoxicose gestionnelle transitoire}
\label{sec:orge709def}
Fréquent (2\% grossesse). 1er trimestre : nervosité, tachycardie, pas de prise de
poids

DD : Basedow (pas Ac anti-récepteur TSH)

\paragraph{Rares}
\label{sec:org92220f2}
Mutations activatrices du récepteur TSH, métastase massives sécrétantes (K
thyroïdiens vésiculaire différencié), tumeurs placentaires/testiculaires, \{sd
résistance hormones thyroïdiennes, adénome hypophysaire\}

\subsubsection{Forme clinique}
\label{sec:org0217aaf}
\begin{itemize}
\item Enfant : généralement Basedow (néonatale/acquise) : avance staturale et
osseuses, hyperactivité \textpm{} signes oculaires
\item Femme enceinte : passage d'Ac \thus hyper- ou hypo-thyroïde. Passage
d'antithyroïdiens de synthèse \thus goitre, hypothyroïdie possible. Contraception !
\item Âgé : évolution discrète (AEG, fonte musculaire, cachexie, insuf
cardiaque). Penser thyrotoxicose si troubles rythme/insuf cardiaque
\end{itemize}

\subsubsection{Traitement}
\label{sec:org08ba339}
\{\} Urgence : crise aigüe thyrotoxicose, cardiothyréose chez âgé/cardiqaue,
orbitopathie maligne, cachexie vieillard, Basedow chez \female{} enceinte

Repos, sédatifs, bêtabloquant, contraception

gls:ATS :
\begin{itemize}
\item -mazole (30-60mg/j), -thiouracile (300-600mg/j) : bloque TPO
\item ES : allergies cut, \inc enzymes hépatiques, neutropénie, agranulocytose++
( !!)
\item surveillance : T4 libre jusque N puis T4L et TSH. NFS 10jours pendant 2 mois (agranulocytose)
\end{itemize}

Chir : thyroidectomie totale sauf si adénome toxique (lobectomie)

Radio-iode : simple, sans risque génétique/cancérisation secondaire (\danger{} orbitopathie\ldots{}). CI : femme enceinte.

\paragraph{Résultats}
\label{sec:orge5fd7ac}
\begin{itemize}
\item Basedow : thyroïdectomie \thus hypothyroïdie définitive. Radio-iode \thus
hypothyroïdie 50\%, risque aggravation orbitopathie. Donc ttt médical (1-2
ans) puis chir/iode si récidive
\item Adénome/goitre multinodulaire toxique : chir, iode
\item Induite par l'iode : arrêt si possible
\item Thyroïdite subaigüe : anti-inflammatoire (AINS/corticoïde)
\end{itemize}

\paragraph{Formes particulières}
\label{sec:orga2fe8e4}
\begin{itemize}
\item Cardiothyréose : propanolol et anticoag. Si insuf cardiaque : tonicardiaque,
diurétiques, vasodilatateurs, betabloquant, anticoag. Pour thyrotoxicose : ATS
puis chir/iode 131
\item Crise aigüe thyrotoxique : soins intensifs, réa, ATS, propanolol, corticoïdes,
iode131 après 24h ATS
\item Orbitopathie : pas d'effet ATS, iode peut aggraver !! Si simple, collyre. Si maligne : cf spécialiste
\item Femme enceinte : si transitoire, repos. Si Basedow : repos si mineur. Si forme
importante : ATS faible dose. Si formes grave, chir (2eme trimestre) possible)
\end{itemize}
\thus surveillance avant et après accouchement  

\subsection{241 \textdagger{} Hypothyroïdie}
\label{sec:org0e3f94a}
\begin{tcolorbox}
Rappel : TRH (hypothalamus) stimule la production de TSH (hypophyse) qui stimule la thyroïde
\end{tcolorbox}

\begin{itemize}
\item Atteinte de la glande thyroïde  : \inc TSH et 
\begin{itemize}
\item soit T4L N : hypothyroïdie frustre
\item soit T4L \dec : hypothyroïdie patente
\end{itemize}
\item Ou hypothalamo-hypophysaire : T4L \dec et 
\begin{itemize}
\item soit TSH légèrement \inc : hypothalamus
\item soit TSH \dec ou N : hypophysaire
\end{itemize}
\end{itemize}

\subsubsection{Sémiologie}
\label{sec:org14b3c10}
Général :
\begin{itemize}
\item sd d'hypométabolisme\footnote{Asthénie, somnolence, hypothermie, frilosité, constipation, bradycardie,
prise poids modeste}
\item peau pâle/jaune, sèche, squameuse, dépilée; cheveux secs cassants
\item myxoedeme cutanéomuqueux : faciès "lunaire", voix rauque, hypoacousie,
macroglossie
\item neuromusc : crampes, myalgies
\item endocrinien : (galactorrhée), troubles règles, troubles libido
\end{itemize}
Cliniques (rare, diag fait avant) :
\begin{itemize}
\item CV : bradycardie sinusale, \dec contractilité, (insuf cardiaques, troubles
rythme V), épanchement péricardique, favorise athérome coronarien
\item neuromusc, neuropsy : dépressif, sd confusionnel, démence, myopathie prox,
apnée sommeil
\item coma myxoedemateux : si hypothyroïdie primaire profonde et
aggression. Convulsion, EEG non spécifique. Hyponatrémie. Pronostic sévère
\end{itemize}

Palpation : glande ferme hétérogène, pseudonodulaire

Grossesse : 
\begin{itemize}
\item complication mère : HTA, prééclampsie, fausse couche, hémorragie post-partum
\item complications foetus : troubles developpement neuro-intellectuel, hypotrophie
\item 1er trimestre : TSH \dec, T4L limite sup. Puis TSH normale, T4L basses
(physiologique !)
\end{itemize}

Anomalies bio :
\begin{itemize}
\item hémato : anémie normocytaire normochrome (si macrocytose, penser anémie de
Biermer) troubles de coagulation,hémostase
\item hypercholestérolémie, \inc CPK, hyponatrémie dilution
\end{itemize}

\subsubsection{Étiologies}
\label{sec:org4cf693b}
\paragraph{Hypothyroïdie primaire}
\label{sec:orge0d0196}
Auto-immunes :
\begin{itemize}
\item Thyroïdite d'Hashimoto : 
\begin{itemize}
\item goitre ferme, irégulier, Ac anti-TPO.
\item infiltration lymphocytaire du parenchyme thyroïdien. Facteurs environnementaux,
terrain génétique.
\item penser à lymphome si \inc rapide du goitre
\item écho thyroïdiennes : hypoéchogène, hétérogène, vasc hétérogène (scint
inutile)
\end{itemize}
\item Thyroïdite atrophique : pas de goitre, Ac anti-thyroidiens moins
élevés. Souvent une évolution d'Hashimoto, > 50 ans.
\item Thyroïdite du post-partum : idem, petit goitre. Normalement résolutif dans
l'année. 5\% des grossesses
\end{itemize}
Non auto-immune :
\begin{itemize}
\item thyroïdite subaigüe de De Quervain : inflammation du parenchyme. Phase de
thyrotoxicose puis hypothyroïdie
\item thyroïdite sans Ac
\item thyroïdite iatrogène : interferon++, amiodarone, ATS, iode131, radiothérapie
cervicale, lithium, ttt anti-tyrosine kinase (cancéro)
\end{itemize}
Autres : carences iodées (endémie++), hypothyroïdie congénitale (dépistage à
naissance + 72h\footnote{Clinique discrète : ictère prolongé, constipation, hypotonie, pleurs
rauques, difficulté succin, fontanelles larges, hypothermie})

\paragraph{Démarche diagnostique}
\label{sec:org9c2a7c9}
TSH puis (T4L (profondeur) et Ac anti-TPO, échographie pour étiologie)
\paragraph{Insuffisance thyréotrope}
\label{sec:orgfbf334c}
\begin{itemize}
\item compression région hypothalamo-hypophysaire (HH) par tumeur (adénome hypophysaire
souvent)
\item séquelle post-chir, post-radio des tumeurs de la région HH
\item séquelles méningite, trauma crânien, hémorragie méningée
\item génétiques (rare)
\end{itemize}

IRM systématique !

\subsubsection{Traitement}
\label{sec:orgaf1874a}
Lévothyroxine (T4) 
\begin{itemize}
\item hypothyroïdie patente : L-T4 50 à 150 \(\mu\)g/j. Si coronarien : \inc progressivement
de 12.5 à 25\(\mu\)g/j. \danger Surveillance ! (ECG hebdo si grave, hospit si coronarien
récent, sinon patient doit consulter si douleurs thoraciques)
\item hypothyroïdie frustre : 3 cas
\begin{itemize}
\item TSH > 10mUI/L ou Ac anti-TPO : ttt
\item TSH < 10mUI/L et pas d'Ac anti-TPO : surveillance
\item si grossesse : dès TSH \(\ge\) 3mUI/L
\item à discuter sinon
\end{itemize}
\end{itemize}

Suivi
\begin{itemize}
\item hypothyroïdie primaire : objectif : TSH \(\in [0.5, 2.5]\) mUI/L (\(\approx\) 10mUI/L pour âgé, et < 2.5mUI/L pour femme eceinte)
\item insuf thyréotrope : suivi sur T4L seulement
\end{itemize}

Situations particulières:
\begin{itemize}
\item grossesse : \inc posologie dès diagnostic grossesse
\item \inc si interférence avec l'absorption intestinale\{sulfate de fer, carbonate de calcium, hydroxyde
d'alimunie, cholestyramine\}, la clairance \{phénobabrital, carbamazépinex, rifampicine,
phénytoïne, sertraline, chlooriqune\}, oestrogenes
\item néonatale : L-T4 à vie
\end{itemize}

\subsubsection{Dépistage ?}
\label{sec:org1499e9a}
\begin{itemize}
\item Adulte : si risque : signes clinique, goitre, hypercholestérolémie, ATCD
thyroïdiens, auto-immunité thyroïdienne, irradiation cervicale, \{amiodarone,
lithium, interféron, cytokines\}
\item Femme enceinte : si signes, contexte thyroïdien (perso/familial), auto-immunité
\end{itemize}

\subsection{242 Adénome hypophysaire}
\label{sec:orgf02ab66}
\subparagraph{Révélé par sd tumoral}
\label{sec:org84a4d7b}
Suspicion sur clinique, confirmé par IRM
\begin{itemize}
\item Clinique : céphalées, "voile" visuel \footnote{Par compression des voies optiques. Fond d'oeil,
acuité visuelle OK.}, quadra-/hémi-anopsie temporale
\item \danger apoplexie hypophysaire \footnote{Céphalées violentes, sd méningé sd confusionnel, troubles visuel.}(rare) thus imagerie en urgence \danger
\item IRM : microadénome (reste hypointense après injection) ou macro adénome (>
10mm, hyperintense après injection).
\item DD : craniopharyngiome intra-sellaire , méningiome intra-sellaire
\end{itemize}

\subparagraph{Révélé par sd d'hypersécrétion (Tab \ref{tab:org6feca11})}
\label{sec:org78d1aaa}
\begin{table}[htbp]
\caption{\label{tab:org6feca11}Syndromes d'hypersécrétion}
\centering
\begin{tabular}{lll}
Hypersécrétion & Signes & Diagnostic\\
\hline
Hyperprolactinémie & galactorrhée et & 1. Confirmer\\
(prolactine) & \female: trouble cycle menstruel & 2. Éliminer grossesse, médicaments, hypothyroïdie\\
 & \male: gynécomastie, troubles sexuels. & périphérique, IR\\
 &  & 3. IRM : microadénome\\
 &  & ou \{macroadénome, tumeur non prolactinique\}\\
\hline
Acromégalie & Sd dysmorphique \tablefootnote{Extrémités élargies, visage (nez élargi, front bombé, lèvres épaisses, tendance prognathisme} & Pas de freinage à HGPO\\
(acrshort:GH) & Signes fonctionnels \tablefootnote{sueurs, céphalées, paresthésies mains, douleurs articulaires, asthénie fréquente, HTA} & IGF-1 \inc\\
 & Hypertrophie myocarde (IC \thus DC ) & \\
 & Diabète, SAOS & \\
 & Goitres, polypes côlon & \\
\hline
Sd Cushing & amyotrophie ceinture et abdomen, & 1. CLU \inc, freinage minute\tablefootnote{Cortisolémie matin après 1mg dexaméthasone à 23h (rétrocontrole négatif des glucocorticoïdes sur cortisol)} négatif\\
(glucocorticoïdes) & peu amincie (mains), ecchymoses, vergétures & 2. Si ACTH \dec: adénome surrénalien ou\\
 &  & corticosurrénalome malin\\
 & graisse facio-tronculaire, bosse de bison, & 3. Sinon freinage fort\tablefootnote{Dexmathéasone toutes les 6h})\\
 & ostéoporose , hyperandrogénie & + test stimulation ACTH (CRH, métopirone\\
 & spanioménorrhée \female, impuissance \male, \dec libido, & - positif : Cushing (adénome hypophysaire)\\
 & HTA, troubles psy & - sinon sécrétion ectopique\\
 &  & DD : stress, dépression, psychose, alcoolisme\\
\end{tabular}
\end{table}

\subparagraph{Révélé par insuffisance antéhypophysaire (Tab \ref{tab:org823ff5b})}
\label{sec:org6e7a908}
Clinique : face pâle, "veillot", dépigmentation aréole mammaire et OGE,
dépilation complète aisselles pubis

IRM si déficit hypophysaire

\begin{table}[htbp]
\caption{\label{tab:org823ff5b}Insuffisance antéhypophysaire}
\centering
\begin{tabular}{lll}
Type & Signes & Diagnostic\\
\hline
Gonadotrope & \male = \{\dec libido, pilosité visage \dec, & \male{} : troubles sexuels \dec testostérone\\
 & petits testicules mou, infertile\} & \\
 & \female = \{aménorrhée, dyspareunie\} & \female{} préménopause : aménorrhée, oestradiol \dec,\\
 &  & gonadotrophines N\\
 & ostéoporose,  (retard pubertaire) & \female{} postménopause  : gonadotrophines \dec \footnotemark\\
Corticotrope & asthénie, hypotension, amaigrissement & test Métopirone,\\
 & pas de déficit en aldostérone ! & cortisol < 200ng/mL si hypoglycémie\\
 & Risque de collapsus CV & (cortisolémie, synacthène, CRH)\\
Thyréotrope & hypothyroïdie modérée & \dec{} T4L sans augmentation de TSH\\
Somatotrope & adulte = \{\dec masse et force musc, adiposité abdo\} & stimulation GH \(\times 2\)\\
 & enfant = retard croissance, hypoglycémies & stimulation GH\\
\end{tabular}
\end{table}\footnotetext[28]{\label{orgb93c69f}Ou dans les valeurs des femmes jeunes}

\subsection{243 Insuffisance surrénale}
\label{sec:orgfdf61bc}
\subsubsection{Insuffisance surrénale lente}
\label{sec:org56032a3}
Rare mais grave 

Surrénales sécrètent :
\begin{itemize}
\item glucocorticoïdes \(\approx\) cortisol \footnote{Stimulé par ACTH, rétrocontrole nég. sur ACTH} : hyperglycémiant, \inc catabolisme protidique et tonus vasculaire, \dec
ADH, anti-inflammatoire et antipyrétique,
Minimum 0-2h, maximum 7-9h
\item minéralocorticoïde \(\approx\) aldostérone : réabsportion Na+ et Cl-, excrète K+
\item androgènes surrénalien (stimulé par ACTH)
\end{itemize}

\begin{table}[htbp]
\caption{\label{tab:org9962f89}Insuffisance surrénale primaire (maladie d'Addison)/secondaire : clinique}
\centering
\begin{tabular}{ll}
Primaire (surrénale) & Secondaire (hypophysaire)\\
\hline
Fatigue, dépression, anorexie, nausées & \\
\dec poids, hypotension, hypotension orthostatique & \\
Hyperpigmentation & Pâleur\\
HyperK, hypoNa (manque sel) & HypoNa (dilution)\\
\end{tabular}
\end{table}

\subparagraph{Diagnostic}
\label{sec:org7902d06}
Cortisol + ACTH \danger ne pas attendre résultats pour commencer traitment 
\begin{itemize}
\item clinique : \ref{tab:org9962f89}
\end{itemize}
\begin{table}[htbp]
\caption{\label{tab:org47c075c}Insuffisance surrénale : diagnostic}
\centering
\begin{tabular}{lll}
 & Primaire & Secondaire\\
\hline
cortisolémie à 8h & basse & basse\\
ACTH & haute & basse\\
aldostérone & basse & N\\
rénine & haute & N\\
Synacthène & réponse insuffisante du cortisol & réponse insuffisante\\
\end{tabular}
\end{table}

\begin{itemize}
\item cortisolémie (max = 8h) puis : ACTH \inc{} si primaire, rénine \inc si primaire
\item test Synacthène\footnote{Analogue de l'ACTH} (+ Métopirone ou hypoglycémie insulinique si doute)

NB : femme enceinte = \{\inc seuil, faisceau d'args\}, enfant : répéter dosages
voire ttt probabiliste
\end{itemize}

\subparagraph{Étiologies de l' acrshort:IS primaire}
\label{sec:org111c8c1}
\begin{itemize}
\item Auto-immune (80\% adulte, 20\% enfant) : gls:PET1, gls:PET2 \thus autoAc anti-21-hydroxylase, scanner (surrénales atrophiques)
\item \emph{tuberculose bilatérale surrénale} (10\%) : transplanté ou ID avec TCD tuberculose
\thus scanner surrénales
\item \emph{VIH} (stade avancé) : iatrogène, infection opportuniste (CMV++), atteinte de
l'hypophyse (lymphome, CMV), corticoïde anti-inflammatoire et ritonavir
\end{itemize}
\danger dénutrition \thus spécialiste
\begin{itemize}
\item autres : \emph{iatrogènes}\footnote{Surrénalectomie bilatére, anticortisolique de synthèse, nécrose hémorragique}, \emph{métastases bilatérales}\footnote{\danger éliminer phéochromocytomes avant biopsie surrénale}, lymphomes, maladies
infiltratives, causes vasculaires
\item enfant : génétiques surtout = \emph{bloc enzymatique} (dépistage obligatoire), adrénoleucodystrophie
\end{itemize}

\subparagraph{Étiologies de l'IS secondaire}
\label{sec:org1584b47}
\begin{itemize}
\item \emph{interruption corticothérapie prolongée} surtout (> 7mg prednisone)
\item autres\footnote{S'associe souvent à d'autres insuffisances de l'axe hypothalamo-hypophysaire} : tumeur région hypothalamo-hypophysaire, hypophysite (auto-immune),
granulomatose, trauma, chir hypophysaire, radiothérapie, sd de Sheehan
\end{itemize}

\subparagraph{Prise en charge}
\label{sec:org4830c2e}
Ttt cause et ttt substitutif :
\begin{itemize}
\item glucocorticoïdes (hydrocortisone) 15-25mg/j
\item minéralocorticoïde (fludrocortisone) 50-150\(\mu\)g/j si IS primaire
\end{itemize}

Éducation du patient : régime normosodé, pas de laxatif, ttt à vie, hydrocortisone en SC si > 2 vomissement/diarrhées en < 1/2 journée

Surveillance clinique : surdosage en hydrocortisone/fludocortisone, cortisolémie et ACTH inutile !!

\subsubsection{Insuffisance surrénale aigüe}
\label{sec:orga7a5ac2}

\subparagraph{Diagnostic}
\label{sec:orgc5a4af1}
Si diagnostic \textbf{non} posé : cortisol + ATCH. Ne pas attendre les résultats 
\begin{itemize}
\item clinique : déshydratation extracellullaire\footnote{Pli cutané, hypotension} , confusion, trouble dig, douleurs musc, fièvre
\item biologie : IR, fonctionnelle++, hypoNa, hyperK++
\end{itemize}

\subparagraph{Causes}
\label{sec:orgc06ac03}
\begin{itemize}
\item Insuf surrénale chronique décompensée++
\item D'emblée si bloc enzymatique surrénalien (21-hydroxylase) complet (néonatale)
ou hémorragie bilat surrénale ou apoplexie hypophysaire
\item Décompensation par n'importe quelle patho intercurrente
\end{itemize}

\subparagraph{PEC}
\label{sec:orgc0564c4}
\danger Urgence extrème 
\begin{itemize}
\item 100mg hydrocortisone (IV, IM, SC) \thus \faHospital (réa)
\begin{itemize}
\item perfusion NaCL (et G30\% si hypoglycémie)
\item ttt facteur déclenchant
\item surveiller : PA, FC, FR, oxymétrie de pouls, diurèse, T, glycémie, CS, ECG
si hyperK
\end{itemize}
\end{itemize}

Ttt préventif : patient doit \inc ses doses, médecin traitant au courant

\subsubsection{Arrêt d'une corticothérapie}
\label{sec:org879418f}
Risque = ebond de la maladie causale, insuf surrénale secondaire (corticotrope), sd de sevrage

À risque : (ttt \(\ge\) 3 semaines par \(\ge\) 20mg prednisone) ou (corticoïdes et inhib enzymatique du
cytochromie P450 (ritonavir)) ou sd Cushing iatrogène
\subsection{244 \textdagger{} Gynécomastie}
\label{sec:orgf9fa618}
Hyperplasie tissue glandulaire mammaire, fréquente. Dû à oestrogène \inc{} et testostérone \dec{}. Regarder aussi TeBG,
SHBG
\subsubsection{Démarche}
\label{sec:org4c24cfc}
\begin{itemize}
\item Clinique : palpation = ferme/rugueux, mobile arrondi, centré par le mamelon (rien si adipomastie)
\item Mammographie si doute : opacité nodulaire/triangulaire (rien si adipomastie). Élimine cancer du sein (rare)
\item Physiologique ? 
\begin{itemize}
\item 2/3 des nouveaux-nés
\item pubertaire : de 13 jusque 20 ans, rétrocède . Palper testicule pour atrophie testiculaire/tumeur
\item fréquente > 65 ans. Palpation testiculaire
\end{itemize}
\end{itemize}
\subsubsection{Étiologie}
\label{sec:org9b1b94a}
\begin{tcolorbox}
Causes fréquentes : médic, idiopathique, cirrhose, insuf testiculaire/gonadotrope, (tumoral)
\end{tcolorbox}
Évidente :
\begin{itemize}
\item insuf rénale chronique, cirrhose, médicaments (surtout spironolactone,
antiandrogène, kétoconazole, neuroleptiques, ATB antirétroviraux, antiulcéreux)
\end{itemize}
Sinon exploration hormonale : T4L, TSH, hCG, testostérone totale, LH, FSH,
prolactine, oestradiol

Causes endocriniennes :
\begin{itemize}
\item hyperthyroïdie
\item insuffisance testiculaire/hypogonadisme périphérique (8\%) : sd de Klinefelter
le plus fréquent
\item hypogonadisme d'origine hypothalamique/hypophysaire: testostérone basse, LH,
FSH normales/abaissées \thus imagerie hypophysaire, dosage
prolactine. Hyperprolactinémie ou tumorale
\item tumeur sécrétant oestrogène : oestradiol \inc, testostérone \dec \thus tumeur
testiculaire (ou surrénalienne rarement) \thus echo testiculaire ou scanner
abdo
\item tumeur sécrétant hCG : \inc hCG \thus écho testiculaire, scanner
cérébrales. Dans les bronches ou le foie parfois. Chimio.
\item Résistance androgènes (exceptionnelle) : testostérone \inc, LH \inc
\item idiopathique (25\%)
\end{itemize}

\subsubsection{Traitement}
\label{sec:org8b7a389}
Traiter la cause. Sinon
\begin{itemize}
\item Pubertaire : ne rien faire
\item Idiopathique : androgènes non aromatisables 3 mois. Si inefficace, chir
plastique possible
\end{itemize}
\subsection{245 Diabète}
\label{sec:orgeb0cfc7}
\begin{tcolorbox}
 Définition : glycémie à jeun \ge 1.26g/L (2 reprises) ou (aléatoire \ge 2g/L et signes hyperglycémie)\footnotemark
\end{tcolorbox}
\footnotetext{Normale à jeûn < 1.10g/L}

Caractéristiques diabète 1 (le diabète 2 s'y oppose) : 
\begin{itemize}
\item ATCD familiaux rares, < 25 ans, début rapide explosif avec symptomatologie bruyante
\item poids normal ou \dec, hyperglycémie majeure > 23g/L
\item souvent cétose
\item pas de complications dégénératives
\item mortalité par insuf rénale (CV pour diabète 2)
\end{itemize}

\subsubsection{Diabète 1}
\label{sec:org0c6bf60}
Prévalence : 1/200 000 (10\% des diabétiques). Peut survenir à tout âge. \inc
incidence. Sex-ratio = 1

\subparagraph{Physiopathologie}
\label{sec:orgead90bf}
Carence en insuline par destruction cellules beta du pancréas. Auto-immun++ ou idiopathique.

Prédisposition génétique, facteurs environnementaux.

\danger 10\% d'autres maladies auto-immunes \thus doser Ac anti-TPO (Basedow,
thyroïdite), anti-surrénale (Addison), anti-transglutaminase \textpm{} anti-endomysium
(coeliaque), anti-paroi gastrique, anti-facteur intrinsèque (Biermer)

\subparagraph{Diagnostic}
\label{sec:orgf864338}
Clinique =  glycémie \(> 2g/L\), maigrissement, cétonurie \textpm{} acrshort:SPUPD
\begin{itemize}
\item NB : signes d'acidose\footnote{Dyspnée de Kussmaul, haleine acétonique} possibles
\item \emph{si doute} : Ac anti-GAD (\textpm{} anti-ilôtsq anti-IA2, anti-insuline, anti-ZnT8)
\item \emph{si négatif} :
\begin{itemize}
\item hérédité dominante : MODY, mutation SUR1/KIR6-2 (si diabète néonatal)
\item sd de Wolfram \footnote{Atrophie optique, surdité, diabète insipide < 20 ans}, mitochondropathie\footnote{Surdité, dystrophie maculaire, cardiomyopathie transmission par la mère}
\item secondaire : cancer pancréas, pancréatite chronique, mucoviscidose, hémochromatose, médicaments
\end{itemize}
\end{itemize}
Formes : diabète 1 lent (LADA\footnote{Latent Autoimmune Diabetes in the Adult}), (révélé par acidocétose), non insulinodépendantes, cétosique du sujet noir d'origine africaine : mécanisme
   auto-immun

\subparagraph{Évolution}
\label{sec:orgf7f4f5b}
Schéma théorique : estruction cellules \(\beta\), clinique (85\% détruites), séquellaire

Diabète instable : 
\begin{itemize}
\item itérations de cétoacidoses ou hypoglycémies sévères, psycho.
\item DD : gastroparésie, déficit systèmes contra-insuliniques, Ac anti-insuline
\end{itemize}

\subparagraph{PEC}
\label{sec:org7a0c698}
Insuline à vie (Table \ref{tab:org8963bff} ) + alimentation variée sans interdits, exercice physique
\begin{itemize}
\item Stylo à insuline (pompe si échec) :analogue lent (1-2/j) et analogue rapide (3-4) \thus
éducation nutritionnelle
\item ES : hypoglycémie, lipoatrophie (immuno), lipohypertrophie (piqûres au même endroit)
\end{itemize}

Objectifs : HbA1c < 7\% (enfants : entre 7.5 et 8.5, complication/sujet âgé : 8\%)
\begin{itemize}
\item 4 glycémies/jour, injection d'insuline, adapter ttt, contrôle de l'alimentation \thus
éducation thérapeutiques
\item Surveillance :
\begin{itemize}
\item HbA1c
\item diabétologue/pédiatre endocrinologue 3/an
\item \{lipides, créat, microalbuminurie\}
\item ophtalmo, cardiologie 1/an (sympto/âgé,compliqué), dentiste 1/an
\end{itemize}

\begin{table}[htbp]
\caption{\label{tab:org8963bff}Traitement insulinique du diabète 1}
\centering
\begin{tabular}{llll}
Type & Injection & Durée & Utilisation\\
\hline
insuline humaine recombinante (Actrapid) & IV, IM, SC & 7-8 & Prandiale, hyperglycémie\\
analogue rapide (Humalog, Novorapid, Apidra) & IV, IM, SC & 4-6h & Pompe\\
forme lente (NPH) & SC & 9-16h & \\
analogue lents (Lantus) &  & 16-40h & \\
\end{tabular}
\end{table}
\end{itemize}


\subparagraph{Cas particuliers}
\label{sec:org6751fed}
\begin{itemize}
\item Enfant/ado : \danger cétoacidoses
\item Femme (cf Sec. \ref{orgf0d741f}
\begin{itemize}
\item oestrogestatif à discuter
\item grossesse : équilibre dès conception !! par analogue de l'insuline.
\item CI absolue : insuf coronaire instable
\end{itemize}
\item Pas d'arrêt de l'insuline (lent si examen à jeun)
\end{itemize}

\subsubsection{Diabète 2}
\label{sec:orgaa41ab8}
90\% de diabète. Prévalence 4\%. 

FR = obèse, anomalie métabolisme glucidique, ATCD familiaux diabète 2, ethnie noire/hispanique.

\subparagraph{Physiopatho}
\label{sec:org3443202}
Insulinorésistance : causée par la génétique, sédentarité, excès pondéral. Au niveau du muscle, foie, lipolyse

\emph{Et} déficit insulinosécrétion. 

\subparagraph{Dépistage}
\label{sec:org8883028}
Signes cliniques de diabètes, > 45 ans (tous les 3 ans), \(\ge\) 1 FR. 
 Non caucasien/migrant, \glslink{sdMetabolique}{sd métabolique}

DD : diabète 1 lent, génétique (MODY2, mitochondrial), secondaire (pancréatopathie, hémochromatose,
mucoviscidose, médicaments)

\subparagraph{Évolution}
\label{sec:orgdf1f597}
Insulinopénie \thus insulinoréquerant. Pronostic selon complications.

\paragraph{Traitement}
\label{sec:org74514eb}
\begin{itemize}
\item Activité physique 3-5/semaine: intensité modérée \(\ge\) 30min/j et intense (> 60\% \(VO_{2max}\)) de 20min
\begin{itemize}
\item CI : insuf coronarienne, rétinopathie proliférante non stabilisée
\item surveiller risque hypoglycémie, pieds !
\end{itemize}
\item Alimentation : équilibrée, objectif = poids -5 à 10\%
\item Metformine++\footnote{ES = diarrhées. CI = insuf rénale (?), hépatique, respiratoire} en oral \textpm{} sulfamide/glinides/inhibiteurs
DPP-4/inhibiteurs \(\alpha\)-glucosidase/analogues GLP-1\footnote{1ere intention = metformine. Si besoin, +sulfamide. Si besoin : DPP4 si
écart < 1\% sinon insuline (ou GLP-1 si IMC > 30)}
\item \textpm{} insulinothérapie quand insulinorequérance, mal équilibré(cf diabète 1)
\end{itemize}
Objectifs : HbA1c < 7\%\footnote{< 6.5\% si découverte} (8\% si grave, 9\% si agé dépendante)
\begin{itemize}
\item autosurveillance glycémique : pas systématique si ttt oral (1-3 cycles/j),
nécessaire si insuline
\end{itemize}

\subsubsection{Complications}
\label{sec:org3dc2e27}
Souffrance vasculaire : micro- (rein, oeil, nef) et macro-angiopathie (\inc
athérosclérose). AOMI x6-10

\subparagraph{Physiopatho}
\label{sec:orgb33d2eb}
Excès de glucose \thus aggression des vaisseaux (endothélial++), inhibition des
mécanismes de défense cellulaires

Conséquences : épaississement des membranes basales, troubles perméabilité
vasculaire, prolif vasculaire (rétine), fibrose (rein)

\subparagraph{Rétinopathie diabétique}
\label{sec:org6cad983}
Cf \hyperref[orgd703da6]{chap 21 d'ophtalmo}
\paragraph{Néphropathie}
\label{sec:orga3ea97d}
\begin{itemize}
\item Diabète = 1ere cause d'IR terminale. Risque CV x10 chez DT1, 30\% DC IR terminale
chez DT1 (5\% chez DT2)
\end{itemize}

\subparagraph{Physiopatho}
\label{sec:orgc74abb0}
\inc pression intra-glomérulaire \thus dilatation des glomérules. Puis sclérose) avec \inc albumine\footnote{Microalbuminurie, macro quand détectable à la BU}.

\subparagraph{Dépistage}
\label{sec:orgb7afc77}
1 BU/an protéinurie, hématurie, infection urinaire), albuminurie/créatinurie 
\subparagraph{Diagnostic}
\label{sec:org98698af}
Rétinopathie , plusieurs excrétions urinaire d'albumine \inc 
\begin{itemize}
\item si doute : ponction-biopsie rénale : 
\begin{itemize}
\item diabète 1 : hypertrophie mésangiale/glomérulaire < épaississement membrane basale, dépôts
mésangiaux < hyalinose artériolaire < glomérulosclérose nodulaire
\item diabète 2 : 1/3 typique\ldots{}
\end{itemize}
\item 5 stades : 
\begin{enumerate}
\setcounter{enumi}{3}
\item Néphropathie incipiens : microalbuminurie\footnote{30-300mg/24h ou 20-200mg/L}
\item Néphropathie : PA élevée, DFG \dec de 10mL/min/an, nodule de sclérose,
hyalinose artériolaire
\item Insuffisance rénale
\end{enumerate}
\end{itemize}

\subparagraph{Traitement}
\label{sec:org8add430}
Prévention 
\begin{itemize}
\item primaire (diabète, FR HTA)
\end{itemize}
\begin{itemize}
\item secondaire : Tab \ref{tab:org7463954}. Surveiller glycémie !! Éviter AINS,
produits contrastes iodés
\end{itemize}
\begin{table}[htbp]
\caption{\label{tab:org7463954}tab:nephro\textsubscript{diabete}}
\centering
\begin{tabular}{lll}
Stade & Objectifs & Moyens\\
\hline
microalbuminurie & : HbA1c < 7\%, PA < 140/85 & \uline{IEC/sartans}\tablefootnote{\danger sténose artère rénales : doser K+, créat}, FR\\
 &  & \emph{et} PEC tabac, régime hypoprotidique, sel < 6g/j\\
macroalbuminurie & PA < 140/85mmHg & IEC/sartan + diurétique thiazidique.\\
 & Protéinurie < 0.5g & \\
IR & PAs < 130mmHg & \\
- DFG \(\in\) [30, 60] &  & adapter poso\\
- DFG < 30mL/min & HbA1c < 8\% & autorisé : insuline, répaglinide, inhib \(\alpha\)-glucosidase,\\
- DFG < 25 &  & autorisé : inhib DPP4\\
\end{tabular}
\end{table}

NB : infections urinaires : \(\times 3\) dont 90\% asymptomatique (basses) \thus
ttt si symptomatique. Risque = contamination du haut appareil urinaire  aggravation néphropathie glomérulaire.

\paragraph{Neuropathie}
\label{sec:orgd87f4a5}
\begin{itemize}
\item Autonome : tardive
\item Périphérique : 50\% des diabètes à 20 ans. FR : grande taille, tabac, âge,
AOMI, carences nutritionnelles/vitaminiques, alcool, insuf rénale
\end{itemize}

Atteinte métabolique et vasculaire.

\subparagraph{Diagnostic}
\label{sec:org7f7f730}
Examen clinique et interrogatoire (+ complémentaires si autonome)

\subparagraph{Sensorimotrice}
\label{sec:org0d9db9b}
En "chaussettes" puis en "gants")
\begin{itemize}
\item Polynévrite symétrique distale++ :
\begin{itemize}
\item hypoesthésie pression/tact/thermique/proprioceptique ignorée
\item \textpm{} paresthésies distales, douleurs "arc électrique"
\item ROT achilléen aboli (puis rotulien)
\item voûte plantaire se creuse (tardivement)
\item complication : pied "cubique" de Charcot
\end{itemize}
\item Plus rare : polynévrite asymétrique proximale, polyradiculopathie
thoracique, mononévrite, multinévrite
\end{itemize}

\subparagraph{Autonome}
\label{sec:org8b5d3fb}
\begin{itemize}
\item CV : tachycardie sinusale, bradycardie, allongement QT
\item Vasomotrice : hypotension orthostatique \emph{sans} accélération du pouls
\item Troubles sudation : sécheresse cutanée MI
\item Digestive : parésie, dysphagie, gastroparésie (fréq), diarrhée, constipation
\item Vésicale : résidu post-miction \thus IU \thus clinique, écho (prostate, vessie)
\item Dysfonction érectile : psychogène, (sd de Leriche\footnote{Thrombus bloquant l'aorte abdominale avant bifurcation}). DD : examen génital, testostérone, prolactinémie.
\end{itemize}

Examen 
\begin{itemize}
\item interrogatoire , inspection pieds, ROT abolis (niveau troubles sensitifs), monofilament, sensibilité épicritique, thermoalgique, vibratoire\footnote{Grosses fibres\label{org3b44ad5}}, proprioceptiques\textsuperscript{\ref{org3b44ad5}}
\item ECG annuel, EMG si atypique
\item \(\Delta\) FC inspiration - expiration\footnote{Sensible mais pas interprétable > 60 ans ou patho bronchorespiratoire}, rapport RR long/court pendant épreuve de
Valsalva, \(\Delta\) FC couché - debout
\end{itemize}

DD : neuropathies métaboliques (insuf rénale, amylose, hypothyroïdie), toxiques
(alcool, tabac, iatrogène), paranéoplasiques, carentielles, inflammatoire,
infectieuse (Lyme, lèpre), autre (Charcot-Marie-Tooth, péri-artérite noueuse)

Traitement : 
\begin{itemize}
\item préventif = glycémie. FR : alcool, tabac, insuf rénale, carence vitamines B, médicaments.
\item Si installées, stabiliser et éviter les complications (mal perforant plantaire++)
\item Antalgiques, hydratation peau
\end{itemize}

\paragraph{Macroangiopathie}
\label{sec:orgeaa975f}
\diameter > 200 \(\mu\)m. Plus fréquente et sévère. Artères visibles sur radio.

Prévention CV = \textbf{problème majeur} des diabétiques 2 : \(\frac{3}{4}\) DC d'une cause
CV. Risque CV \texttimes{}2-3 (\texttimes{}3-4 chez \female). 

\subparagraph{Dépistage}
\label{sec:org63de8a2}
Risque > 1\% = élevé\footnote{Calcul par les études UKPDS ou SCORE (mais \texttimes{}2-4 pour ce dernier)}

\begin{enumerate}
\item \emph{FR} :
\begin{itemize}
\item CV : âge > 50 ans \male \footnote{> 60 \female}, diabète > 10 ans, ATCD IDM/mort
subite\footnote{< 55 ans \male, < 65 ans \female}, ATCD AVC \footnote{< 45 ans},  tabac, HTA permanente, HDLc < 0.4g/L, microalbuminurie > 30mg/24h
\item autres : obésité abdominale \footnote{> 102cm \male, > 88cm \female}, IMC > 30k/m\textsuperscript{2}, sédentarité, > 3 verres vin/j, pyschosociaux
\end{itemize}

\item Montrer atteinte artérielle : 
\begin{itemize}
\item coronaropathie : ECG repos annuel, scinti avec épreuve d'effort ou coronarographie
\item carotides ? auscultation \thus écho si AIT possible
\item AOMI ? pieds, pouls, claudication, IPS cheville/bras < 0.7 ? Écho-doppler
\end{itemize}
\end{enumerate}

\subparagraph{Diagnostic}
\label{sec:org94d40ac}
\begin{itemize}
\item \emph{Ischémie myocardique} silencieuse fréquente ! \thus dépistage systématique si
trouble dig, asthénie effort\ldots{}
\item AOMI : 1/3 proximale (HTA), 1/3 distale sous genou (glycémie, tabac), 1/3 proximale et distale
\item : diabète + microangiopathie sévère, diabète + atteinte vasculaire
\end{itemize}

\subparagraph{Traitement}
\label{sec:org397e18d}
Objectif HbA1c < 6.5\% (7\% si âgé ou à risque.
\begin{itemize}
\item Activité physique
\item LDL < 1.3g/L (1.0 si risque CV élevée ou néphropatie) : statines \uline{ou} fibrates
\item PAs \(\in\) [130, 139] et PAd < 90mmHg.
\item Poids : IMC < 25kg/m\textsuperscript{2}
\item Arrêt tabac,
\item Prévention thrombose si \(\ge\) 1 FR : aspirine 75-150mg
\end{itemize}
Si revascularisation : stents par défaut  et chir si atteinte 3 coronaires.

\paragraph{Pied diabétique}
\label{sec:org5817b73}
1 patient sur 10 à risque d'1 amputation d'orteils. Éviter les plaies pour prévenir l'amputation

\subparagraph{Mal perforant plantaire (MPP)}
\label{sec:org4b30b5d}
Neuropathies \thus hypoesthésie, déformations ostéoarticulaires \thus durillons
puis fissure et infection \thus dermo-hypodermite.

Autres
\begin{itemize}
\item Ischémie/nécrose : peau froide, fine, dépilée, livedo. \thus revasculariser en urgence
\item Nécrose + MPP
\item Dermo-hypodermite nécrosante : très rare, \thus débrider en urgence, ATB. 
Cas particulier : gangrène gazeuse à \bact{perfringens} \thus urgence vitale \danger
\end{itemize}

CAT
\begin{itemize}
\item Radio pieds bilatérale (ostéite ?), si infection : NFS, iono, CRP
\item décharge, excision kératose à domicile si suffit
\item réhydratation \textpm{} équilibre glycémie, anticoag, accin anti-tétanos !
\item si infection : parage et drainage, ATB (cocci G+ si récent, sinon bacille G-)
\item si artério : revascularisation
\item si ostéite : résection chirurgicale ou ATB 6-12semaine et sans l'appui
\end{itemize}

\paragraph{Autres}
\label{sec:orgac09c50}
\begin{itemize}
\item Peau : nécrobiose lipoïdique (rare), dermopathie diabétique (fréquente),
lipodystrophie, acanthosis nigricans, vitiligo, xanthomatose éruptives
\item Infections : otite nécrosante (urgence !), mucormycose (urgence !)
\item Foie : hépatologue dès anomalie transaminases ou \(\gamma\)-GT
\item Articulations : capsulite rétractile, maladie de Dupuytren\footnote{Sclérose réractile de l'aponévrose palmaire moyenne}, Chéiroarthropathie, arthrose
\item Dents : maladie parodontale \thus dentiste tous 6 mois
\end{itemize}

\begin{tcolorbox}
\begin{itemize}
 \item annuel : FO, ECG repos 
 \item tous les 5 ans : écho-doppler MI (si > 40 ans, diabète > 20 ans) tous 5 ans
 \item bio : HbA1c 4/an, glycémie veineuse, lipides 1/an, microalbuminurie 1/an,
   créatininémie jeun, clairance créat 1/an, TSH
\end{itemize}
\end{tcolorbox}

\paragraph{Complications métaboliques}
\label{sec:org4d877eb}
\subparagraph{Coma cétoacidosique}
\label{sec:org28bfa52}
\begin{itemize}
\item acétonurie, glycosurie, glycémie 2.5g/L, pH veineux < 7.25, bicarbonate <
15mEg/L
\item cause : déficit insuline absolu/relatif, inconnue
\item évolution : cétose puis cétoacidose (Kussmaul, stupeur, déshydratation mixte)
\item gravité : âgé, ph < 7, kaliémie 4-6 mmol/L, coma profond, TA instable, pas de
diurèse après 3h, vomissements incoercibles
\item DD : urgence abdo, coma hyperosmolaire
\item Régression sous ttt en 24-48h : 
\begin{itemize}
\item éducation : si cétose, maintenir injections, supplément insuline rapide,
acétonurie si glycémie > 2.5g/L
\item curatif : insuline rapide IV, recharge volumique, K+, glucose
si besoin, facteur déclenchant
\end{itemize}
\end{itemize}

\subparagraph{Coma hyperosmolaire :}
\label{sec:orgc2a43b6}
\begin{itemize}
\item glycémie > 6g/L, osmolalité > 350mmol/kg, natrémie corrigée > 155mmol/L, pas
de cétose ni d'acidose
\item FR : > 80 ans, infection aigüe, diurétique, \uline{pas d'accès aux boissons}, corticothérapie
\item ttt : réhydratation prudente, lente, insuline IV, surveillance, héparine
préventive, ttt causal
\end{itemize}

\subparagraph{Hypoglycémie}
\label{sec:org35012ea}
Inévitable  mais pas mortelle
\subsection{246 \textdagger{} Prévention par la nutrition}
\label{sec:orged1700f}
\subsection{247 \textdagger{} Modifications thérapeutiques du mode de vie}
\label{sec:org004c1da}
\subsection{248 \textdagger{} Dénutrition (à compléter)}
\label{sec:org612622b}
\subsection{249 \textdagger{} Amaigrissement}
\label{sec:orgc7e1fa0}
Fréquent

\subsubsection{Interrogatoire}
\label{sec:org0107177}
\begin{itemize}
\item Histoire pondérale, conditions, de vie, psychologique, activité physique excessive et apports alimentaires insuffisants
\item Anorexie, \{troubles digestifs, palpitations, sd polyuro-polydipsie\}, troubles
libido/érection, amnénorrhée (anorexie mentale ou hypothalamique
fonctionnelle), médicaments (nausée, anorexie), dépression masquée++
\end{itemize}

\subsubsection{Examens :}
\label{sec:org30cc3e0}
Clinique : poids, taille, IMC, pli cutané, fonte musculaire, carences vitamines,
pâleur cutanéomuqueuse

Complémentaires :
\begin{itemize}
\item bio : NFS (anémie), VS/CRP (inflammatoire), iono (hyponatrémie \thus insuf
surrénale), BU (glycosurie), calcémie, \{transaminase, \(\gamma\)-GT\}(foie), TSH
(hyperthyroïdie), \{B12, folates, TP, albuminémie\}, graisses fécales ?
(pancréatite chronique calcifiante), dénutrition\footnote{(Pré-)Albumine, IGF-1, ferritine sérique}
\item Radio thoracique (tuberculose), écho abdo (abcès/tumeur), fibro (obstacle),
DEXA (composition corporelles)
\end{itemize}

\subsubsection{Étiologie}
\label{sec:org9a8780c}
\begin{itemize}
\item Poids stables, apports nutritionnels normaux, examens normaux :  maigreur
constitutionnelle
\item Si perte de poids confirmée, éliminer anorexie mentale, maladies digestives,
iatrogène, cancer extradigestif, maladies infectieuses, neuro, grande
défaillance cardiaque/rénale/respi/hépatique, alcool
\item Sinon, causes endocrines :
\begin{itemize}
\item diabète 1 ou 2 : glycémie, HbA1C
\item hyperthyroïdie : TSH \dec\dec, hormones thyroïdiennes \inc
\item hypercalcémie : si gls:PTH inadaptée, hyperparathyroïdie primaire
\item insuf surrénalienne : cortisol, ACTH plasmatique
\item panhypopituitarisme\footnote{Insuffisance antéhypophysaire complète} : cortisol \dec
\item phéochromocytomes : (nor)métanéphrines dans urines 24h, imagerie surrénales
\end{itemize}
\end{itemize}
\subsection{250 \textdagger{} Troubles nutritionnels chez sujet âgé (à compléter)}
\label{sec:orga34c683}
\subsection{251 \textdagger{} Obésité(à compléter)}
\label{sec:orga8c0e37}
\subsubsection{Adulte}
\label{sec:org729c469}
Surpoids = IMC \(\in\) [25, 29.9]kg/m\textsuperscript{2}. Obésité :
\begin{itemize}
\item grade 1 : IMC \(\in\) [30, 34.9]kg/m\textsuperscript{2}.
\item grade 2 : IMC \(\in\) [35, 39.9]kg/m\textsuperscript{2}.
\item grade 3 : IMC \(\ge\) 40kg/m\textsuperscript{2}.
\end{itemize}

Limites : sous-estimé chez asiatiques. Seulement pour [18,65] ans

Phases : prise de poids, constituée, perte, rechutes

Localisation : viscéral (scanner, IRM), sous-cutanée, ectopique (muscle, foie)

Épidémio : +27.5\% 1980-2013 (monde). France : de plus en plus jeune, \inc chez >
65 ans

Étiologie :
\begin{itemize}
\item génétique: envisager si précoce (naissance +24 mois), troubles du
comportement alimentaire
\item obésités communes liées à des facteurs environementaux (majorité) : surtout
déséquilibre apport caloriques- dépense
\begin{itemize}
\item antipsychotiques, glucocorticoïdes, antidépresseurs,
antiépileptiques, antidiabétiques
\item arrêt du tabac, privation de sommeil (?), hypothalamique (rare)
\end{itemize}
\end{itemize}

Complications : \inc RR mortalité, métabolique, CV, respi, ostéoarticulaire,
digestive, rénale, gynéco, cutanée, néoplasiques, psychosociale

\paragraph{Clinique}
\label{sec:orgd5c0f81}
\begin{itemize}
\item Interrogatoire : 
\begin{itemize}
\item ATCD familiaux d'obésité, poids naissance, âge surpoids,
poids max et min, circonstances déclenchantes, tentatives antérieures, phases
\item Comportement alimentaire (carnet), évaluation dépense énergétique, pyscho-comportementale
\item Complications (SAS)
\end{itemize}
\item Examen : poids, taille, PA, tour de taille\footnote{Obésité abdominale : > 88cm \female, > 102cm \male}, obésité secondaire
\item Complémentaires : glycémie à jeune, lipides, hépatique, uricémie, ECG repos
\end{itemize}

\paragraph{Traitement}
\label{sec:orgee6fc98}
\begin{itemize}
\item Diététique, activité physique (\(\forall\) IMC)
\item Psychologique
\item Médicaments (IMC \(\ge\) 30 ou (\(\ge\) 27 et comorbidités)) : orlistat
\item Chir bariatrique :  \{anneau gastrique ajustable, sleeve gastrectomie\},
\{court-circuit gastrique, dérivation biliopancréatique\} : < 65 ans. Prise en
charge 6 mois avant et post-op à vie (carences vitaminiques)\footnote{CI : troubles cognitifs sévères, troubles sévères non stabilisés du
comportement alimentaire, dépendances à l'alcool / substances psychoactives, pas
de PEC médicale, pronostic vital mis en jeu, CI à l'anesthésie générale,
incapacité à faire un suivi médical prolongé}
\end{itemize}
\subsubsection{Enfant/ado}
\label{sec:org16431ff}
\danger{} évolutivité. Surpoids : IMC > 25. Obésité 
\begin{itemize}
\item grade 1 : > 30kg/m\textsuperscript{2}
\item grade 2 : > 35kg/m\textsuperscript{2}
\item grade 3: > 40kg/m\textsuperscript{2}
\end{itemize}

Épidémio : stabilisation mais obésités sévères \texttimes{}4

\paragraph{Étiologies}
\label{sec:orgf723b15}
\begin{itemize}
\item génétiques : mutation sur récepteur de la mélanocortine type 4 = 2.5-5\%
\item communes (majorité) : facteurs environnementaux et prédisposition génétique
\begin{itemize}
\item repond d'adiposité à 6 ans. Risque d'obésité \(\propto\) précocité du rebond
\item tour de taille/taille > 0.62 = forte valeur prédictive
\item FR : surpoids parent, poids excessif/tabac pendant grossesse, anomalie de
croissance foetale, \inc\inc poids à naissance + 2ans, difficulté
socio-éoc, manque d'activité physique, troubles sommeil, psychopatho
\end{itemize}
\item secondaires (rare) : ralentissement de la vitesse de croissance naturelle
\end{itemize}

\paragraph{Complications}
\label{sec:org198271a}
\begin{itemize}
\item HTA : > 97e percentile + 10mmHg
\item Insulinorésistance avec glycémie normale fréquente
\item \inc TG et \dec HDL
\item Stéatose hépatique non alcoolique
\item Rachialgies, gonalgies, troubles statique vertébrales. Penser à l'épiphysiolyse
de la tête fémorale : garçons [10,15] ans avec douleur mécanique de hanche
\thus radio de profil
\item Psychologique
\end{itemize}

\paragraph{Clinique}
\label{sec:org596c5d4}
Interrogatoire : 
\begin{itemize}
\item ATCD familaux,
\item personnels : poids, taille naissance, âge d'appartition, changements environnementaux, tentatives antérieures, troubles des règles
\item comportement alimentaire (difficile)
\end{itemize}
Examen clinique :
\begin{itemize}
\item poids, taille, PA, tour de taille, pli-cutané (masse grasse < 20\% après 5 ans), courbes de
croissance (ralentissement = pathologique !), dermato (acanthosis nigricans =
insulinorésistance, vergétures= hypercorticisme, intertrigo, mycose)
\end{itemize}
Pas d'examens complémentaires !

\paragraph{Traitement}
\label{sec:org19ad848}
Prévention surtout. Modifier style de vie (efficacité faible). Chir possible
avec équipes spécialisées
\subsection{252 \textdagger{} Diabète gestationnel + nutrition et grossesse (à compléter)}
\label{sec:orgd07b208}
\label{orgf0d741f}
 Physio chez femme enceinte selon moitié:
\begin{itemize}
\item non diabétique : (\inc insulinéme, insulinosensibilité) puis (insulinorésistance
\thus hyperinsulinisme ou diabète gestationnel)
\item à risque de diabète : (hypoglycémie, cétose) puis (insulinosécrétion
postprandiale insuffisante)
\end{itemize}

\subsubsection{PEC du diabète pré-gestationnel}
\label{sec:orga7730cb}
Grossesse à risque mais fécondité normale (sauf si sd ovaires polykystiques).

\danger{} Normalisation glycémie préconception \(\rightarrow\) accouchement
\begin{itemize}
\item HbA1c \(\le\) 6.5\%
\item glycémie à jeun \(\in\) [0.6, 0.9]g/L
\item glycémie repas + 1h < 1.40g/L et +2h 1.20g/L
\end{itemize}

\paragraph{Risque foetus}
\label{sec:org1bfe4da}
\begin{itemize}
\item Fausses couches spontanées \texttimes{}2, \(\propto\) hémoglobine glycquée
\item Malformation congénitales \texttimes{}2, constituée pendant 8 premières semaines :
cardiaque, neuro, rénale \thus \inc fausses couches spontanées, mortalité
foeatale/néonatale, malfomations
\item 2e trimestre : macrosomie, hypoxie tissulaire, retard maturation pulmonaire,
hypertrophie cardiaque septale
\item 3e trimestre : mort foetale
\item Accouchement : \inc prématurés, césariennes. Danger : trauma foetal,
hypoglycémie sévère, hypocalcémie, hyperbilirubinémie/polyglobulie, détresse
respi transitoire, maladie des membranes hyalines
\item Long terme : surpoids/obésité et diabète 2
\end{itemize}
\paragraph{Risque mère}
\label{sec:org2f22a54}
\begin{itemize}
\item HTA (30\%) : si > 20 SAc, risque de toxémie gradivique. \texttimes{}5 si
diabète 1. Risque vital
\item Rétinopathie : ttt préalable si rétinopathie proliférative. CI : rétinopathie
proliférative floride non traitée
\item Néphropathie : 
\begin{itemize}
\item FR = \{HTA, déséquilibre glycémique, rétinopathie évoluée dès
départ, diabète ancien, insuf rénale, hydramnios, correction trop rapide d'une
hyperglycémie chronique\}.
\item Insuf rénale \thus hypotrophie foetale, prééclampsie. Si IR préexistante : 50\%
mortalité foeatale \textbf{in utero}
\item dépistage : créat plasmatique, microalbuminurie, protéinurie
\item IEC contre-indiqués
\end{itemize}
\item Coronaropathie : exceptionnelle mais gravissime. Dépister si diabète ancien et
complications microvasculaire (ECG, effort)
\item Infection urinaire \inc, risque pyélonéphrite, décompensation diabétique
\item Diabète 1 : \inc risque dysfonction thyroïdiennes
\end{itemize}

\paragraph{PEC}
\label{sec:org5669fef}
\begin{itemize}
\item Avant grossesse : glycémie \(\in\) [0.7, 1.20] préprandial, \(\in\) [1, 1.4]
postprandial et HbA1c < 7\%
\begin{itemize}
\item diabète 1 : \inc insuline
\item diabète 2 : insuline si régime ne suffit pas/arrêt ttt oral
\end{itemize}
\item Pendant    
\begin{itemize}
\item équilibre glycémique++ (6 glycémies capillaires/jour)
\begin{itemize}
\item \danger variations physiologiques : insuline \dec puis \inc puis \dec\dec
\item cétonémie/cétonurie si glycémie > 2g/L
\end{itemize}
\item \(\ge\) 1600kcal/j 2eme et 3eme tri
\item surveiller poids, PA, créat plasmatique, microalbuminurie, protéinurie, FO,
BU, protéinurie
\item surveillance obstétricale : dater++ (12-14SA), malformations (22-24), placenta et liquide
amniotique (32-34SA), cardiomyopathie hypertrophique (32-34SA), bien-être
foetal
\item pas de bêtamimétique si prématuré
\end{itemize}
\end{itemize}
\paragraph{(Post)partum}
\label{sec:orge08e7b8}
Accouchement programmé souvent, facilité si rétinopathie sévère, insuline
  SC/IV et glucosé avec surveillance horaire

Puis : insuline selon besoin pré-grossesse (D1) ou arrêt (D2)

\subsubsection{Diabète gestationnel}
\label{sec:org4030fa7}
Si lié à la grossesse, apparait en 2eme partie. Risque : pré-éclampsie,
césarienne (\(\propto\) hyperglycémie matenrelle). FR : surpoids 

Même complications liées à l'hyperinsulinisme que pré-gestationnel

\paragraph{Dépistage}
\label{sec:org504134b}
Si FR seulement : 
\begin{itemize}
\item \(\ge\) 35ans
\item IMC \(\ge\) 25kg/m\textsuperscript{2}
\item ATCD : diabète gestationnel, macrosomie, diabète chez parents 1er degré
\end{itemize}
Diagnostic :
\begin{itemize}
\item début de grossesse si glycémie jeun \(\ge\) 0.92g/L \thus PEC immédiate
\item sinon à 24-28SA et (glycémie jeun < 0.92g/L ou non faite) : hyperglycémie
provoquée oralement
\end{itemize}

\paragraph{Traitement}
\label{sec:orgab55548}
\begin{itemize}
\item Diététique (30-35kcal/kg [25 si surpoids]), activité physique, antidiabétique
CI , insuline si régime ne suffit pas après 8 jours
\item Surveillance : glycémie (6/jour puis 4/jour), cétonurie (si glycémie > 2g/L),
HTA
\item Objectif : glycémie jeun < 0.95g/L et postprandiale +2h < 1.20g/L
\end{itemize}

Post-partum : arrêt insuline et surveillance glycémie (diabète antérieur
?). Vérifier glycorégulation à 3 mois. Risque de récidive si grossesse
\subsection{253 \textdagger{} Nutrition chez le sportif}
\label{sec:org644dd65}
\subsubsection{Examen d'aptitude}
\label{sec:org5f2eafd}
Dépister les pathologies induisant un risque vital/fonctionnel grave : mort
subite (1-4/100 000 après 35 ans)
Obligation légale si compétition (licencié ou non)\footnote{Médecin qualifé pour : alpinisme, armes à feu, mécaniques, aériens, sous-marins, de combat
avec HS}

Examen :
\begin{itemize}
\item ATCD sportif, médicaux familiaux (CV, hypercholestérolémie familiale),
conduites à risque, alimentaire, ttt, toxiques
\item Clinique : 
\begin{itemize}
\item poids, taille, IMC, (courbe de croissance)
\item maturation pubertaire
\item ostéoarticulaire, cardiorespiratoire, test dynamique sous-maximal
(Ruffier-Dickson)
\end{itemize}
\item Complémentaire : ECG repos\footnote{Pour 1er certificat puis tous les 3 ans puis tous les 5
ans jusque 35 ans}, CV
\end{itemize}

\subsubsection{Bénéfices/inconvénients}
\label{sec:orgd018590}
Adulte :
\begin{itemize}
\item Bénéfices :
\begin{itemize}
\item maintien santé : \dec mortalité prématurée, \inc qualité de vie, \inc
autonomie (âgé), régule poids
\item prévention : cancers (colon, sein), CV, métabolique, ostéoporose \female
\item ttt : anxiété, cardiomyopathie ischémique, BPCO, obésité, diabète 2, neuro,
rhumatismales, dégénératives
\end{itemize}
\item Surveillance : dépistage d'insuf coronarienne > 40 ans, \danger nutrition et
hydratation si > 3h/semaine
\item Recommandation : 150min/semaine (modéré) ou 75min/semaine (soutenu). Idéal : x2
\end{itemize}

Enfant : 
\begin{itemize}
\item Bénéfices :
\begin{itemize}
\item dev psychosocial : \dec stress, anxiété, \inc intégration sociale, \inc
confiance en soi
\item dev psychomoteur : concentration, coordination, équilibre
\item \inc masse maigre, \inc densité osseuse
\item prévention : sd métabolique, surpoids, CV
\end{itemize}
\item Surveillance : nutrition (éviter retards de croissance/pubertaire), attitude
alimentaires restrictives
\item Recommandation : 60min/jour (modéré-soutenu) et renforcement musculaire,
osseux 3x/semaine
\end{itemize}

\subsubsection{Besoins nutritionnels}
\label{sec:orgd9eb3fb}

\begin{center}
\begin{tabular}{llll}
Intensité & durée & Energie & Limitation\\
\hline
Très intense & secondes & ATP, P-Cr & \\
Intense & minutes & Glycogène musculaire & Lactate\\
Faible-élevée & prolongée & glycogène musculaire/lipides & VO\textsubscript{2} max\\
\end{tabular}
\end{center}

Macronutriments :
\begin{itemize}
\item Glucides : détermine l'épuisement si endurance \thus index glycémique faible à
distance, IG élevé juste avant. Pendant : maintenir glycémie. Après :
reconstituer les stocks de glycogène
\item Lipides à limiter si intensité élevé/compétition
\item Protides : endurance 1.2-1.4g/kg/j, force : 1.3-1.5g/kg/j si maintien masse, sinon jusque 2.5g-kg/j
\end{itemize}

Hydrosodé : avant = 500ml en 2h (prévention). Pendant : NaCl si \(\ge\) 1h selon
intensité (jusque 1.5L/h). Après : 150\% perte pondérale.

Minéraux, vitamines:
\begin{itemize}
\item attention situation à risque : déficit en fer, contrainte de poids,
alimentation glucidiques mais faible densité nutritionnelle, exclusion de
groupes d'aliments
\item endurance : vit B énergétiques\footnote{Thiamine, riboflavine, niacine, B6} , vit. "antioxydantes"\footnote{Vit C, E, \(\beta\)\{xcarotène}
\item force : \inc vit B6, \inc "antioxydantes"
\end{itemize}

\paragraph{Enfant}
\label{sec:org5753d0a}
Apport insuffisants \thus retard croissance staturo-pondéral ou pubertaire,
\dec masse musculaire, déminéralisation osseuse, déficit immunitaire.

Surveiller calcium, vit D, fer.

\subsection{265 \textdagger{} Hypocalcémie, dyskaliémie, hyponatrémie}
\label{sec:org47ebbaf}

\subsubsection{Hypocalcémie (hypoCa)}
\label{sec:orgbfe1d21}
Éliminer fausses hypoCa dues à l'hypoalbuminémie\footnote{Une partie du calcium est lié à l'albumine}.
Calcémie = équilibre absorption intenstinale, résorption osseuse, excrétion
rénale. Régulé par PTH, calcitriol

Clinique : 
\begin{itemize}
\item hyperexcitabilité neuromusc : paresthésie main, pieds, péribuccales
(spontanées/effort), signe de Trousseau ("main d'accoucheur"), signe de
Chvosteck (peu spécifique), crises de tétanie (paresthésie, fasciculation
pouvant entraîner arrêt respi)
\item chronique : sd de Fahr\footnote{Cataracte sous-capsulaire, calcification des noyaux gris centraux} \thus signes extrapyramidaux, crises comitiales
\item \inc QTc \thus troubles du rythmes
\item dans l'enfance : musc, neuro, cardiaques
\end{itemize}

\paragraph{Principales causes}
\label{sec:orgf110e8d}
\begin{itemize}
\item Hypoparathyroïdes : anamnèse et \{hypoCa, PTH \dec, phosphatémie
normale/haute\}.
\begin{itemize}
\item post-chir++ : parathyroïdectomie totale
\item congénitale : sd Di George++\footnote{Hypoplasie des parathyroïdes et du thymus, dysmorphie faciale, anomalies cardiaques}
\end{itemize}
\item Pseudoparathyroïdies : génétiques : résistance à la PTH \thus PTH
\inc. Chondrodysplasie possible
\item Anomalie vitamine D
\begin{itemize}
\item carence vit D = 1ere cause hypoCa chez nourrisson \thus rachitisme
carentiel. Chez l'adulte, seulement si déficit prolongé et profond
\item malabsorption digestive, insuf rénale chronique, cirrhose
\end{itemize}
\end{itemize}

\paragraph{TTt}
\label{sec:org54d9526}
\begin{itemize}
\item Aigüe = urgence  \thus calcium IV lente (2-3x10ml). Suspension des ttt qui \inc
QTc, réduction digoxine
\item Chronique : vit D (ou dérivés actifs) et calcium per os
\end{itemize}

\subsubsection{Hyper-/hypo-kaliémie,}
\label{sec:orga27c026}
Retentissement cardiaque \thus vital 

\paragraph{HyperK}
\label{sec:orga7c663c}
Principales causes
\begin{itemize}
\item Acidose (sort K+ de la cellule) et insulinopénie (réduit entrée K+) : ttt par
insuline à risque d'hypoK \danger \thus apport K+ dès normokaliémie
\item Hypoaldostéronisme
\begin{itemize}
\item insuf surrénale périphérique
\item secondaire : chez > 65 ans, diabétiques. Risque = aggravation si IEC ou ARA II
\end{itemize}
\item Pseudo-hypoaldostéronisme : résistance à l'aldostérone (génétique)
\end{itemize}

\paragraph{HypoK}
\label{sec:orgf25196e}
\begin{itemize}
\item Dénutrition sévère : anorexique, post-chir bariatrique sans suivi
\item Insulinothérapie : si cétoacidose et troubles digestifs majeurs \thus
insulinothérapie seulement après normokaliémie, sinon arrêt cardiocirculatoire
\item \inc activité \(\beta\)adrénergique
\item Paralysie périodique famililiale : exceptionnelle, paralysie brutale
transitoire des 4 membres
\item Hyperaldostéronisme ou hypercorticisme : y penser si HTA (non constante) et hypoK avec
kaliurèse \inc
\item Polyurie : hyperglycémie \inc
\item Hypomagnésémie : si Mg \dec, malabsorption, pertes digestives causées par
IPP. Sinon : pertes urinaires acquises/génétique
\item Bloc 11-\(\beta\)hydroxystéroïde déshydrogénase : tableau similaire à
hyperaldostéronisme primaire mais avec aldostérone \dec. Si HTA et hypoK,
vérifier réglisse et pastis (glycyrrhizine)
\end{itemize}

\subsubsection{Hyponatrémie endocrinienne}
\label{sec:orgd3b9c58}

HypoNa = anomalie électrolytique la plus commune chez hospitalisés

Osmolarité (mosm/L) : 2\texttimes{}([Na+] + [K+]) + glycémie + urée

\begin{figure}[htpb]
  \centering
  \resizebox{0.9\linewidth}{!}{
    \tikz \graph [
  % Labels at the middle 
      edge quotes mid,
  % Needed for multi-lines
      nodes={align=center},
      sibling distance=3cm,
      level distance=1.5cm,
      edges={nodes={fill=white}}, 
    layered layout]
    {
      Osmolalité -> {
        Augmentée -> "Hyperglycémie";
        Normale -> "HyperTG\\Hyperprotidémie";
        Diminuée -> Volémie -> {
	  "Augmentée\\(hyperhydrat. extracell)" -> "Insuf cardiaque\\Cirrhose\\Sd néphrotique"
	  -> "Sérum salé\\isotonique";
	  "Normal\\(hyperhydrat intracell)" -> "Hypothyroïdie\\Insuf corticotrope\\SIADH"
          -> "Sérum salé\\hypertonique";
	  "Diminué\\(déshydrat extracell)" -> "Perte digestives\\rénales, cérébrales\\Insuf corticosurrénales aigüe"
          -> "Restriction hydrosodée";
	};
      };
    };
  }
  \caption{Démarche diagnostique et ttt devant une hyponatrémie}
\end{figure}

Physiopatho : hormone anti-diurétique (ADH) : répond au stimulus osmotique,
volémique et stress etc. Action vasoconstrictive, corticotrope (stress),
antidiurétique

\paragraph{SIADH}
\label{sec:orgfdae7c5}
PA et FC normale, pas de pli cutané, (déshydratation extra-cellulaire) ni
d'oedème (hyperhydratation extra-cellulaire)

DD : cf figure. Si hyponatrémie hypoosmolaire normovolémique :
\begin{itemize}
\item insuf corticotrope : cortisolémie et ACTH
\item insuf surrénale aigüe
\item hyporthyroïdie proto-thyroïdienne : TSH \inc
\item hypopituitarisme antérieure : cortisolémie, TSH, T4L
\end{itemize}

Étioliogies :
\begin{itemize}
\item iatrogènes : neuroleptiques, antidépresseurs, chimio, carbamazépine,
desmopressine
\item quasi toutes affections neuro, notament intervention trans-sphénoïdale
(adénome corticotrope)
\item pulmonaires
\item tumeurs malignes : cancer bronchique à petites cellules++
\item rares : mutation récepteur V2 ADH, marathonien, VIH
\item Intoxication aigüe à l'eau
\end{itemize}

\paragraph{Traitement}
\label{sec:orgb23461c}
Urgence si < 115mmol/L ou \{délire, coma, convulsion\}  \thus sérum salé
hypertonique jusque Natrémie = 120mmol/L (puis restriction hydrique).

\danger{} < 12mmol/24h sinon tableau d'AVC (myélinolysie centropontine) !! 

Thérapeutique :
\begin{itemize}
\item restriction hydrique : mal tolérée
\item déméclocycline : induit diabète inspidie néphrogénique
\item aquarétique (tolvaptan)
\end{itemize}

Indications :
\begin{itemize}
\item symptômes cliniques sévères/récent : sérum salé hypertonique
\item symptômes plus modérés : sérum et tolvaptan
\item sinon restriction hydrique et tolvaptan (ou déméclocycline)
\end{itemize}
\subsection{266 \textdagger{} Hypercalcémie}
\label{sec:org140c532}
Diagnostic = double dosage calcémie. Étiologie selon parathormone (PTH)

Physio : calcémie régulée par PTH et calcitriol
\begin{itemize}
\item PTH: \inc absorption intestinale du calcium et phosphore, \inc résorption
osseuse, \dec réabsorption phoshpore et \inc absorption calcium (rein)
\item PTH régulée par récepteur sensible au calcium (CaSR)
\end{itemize}

Bio : calcémie totale = \{calcium ionisé, calcium lié = \{lié à l'albumine,
complexé aux anions\}\}. Calcium ionisé \(\approx\) 50\% calcium total\footnote{Sauf si acidose, hyperprotidémie, \inc phosphore/sulfate sériques}

Clinique : asthénie, \{polyuro-polydipsie, lithiases rénales\}, \{anorexie,
constipation, nausées\}, \{apathie, somnolence, confusion, psychose, coma\}, \{HTA,
\dec QT\}

\danger hyperglycémie maligne = urgence \{\} avec déshydratation, \{confusion,
coma, insuf rénale\} et risques de troubles du rythme cardiaque, bradycardie avec asystolie

\subsubsection{Étiologies}
\label{sec:orgea34c39}
\paragraph{Hypercalcémie PTH dépendante (PTH N ou \inc)}
\label{sec:org40cf09f}
\begin{itemize}
\item \textbf{Hyperparathyroïdie\footnote{La PTH est produite par la parathyroïde\ldots{}} primaire} (55\%) : lésion parathyroïde. 
\begin{itemize}
\item Signes cliniques précédents avec rénaux, osseux (clinique et radio\footnote{Ostéite fibrokystique de von Recklinhausen, exceptionnelle.}) \thus créatinine
plasmatique, scénal rénal non injecté
\item Surtout densité osseuse (tier distal du radius)
\item Bio : hypercalcémie et PTH non adaptée (N ou \inc). 
\begin{itemize}
\item \danger corriger déficit vitamine D avant doser calcémie.
\item \danger DD : sd hypercalcémie-hypocalciurie familiale,
hyperparathormonémie avec ttt au lithium
\item calcémie et phosphorémie n'ont de sens qu'avec une fonction rénale normale
\item calciurie : si augmentée, enlève les DD précédents
\end{itemize}
\item imagerie : bio primaire mais sert si indication opératoire seulement (écho, scinti)
\item étiologie :
\begin{itemize}
\item majorité : sporadique, isolé
\item NEM1\footnote{Néoplasie endocirinienne multiple de type 1} (1\%) : hyperparathyroïdie primaire = 95\%. Recherche tumeurs
endocrines pancréas et duodenum, adénomes hypophysaires
\item NEM2 : cancer médullaire de thyroïde puis phéochromocytome bilat et
hyperparathyroïdie primaire avec atteinte multiglandulaire
\item \danger hyperparathyroïdie primaire chez jeune = suspicion transimission
génétiques
\item hyperparathyroïdie \emph{secondaire} : adaptation à hypocalcémie (chercher chez
insuf rénaux chronqiue)
\item hyperparathyroïdie \emph{tertaire} : insuf rénaux chronique
\end{itemize}
\end{itemize}
\item \textbf{hypercalcémie-hypocalciurie familiale bénigne} : hypercalcémie,
hypophosphorémie, (hypermagnésémie), calciurie \dec\dec{}, PTH inadaptées
(N ou \inc)
\item Lithium
\end{itemize}

\paragraph{hypercalcémie PTH-indépendante}
\label{sec:org86ed114}
\begin{itemize}
\item \textbf{Hypercalcémie des affections malignes} (30\%) : PTH \dec\dec{}. 
\begin{itemize}
\item Tumeurs : poumon, sein, rein, tractus digestif
\item Production tumorale de PTHrp (mime PTH)
\end{itemize}
\item Autres : 
\begin{itemize}
\item granulomatose : hyperphosphorémie, PTH \dec
\item iatrogènes : vitamine D (hypercalcémie, hyperphosphorémie, PTH passe),
vitamine A (asthénie sévère, douleurs musc et osseuse, alopécie des
sourcils, chéilite fissuraire), diurétiques thiazidique, buveurs de laits
(plutôt fortes doses d'antiacide ou carbonate de calcium)
\end{itemize}
\end{itemize}

\subsubsection{Traitement}
\label{sec:org287d216}
Hyperparathyroïdie primitive : guérison par ablation des adénome(s) par chir
conventionnelle ou mini-invasive (faire imagerie avant !)

Sinon, traitement palliatif : bisphosphonates (inhibe résorption osseuse),
calcimimétiques (\dec PTH), 

\danger Hypercalcémie maligne = urgence \{\} : 
\begin{itemize}
\item sérum phy
\item bisphosphonate en perf lente ou corticothérapide IV (myélome/hémopathie) ou dialyse (maligne)
\end{itemize}
\subsection{303 \textdagger{} Tumeurs de l'ovaire (hormono-sécrétante)}
\label{sec:org28b5773}
\subsubsection{Sécrétant des oestrogènes}
\label{sec:orgb7fb0bb}
Tumeurs de la granulosa : 
\begin{itemize}
\item malignes, les plus fréquentes des tumeurs des cordons sexuels et du stroma.
\item plutôt femmes [30,50] ans
\item jeune fille : pseudo-puberté précoce. Femme :
aménorrhées/ménométrorragie. Ménopausée : saignement vaginal dû à hyperplasie
endométriale\footnote{Tumeurs souvent > 10cm, kystique, multiloculaire, unlatérale}
\item ttt : ovariectomie unilatérale mais récidives 10-33\%
\end{itemize}

Thécomes : 
\begin{itemize}
\item très rare, surtout péri-/post-ménopause
\item Tumeurs solides, bénignes \thus exérèse = guérison
\end{itemize}
Sd Peutz-Jeghers (très très rare)

\subsubsection{Sécrétant des androgènes}
\label{sec:org4832623}
Tumeurs à cellules de Sertoli-Leydig 
\begin{itemize}
\item sécrète testostérone
\item rare. Y penser si hirsutisme récent avec signes de virilisation
\item DD : corticosurrénalome (faire scanner surrénales), sd Cushing (faire freinage
minute), block 21-hydroxylase (doser 17-hydroxyprogestérone)
\item femme 30-40ans
\item détecté à l'écho ovarienne vaginale ou IRM pelvienne
\item si < 5 cm, bon pronostic \thus ttt conservateur chez femme jeune
\end{itemize}

Tumeurs à cellules de Leydig
\begin{itemize}
\item cristaux de Reinke (caractéristique)
\item typiquement : virilisantes chez ménopausée
\item petite taille, bénigne \thus ovariectomie bilatérale
\end{itemize}

Tumeurs germinales sécrétantes
\begin{itemize}
\item tumeur ovarienne sécrétant de l'hCG : chez femme jeune, aménorrhée, douleurs
abdo/métrorragie. Tttt : conservateur si jeune, chimio si étendu
\item gonadoblastome : chez sd de Turner avec mosaïque et chromosome Y (risque
7-20\%) \thus gonadectomie préventive
\item autres : sécrétant hCG, T4, sérotonine
\end{itemize}

\subsection{305 \textdagger{} Tumeurs du pancréas (endocrine)}
\label{sec:orge95d03f}
Rare, concerne pancréas et duodénome. Diagnostic histologique, compléteté par
immunohistochimie

Pronostic péjoratif : > 2 cm, invasion vasculaire, dissémination métastase

\begin{table}[htbp]
\caption{Caractéristiques des tumeurs endocrines duodéno-pancréatiques}
\centering
\begin{tabular}{ll}
Sécrétion & Clinique\\
\hline
Insuline & Hypoglycémie organiques\\
Gastrine & Ulcère oestro-gastro-duodénaux, diarrhées\\
ACTH & Cushing\\
Glucagon & Diabète, érythème migrateur, diarrhée, amaigrissement, thromboses\\
VIP & Diarrhée hydroélectrolytique profuse, hypokaliémie\\
GHRH & Acromégalie\\
\end{tabular}
\end{table}

Imagerie : scanner spiralé TAP \textpm{} IRM abdo

Formes familiales : NEM1, neurofibromatose 1, von Hippel-Lindau

\subsection{310 \textdagger{} Tumeurs du testicule (aspects endocriniens)}
\label{sec:orga6c5484}

Prévalence : 9/100 000, ado/adulte jeune

\subsubsection{Tumeurs stromales}
\label{sec:org19f5d80}
Cellules de Leydig. Unilatérales, bénignes
\begin{itemize}
\item Garçon < 9 ans : pseudo-puberté précoce \thus testostérone plasmatique, écho
testiculaire
\item Adulte : féminisation, infertilité \thus oestradiol \inc, testostérone N ou \dec
\end{itemize}

Cellules de Sertoli : rares (enfant) ou exceptionnelles
(adulte). Féminisation/pseudo-puberté précoce à 50\%. Testostérone/oestradial
\inc, LH et FSH \dec, inhibine B \inc.\footnote{Peut appartenir à : complexe Carney, sd Petuz-Jeghers}

\subsubsection{Autres}
\label{sec:org41b2e35}
\begin{itemize}
\item Tumeurs germinales : fréquentes, écho testiculaire
\begin{itemize}
\item séminomateuses : fréquentes, pronostic bon
\item non séminomateuses : pronostic réservé
\end{itemize}
\item Inclusion surrénaliennes : par excès ACTSH. Marqueur : 17-hydroxyprogestérone
\end{itemize}

\subsubsection{PEC}
\label{sec:org2d75297}
Glucocorticoïdes si inclusion surrénaliennes. Sinon chir 1ere intention. Chimio si métastases pulmonaires/ganglionnaires.


\subsection{Annexes}
\label{sec:org86b575d}
\subsubsection{Hormones}
\label{sec:orgeb64c0a}


\begin{figure}[htpb]
  \centering
  \resizebox{!}{5cm}{
    \tikz \graph [decision]
    {
      ht/hypothalamus[organ] -> ["CRH"] hh/hypophyse[organ] -> ["ACTH"] cs/corticosurrénale[organ];
      cs -> cort/cortisol;
      cort ->[bend left=60, "-"] ht;
      cs -- ["+"] hh;
      hh --["+"] ht;
    };
  }
\resizebox{!}{5cm}{
    \tikz \graph [decision, layer distance=1.5cm]
    {
      ht/hypothalamus[organ] -> ["GnRH"] hh/hypophyse[organ] -> "FSH, LH" -> {
        testicules[organ] -> test/testosterone;
        ovaires[organ] -> est/estrogène;
      };
      test -> [bend left=70, "-"] ht;
      test -> [bend left=60, "-"] hh;
      est -> [bend right=70, "-"] ht;
    };
}
\resizebox{!}{5cm}{
    \tikz \graph [decision, layer distance=1.5cm]
    {
      ht/hypothalamus[organ] -> ["TRH"] hh/antéhypophyse[organ]
      ->["TSH"] th/thyroide[organ] -> ["T4"] "foie,muscles" -> "T3";
      th -> [bend left=60, "-"] hh;
      th -> [bend left=70, "-"] ht;
    };
}
\resizebox{!}{5cm}{
    \tikz \graph [decision, layer distance=1.5cm]
    {
      ht/hypothalamus[organ] -> ["GHRH"] hh/hypophyse[organ] -> ["GH"] Foie[organ] -> "IGF-1" -> {
        os;
	muscle;
	graisse..;      
      }
    };
}
\end{figure}

\begin{table}[htbp]
\caption{Hormones produite par les surrénales (du moins au plus profond)}
\centering
\begin{tabular}{ll}
Zone de la surrénale & Hormones\\
\hline
corticale (glomerusa) & minéralocorticoïdes (aldostérone)\\
corticale (fasciculata) & glucocorticoïdes (cortisol)\\
corticale (reticularis) & androgènes\\
médullaire & épinephrine, norepinéphrine\\
\end{tabular}
\end{table}

\subsubsection{Syndromes génétiques}
\label{sec:orgd12d40e}
\begin{center}
\begin{tabular}{llll}
 & Klinefelter & Turner & Kallmann\\
\hline
Sexe & \male & \female & \male{}, \female\\
Frequence & 1/500 & 1/2000 \female & 1/30 000 \male{}, 1/250 000 \female{}\\
Caryotype & 46 XX,Y & genes manquants & \\
 &  & sur bras court d'un chr. X & \\
Caractéristiques & Hypogonadisme, infertilité & Petite taille & Anosmie\\
 & gynécomastie & Aménorrhée & \male : Micropénis, cryptorchidie\\
 & trouble apprentissage, comm. & Pas de seins ? & Pas de dev. sexuel secondaire\\
 &  &  & \\
\end{tabular}
\end{center}

\section{Pneumologie}
\label{sec:org62f7388}
\subsection{73 Addiction au tabac}
\label{sec:org57839e1}
Tabagisme : primaire (inspiré), secondaire (passif++), tertiaire (exhalé)
Produits : nicotine (dépendance), fumée de tabac (0.3 \(\mu\)m), goudrons
(cancérigène), CO (hypoxie, risque ischémie)

16 millions de fumeurs en 2013 (France). Éducation et cat. sociale faible = plus
tabagiques
\subparagraph{Pathologies liées au tabac}
\label{sec:org80f10aa}
\begin{itemize}
\item K : \emph{broncho-pulmonaires} (90\% dû à actif, 25\% au passif). Voies aérodigestives sup, vessie, pancréas, rein, col de l'utérus
\item Respiratoires : BPCO, asthme
\item CV: cardiopathies ischémiques, artérite, HTA, cérébro-vasculaires, athérome,
\end{itemize}
maladie coronaire
\begin{itemize}
\item Autres : digestives, kératites, retard consolidation os, agueusie, anosmie
\end{itemize}

Passif :
\begin{itemize}
\item +25\% risque cancer bronchique, +25\% maladies CV, aggrave asthme, BPCO
\item nourrisson : RCIU\footnote{Retard croissance intra-utérin}, \(\nearrow\) risque
infections respi, mort-subite (1ere cause identifiée)
\end{itemize}

\subparagraph{Prise en charge}
\label{sec:org432b82d}
Évaluer : consommation (paquets/années), dépendance, autres (alcool = déclencheur,
cannabis), motivation, comorbidités (psy, CV, respi)
\subparagraph{Traitement}
\label{sec:org9184a24}
Conseil, motivationnel.

Substitut nicotinique, varénicline [\danger suivi], bupropion), TCC, cigarette électronique

Sevrage : réussi si \(\ge 1\) an

\subsection{108 - Troubles du sommeil}
\label{sec:orgca9fc5d}
\subparagraph{Définitions}
\label{sec:org7a10c07}
SAOS =
\begin{itemize}
\item somnolence diurne non expliquée
\item \emph{ou} 2 parmi \{ronflements, étouffement, éveils répétés, sommeil non réparateur, fatigue diurne, trble concentration, nycturie)\}
\item \emph{et} IAH\footnote{Index d'Apnées et Hypopnées (nb par heure)} \(\ge\) 5 \footnote{Sévère si IAH \(\ge\) 30.}
\end{itemize}
NB: Apnée obstructive (arrêt débit), centrale (idem, \uline{sans} efforts ventilatoires) ou mixte

\subparagraph{Épidémiologie}
\label{sec:org0d456d7}
2\% des \female, 4\% des \male.

FR : obésité, homme, âge, anomalie voie aériennes supérieures. 

Comorbidités : \emph{somnolence diurne excessive}, HTA, AVC, maladie coronaire, sd
métabolique \footnote{Obésité abdo et (2 parmi : HTA, glycémie \(\ge\) 5.6 mmol/L, HDL bas, hypertriglycéridémie)}, diabète, dylipidémie
\subparagraph{Diagnostic}
\label{sec:orgf8c814a}
Suspicion clinique, confirmation par polygraphie 
\begin{itemize}
\item clinique : nuit = \{ronflements, pause respi, étouffement, nycturie\}, journée =
somnolence excessive (questionnaire d'Epworth)
\item examen : IMC, obésité abdo (\(\ge\) 94 cm \male, 80cm \female), ORL\footnote{Rétrognathisme, macroglossie, hypertrophie du palais mou ou des
amygdales, obstruction nasale}, CV, respi
\item polygraphie \footnote{Flux naso-buccal, mouvement thoraco-abdominaux, capteur de son, SpO\textsubscript{2}, ECG} ou polysomnographie\footnote{Enregistre en plus EEG, électro-oculogramme, EMG muscles houppe du menton} (cher++)
\item faire aussi EFR (dépistage BPCO) et gaz du sang (complications BPCO + SAOS)
\end{itemize}
DD : hypersomnie diurne (insomnie, sd dépressif, sédatif, hygiène de sommeil, neuro)
\subparagraph{Traitement}
\label{sec:org83b7850}
\begin{itemize}
\item Général : PEC\footnote{prise en charge} surpoids, éviction \{benzodiazépines, myorelaxants, morphiniques\}, PEC CV
\item \emph{Pression positive continue}\footnote{Remboursement si IAH > 30/h (ou IAH > 10/h sur polysomnographie)}. Sinon : orthèse d'avancée mandibulaire, voire chir
\end{itemize}

\subparagraph{Autres}
\label{sec:org87ad9b3}
\begin{itemize}
\item Sd d'apnée de type central : IC sévère, atteinte tronc cérébral, séjour altitude, morphiniques
\item Hypoventilation alvéolaire : Traitement = VNI
\item Sd obésité hypoventilation\footnote{(PaCO\textsubscript{2} > 45 mmHg) et PaO\textsubscript{2} < 70 mmFg) et (IMC > 30 kg/m\textsuperscript{2} sans d'autre cause.} : Traitement = VNI
\end{itemize}

\subsection{151 \textdagger{} - Infections broncho-pulmonaires communautaires}
\label{sec:orge1d98f3}
\subsubsection{Bronchite aigue}
\label{sec:orga8ebb9d}
Très fréquente, virale 90\%. 

Diagnostic clinique : épidémie, toux sèche \(\to\) productive, expectorations,
\uline{pas} de crépitants.\\

Ttt symptomatique seulement ! (pas d'ATB, ni corticoïdes\ldots{})

\subsubsection{EBPCO (cf chap BPCO)}
\label{sec:orgbf680d8}

\subsubsection{Pneumonie aigue communautaire}
\label{sec:org7ad1eb0}

\subparagraph{Diagnostic}
\label{sec:orgc48490c}
Radio
\begin{itemize}
\item Clinique : \{toux, expect purulentes, dyspnée\} + \{fièvre, asthénie\} + crépitants
\end{itemize}

\subparagraph{PEC}
\label{sec:org8a0a7c8}
Si \faHospital : hémocultures, ECBC, antigenurie pneumocoque (\textpm{} PCR,
antigenurie légionelle)\\
\faHospital : signes de gravité (score CRB65\footnote{Confusion, fréquence Respiratoire \(\ge\) 30min, (Blood) PAs < 90mmHg ou PAd \(\ge\) 60mmg, Age \(\ge\) 65 ans} ge  ) / incertitude / échec domicile / comorbidité / inobservance

\subparagraph{Étiologies (Tab \ref{tab:org0bf695a}}
\label{sec:orgce5d6bd}
\begin{table}[htbp]
\caption{\label{tab:org0bf695a}Orientation clinique (non discriminant !)}
\centering
\begin{tabular}{llll}
 & Pneumocoque & Atypique & Légionellose\\
\hline
Début & Brutal & Progressif & Rapidement progressif\\
Signes & Thoraciques & Extra-thoraciques & Myalgie, digestif\\
Biologie &  & Anémie hémolytique & Hyponatrémie, rhabdomyolyse\\
Microbio & CG+ chainettes/Ag pneumocoque + &  & Ag légionelle +\\
RX thorax & condensation systématisée & Opacités multifocales & CS ou OM\\
\end{tabular}
\end{table}
\begin{itemize}
\item Pneumocoque : fréquent+++. Pas de transmission interhumaine
\item Atypique : \bact{mpneumoniae}, \bact{cpneumoniae}, \bact{psitacci}
\item Legionnella : pas d'isolement. DO
\item Pneumonie virale : signes respi + sd grippal. Grippale = diagnostic PCR, traitement = inihbiteur neuramidase \danger \bact{dore}
\end{itemize}

\subparagraph{Traitement}
\label{sec:org67e5a1e}
Urgent, probabiliste. Oral, 7j\footnote{(8-14 si légionellose, 21j si légionelles graves/ID)}  Réévalué 48-72h

\begin{itemize}
\item Ambulatoire : \emph{Amoxicilline} ou \emph{macrolide} \(\to\) switch
\item \faHospital  \emph{Amoxicilline} \(\to\) réévaluation
\item Réa : \emph{C3G IV + macrolide IV} / \emph{FQ} pour pneumocoque
\end{itemize}

Échecs : 
\begin{itemize}
\item épanchement pleural, abcès, obstacle
\item observance, pharmacocinétique, hors spectre
\item diagnostic (EP, PID aigǜes, K, tuberculose pulmonaires, infarctus pulmonaire, vascularite\ldots{})
\end{itemize}

\subparagraph{Prévention}
\label{sec:org135ea13}
Vaccin : 2 doses + rappel enfant, 2 doses adulte (ID, comorbidité)

\subparagraph{Immunodéprimé}
\label{sec:orgdc0a268}
Pneumocystose pulmonaire : 

\begin{itemize}
\item RX : sd interstitiel diffus bilatéral symétrique.
\item ATB cotrimoxazole 21j
\end{itemize}

\subsection{180 \textdagger{} - Accidents du travail}
\label{sec:org935883d}
Maladie professionnelle = lente, prolongée (\(\neq\) accident du travail)
\subsubsection{Maladies}
\label{sec:org0943699}
\paragraph{Asthmes}
\label{sec:orgf0c9158}
10-15\% des asthmes. On a

\begin{itemize}
\item asthme professionnel : (avec latence [immunologique]) ou (sans latence
[exposition unique])
\item asthme aggravé par le travail
\end{itemize}

Métiers : boulangers, santé, coiffeurs, peintres pistolets, bois, nettoyage\\
Diagnostic : de novo, profession, rythme
\paragraph{BPCO}
\label{sec:org09b2658}
\gls{TVO} lente progressive + inflammation
poumons. Facteurs professionnels = 10-20\% BPCO\\
Métiers : mines, BTP, fonderie, textile, agricole
\paragraph{Cancers}
\label{sec:org09be2a2}

\begin{itemize}
\item mésothéliome : 2/100 000habi/an, amiante++, DO
\item K bronchique primitif
\end{itemize}

\paragraph{Pneumopathie interstitielle diffuse (PID)}
\label{sec:org10543b4}

\begin{itemize}
\item Pneumopathie d'hypersensibilité : agricole
\item Silicose (+cancer bronchique primitif)
\item Bérylliose : adénopathies médiastinales, sd infiltrant parenchymateux
\item Sidérose : fumées d'oxyde de fer, micronodule \textpm{} emphysème
\item Asbestose : fibrose
\end{itemize}

\paragraph{Liées à l'amiante}
\label{sec:org11547d3}

\begin{itemize}
\item Cancer : mésothéliome, cancer bronchique primitif
\item Pleural (plaques, épaississements, pleurésies), parenchyme (fibrose)
\end{itemize}

\subsubsection{Reconnaissance}
\label{sec:org7efcee4}
3 conditions : médicale, administrative, professionnelle. 

Déclaration à la CPAM (employeur si AT, sinon victime)

Indemnité\footnote{Pour l'amiante : caisse d'assurance malaide ou FIVA}

\begin{itemize}
\item 60\% salaire si < 28 jours, 80\% sinon
\item si séquelles, selon taux incapacité permanente : capital (< 10\%), rente
(> 10\%), rentre et autres (> 40\%)
\end{itemize}

\subsection{182 \textdagger{} - Hypersensibilités et allergies respiratoires}
\label{sec:org47f0cfe}
\label{sec:182_hypersensibilites_et_allergies_respiratoires}

\begin{figure}[htpb]
  \centering
  \resizebox{0.5\linewidth}{!}{
    \tikz \graph [
    % Labels at the middle 
    edge quotes mid,
    % Needed for multi-lines
    nodes={align=center},
    sibling distance=3cm,
    edges={nodes={fill=white}}, 
    layered layout]
    {
      HS non allergique;
      HS allergique -> {
        Non IgE;
        IgE -> {
          atopique -> "insectes\\helminthes\\médicaments";
          non atopique -> "rhinite\\asthme\\alimentaire\\professionelles";
        };
      };
    };
  }
  \caption{Hypersensibilité}
\end{figure}
\subsubsection{Définitions}
\label{sec:orgb9ab656}
Atopie : prédisposition héréditaires à IgE face à des allergènes.\\
Allergie : réaction d'HS par mécanismes immunologiques\\
Sensibilisation : test cutané positif à un allergène\\
Allergène : capable d'induire une réaction d'HS = pneumallergènes (aéroportés),
trophallergènes (alimentaires), professionnels, recombinants

Hypersensibilité :

\begin{itemize}
\item type 1 (immédiate) : la plus fréquente, IgE. Rhinite, asthme
\item type 2 : complément et phagocytose. Réactions médicament.
\item type 3 : complexes immuns. Pneumonies d'HS
\item type 4 : LT, cytotoxiques 48-72h. Granulome épithélioïde gigantocellulaire.
Dermato.
\end{itemize}


\paragraph{Asthme et rhinite allergique}
\label{sec:org063bd56}
Polygénique.\\
Environnement : infections virales, sensibilitations pneumallergènes, tabac dès
conception, pollution air intérieur. 

\paragraph{Anomalies voies aériennes (asthme)}
\label{sec:org79418bb}
Remodelage bronchique : épaississement (membrane basale et muscle lisse \(\wedge\) oedème
bronchique) et obstruction \(\diameter\) (mucus)

\paragraph{Réaction IgE}
\label{sec:org2b3f938}

\begin{itemize}
\item Sensibilisation : synthèse IgE spécifiques (lymphocytes B)
\item effectrice : fixation de l'allergène \thus activation (histamine,
cytokines) \thus cascade allergique
\end{itemize}


\subsubsection{Épidémiologie}
\label{sec:org2101787}
Atopie = 30-40\% population. HS médicament = 7\% pop. Allergie alimentaire : 2\%
des 9-11 ans

FR : 

\begin{itemize}
\item génétique (enfants à risque)
\item environnement : fréq \(\nearrow\) temps, alimentaires, tabagisme passif
maternel, allergènes, pollution atmosphérique \danger niveau de preuve
\end{itemize}

Morbidité forte, mortalité encore forte pour l'asthme.

\subsubsection{Diagnostic}
\label{sec:orge27bed1}
Clinique : asthme, rhinite, conjonctivite

\paragraph{Diagnostic}
\label{sec:orged7db49}
Unité de temps, lieu et action (perannuels [acariens,
blattes, phanères d'animaux, végétaux d'intérieur, moisissures] ou saisonniers
[pollens]) \(\wedge\) IgE spécifiques

Prick-test = référence.

\begin{itemize}
\item \$\diameter \(\ge\) 3\$mm / témoin
\item acariens, pollens, phanères d'animaux, blatte, moisissures chez l'adulte
\item arachide, blanc d'oeuf, poisson, lait de vache si < 3 ans
\item CI : antihistaminiques, \$\(\beta\)\$-bloquants, eczéma, grossesse si
allergie médic
\end{itemize}


Autres tests : dosage IgE spécifique (moins sensible), multiallergénique (sensible mais
pas quantitatif)

\paragraph{Professionnelles}
\label{sec:org7fa3181}
Boulangers, santé, coiffeurs, peintres pistolets, bois, nettoyage

\paragraph{Autres}
\label{sec:orgdfe315b}
\begin{itemize}
\item Test de provocation = certitude mais dangereux
\item Très sécifique : dosage IgE totale, dosage éosinophiles sanguin, dosage tryptage sérique
\end{itemize}

\subsubsection{Traitement}
\label{sec:org432cb97}
\paragraph{Éviction allergènes Toujours.}
\label{sec:org09ef763}

\paragraph{Symptomatiques}
\label{sec:org4491574}

\begin{itemize}
\item antihistaminiques : rhinite, conjonctivite, prurit
\item corticoïdes : systématique si urgence (prednisone, prednisolone), local
si traitement fond
\end{itemize}


\paragraph{Immunthérapie spécifique}
\label{sec:org8d92318}
Faibles doses croissantes d'allergènes :

\begin{itemize}
\item Sous-cutanées/sub-linguale : acariens, pollens, hyménoptères
\item orale : pollen
\end{itemize}

CI : maladies allergiques, dysimmunités, grosses (induction), asthme sévère non
contrôlé, mastocytoses, \$\(\beta\)\$-bloquants

ES : 

\begin{itemize}
\item syndromique (asthme, rhinite, urticaire) = alerte
\item générale (hypotensions, bronchospasme, choc anaphylactique) =
interruption
\end{itemize}

\subsection{184 \textdagger{} - Asthme, rhinite}
\label{sec:orgdd8389b}
Asthme 6\%, mortalité 1000 DC/an, en baisse. Morbidité en hausse

Rhinite allergique 24\%.
\subsubsection{Définitions}
\label{sec:orgdb90174}
Asthme : 

\begin{itemize}
\item inflammation chronique modifiant les VAS avec symptômes respi et obstruction voies aériennes réversible
\item interaction gènes-environement, déclencheurs : exercice, HS aspirine/AINS,
irritants inhalés
\end{itemize}


TVO : 

\begin{itemize}
\item \gls{VEMS}/CVF < 0.7
\item réversible si +200mL et +12\% après BDCA\footnote\{Broncho-dilatateurs à courte durée
d'action\}
\end{itemize}


Hyperréactivité bronchique : -20\%VEMS après métacholine/air sec (VPP = 100\%)

Débit expiratoire de pointe (pour urgence)

\subsubsection{Diagnostic}
\label{sec:org1e9587c}
\paragraph{Asymptomatique}
\label{sec:org9f86903}
Diagnostic :

\begin{itemize}
\item symptômes caractéristiques (20 min, réversible, variable)

\begin{itemize}
\item plusieurs parmi \{gêne respiratoire, dyspnée,
sifflements, oppression thoraciques\}
\item \(\nearrow\) nuit/réveil, variable,réversibles, déclenché par \{rire, exercice
virus, allergènes, irritants\}
\end{itemize}

\item \textbf{et} obstruction bronchique : sibilant, TVO réversible ou apparaît avec métacholine
\end{itemize}

Sévérité évaluée à 6 mois
\paragraph{Exacerbation}
\label{sec:org2d34cd3}
\nearrow{} symptômes > 2 jours, non calmée, sans retour état habituel

Signes de sévérité :

\begin{itemize}
\item mots (au lieu de phrases), assis en avant, agité
\item FR > 30/min
\item muscles respi annexes
\item FC > 120/min, SpO\(_{\text{2}}\) < 90\%
\item DEP < 50\%
\item silence auscultatoire
\item respi paradoxale
\item troubles conscience, bradycardie, collapsus
\end{itemize}

\paragraph{DD}
\label{sec:orgb735564}
Sans TVO : cordes vocales, sd hyperventilation

TVO non réversible : \{BPCO, bronchectasies, mucoviscidose, bronchiolites
constrictives\}, autres (corps étranger, tumeur, insuf. cardiaque)
\paragraph{Bilan}
\label{sec:org7066002}
Facteurs favorisants, radio thorax, EFR (+test métacholine)

\subsubsection{Traitement}
\label{sec:orgcdc18b6}

\paragraph{Long cours}
\label{sec:org2841791}
Ttt de fond : cf Table\textasciitilde{}\ref{tab:ttt_asthme}

\begin{table}
  \centering
  \begin{tabular}{ccccc}
    \toprule
    Palier 1 & Palier 2 & Palier 3 & Palier 4 & Palier 5 \\
    \midrule
    CSI faible & ALT & CSI moyen/fort & triotropium & CSO faible\\
             & & ou (CSI faible et ALT) & ou (CSI fort et ALT) &\\
  \bottomrule                                                              
  \end{tabular}
  \caption{Ttt de fond de l'asthme (\textbf{BDCA à la demande}).\\
    CSI = Corticostéroïdes inhalés. ALT =
    anti-leucotriènes. CSO = Corticostéroïdes oraux}
  \label{tab:ttt_asthme}
\end{table}

Autre : 

\begin{itemize}
\item activité physique (sauf plongée)
\item facteurs favorisants : rhinite, allergie, tabac, \(\beta\)-bloquant
(aspirine/AINS), RGO, comorbidités
\item vaccins grippe (pneumocoque si asthme sévère)
\end{itemize}

Efficace ?

\begin{itemize}
\item symptômes contrôlés (diurnes < 2/sem, pas de réveil nocture, BDCA < 2/sem, pas
limitation d'activité)
\item exacerbations < 2 corticostéroïdes systém./an
\item VEMS/CV > 0.7 et VEMS \(\ge\) 80 \%
\end{itemize}

Si non contrôlé

\begin{itemize}
\item CSI faible + BDLA\footnote{Broncho-dilatateur à longue durée d'action} < CSI
moyen/fort + BDLA < centre spécialisé
\end{itemize}

Suivi : périodique, +3mois si changement ttt, mensuel si grossesse

\paragraph{Urgence cf figure\textasciitilde{}\ref{org1b938d5}}
\label{sec:org0db479d}

\begin{figure}[htpb]
  \centering
  \tikz \graph [
  % Labels at the middle 
  edge quotes mid,
  % Needed for multi-lines
  nodes={align=center},
  sibling distance=3cm,
  edges={nodes={fill=white}}, 
  layered layout]
  {
    "Évaluation" -> {
      "Exacerbation\\
      sans signes de gravite" -> 
      "\(\beta_2\) mimétique forte doses\\
      chambre inhalation\\
      \textbf{Corticothérapie orale}  7j" [draw]
      -> Réévaluation 1H;
      "Exacerbation \\
      avec signes de gravite" -> 
      "\(\beta_2\)-mimétique forte dose" [level distance=2cm] -> {
        "O\(_2\)\\
        \(\beta_2\) mimétique nébulisés (5mg/20min)\\
        $\pm$ ipratropium\\
        \textbf{Corticothérapie orale} " [>"\faHospital", draw];
      };
      Perte de contrôle ->
      "\(\beta_2\) mimétique" [draw] -> Réévaluation 1H;
    };
  };
  \caption{Asthme : traitement d'urgence}
\label{org1b938d5}
\end{figure}

Pas d'ATB sauf si suspicion bactérienne. Adrénaline seulement pour choc anaphylactique
\danger pas de BDLA 

\subsubsection{Rhinite allergique}
\label{sec:orgead8341}
Diagnostic :

\begin{itemize}
\item PAREO : prurit, anosmie, rhinorrhée, éternuement, obstruction nasale
\item fosses nasales au speculum inflammées
\item allergique (argumenter !!)
\end{itemize}

Sévère si persistant (> 4 semaines/an) et retentissement qualité de vie

TDM

Ttt : laval nasal, allergie, antihistaminique/corticoïdes nasaux, tabac, stress

\subsection{188 \textdagger{}, 189 Pathologies auto-immunes}
\label{sec:org3e8a639}
\label{sec:pathologies_auto_immunes}

Manifestations respi. viennent de (par priorité décroissante) :

\begin{itemize}
\item infectieuse (favorisé par ttt)
\item toxicité médicamenteuse
\item spécicifique
\item indépendant
\end{itemize}


\subsubsection{Complications infectieuses}
\label{sec:org334a16a}
Immunodépression :

\begin{itemize}
\item corticoïdes forte dose > 1 mois
\item méthotrexate
\item cyclophosphamide
\item anti-TNF\(\alpha\) (infliximab)
\end{itemize}

Tuberculose : 

\begin{itemize}
\item clinique semblable à l'ID = 50\% extrapulmonaire, 25\% disséminé
\item prévention si TNF-\(\alpha\) : INH 9 mois ou INH-RMP 3 mois
\end{itemize}

Pneumocystose : si corticoïdes forte dose ou méthotrexate ou
cyclophosphamide

\begin{itemize}
\item clinique : début brutal, insuf. respi, mortalité élevée
\item radio thorax : condensation alvéolaire/verre dépoli bilat
\item penser co-infections
\end{itemize}

\subsubsection{Médicaments}
\label{sec:org45eefe5}
Méthotrexate : plus fréquent, pneumopathie d'HS (opacités diffuses), \gls{LBA}
lymphocytaire. Évolution favorable à l'arrêt + corticothérapie

Inhibiteurs TNF-\(\alpha\) : PID/granulomatoses, anaphylactique

\subsubsection{Connectivites}
\label{sec:org3b64835}
Polyarthrite rhumatoide

\begin{itemize}
\item PID\footnote{Pneumopathie interstitielle diffuse} : radio =
réticulations, rayons de miels, bronchectasies. Surveillance seulement
\item pleurésie rhumatoïde : unilatérale, peu abondante, exsudative. Évolution
favorable
\item nodules pulmonaires rhumatoïdes.
\item bronchiolite oblitérante
\end{itemize}

Sclérodermie systémique : CREST (Calcinose, Raynaud, dyskinésie oEsophagienne,
Sclérodactyie, Télangiectasie), Ac anti-nucléraires

\begin{itemize}
\item PID : semblable à PINS\footnote\{Pneumopathie interstitielle non
  spécifique\} : opacités en verre dépoli, bronchectasies par traction. Survie
85\% à 5 ans
\item HTA pulmonaire : dyspnée. Echo. cardiaque +
cathéterisme cardiaque
\end{itemize}

Lupus érythémateux disséminé 

\begin{itemize}
\item pleurésie lupique : peu abondant, svt bilatéral, svt + péricardite
\item infectieux, sd hémorragie alvéolaire
\end{itemize}

Dermato-, polymyosite

\begin{itemize}
\item PID chronique : 1ere cause DC, opacités verre dépoli, \gls{LBA} lymphocytaire.
Ttt : corticothérapie + IS
\item PID (sub)aigüe
\end{itemize}

Sd de Gougerot-Sjögren

\begin{itemize}
\item bronchite lymphocytaire chronique : toux sèche chronique
\item PID, lymphome pulmonaire primitif
\end{itemize}


\subsubsection{Vascularites}
\label{sec:org9d19f2c}
Granulomatose avec polyangéite : 40-50ans, début ORL+ poumon (+ rein)

\begin{itemize}
\item radio : nodules s'excavant, opacités verre dépoli
\item Ttt urgent = corticothérapie, cyclophosphamide
\end{itemize}

Granumolatose éosinophilique avec polyangéite :  hyperéosinophilie, pneumopathie
éosino.

Polyangéite microscopique : sd hémorragique alvéolaire

\subsection{199 \textdagger{} Dyspnée aigüe et chroniques}
\label{sec:orgc31b98b}
\label{sec:199_dyspnee_aigue_et_chronique}
Examens : ECG, RX thorax, gaz du sang, D-dimère, BNP, NFS a minima

\subsubsection{Aigüe}
\label{sec:org88e6d1d}
= quelques heures/jours

Détresses respi aigüe :

\begin{itemize}
\item cyanose
\item sueur
\item FR > 30/min ou < 10/min
\item tirage, muscle respi accessoires
\item respi abdo paradoxale
\end{itemize}

Hémodynamique :

\begin{itemize}
\item FC > 110/min
\item choc (marbrures, oligurie, angoisse, extr. froides)
\item PAS < 80 mmHg
\item insuf. ventriculaire droite (turgescence jug, OMI, signe Harzer)
\end{itemize}

Urgence !!

Voir table\textasciitilde{}\ref{tab:dyspnee_aigue}. Autres :

\begin{itemize}
\item cardiaque : tamponnade\footnote\{orthopnée, tachy, assourdissement bruits,
ascult pulmonaire normale, turgescence jugulaire, pouls paradoxal\}, troubles
\end{itemize}
(supra)-ventriculaire, choc cardiogénique
\begin{itemize}
\item pulmonaire : SDRA, décompensation aigüe, atélectasies, trauma
\end{itemize}


\begin{table}[htbp]
\caption{Étiologies de dyspnée aigüe}
\label{tab:dyspnee_aigue}
\centering
\begin{adjustbox}{max width=\textwidth}
\begin{tabular}{lll}
\toprule
Inspiratoire & Expiratoire & Sinon\\
\midrule
corps étranger (enfant) & asthme (jeune, allergie, sibilants) & EP
                                                                (ascul. normale,
                                                                douleur
                                                                thoracique, phlébite)\\
épiglottite (enfant) & BPCO (tabac, bronchite aigüe, sibilants) & pneumothorax,
                                                                  épanchement
                                                                  pleural (sd
                                                                  pleural,
                                                                  douleur thoracique)\\
laryngite (enfant) & OAP (âgée, crépitants, expector mousseuse) & pneumopathie
                                                                  infectieuse
                                                                  (sd
                                                                  infectieux,
                                                                  douleur thoracique)\\
\oe{}dème de Quincke (terrain)&  & OAP (âgé, orthopnée, crépitants, expector mousseuse)\\
 &  & \\
\bottomrule
\end{tabular}
\end{adjustbox}
\end{table}

\subsubsection{Chronique}
\label{sec:orge9e3394}
cf table\textasciitilde{}\ref{tab:dyspnee_chronique}.
Autres :

\begin{itemize}
\item Cardiaque:constriction péricardique
\item Pulmonaire : (restrictif) pneumoconioses, post-tuberculose, paralysie phrénique, cyphoscoliose, obésité morbide
\item HTAP
\item HT pulmonaire post-embolique
\end{itemize}


\begin{table}[htbp]
\caption{Étiologies de dyspnée chroniques}
\label{tab:dyspnee_chronique}
\centering
\begin{adjustbox}{max width=\textwidth}
\begin{tabular}{lll}
\toprule
Sibilants & Crépitants & Auscult normale\\
\midrule
BPCO& PID (toux sèche, maladie systémique) & EP/maladie vasculaire pulmonaire \\
asthme& Insuf cardiaque gauche & Neuromusc \tablefootnote{signe neuro, orthopnée, respi abbdo paradoxale} \\
Insuf cardiaque gauche (ATCD cardiaque, orthopnée, toux) &  & Parétiale
                                                              (obésité, scoliose)\\
          && Hyperventilation\tablefootnote{C normal, vertige, $\ne$ effort, paresthésie}\\
\bottomrule
\end{tabular}
\end{adjustbox}
\end{table}
Quantification : échelle Borg [0-10] (aigüe) ou MRC [0-4] (chronique)
\subsection{200 \textdagger{} Toux chronique}
\label{sec:orgd3cee69}
\label{sec:200_toux_chronique}

\danger Éliminer toux post-infectieuse (< 3 semaines)

Signes de gravité : 

\begin{itemize}
\item AEG, sd infectieux
\item dyspnée d'effort, hémoptysie
\item modification toux chez fumeur
\item dysphonie, dysphagie, fausses routes
\item adénopathies cervicales suspectes
\item anomalies cardiopulmonaires
\end{itemize}

\begin{figure}[htpb]
  \centering
  \caption{PEC initiale d'une toux chronique}
  \resizebox{0.6\linewidth}{!}{
    \tikz \graph [
    % Labels at the middle 
    edge quotes mid,
    % Needed for multi-lines
    nodes={align=center},
    level distance=40pt,
    sibling distance=3cm,
    edges={nodes={fill=white}}, 
    layered layout]
    {
      Signe de gravité -> {
        Exploration ["oui"];
        "Médicaments ?" ["non"] -> {
          Test d'éviction ["oui"];
          "Coqueluche ?" ["non"] -> {
            Test diagnostique ["oui"];
            "Radio thorax anormale ?" -> {
              Bilan spécialisé ["oui"];
              "cf~\nameref{subsec:toux_orientation}" ["non"];
            };
          };
        };
      };
    };
  }
\end{figure}

\subsubsection{Orientation diagnostique}
\label{sec:org39e0c38}
\label{subsec:toux_orientation}
\begin{tcolorbox}
Rhinorrhée chronique, RGO, asthme, tabac, médicaments\footnote{IEC, inhib angiotensine II, \beta{}-bloquants}, coqueluche
\end{tcolorbox}
\textbf{ORL}

\begin{itemize}
\item rhinosinusiens : sd rhinorrée postérieur++, obstruction nasale chronique
\item carrefour aérodigestif : diverticule de Zenker, laryngite chronique
\end{itemize}

\textbf{Respiratoire} 

\begin{itemize}
\item Asthme : TVO réversible/hyperréactivité bronchique
\item BPCO : TVO non réversible
\item Cancer bronchique, tumeurs, bronchectasise (cf
\nameref{sub:bronchectasies})
\end{itemize}


\textbf{RGO} : pyrosis. Endoscopie digestive si FR, pHmétrie des 24h

\textbf{Allergique} 

\textbf{Systémique} : sd Gougerot-Sjögren, polychondrite atrophiante, maladie
de Horton, granulomatose avec polyangéite, rectocolite hémorr., maladie de
Crohn.

\textbf{Comportement} : dernière étiologie

\paragraph{Traitement d'épreuve (ordre d'échec):}
\label{sec:org5afef3d}
\begin{enumerate}
\item RGO : bromphéniramine + pseudoéphédrine 3 sem
\item asthme si TVO réversible, corticoïdes inhalés ou bronchodilatateurs inhalés si
hyperréactivité bronchique
\item avis spé
\end{enumerate}

Traitement symptomatique : arrêt tabagisme. Éviter si possible

\begin{itemize}
\item si toux sèche : opiacés, antihistaminique anticholinergique, non
antihistaminique non opiacés
\item si toux productive : mucomodificateurs, kiné
\end{itemize}


\subsubsection{Bronchectasies (DDB)}
\label{sec:org14cd5bf}
\label{sub:bronchectasies}
Types : bronchectasies (élargissement \(\diameter\)), bronchocèle (pus), "par
traction" (NB: pas des vraies bronchectasies)

Étiologie :

\begin{itemize}
\item infection respi sévères : coqueluche, tuberculose (virales respi enfant,
pneumonie bact, suppuration suite sténose)
\item mucoviscidose
\item non infectieux (poumon radique, aspergillose allergique, SDRA,
systémique, déficit immunitaire)
\end{itemize}

Évolution : colonisation bactérienne, hémoptysie, TVO (car dilatation seulement
proximale), insuffisance respi

Clinique :

\begin{itemize}
\item toux productive quotidienne depuis l'enfance
\item hémoptysie
\item EC : (râles bulleux à l'auscult), (hippocratisme digital)
\item infections à \{\bact{influenzae}, \bact{pneumocoque}\} puis \{\bact{dore},
\bact{aeruginosa}++\}
\end{itemize}


Diagnostic : TDM (certitude) = \(\diameter_\text{bronche}\) > \(\diameter_\text{artère}\), lumière
bronchique > 1/3 parenchyme, pas de réduction du \(\diameter\), grappes de kystes,
opacités tubulées

Traitement : ATB si exacerbation, complications parenchymateuses. Macrolides
pour l'inflammation. Chir si très local + compliqué.

\subsection{201 \textdagger{} Hémoptysie}
\label{sec:orgac7dc31}
\label{sec:201_hemoptysie}

Urgence 

\begin{enumerate}
\item Est-ce une hémoptysie ? Hématémèse ou ORL possibles
\item Gravité ? Suivant abondance, terrain, persistance \thus risque = hématose et asphyxie
\end{enumerate}

\subsubsection{Étiologies}
\label{sec:org9424d89}

\begin{itemize}
\item Tumeurs bronchopulmonaires++
\item Bronchectasies++
\item Tuberculose (évolutive/séquelles)++
\item Idiopathique++
\item Infections : aspergillaires, pneumopathie infectieuses nécrosante
\item Vasc : embolie pulmonaire, HT pulmonaire, anévrysmes/malformations
\item Hémorragie alvéolaires : insuf. cardiaque gauches, médicaments/toxiques,
vascularites, collagénose, sd Goodpasture
\end{itemize}


\subsubsection{Diagnostic}
\label{sec:org8c3fa23}
Localisation (important !)

Interrogatoire : ATCD respi, cardiaque, histoire médicale

Examens 

\begin{itemize}
\item clinique : \{\(SpO_2\), tension, pouls\}, mauvaise tolérance respi, gêne
latéralisée, hippocratisme digital
\item radio thorax pour siège (verre dépoli/sd alvéolaire), lésion
\item scanner plus précis : nature, localisation, carto. vasc.
\item (endoscopie : hématémèse, multiples lésions, tumeur proximale)
\item (artériographie bronchique : ttt par embolisation)
\item autres : gaz du sang, dosage Hb, bilan coagulation, groupe sanguin, \{BK,
ECG\} si suspicion OAP hémorragique
\end{itemize}


\danger BPCO \(\centernot\implies\) hémoptysie

\subsubsection{Traitement}
\label{sec:orgc113f52}
\(O_2\) + vasconstriction IV, protection voies aériennes (décubitus latéral,
ventilation mécanique)

Embolisation artérielle bronchique

Chir si localisé, fonction respi OK et "à froid"

\subsection{202 Épanchement pleural}
\label{sec:orgf2250f5}
\subparagraph{Diagnostic}
\label{sec:orgbcc5852}
Suspicion clinique, confirmé par imagerie
\begin{itemize}
\item douleur thoracique \emph{dépendant de la respiration}, dyspnée, toux sèche \emph{au
changement de position}, hyperthermie
\item examen : sd pleural liquidien (silence auscult, matité, \(\emptyset\) transmission
corde vocales, souffle pleurétique)
\item \danger Signes de gravité : détresse respi, choc septique, choc hémorragique
\end{itemize}
Imagerie
\begin{itemize}
\item RX : opacité dense, homogène, non sytématisé, limité par ligne concave\footnote{DD atélectasie : médiastin dévié vers l'opacité}
\item échographie pleurale (différencie pleurésie et collapus, guide ponction)
\item TDM : en urgence si embolie pulmonaire ou hémothorax
\end{itemize}

\subparagraph{Causes}
\label{sec:orgd050d8e}

\emph{Transsudats (protides < 25g/L)} : IC G, cirrhose, sd néphrotique, atélectasie (EP)

\emph{Exsudats (protides > 35 g/L)}\footnote{Ou critère de Light : (LDH > 200 UI/L) ou (protides pleuraux/sérique > 0.5) ou (LDH
pleuraux/sérique > 0.6)} 
Cf tab \ref{tab:org24ae670}
\begin{table}[htbp]
\caption{\label{tab:org24ae670}Étiologies des exsudats (néoplasique, infectieux)}
\centering
\begin{tabular}{llll}
Catégorie & Cause &  & CAT\\
\hline
Néoplasiques & pleurésie métastatique++ & poumon, sein, \oe{}sophage, colon & scanner, biopsie aveugle/vue                                                                           biopsie aveugle/vue\\
 & mésothéliome & exposition amiante & biopsie++ (thorascopie++)\\
 &  & RX: épaississement pleural circonférentiel & \\
 &  & , rétraction hémithorax & \\
Infectieux & bactérien &  & ATB \textpm{} évacuation si compliqué\tablefootnote{Épanchement abondant, germes, liquide purulent}\\
 & virale &  & \\
 & tuberculose & progressif, amaigrissement & biopsie pleurale++\\
\end{tabular}
\end{table}

Autres : EP, bénigne liée à l'amiante (exclusion++), post trauma (rupture
oesophagienne, sous-diaphragme), systémique (lupus, polyarthrite rhumatoide)


\subparagraph{Ponction}
\label{sec:org9b2cb09}
En majorité (sauf si peu abondant et insuf cardiaque G\footnote{Sauf si unilat/asymétrique ou douleur pleurale/fièvre ou traitement insuffisant)} ). En urgence si
fébrile, hémothorax ou mauvaise tolérance. Tout évacuer seulement si étiologique ou non cloisonné

\subparagraph{Biologie 1ere intention}
\label{sec:org7a39481}
Biochimie (trans- ou exsudat), cytologie (cf \ref{tab:org954af4a}), recherche germes pygènes, mycobactéries

\begin{table}[htbp]
\caption{\label{tab:org954af4a}Épanchements pleuraux avec exsudats : étiologies}
\centering
\begin{tabular}{lllll}
 & Cellules tumorales & Neutrophiles & Lymphocytes & Éosinophiles\\
\hline
Néoplasique & Métastasique &  & Cancer & Cancer\\
 & Mésothéliome &  & Lymphome & \\
 & Hémopathies malignes &  &  & \\
\hline
Infectieux &  & Parapneumonique & Tuberculose & Parasitose\\
\hline
Autres &  & Embolie pulmonaire & Sarcoïdose & Hémothorax\\
 &  & Pancréatite & Chylothorax & Pneumothorax\\
 &  & Sous-phrénique & PR, lupus & Embolie pulmonaire\\
 &  & Oesophage &  & Asbestosique bénigne\\
 &  &  &  & Médicament\\
\end{tabular}
\end{table}

\subsection{203 \textdagger{} Opacités et masses thoraciques}
\label{sec:org8b44ee6}
\label{sec:203_opacites_et_masses_thoraciques}


\begin{itemize}
\item < 3mm : micronodules
\item \$[3,30]\$mm : nodules
\item > 30mm : masses
\end{itemize}


\subsubsection{Nodules}
\label{sec:org4fd0cc2}
Origine maligne probable si :

\begin{itemize}
\item homme, > 50 ans, fumeur
\item carcinogènes professionnels, > 1cm (> 3cm ++)
\item contours spiculés, polylobés, irrégulier
\item attire structures proches
\item augmente de taille
\item pas de calcifications
\item fixe TED-FDG
\end{itemize}

Certitude = histologie (ponction transpariétale à l'aiguille)

\textbf{Tumeurs malignes}  

\begin{itemize}
\item cancers bronchopulmonaires primitifs : > 50 ans, fumeur, souvent nodule
solitaire
\item secondaires (métastases) : opacités rondes régulières
\end{itemize}


\textbf{Tumeurs bénignes} 

\begin{itemize}
\item Hamartochondrome (freq++) : "pop-corn", pathognomonique
\item Tumeurs  carcinoïdes
\end{itemize}


\textbf{Non tumorales}

\begin{itemize}
\item infectieux 

\begin{itemize}
\item abcès à pyogène (contexte aigü fébrile)
\item bactérie filamenteuse crossance lente
\item tuberculome (\thus prélèvements)
\item kytes hydatiques ("membrane flottante")
\item aspergillome (opacité ronde + croissant gazeux)
\end{itemize}

\item granumolatose avec polyangéite
\item nodules rhumatoïdes
\item atélectasies
\item masses pseudo-tumorale silicotiques (micronodules, confluents ?)
\item malformations artérioveineuses
\end{itemize}


\paragraph{Examens}
\label{sec:orgb3239d0}

\begin{itemize}
\item TDM et TEP
\item Fibroscopie pronchique systématique
\item Ponction transpariétale sous TDM (sauf insuffisance respi)
\item Autres : thoracotomie, médiastinoscopie si ADP médiastinales fixant en TEP-FDG
\end{itemize}

Prélevement si solide, > 8mm, hypermétabolique. Sinon surveillance TDM (sauf non solide et image résolutive à 6 semaines)

\subsubsection{Masses/tumeurs du médiastin}
\label{sec:org1805799}
Diagnostic : opacité avec limite externe nette, raccord pente douce, limite interne non
visible, tonalité hydrique

DD: intraparenchymateux, pariétal \thus TMD

\paragraph{Médiastin antérieur}
\label{sec:orgdbb0df6}

\begin{itemize}
\item Supérieur : goître plongeant \thus TMD : continuité glande thyroïde
\item Moyen : 

\begin{itemize}
\item tumeurs thymiques : épithéliales (thymomes, carcinomes thymiques),
lymphomes thymiques, kystes, tumeurs bénignes
\item Tumeurs germinales : bénignes, séminomateuses, non séminomateuses
(carcinomes embryonnaires, vitellines, choriocarcinomes)
\end{itemize}

\item Inférieur : kystes pleuropéricardiques
\end{itemize}


\paragraph{Médiastin moyen}
\label{sec:org5bd59d2}

\begin{itemize}
\item Tumoral : cancer bronchopulmonaires, lymphomes, LLC, cancers
extra-thoraciques
\item Non tumoral : sarcoïdose, tuberculose, silicose, infections
parenchymateuses chroniques, histoplasmose (Amérique du Nord)
\item Autres : insuf. cardiaque gauche
\end{itemize}


\paragraph{Médiastin postérieur "neurogènes"}
\label{sec:org6c7ea20}

Diagnostic :

\begin{itemize}
\item médiastin antérieur : 

\begin{itemize}
\item \$\(\alpha\)\$-foetoprotéine (tumeurs vitellines), HCG\footnote{Hormone gonadotrophine chorionique} (choriocarcinomes)
\item ponction transpariétale
\item médiastinotomie
\item chir si complète et peu mutilante
\end{itemize}

\item médiastin moyen : médiastinoscopie ou ponction transbronchique
\item médiastin postérieur : 

\begin{itemize}
\item ponction transpariétale, transoesophagienne
\item chir si complète et peu mutilante
\end{itemize}
\end{itemize}



NB : urgence si jeune et suspicion de tumeur germinale non séminomateuses

\subsection{204 \textdagger{} Insuffisance respiratoire chronique}
\label{sec:org93330f9}
\label{sec:org6d633b6}
\subsubsection{Mécanismes}
\label{sec:org3965a3c}
\label{sec:org47f478d}
Hypoxémie

\begin{itemize}
\item Inadéquation ventilation/perfusion :

\begin{itemize}
\item effet shunt (mauvaise ventilation) => \(O_2\) corrige
\item shunt vrai (communication anat. ou non ventilé) => \(O_2\) ne corrige
\end{itemize}
pas

\item Hypoventilation alvéolaire : pure (commande, neuromusc) ou effet "espace mort"
\end{itemize}

(mauvaise perfusion)

\begin{itemize}
\item Atteinte de la surface d'échange
\end{itemize}

Hypercapnie : hypoventilation alvéolaire (pompe ventilatoire/commande centrale
ou effet espace mort)

\subsubsection{Conséquences}
\label{sec:org6bf628f}
\label{sec:orgf6b3986}
Hypoxémie : Polyglobulie, rétention hydrosodée (fréquente), hypertension pulmonaire

Hypercapnie : compensée par le rein

\subsubsection{Étiologies}
\label{sec:org1d34163}
\label{sec:org5310651}
Hypoxémie si PaO\(_{\text{2}}\) < 70mmHg (arbitraire). Voir table\textasciitilde{}\ref{tab:etio_irc}.
\begin{table}
\begin{center}
  \begin{tabular}{llllll}
    \toprule
    TV ? & Obstructif & Restrictif & Restrictif & Mixte & Non\\
         &            &  $\frac{T_{LCO}}{V_a}$ bas & $\frac{T_{LCO}}{V_a}$ normal & & \\
    \midrule
    Patho. & BPCO & Interstitielles & Sd obésité-hypoventil & DDB & HTP\\
       & asthme &  & Atteinte cage thoracique & Muscoviscidose & \\
       & bronchiolite &  & &  & \\
    Mécanisme & $\frac{V_a}{Q}$ & Surf d'échange & Hypoventilation & $\frac{V_a}{Q}$ & Surf d'échange\\
    Atteinte & échangeur & échangeur & pompe/central & échangeur & vasculaire\\
    \bottomrule
  \end{tabular}
\end{center}
\caption{Insuffisance respiratoire chronique : diagnostic simplifié (selon
  EFR). $V_a$ ventilation alvéolaire, $Q$ débit sanguin, $T_{LCO}$ capacité de
transfert du CO}
\label{tab:etio_irc}
\end{table}


\subsubsection{Diagnostic}
\label{sec:org407304d}
\label{sec:orgf8776e2}
\paragraph{Symptômes}
\label{sec:org7b8c6de}
\label{sec:org35a8cf3}
IRC = dyspnée (sous-évaluée), neuropsy + patho initiale

Physique : 

\begin{itemize}
\item IRC : cyanose, insuf. cardiaque D (turgescence jugulaire, oedeme MI, reflux
hépato-jugulaire)
\item patho : 

\begin{itemize}
\item obstructive : distension thoracique, dimin. bilat murmure vésicul
\item restrictive : râle crépitant des bases, hippocratisme digital
\end{itemize}
\end{itemize}



\paragraph{Diagnostic}
\label{sec:org172de7e}
\label{sec:org7599c0f}
PaO\(_{\text{2}}\) < 70 mmHg (gaz du sang : hypercapnie)

Étiologie :

\begin{itemize}
\item EFR donne TVO (VEMS/CVF < 70\% => BPCO), TVR (échangeur/pompe) ou
\end{itemize}
mixte (DDB, mucov)
\begin{itemize}
\item Radio thorax
\item Autres : NFS (polyglobulie), ECG (dextrorotation, BDB droite, repolarisation),
\end{itemize}
écho cardiaque systématique (éval. ventricule D, dépistage du G)


\subsubsection{Traitement}
\label{sec:org3780f55}
\label{sec:org870a2d5}
Cause, arrêt tabac, vaccins (grippe, pneumocoque), réhabilitation respi.

Oxygénothérapie de longue durée : indiquée si 2 mesures à 2 semaines avec

\begin{itemize}
\item obstructive = PaO\(_{\text{2}}\) < 55mmHG ou ( \(\in\) [55, 60] mmHG et hypoxie
tissulaire\footnote\{Ht > 55\%, HTP, insuf. ventricule D, SpO\(_{\text{2}}\) nocturne \(\le\) 88\%\})
\item restrictive = PaO\(_{\text{2}}\) < 60 mmHg
\end{itemize}

Efficace si  IRC après BPCO, 15h/jour, \(O_2\) gazeux ou liquide

Ventilation long cours : IRC restrictive, la nuit.

Chir rare (200/an)

\subsubsection{Pronostic}
\label{sec:orga882ed2}
\label{sec:org2ec66b4}
Irréversible, risque = insuf. respi aigüe (surtout causée par insuf. respi. basse, dysfonction
cardiaque G, EP)


\subsection{205 BPCO}
\label{sec:orgcf36f1d}

\subparagraph{Épidémio}
\label{sec:org5ef4651}
\inc dans le monde. 

FR : \{tabac++, aérocontaminants professionnels\}, génétique: \(\alpha\)-1 antitrypsine

\subparagraph{Diagnostic \footnote{On peut inclure dans BPCO \textbf{si TVO} : bronchite chronique (toux
productive quotidienne \(\ge\) 3 mois/an et \(\ge\) 2 ans), emphysème (élargissement espaces aériens distaux + destructions parois
alvéolaires) inclus dans BPCO}}
\label{sec:org18a4b1f}
Évoqué sur la clinique, confirmé par EFR
\begin{itemize}
\item dyspnée/, toux, expectorations.
\item Signes physiques  : \nearrow temps expiratoire, \searrow murmure vésiculaire, \(\searrow\) bruits coeur, distension thoracique
\item EFR: \uline{TVO (VEMS/CVF < 0.7) persistante} après bronchodilateur
\end{itemize}

Voir tab \ref{tab:org62246bc}

\begin{table}[htbp]
\caption{\label{tab:org62246bc}Différences asthme-BPCO}
\centering
\begin{tabular}{ll}
Asthme & BPCO\\
\hline
Obstructive non réversible & Obstructive réversible\\
Jeune, atopique & Fumeur, > 40 ans\\
Survient \textasciitilde{}40 ans & Enfance\\
\end{tabular}
\end{table}

Sévérité : échelle GOLD\footnote{Pour l'obstruction : stade 1 (VEMS \(\ge 80\%\)) à 4 (VEMS < 30\%)}, MRC\footnote{Pour dyspnée : 0 = efforimportant, 1 = à plat, 2 = doit s'arrêter pour
marche à plat, 3 = qq minutes à plat, 4= pour s'habiller}, fréquences exacerbations (\(\ge\) 2 /an = grave)

\subparagraph{Complémentaire}
\label{sec:org825b382}
ECG si VEMS < 50\%, NFS, \(\alpha\)-1 antitrypsine si besoin
\subparagraph{Évolution}
\label{sec:org43f1352}
Perte fonction respi, exacerbations, handicap respi, risque
d'insuffisance respi, comorbidité CV = 1ere cause de mortalité

Score BODE\footnote{Body mass index, Obstruction, Dispnea, Exercice} pour la prédiction.
\subparagraph{Traitement}
\label{sec:org1dea37c}
Cf fig\textasciitilde{}\ref{fig:ttt_bpco}. 
\emph{Arrêt tabac}, vaccins grippe et pneumocoques, réhabilitation respiratoire, O\textsubscript{2}, chirurgie possibles
\begin{figure}[htpb]
  \centering
 \resizebox{0.5\linewidth}{!}{
  \tikz \graph [
  % Labels at the middle 
  edge quotes mid,
  % Needed for multi-lines
  nodes={align=center},
  sibling distance=4cm,
  level distance=2cm,
  edges={nodes={fill=white}}, 
  layered layout]
  {
    "Dyspnée/exacerbations" -> {
      BD longue durée[>"oui", draw]-> 
      {
        cBD/"Cortico. inhalé\\ + BD longue durée"[draw, >"exacerbation"]
	-> ["insuffisant"] c2BD/"Cortico. inhalé\\+ 2 BD longue durée"[draw]
	-> reeval/"Réévaluation"[>"dyspnée"];
        2BD/"2 BD longue durée" [draw, >"dyspnée"] -> {
	  c2BD[draw, >"exacerb"];
	  reeval[>"dyspnée"];
        };
      };
      BD courte durée [>"non", draw];
    }
  };
  }
  \caption{Traitement BPCO}
  \label{fig:ttt_bpco}
\end{figure}

\subsubsection{Exacerbations BPCO}
\label{sec:org77c76f8}
Exacerbation aigüe = aggravation \(\ge 2\) jours
\subparagraph{Diagnostic}
\label{sec:org98e4c08}
Soit BPCO connu avec \inc dyspnée, toux/expect, soit voir section \hyperref[org7752461]{item 354}

Déclenchants : \emph{infectieux}\footnote{Mais souvent pas de facteur précis} 
(\bact{influenzae}, \bact{pneumocoque}, \bact{catarrhalis})

DD : PAC, dysfonction cardiaque gauche, EP, pneumothorax, médicaments CI, trauma/chir thoracique, insuffisance cardiaque gauche aigüe.

\subparagraph{Explorations (si sévère)}
\label{sec:orgab229aa}
Imagerie thorax, ECG, NFS, CRP, iono, créat, gazométrie

\subparagraph{Traitement}
\label{sec:orgeab49af}
\emph{Bronchodilatateurs \(\beta\)-2 agonistes courte-durée} 
\begin{itemize}
\item \textpm{} ATB 5-7 jours (si expectoration
purulente ou gravité ou BPCO sévère) 
\begin{itemize}
\item amox + acide clav/CG3/fluoroquinolones si FR
\item amox \textpm{} acide clav/pristinamycine/macrolides sans FR
\end{itemize}
\item \faHospital : oxygénothérapie, kiné, HBPM, (assistance ventilatoire)
\end{itemize}

\subsection{206 \textdagger{} Pneumopathies infiltrantes diffuses}
\label{sec:orga170edf}
\subsubsection{Présentation}
\label{sec:orgb025018}
Clinique : dyspnée d'effort prgorsessive.\\
EFR : \gls{TVR} ( CPT < 80\% et VEMS/CVL >
70\% ) et TLCO < 70\%, hypoxémie, désaturation\\
Radio : opacités parenchymateuse non systématisées bilatérales

\subsubsection{PID aigüe}
\label{sec:orgd24fcc5}
\paragraph{Étiologies}
\label{sec:org24942f7}
Causes connues : lymphangite carcinomateuse, insuf. cardiaque gauche, médicamenteuse\\
Causes inconnues : sarcoïdose, fibrose plumonaire idiopathique
\paragraph{Démarche}
\label{sec:org5e0ab6b}

\begin{itemize}
\item Contexte (ATCD, ID, exposition)
\item ECG, BNP, echo cardiaque
\item LBA si possible
\item PEC thérapeutique (réa si détresse respi, \(O_2\), ATB probabiliste si fièvre, arrêt de médic. pneumotoxiques)
\end{itemize}


\subsubsection{PID subaigüe/chronique}
\label{sec:orgd42ed33}
\paragraph{Démarches}
\label{sec:org192e3ee}
Interrogatoire++ : terrain (sarcoïdose=25-45 ans, \gls{FPI} si > 60 ans), tabac (histiocytose langerhansienne, \gls{DIP}), toxico, médic, ATCD radio, exposition

Clinique : état général, signes de connectivite

\begin{table}[htbp]
  \caption{Biologie PID subaigüe}
  \centering
  \begin{tabular}{ll}
    \toprule
    Examen & Maladie\\
    \midrule
    NFS, CRP & Sd inflammatoire\\
           & Hyperéosinophilie, lymphopénie\\
    BNP & Insuf. cardiaque\\
    Créat & Insuf. rénale\\
    Précipitines sériques & Hypersensib. (si contexte)\\
    CEA, calcémie, calciurie & Sarcoïdose\\
    Facteur rhumatoïdes etc & Connectivites\\
    ANCA & Vascularite\\
    Séro VIH & Opportuniste\\
    \bottomrule
  \end{tabular}
\end{table}

\begin{table}[htbp]
  \caption{LBA PID subaigüe}
  \centering
  \begin{tabular}{ll}
    \toprule
    Normal & 80\% macrophages\\
           & < 15\% lymphocytes\\
           & < 5\% PNN\\
           & < 2\% PNE\\
    \midrule
    Alvéolite & Hypercellularité totale\\
    Histiocytose langerhansienne & Macrophage\\
    Sarcoïdose, PHS & Lymphocytaire\\
    P. à éosinophiles & Éosinophilique\\
    \gls{POC} & Panachées\\
    Hémorragie alvéolaire & Rosé\\
    Protéinose alvéolaire primitive & Laiteux\\
    \bottomrule
  \end{tabular}
\end{table}

Examens complémentaires :
\begin{enumerate}
\item fibro et LBA (+ biopsie bronchique)
\item soit biopsie pulmonaire chir (pas de diagnostic), soit biopsie
transbronchique et ADP médiastinales
\end{enumerate}

\paragraph{Oedème pulmonaire}
\label{sec:org7702a11}
Mécanisme : Surcharge hémodynamique\\
Clinique : HTA, coronaropathie, valvulopathie mitrale\\
Diagnostic : ECG, BNP, écho coeur\\
Imagerie : Péri-hilaire

\paragraph{Tuberculose}
\label{sec:orga64806f}
Mécanisme : BK\\
Clinique : Contage, AEG, hémoptysie\\
Diagnostic : Expectorations (ED, culture, biopsie transbronchique)\\
Imagerie : 

\begin{itemize}
\item pulmonaire = nodules, infiltrats, excavations
\item miliaire = micronodules diffus
\end{itemize}


\paragraph{Médicaments}
\label{sec:org2219fe4}
Imagerie : condensations, verre dépoli, épanchement pleural

\paragraph{Pneumopathies d'hypersensibilité}
\label{sec:orge1b1184}
Mécanisme : Ag organiques\\
Clinique : 

\begin{itemize}
\item aigüe : sd peudo-grippal quelques heurs
\item subaigüe : semaines/mois avec toux, fébricule, râles crépitants, squeaks
\item chronique : dyspnée, toux sèche
\end{itemize}

Diagnostic : Sérologie, LBA\\
Imagerie : Micronodules centrolobulaires flous, verre dépoli (lobes supérieurs)\\
Traitement : éviction Ag

\paragraph{Pneumoconioses}
\label{sec:orge4eb82e}
Mécanisme : Amiante, silice\\
Clinique : Exposition\\
Imagerie : 

\begin{itemize}
\item silicose : opacités micronodulaires diffuses \(\implies\) masses pseudotumorales. Peut donner un cancer bronchique
\item asbestose : opacités linéaires non septales des bases \(\parallel\) ou \(\bot\) plèvre, réticulations et rayons de miels comme FPI. Évolue vers insuf respi chronique.
\end{itemize}


\paragraph{Sarcoïdose}
\label{sec:org20e623a}
Mécanisme : Signes extra-respiratoires\\
Diagnostic : anapath : extra-pulmonaire, biopsie éperons bronchiques et transbronchique. ADP médiastinales \\
Imagerie : Nodules, micronodules (ditribution lymphatique), adénopathie, hyperdensités, distorsions bronchiques

\paragraph{Fibrose pulmonaire idiopathique}
\label{sec:org64999c9}
Clinique : Dyspnée d'effort progressive, toux sèche, hippocratisme digital, crépitants sec base\\
EFR : TVR, diminution TLCO\\
Imagerie : Réticulations, bronchectasies, rayons de miel. Domine sous-pleural et bases

\paragraph{Connectivites}
\label{sec:org3080327}
Mécanisme : Dysimmunitaire\\
Clinique : Extra-respi (polyarthrite rhumatoide, sclérodermie, lupus, vascularite)\\
Diagnostic : Ac spécifiques\\
Imagerie : Réticulations, hyperdensités, bronchectasies

\paragraph{Pneumopathie interstitielle non spécifique}
\label{sec:orga57dfcc}
Origine : connectivite, médicaments (idiopathique)\\
Imagerie : verre dépoli, réticulations, bronchectasies (sauf extrême périphérie du poumon)

\paragraph{Proliférations tumorales}
\label{sec:orgbd964cf}
Lymphangite carcinomateuse : toux sèche, rebelle. \\
Radio : épaississements nodulaires des septas intralobulaires.\\
Diagnostic : biopsies des éperons\\
Carcinome lépidique : verre dépoli. 

\subsection{207 \textdagger{} Sarcoidose}
\label{sec:orga3b5d2d}
Maladie : systémique, cause inconnue, hétérogène, ubiquitaire. Début 25-45ans
dans 2/3\\
Atteinte médiastino-pulmonaire 90

\subsubsection{Expression}
\label{sec:orgd63c43b}
\label{sec:org39048da}
\paragraph{Pulmonaire}
\label{sec:orgf577e00}
\label{sec:org4ec1d7e}
Toux (dyspnée)
Radio : 4 stades

\begin{itemize}
\item I : adénopathies hilaires bilatérales symétriques
\item II : + atteinte parenchyme (micronodulaire diffus, parties moyennes supérieures)
\item III : atteinte parenchyme isolée
\item IV : fibrose = opacités parenchymateuses rétractiles + ascension hiles, distorsion bronchovasc (sup et post)
\end{itemize}

TDM : atteinte parenchyme = micronodule selon lymphatiques. Utile pour : formes atypiques ou détection précoce (fibrose, complications [greffe aspergillaire])\\
EFR : sd restrictif, DLCO \$\searrow\$\\
(Endoscopie bronchique : normal/muqueuse en "fond d'oeil".)\\
Biopsie : \{éperons, LBA\} > \{ponction ganglions médiastinaux, transbronchique\} > médistanoscopie\\
Formes atypiques : TVO, cavitaires, pseudonodulaires/alvéolaires

\paragraph{Extra-pulmonaire}
\label{sec:org842c2a5}
\label{sec:org316c190}

\begin{itemize}
\item Oeil : uvéite antérieure aigue (toujours cherche uvéite postérieure)
\item Peau : nodules cutanés, lupus pernio, érythème noueux
\item ADP
\item Foie
\item Moins fréquentes : nerveux (sd méningé, paires craniennes), ORL (obstruction
nasale,
\end{itemize}

sd Mikulicz, sd Heerfordt), ostéo-articulaire (bi-arthrite cheville =
spécfique++), 
coeur (BAV, bloc branche droit), rein (\(\nearrow\) créatininémie)

\begin{itemize}
\item Généraux : asthénie (pas de fièvre sauf sd de Löfgren)
\end{itemize}

Sd de Löfgren = érythème noueux + ADP hilaires médiastinales (+ fièvre)
\paragraph{Biologie}
\label{sec:orgfb503e0}
\label{sec:orgbfe5d87}

\begin{itemize}
\item Hypercalciurie
\item Lymphopénie CD4
\item Hypergammaglobulinémie
\item Enzyme de conversion de l'angiotensine sérique (ECA)
\end{itemize}

\subsubsection{Diagnostic}
\label{sec:org62cfdb3}
\label{sec:org6670330}
Clinique + radio + lésions granulomateuses tuberculoides sans nécrose caséeuse +
élimination DD
\subsubsection{Évolution}
\label{sec:org8f09bcc}
\label{sec:org707f9ea}
< 2 ans : évolution favorable sans traitement.\\
Chronique > 2 ans : attention au vital/fonctionnel \\
Suivi : 3-6 mois\\
Pronostic : 80\% favorable sans traitement, 10\% séquelles, 5\% DC

\begin{table}[htbp]
  \caption{Pronostic de la sarcoidose}
  \centering
  \begin{tabular}{ll}
    \toprule
    Négatif & Positif\\
    \midrule
    > 40 ans & Érythème noueux\\
    Chronicité & Forme aigüe\\
    Stade III, IV & Stade 1 asymptomatique\\
    Extra-respi grave & \\
    \bottomrule
  \end{tabular}
\end{table}

Atteintes :

\begin{itemize}
\item pulmonaire : insuf. respir chronique, principace cause DC
\item extra-thoracique : attention fonctionnel/vital
\end{itemize}


\subsubsection{Traitement}
\label{sec:org33db1b0}
\label{sec:orgc05b9f6}
Atteinte respi : pas de ttt si sd de Löfgren ou stade I asymptomatique\\

\begin{itemize}
\item 1ère intention : Corticoïdes > 12 mois à 0.5mg/kg (décroissance par 6-12 semaines)
\item 2eme intention : hydroxychloroquine, méthotrexate, azathioprine
\item 3eme intention : cyclophsamide, anti-TNF-\(\alpha\)
\end{itemize}

\subsection{222 \textdagger{} Hypertension artérielle pulmonaire}
\label{sec:orgf659f4e}

\subparagraph{Physiopatho}
\label{sec:org3b54a24}
Circulation : (basse pression, faible résistante) \thus forte résistance

\subparagraph{Définition}
\label{sec:org650575f}
acrshort:PAPm \(\ge 25\) mmHG et
$\begin{cases}
  \text{\gls{PAPO}} \le 15 & \text{mmHG si précapillaire} \\
  \text{PAPO} > 15 & \text{mmHG si postcapillaire}
\end{cases}$

5 groupes :
\begin{enumerate}
\item HT \uline{A} P\footnote{Idiopathique, héritable, médicaments, maladie veino-occlusive, hémangiomatose capillaire
pulmonaire, HTP persistante du nouveu-né} :  pré-capillaire
\item \textbf{cardiopathie gauche}  : post-capillaire (fréq+++)
\item HTP maladie \textbf{respiratoire chronique}  : pré-capillaire (freq++)
\item HTP post-embolique chronique : pré-capillaire
\item HTP multi-factorielles : pré-capillaire
\end{enumerate}

\subparagraph{Pronostic}
\label{sec:orgddfab51}
6 cas/million (idiopathique), femme. Survie avec ttt : 58\% à 3 ans

\subparagraph{Diagnostic}
\label{sec:orgaed950c}
Cathétérisme cardiaque D/ avec \textbf{PAPm \(\ge 25\) mmHg}. Découvert sur dyspnée, dépistage
\begin{itemize}
\item Interrogatoire : ATCD, anorexigènes, toxiques, maladie (sclérodermie : sd de
Raynaud, dysphagie, dyspepsie)
\item dyspnée d'effort+++ progressive (lipothymie à l'effort,
syncope, asthénie, douleurs angineuses, palpitations, hémoptysies)
\item HTP (signe de Carvallo\footnote{Souffle holosystolique d'insuf. tricuspid majoré à l'inspiration}, éclat B2, souffle diastolytique d'insuf pulmonaire), insuf cardiaque D\footnote{Tachy, galop, turgescence jugulaire, reflux hépato-jugulaire, HMG, OMI, anasarque}
\item RX thorax normale ou dilatation artères pulmonaires, élargissement coeur D
\item ECG : hypertrophie D, trouble rythme
\item \emph{Écho cardiaque transthoracique} = non invasif de référence.
\end{itemize}

\subparagraph{Démarche}
\label{sec:orgaec6735}
Si écho cardiaque compatible avec HTP :

\begin{itemize}
\item Cardiopathies G, maladies respi + exams pour groupe 2 et 3. Si confirmé : \faStop
\item Sinon regarder signes thrombo-embolie chronique (scinti, angioscan) : Si
groupe 4, \faStop
\item Sinon confirmer HTP précapillaire
\item Si confirmée, tester pour groupe 1 (connectivites, médicaments, VIH,
cardiopathie congénitale, HT portale, schistosomiase) ou groupe 5
\end{itemize}

\subsubsection{Enfant}
\label{sec:orgf884a7e}

\subparagraph{Physiopatho}
\label{sec:org5daf854}
Augmentation du débit ou des résistances

\subparagraph{Clinique}
\label{sec:org0511d89}
\begin{itemize}
\item Nouveau-né : cyanose réfractaire à l'\(O_2\), détresse respi/circ \thus échocardio en
\end{itemize}
urgence \danger
\begin{itemize}
\item Enfant : dépistage si cardiopathie congénitale, patho. respi. chronique, maladie de
\end{itemize}
systèmes, ATCD familiaux (signes tardifs !\footnote{Dyspnée d'effort, syncope, fatigue} )
\subparagraph{Diagnostic}
\label{sec:org7f11921}
Échocardio, confirmé par cathétérisme 

Scanner (parenchyme), écho hépatiques (shunt porto-cave)


\subsection{224 \textdagger{} Embolie pulmonaire et thrombose veineuse profonde}
\label{sec:orgf5dc94f}
\label{sec:224_embolie_pulmonaire_et_thrombose_veineuse_profonde}
\gls{MTEV} = \{\gls{EP}, \gls{TVP}\}

Maladie fréquente (1 cas /10 000 si < 40 ans, 1/100 si > 75 ans) et grave.
3eme cause de DC en france.

TVP : obstruction thrombotique d'un tronc veineux profond. EP : idem mais
artères pulmonaires ou leurs branches (secondaires TVP à 70\%)

\paragraph{Physiopatho stase veineuse, lésions pariétales, anomalies de l'hémostase \thus}
\label{sec:org280b544}
obstruction/lyse. Si migre dans les artères pulmonaires : symptômes si
obstruction à 30-50\% \thus anomalie hémodynamique \thus insuf respi

\paragraph{Évolution}
\label{sec:org752ada9}
TVP distales asymptomatique (postop) : 20\% deviennent proximales\\
TVP distales symptomatique : récidive (9\% si ttt anticoagulant)\\
TVP proximal symptomatique : risque important !\\
EP : TVP + 3/7 jours, mortelle dans l'heure à 10\%

Facteurs de risque :

\begin{itemize}
\item acquis : majeurs = chir < 3mois, trauma MI\footnote{Membres inférieurs},
\faHospital{} aigü, cancer en cours  ttt, sd antiphospholipide, sd
néphrotique
\item constit : rare = déficit (antithrombine, prot. C, S), fréq = (mutation \{Leiden,
prothrombine\}, facteur VIII > 150\%)
\end{itemize}

Complications : DC, récidive (mortelle ou non), séquelle (HTP thrombo-embolique
chronique si EP, sd post-phlébite si TVP)

Risque de récidive dépend de la clinique : élevé si (non provoqué par facteur
majeur ou modéré) ou (\og facteur persistant)

Conséquence :

\begin{itemize}
\item hémodynamique : \(\nearrow\) pression artérielle pulmonaire, dilatation
VD\footnote{Ventricule droit}, compression VG
\item respiratoire : hypoxémie (effet espace mort (non perfusé) \thus effet shunt
(ventil/perfusion diminué))
\end{itemize}


\subsubsection{Diagnostic de l'embolie pulmonaire}
\label{sec:orgd3ef0b5}

\paragraph{Clinique pas spécifique : dyspnée, douleur thoracique, syncope,}
\label{sec:org1088d08}
crachats hémoptoïques, asymptomatique

Radio thoracique, ECG : élimine les DD

Gaz du sang : hypoxie-hypocapnie

\thus score probablitié (Genève, Wells)

\begin{figure}[htpb]
  \centering
  \resizebox{0.3\linewidth}{!}{
    \tikz \graph [
    % Labels at the middle 
    edge quotes mid,
    % Needed for multi-lines
    nodes={align=center},
    level distance=2cm,
    sibling distance=3cm,
    edges={nodes={fill=white}}, 
    layered layout]
    {
      "Proba. clinique" -> {
        "D-dimères" [>"faible", draw] -> {
          "Pas de ttt" [>"négatif"];
          "\texttt{\$examen}" [>"positif", draw];
        };
        "\texttt{\$examen}" [>"forte", draw] -> {
          Ttt [>"positif"];
        };
      };
    };
  }
  \caption{Diagnostic général pour l'EP, TVP}
  \label{fig:ep-diag}
\end{figure}


\paragraph{Examens}
\label{sec:org43f5608}
Voir la figure\textasciitilde{}\ref{fig:ep-diag} avec \texttt{examen} = angioscanner.

D-dimère positifs = \$\(\max\)(\text{âge}, 50) \texttimes{} 10 \(\mu\)\$g/L

CI à l'angioscan : insuf. rénale sévère (< 30ml/min) \thus scintigraphie

Si angioscan ou scintigraphie négatif : pas d'EP !

\begin{tcolorbox}
\danger{} si EP grave (état de choc) : angioscan immédiatement ou ETT en
attendant.

{} Thrombolyse/embolectomie après écho si pas d'accès à l'angioscan
\end{tcolorbox}

Sur l'écho, chercher dilatation cavité D, HTP, septum paradoxal

\emph{Si grossesse}  : D-dimères \thus écho. veineuse \thus angioscanner
(prévenir le pédiatre )

\paragraph{Évolution favorable. Complications : choc cardiogénique réfractaire}
\label{sec:org26bffa5}
(DC), rédicive, HTAP chronique post-embolique (rare mais grave)

\subsubsection{Diagnostic de TVP}
\label{sec:orga38c8d7}
\paragraph{Clinique = orientation}
\label{sec:org608e7d5}
Douleur du MI, oedème unilatéral, signes inflammatoires, dilatation veines
superficielles ou asymptomatique

\emph{DD}  : traumatisme, claquage musc/, kyste synovial, \{SPT, insuf veineuse
primaire\}, \{sciatique, compression extrinsèque\}, \{érysipèle, lymphangite,
cellulite\}, lymphoedeme, insuf cardiaque droite/rénale/hépatique

Score de probablitié clinique : Wells (faible/interm/forte)

\paragraph{Diagnostic}
\label{sec:orgb488bbd}
Voir la figure\textasciitilde{}\ref{fig:ep-diag} avec \texttt{examen} = échographie veineuse
des MI

\paragraph{Étiologies}
\label{sec:orga8eeb1b}

\begin{itemize}
\item FR transitoire (chir, fracture < 3 mois, immobilisation > 3 j) ? Sinon, "non
\end{itemize}
provoquée"
\begin{itemize}
\item Recherche de thrombophilie si

\begin{itemize}
\item 1er épisode :  (non provoqué < 60 ans) ou (femme âge procrééer)
\item ou récidive : (MTEV proximale) ou (TVP distale non provoquée)
\end{itemize}

\thus 1ere intention : \gls{AT}, prot. C et S, mutation G20210A,
homocysténiméie, Ac antiphospholipides
\item Recherche cancer : > 40 ans ou bilan thrombophilie négatif
\end{itemize}


Formes particulières :

\begin{itemize}
\item thromboses veineuse superficielles : (sur trajet saphène, douloureux,
rouge, inflammatoire, cordon induré \thus écho-doppler
\item TVP pelvienne
\item thrombose veine cave inférieure
\item phlébite bleue : très rare mais grave
\end{itemize}


Si grossesse : bilan thrombophilie si ATCD familaux/personnels MTEV

Si cancer : HBPM long cours

\paragraph{Évolution : favorable si bien conduit mais récidive toujours}
\label{sec:org7e03938}
possible. Complications : 

\begin{itemize}
\item SPT : lourdeur de jambes, oedeme de cheville, dilat veineuses
superficielles, troubles trophiques sans ulcère, ulcères sus-malléolaires
\item EP
\end{itemize}




\subsubsection{Traitement (TVP + EP)}
\label{sec:orgb70c23a}
\paragraph{Principes}
\label{sec:orgdf4cab3}
Urgence  \thus anticoagulant

CI : coagulopathie sévère, hémorragie intracrânienne spontanée, hémorragie
active difficilement contrôlable, chir récente, (thrombopénie à l'héparine)

\paragraph{Types de traitement}
\label{sec:orge82a236}
Option 1 : Héparines + relais AVK dès injection IV

\begin{itemize}
\item HBPM, fondaparinux > HNF, sauf si IR sévère (< 30ml/min)
\item arrêt si 5 jours d'AVK et IV (ensemble) \(\wedge\) INR \(\in [2,3]\) à 24h
\end{itemize}

Option 2 : Anticoagulants oraux directs : rivaroxaban, apixaban (France)

\begin{itemize}
\item rapide, demi-vie courte
\item facteur X
\item CI : IR sévère, grossesse, interaction médic (cytochrome 3A4 ou
P-glycoprroténie)
\end{itemize}

Éucation thérapeutique\\
Autres :

\begin{itemize}
\item filtre cave (si CI absolu aux anti-coagulant ou EP récidivant)
\item fibrinolyse (si EP + choc, sauf si hémorragie active,
AIC\footnote{Accident ischémique cérébral}< 2 mois,
hémorragie intracrânienne)
\item embolectomie (très rare)
\item contention veineuse (sauf si EP sans TVP)
\item lever +1h
\end{itemize}

\paragraph{Stratégie}
\label{sec:org707a6a1}
Score sPESI = \texttt{90 100 110 C C} \footnote{Sat < 90\%, PAS < 100 mmHg, FC > 110/min, Cancer, insuf Cardiaque chronique}:

Risque faible (sPESI = 0) : \faHospital{} courte < 48h, anticoagulation\\
Risque intermédiaire (sPESI > 0)

\begin{itemize}
\item dysfonction VD ou élévation biomarqueurs\footnote{eBNP, NT-pro-BNP, troponine} : \faHospital{}
médecine, anticoagulation
\item dysfonction VD \(\wedge\) élévation biomarqueurs : urgence \danger{}
\thus USI

\begin{itemize}
\item \(O_2\), scope
\item anticoag : HNF/HBPM puis AVK/AOP à 48-72h\\
\end{itemize}
(trombolyse si choc)
\end{itemize}



Haut risque (choc : PAS < 90mmHG ou -40mmHg) : urgence  \thus réa

\begin{itemize}
\item \(O_2\) (ventilation méca.), scope
\item anticoag HNF (!)
\item thrombolyse avec arrêt HNF tant que TCA > 2x témoin
\item (embolectomie)
\end{itemize}


\paragraph{Durée}
\label{sec:org17c8ff0}
3 mois si 1ere EP/TVP provoqué par facteur majeur transitoire ou risque
hémorragique élevée. 6 mois ou plus sinon

\paragraph{Étiologie}
\label{sec:org142300e}
Chercher cancer occulte dans tous les cas (clinique, radio poumon, NFS VS,
dépistage globux)

Bilan coag : \{antithrombine, prot C, S\}, mutation \{Leiden, prothrombine\}, sd
antiphospholipides

\paragraph{Prophylaxie}
\label{sec:org6769c2c}

\begin{itemize}
\item Post-op : (chir et > 40 ans) ou (chir hanche/genou/caricinologique, anomalie
coag, (> 40 ans et ATCD MTEV))
\item polytrauma ou (\{rhumato, inflammatoire intestin, infectin\} + 1 FR)
\end{itemize}



\paragraph{Cas particuliers}
\label{sec:org5fad299}

\begin{itemize}
\item 
\end{itemize}
TVP distale : symptomatique : anticoagulement 6 semaines seulement si premier épisode. Dans tous les cas,
compression \(\ge 2\) ans
\begin{itemize}
\item \gls{TVS} : \textbf{pas} d'AINS, antiocoagulants en curatif, chirurgie
\item cancer : HBPM, arrêt si plaquettes < 50g/L
\end{itemize}

Prévention : 

\begin{itemize}
\item risque modéré : HBPM/HNF/fondaparinux, compression
\item risque élevé : idem sans HNF
\end{itemize}


Nouveau anticoagulant oraux (rivaroxaban) :

\begin{itemize}
\item prévention chir hanche-genou
\item curatif TVP, EP
\item pas si insuf rénale sévère ou insuf hépatique
\item pas de surveillance bio
\end{itemize}

\subsection{228 \textdagger{} Douleur thoracique aigüe et chronique}
\label{sec:org1443bbb}
\label{sec:228_douleur_thoracique_aigue_et_chronique}

\subsubsection{Signes de gravité}
\label{sec:org63f8e83}

\begin{itemize}
\item Respi : cyanose, tachypnée, lutte avec tirage, balancement thoraco-abdominal
\item CV : pâleur, tachycardie, hypotension, choc (marbrures, extrémités
froides)
\item neuro : lipothymie/syncope, agitation/trble vigilance, général (sudation)
\end{itemize}


\danger arrêt cardio-respi ! Y penser si bradypnée, (bradycardie et choc et troubles vigilance)

\subsubsection{Examens}
\label{sec:org61bb6cc}
Fréquence respi, \(SpO_2\), radio thorax, ECG.
Si \{brady, tachy\}pnée ou \(SpO_2 < 95\%\), gaz du sang

\subsubsection{Urgences vitales}
\label{sec:org18f549a}
Cf table\textasciitilde{}\ref{tab:urgences_douleur_thoraciques}

\begin{table}
  \centering
  \begin{tabular}{ll}
    \toprule
    Urgence & Orientation\\
    \midrule
    Syndrome coronaire aigü (fréquent++ 1/3) & ECG + troponines\\
    Embolie pulmonaire (fréquent) & Suspicion si douleur thorax, pas d'anomalie ascult\\
            & RX thorax "normale" \textit{surtout}  si hypoxémie + FR\\
            & Dyspnée ou douleur thoracique aigüe chez TVP = EP\\
    dissection aortique (exceptionnelle) & Échocardio + angioscanner\\
    Tamponnade (peu fréq) & suspicion (hypotension réfractaire, insuf. cardiaque D
                            aigüe,\\
            & microvoltage + alternance ECG) \thus echo cardiaque\\
    Pneumothorax & ATCD (!), radio thorax \\
    pneumomédiastin (rare) & scanner\\
    \bottomrule
  \end{tabular}
  \caption{Urgences vitales pour douleur thoraciques}
  \label{tab:urgences_douleur_thoraciques}
\end{table}

\begin{itemize}
\item Embolie pulmonaire (fréquent) : suspicion si douleur thorax, pas
d'anomalie ascult, RX thorax "normale" \emph{surtout}  si hypoxémie +
facteurs de risque\\
Dyspnée ou douleur thoracique aigüe chez TVP = EP
\item dissection aortique (exceptionnelle) : échocardio + angioscanner
\item Tamponnade (peu fréq) : suspicion (hypotension réfractaire, insuf.
cardiaque D aigüe, microvoltage + alternance ECG) \thus echo cardiaque
\item Pneumothorax : ATCD (!), radio thorax \\
pneumomédiastin (rare) : scanner
\end{itemize}


\subsubsection{Non urgent}
\label{sec:org2cdea29}
Rythmées par respiration :

\begin{itemize}
\item post-traumatique
\item pneumonie infectieuses : radio thorax
\item épanchement pleural (douleur latéral-base, majorée inspiration, toux)
\item infarctus pulmonaire (douleur basithoracique, faible hémoptysie)
\item trachéobronchite aigüe
\item musculosquelettique, nerfs : tumeurs costales, lésions vertèbres,
névralgies cervicobrachiales
\end{itemize}

Non rythmées :

\begin{itemize}
\item angor d'effort stable (calmée 2-5min post-effort)
\item péricardite (viral si aigü, tuberculose/néoplasie sinon)
\item cocaïne (fréquente) : SCA, myopéricardite,pneumothorax
\item zona thoracique (brûlures, hyperesthésie 24h avant)
\item digestives : reflux gastro-oesophagen, spasmes oesophagiens. Exclure SCA

\item psychogènes
\end{itemize}

\subsection{306 \textdagger{} Tumeurs du poumon}
\label{sec:orgcb5866f}
\label{sec:306_tumeurs_du_poumon}

\subsubsection{Épidémiologie, étiologies}
\label{sec:orgb4d409c}
1ere cause de mortalité par cancer en France (\male : constante, \female
\(\nearrow\))

Tabac : 90\% (âge de début et durée) pour actif, 25\% passif

Professionel : 15\%

\paragraph{Histologie}
\label{sec:org9c2a737}

\begin{itemize}
\item "Non à petites cellules" (CBNPC) (80\%) : adénocarcinomes (périphérie),
carcinomes épidermoïdes (proximal), indifférenciés à grandes cellules
\item neuro-endocrine "à petites cellules" (CBPC) (proximal, médiastin)
\end{itemize}


\subsubsection{Manifestations}
\label{sec:org52fd2fe}
\begin{tcolorbox}
Y penser si (AEG et tabagique) ou (symptôme fonctionnel respiratoire
  chez tabagique > 40A)
\end{tcolorbox}

\emph{Respiratoires}  : toux (souvent révélatrice, sèche, quinteuse), hémoptysie (<
10\%), bronchorrée (propre), dyspnée, pneumonie/bronchite (\danger{} si récidive
dans même territoire), douleur thoracique 

\emph{Locorégionales}  : pleurésies, dysphonies, sd cave supérieur, douleurs thoraciques
(fixes, tenaces), sd de Pancoast-Tobias\footnote\{Névralgie cervicobrachiale avec
  douleurs radiculaire C8-D1, sd Claude-Bernard-horner\}, paralysie phrénique

\emph{Extrathoracique}  : thromboses inexpliquées (y penser !), métastase (SNC,
foie, os, surrénales)

\emph{Sd paranéoplasiques} (à distance, réapparition \thus rechute) :
hippocratisme digital, hypercalcémie, hyponatrémie, sd de Cushing, neurologique
(pseudomyasthénie de Lamert Eaton, neuropathies périph, encéphalopathies)

\subsubsection{Imagerie}
\label{sec:orge40e48f}
Initial = radio thorax chez fumeur > 40 ans
\paragraph{Radiothorax}
\label{sec:orge3f2deb}

\begin{itemize}
\item projection (juxta-)hilaires
\item atélectasies
\item opacités arrondies intraparenchymateuses
\item cavitaires néoplasiques
\end{itemize}

\paragraph{TDM injection}
\label{sec:orgf507a02}
Stade N0 sans adénopathie, N1 si hilaire, N2 si médiastin homolatéral, N3 si médiastin
controlatéral

\paragraph{TEP au 18-FDG}
\label{sec:org8389ada}
Pas dans le cerveau !

Faux positifs : ganglions inflammatoires, infectieux

Faux négatifs : < 1 cm, non solide (verre dépoli)

\subsubsection{Diagnostic histologique}
\label{sec:orga83b2fc}

\begin{itemize}
\item lésion centrale : fibroscopie bronchique
\item périphérique : ponction transpariétale
\item entre les 2 : fibro ou ponction ou thoracotomie
\end{itemize}

Chercher mutations EGFR, réarrangements ALK, ROS1

\subsubsection{Bilan préthérapeutique}
\label{sec:orgb2b2f8a}
\emph{Extension}  : T(tumeur) = locale, N (node) = ganglionnaire, M
(métastase) = à distance. \\
si M : TDM abdo, IRM cérébrale, TEP-TDM

État général (Performance Status de 0 à 4), nutritionnel (-5\% en 1 mois ou
-10\% en 6 mois = \frownie{})

\emph{Cardiorespiratoire} : ECG, écho cardio, épreuve d'effort, doppler artériel des MI/cou,
coronarographie.\\
Règle : 1 lobe = -25\% de VEMS. Opération ssi VEMS post-op > 30\%
\subsubsection{Traitement}
\label{sec:org265564f}
\paragraph{CBPNC}
\label{sec:org12c03a0}

\begin{itemize}
\item localisé (stade I, II) : local (lobectomie ou radiothérapie)
\item localement avancé (stade III) : systémique (chimio) + local (radio ou
chir)
\item disséminé (stade IV) : systémique =
\end{itemize}
$\left \{
  \begin{array}{lr}
    \text{\gls{ITK} si altération moléculaire ciblable} & \text{Médiane > 2 ans}\\
    \text{chimio IV et 3e génération sinon} & \text{Médiane 12 mois}\\
  \end{array}
\right.$
\paragraph{CBPC\footnote{Chimiosensible rechutes fréquentes et rapides}}
\label{sec:orgdcd89e6}
Pas de chir !
$
\left \{
  \begin{array}{l}
    \text{radio + chimio (+ irradiation encéphale) si limitée}\\
    \text{chimio sinon}
  \end{array}
\right.
$
\paragraph{Symptomatique}
\label{sec:orgd80429f}
\emph{Douleur}  : Antalgiques, radiothérapie, chir, AINS, vertébroplastie

\emph{Dyspnée}  : lymphangite carcinomateuses (difficile), obstruction bronchique
(bronchoscopie), pleurésie exsudative (pleuroscopie), sd cave supérieur
(anticoag, corticoïdes)

\paragraph{Plan cancer}
\label{sec:org1f260dd}
Réunion de concertation pluridisciplinaire, consultation d'annonce, plan
personnalisé de soin
\paragraph{Suivi}
\label{sec:org42ee7c9}
CBNPC : arrêt tabac, TDM 2-3 mois, clinique, bio (rein, NFS, hypercalcémie,
hyopnatrémie)

CBPC : médiane de survie = 16-20 mois si limitée, 8-12 mois si métastatique
\subsubsection{Cancers secondaires}
\label{sec:org03e84bd}
Clinique : pauvre (dyspnée, toux, douleur thoracique, généraux), chercher extension ganglionnaire

Imagerie : 

\begin{itemize}
\item lâcher de ballons ou miliaire
\item épanchement pleural
\item interstitiel (lymphangite carcinomateuses)
\end{itemize}


Diagnostic : clinique, 

\begin{itemize}
\item inconnu : chercher accessibles à ttt spécifique, TEP. Sinon :

\begin{itemize}
\item épidermoïde, adénocarcinome
\item sinon : \female{} = \{gynéco, mammo\}, \male{} = \{PSA, TR, écho prostate\}
\end{itemize}

\item connu : radio peut suffir
\item ancien : métastases 10 ans plus tard possibles
\end{itemize}

\subsection{333 \textdagger{} Oedème de Quincke et anaphylaxie}
\label{sec:orgadea966}
\label{sec:333_oedeme_de_quincke_et_anaphylaxie}

\subsubsection{Définition, épidémiologie}
\label{sec:org05162b4}
HS systémique immédiate sévère. 2 types :

\begin{itemize}
\item allergique : production IgE spécifiques. Libération de médiateurs par
mastocytes et basophiles \thus vasodilatation, bronchoconstriction
\item non-allergique : pas d'exposition préalable, moins sévère
\end{itemize}

Déf : symptômes CV, respi, cutané/digestif mettant en
jeu le pronostic vital immédiatement après contact

\paragraph{Épidémiologie}
\label{sec:org2fdceca}
35 DC (alimentaire), 40 DC (piqûre) / an\\
Agents :

\begin{itemize}
\item aliments (90\% enfant) : arachide++, protéine oeuf/lait, fruits exotiques
\item médicaments (15-20\%)
\item venin hyménoptères (15-20\%)
\item autres : latex, effort
\end{itemize}


\subsubsection{Clinique}
\label{sec:org6fe3d1b}
Délai < 1h (5min si médic. IV, 15 si piqûre, 30 si alimentaire).
Manifestations :

\begin{itemize}
\item CV : PAS < 100mmFg (ou -30\%), tachycardie, pâleur, hypotonie, malaise,
perte connaissance, trouble rythme/conduction, ischémie myocardique
\item respi : de la toux à l'asthme (mortalité \(\nearrow\) si asthme mal
contrôlé)
\item cutanées, muqueuses : prurit, rash cutané érythémateux, urticaire et
surtout 

\begin{itemize}
\item angioedeme : gonflement mal limité ferme, non érythémateux, sans
prurit
\item oedème de Quincke : angioedeme larynx et cou : gêne respi haute
(dysphonie++, dysphagie++). \\
Clinique : gonflement langue, luette, paupière lèvre ou face\\
Possiblement létal
\end{itemize}

\item digestives
\end{itemize}

\textbf{Hautement probable}  si 

\begin{itemize}
\item gêne respi haute ou asthme ou choc
mettent en jeu le pronostic vital
\item cutanéomuqueux
\item début brutal, progression rapide
\end{itemize}

Sévérité
I. cutanéomuqeux généralisé
II. multiviscérale modérée
III. multiviscérale sévère
IV. arrêt respiratoire
V. DC


\paragraph{Diagnostic}
\label{sec:orgddd81d3}
Clinique + contexte.

Doser systématiquement tryptase sérique (confirmation) : H+0, H+1, H+24

Étiologie : interrogatoire, \{prick-tests, IDR\}, dosage IgE

DD : 

\begin{itemize}
\item choc anaphylactique : choc \{vagal, septique, cardiogénique\}, hypoglycémie, mastocytose
\item Oedeme de Quincke : sd cave sup, érysipèle du visage, angi-eodeme à
bradykinine, inhalation de corps étranger (toujours y penser !!)
\end{itemize}


\subsubsection{Traitement}
\label{sec:org52ac785}
\paragraph{Urgence}
\label{sec:org5322e6c}
Adrénaline : 0.01mg/kg (< 0.5mg) adulte à \faHospital{} ou  0.3mg
auto-injection \(\rightarrow\) répéter /5 min jusque stabilisation\\
Voie IM (pas de SC ou d'IV \danger{})

Remplissage vasc : 50-100mL adulte

\(O_2\), libérer voie aériennes, bronchodilatateurs (courte durée)

Glucagon si adrénaline ne fonctionne pas. 

\paragraph{Hors urgence}
\label{sec:org3e97b2b}
Anithistaminique, corticoïdes, arrêt agent.

NB : alerter, allongé jambes relevées (pas vertical \danger{} ), PLS si
inconscient

\paragraph{Préventif}
\label{sec:orgfd190d3}
Trousse avec adrénaline si (absolument) :

\begin{itemize}
\item anaphylaxie antérieure (aliment, insecte ou latex), induite par l'effort
ou idiopathique
\item allergie alimentaire et asthme persistant modéré non contrôlé
\item allergie hyménoptères avec réaction systémique antérieure (adulte) ou
plus sévère que cutanéomuqueux (enfant)
\end{itemize}

\subsection{354 \textdagger{} Corps étranger des voies aériennes}
\label{sec:orge260f88}
\label{sec:354_corps_etranger_des_voies_aeriennes}

Pics = 

\begin{itemize}
\item < 3 ans : graines d'oléagineux avec symétrie G-D
\item âgé si troubles déglutition, mauvaise dentition. 2 tableaux : asphyxie
aigüe (viande) ou pneumonie à répétition/suppuration bronchique
\item rarement chez l'adulte/ado : trauma facial (dents), bricolage, trouble
de conscience. Plutôt à droite
\end{itemize}


\subsubsection{Diagnostic}
\label{sec:orgc517a6d}
\paragraph{Clinique et radio}
\label{sec:org27214f0}
Dans les heures post-inhalation :

\begin{itemize}
\item CE mobile (sd de pénétration) : toux quinteuse, suffocation avec tirage,
cornage, cyanose. Résolutif en qq secondes
\item soit CE expulsé (pétéchies visage/tronc ou bouche/conjonctive)
\item plus rarement enclavement 

\begin{itemize}
\item proximal (enfant) : diminution murmure vésiculaire, wheezing
\item distal (adulte) : asymptomatique
\item exceptionnellement : oropharynx, larynx, lumière trachéale
\end{itemize}
\end{itemize}


Radio : le plus souvent normale. Sinon : atélectasie, hyperclarté \(\nearrow\)
expiration

Mois/années post-inhalation : à évoquer si respi chronique/récidivante ne
répondant pas au tt ou anomalies radio persistantes dans même territoire

\paragraph{DD}
\label{sec:orgf6c9b7f}
(sub)aigü : 

\begin{itemize}
\item détresse respi aigüe à début brutal, tirage, cornage : épiglottite aigüe
(fièvre, modif voix, hypersaliv)
\item infection respi basse : PAC, bronchite sifflante (< 3 ans)
\end{itemize}

Chronique/récidive :

\begin{itemize}
\item trouble ventilation persistant : tumeur bronchique obstructive, sténose
bronchique
\item infection respi même territoire : tumeur bronchique obstructive, foyer
bronchectasies
\end{itemize}


\subsubsection{CAT}
\label{sec:org57ee088}
\paragraph{Asphyxie aigüe}
\label{sec:org68eb853}
Toux ou Heimlich (si conscient) ou réanimation. Si < 2 ans, tapes dos puis
pressions sternum.
\paragraph{Sd pénétration non régressif}
\label{sec:org226af06}
Supposer que CE toujours présent !! (sauf confirmation entourage). 

Clinique : dimination unilat murmure, wheezing, persistance \{toux, dyspnée,
  cornage, tirage\}

RX  : piégage (expiration) \(\wedge\) diminution unilatérale murmure \thus CE
endobronchique (90\%)

\paragraph{Extraction}
\label{sec:org8d699c5}
Bronchoscopie rigide (ou souple). Chez l'enfant, centre spécialisé !

\subsection{354 \textdagger{} Détresse respiratoire aigüe}
\label{sec:org77d4a95}
\label{org7752461}

Définition : Inadéquation charge - capacité de l'appareil respiratoire

\subsubsection{Dianostic}
\label{sec:orge94485a}
Signes de luttes

\begin{itemize}
\item polypnée superficielle
\item recrutemente muscles (scalènes \(\wedge\) intercostaux), abdominaux,
dilatateurs des voies aériennes supérieures (très petit enfant)
\end{itemize}

Signes de faillite

\begin{itemize}
\item respi abdo paradoxale \{\} défaillance court terme
\item cyanose (hypoxémie profonde) \thus \(O_2\) ASAP
\item neuro : astérixis, altérations comportement/vigilance. Glasgow < 9
\thus assitance ventilatoire
\end{itemize}

Appareil circulatoire

\begin{itemize}
\item coeur pulmonaire aigu\footnote\{Turgescence jugulaire, reflux
hépato-jugulaire, hépatomégalie douleureuse, signe de Harzer\}
\item pouls paradoxal
\item hypercapnie : \{céphalées, hypervasc des conjonctives\}, \{tremblements
sueurs, tachy, hypotension\}
\end{itemize}

État de choc 

\begin{itemize}
\item peau froide, marbrures, \(\nearrow\) temps recoloration cutanée
\item PAs < 90 mmHg ou -50mmHg
\item tachy > 120min, FR > 25-30/min
\item oligurie
\item confusion, altération vigilance
\end{itemize}


\subsubsection{PEC}
\label{sec:org67d2af7}
Surveillance, \(O_2\), voie veineuse gros calibre, assistance ventilatoire,
ventilation (non)invasive

Pour l'étiologie : radiothorax, ECG, prélèvement (gaz du sang, NFS, iono, urée,
créat, acide lactique)

\subsubsection{Diagnostic}
\label{sec:orgf8a9f90}

\begin{figure}[htpb]
  \centering
  \resizebox{0.8\linewidth}{!}{
    \tikz \graph [
    % Labels at the middle 
    edge quotes mid,
    % Needed for multi-lines
    nodes={align=center},
    sibling distance=3cm,
    edges={nodes={fill=white}}, 
    layered layout]
    {
      "Obstruction VAS ?" ->["non"]  Radiothorax -> {
        Anormale -> {
          Atélectasies [draw];
          "Anomalies\\pleurales" ->
          "Épanchement\\
          pleural compressif\\
          Pneumothorax" [draw];
          "Opacités\\parenchyme" -> 
          "Pneumonie infect\\
          OAP\\
          Patho. infiltrative" [draw];
        };
        Normale -> {
          "Asthme\\
          aigu grave\\
          EP" [draw];
          "BPCO\\
          Anomalie paroi\\
          Neuromusc" [draw];
        }
      };
    };
  }
  \caption{Diagnostic détresse respiratoire aigǜe}
  \label{fig:diag_detresse_respi}
\end{figure}

Précision de la figure\textasciitilde{}\ref{fig:diag_detresse_respi}

\begin{enumerate}
\item \emph{Obstruction}  : corps étranger, laryngite, oedème de Quincke, sténose
trachée, tumeur laryngées
\item \emph{Anomalie radio} Priorité = pneumonie inf, OAP,
PTX\footnote{Pneumothorax} sous
tension\\
Indication =

\begin{itemize}
\item OAP : \{ATCD = insuf. cardiaque\}, \{expectoration mousseuse, orthopnée\},
\{opacités bilat diffuses, périhilaire\}
\item Pneumonie infectieuse : \{début brutal\}, \{expector. purulente\}, \{sd
inflammatoire\}
\end{itemize}

\item \emph{Radio normale} de novo = EP, PTX compressif, asthme. Si
chronique, dans l'ordre BPCO, paroi thoracique (obésité, déformation),
neuromusc\\
Orientation :

\begin{itemize}
\item EP : fébricule, turgescence jugul, radio normale
\item Pneumothorax : \{longiligne\}, \{pas de vibration vocale\}, murmure vésicul,
\{hyperclarté\}
\item asthme : ATDS, toux purulente, râles ou silence
\end{itemize}
\end{enumerate}

\subsubsection{SDRA}
\label{sec:orge803ab9}
Détresse respi < 7 jours sans défaillance cardiaque, sans surcharge volémique
avec opacités radio bilat diffuses

Mécanisme = oedème lésionnel.

Étiologies :

\begin{itemize}
\item exogène = infectieuse, toxique
\item endogènes : réponse inflammatoire systémique
\end{itemize}


\subsection{356 \textdagger{} Pneumothorax}
\label{sec:orga05665b}

\subsubsection{Définitions}
\label{sec:org8d9d7d8}
Air dans l'espace pleural \thus collapsus 

\paragraph{Spontané}
\label{sec:org0b52668}
Primaire = poumon sain, sujet jeune, non grave

Secondaire = patho, > 40 ans, peut décompenser


\begin{itemize}
\item Blebs (< 1 cm, plutôt primaires)
\item bulles d'emphysème : à la corticalité, tabagismes
\item lésions kystiques
\item cataménial
\end{itemize}


\paragraph{Traumatique}
\label{sec:orge171399}
Femé (côte fracturée) ou ouverts

\paragraph{Épidémiologie}
\label{sec:orgc254ea2}
Spontané primaire : < 35 ans, masculin, longiligne et de grande taille, fumeur !

Spontané secondaire : BPCO (plus rarement asthme, mucoviscidose)

facteurs favorisants : \(\Delta P_{atm}\), vols aérien, plongée subaquatique,
tabagisme actif (PAS
efforts physiques)

\subsubsection{Diagnostic}
\label{sec:org295d13a}
Clinique : 

\begin{itemize}
\item fonctionnel : douleur thoracique (brutale, homolatérale, latérale,
augmente toux), toux sèche irrit
\item physique : diminution murmure vésicul, pas de vibration vocales,
tympanisme percusion
\end{itemize}

Radio : pas en expiration ! 3 catégorise : apical, axillaire, complet

Grave si dyspnée sévère ou collapsus tensionnel (! pas déviation médiastin) 

\thus le plus souvent (pneumothorax compressif secondaire à fistule à
soupape) ou (PTX avec réserve ventilatoire réduite)

\paragraph{Atypique}
\label{sec:org468dcde}
Récidive : 30\% (premier épisode), 50\% (second épisode)

Rarement avec pneumomédiastin

Sous ventilation mécanique : y penser si \(\nearrow\) brutal pression
d'insufflation, collapsus brutal, plaie plèvre

\paragraph{DD}
\label{sec:org6d37e7d}
Facile : douleur thoracique "respiro-dépendante".

Difficile : Dyspnée aigüe sans sd pleural 

\subsubsection{Traitement}
\label{sec:org8e79895}

\begin{itemize}
\item Rien si décollement < 2cm
\item PTX compressif = urgence \thus aiguille simple + immédiat
\item Pneumotthorax spontané primaire (bien ou mal toléré ) :
exsufflation(petit cathéter sur voie thoracique antérieur) puis drain
si échec
\item Pour tous les autres : drain
\end{itemize}


Prévention récidive par pleurodèse (accolement des feuillets) : si récidive
homolat, PTX persistant 5 jours, compliqué ou bilat

Sevrage tabac !!

Attention altitude si PNO existant (seulement existant)

VIH: traitement aussi pneumocystose


\subsection{Gaz du sang}
\label{sec:org0453bba}
\label{appendix:gds}
Le pH est déterminé par l'équilibre entre les bicarbonates (\ch{HCO_3-}) et
\(PCO_2\) :
\begin{equation}
  pH = K_1 + log\frac{[\ch{HCO_3-]}}{K_2 p_{CO_2}}
\end{equation}
avec \(K_1\), \(K_2\) constante.

Un déséquilibre sur un terme induit une compensation sur l'autre. Si le
déséquilibre n'est pas compensé, on aboutit à une acidose ou une alcose.

\begin{table}[htpb]
  \centering
  \caption{Gaz du sang artériel}
  \label{tab:gds}
  \begin{tabular}{ll}
    \toprule
    \(PO_2\) & [80, 100] mmHg\\
    \(SaO_2\) & [95, 98] \%\\
    \(PCO_2\) & [35, 45] mmHg\\
    \ch{HCO_3^-} & [22, 29] mmol/L\\
    pH & [7.38, 7.42]\\
    \bottomrule
  \end{tabular}
\end{table}

Interprétation :
\begin{enumerate}
\item Déterminer si acidose ou alcolose selon le pH
\item Respiratoire ou métabolique ? Si \(PCO_2\) essaie de compenser (en sens
inverse du pH), c'est respiratoire. Sinon métabolique.
\end{enumerate}

NB : la compensation ne normalise pas le pH!
NB : 

\begin{itemize}
\item acidose pulmonaire : hypoventilation alvéolaire
\item alcalose pulmonaire : hyperventilation alvéolaire
\item alcalose métabolique : pertes digestives [vomissements]
\end{itemize}


\subsection{Physiologie}
\label{sec:orge5ff730}

\(P_{alv}\) = pression alvéolaire, \(P_{ip}\) = pression interpleurale (dans la
plèvre), pression transpulmonaire = \(P_{ip} - P_{alv}\).

\begin{figure}[htpb]
  \centering
  \caption{Inspiration (gauche), expiration (droite)}
  \tikz \graph [ nodes={align=center}, layered layout]
  {
    Contraction diaphragme -> Expansion thorax -> "$P_{ip} < P_{atm}$"
    -> Hausse pression transpulmonaire -> Expansion poumons 
    -> "$P_{alv} < P_{atm}$"
    -> Arrivée d'air dans les alvéoles;
  };
  \tikz \graph [ nodes={align=center}, layered layout]
  {
    "Arrêt contraction\\ diaphragme et intercostaux" -> Rétraction thorax 
    -> "Valeur initiale de $P_{ip}$"
    -> Valeur initiale pression transpulmonaire 
    -> Rétraction poumons 
    -> "$P_{alv} > P_{atm}$"
    -> Expulsion d'air depuis les alvéoles;
  };

\end{figure}

\section{Ophtalmologie}
\label{sec:org42f80e3}
\subsection{21 \textdagger{} Rétinopathie diabétique}
\label{sec:org872083d}
\label{orgd703da6}
 Diabète = défini par risque d'une rétinopathie. Complications dont on peut éviter la
 cécité !

30\% diabétique en ont une. Diabète 1 : 90\% après 20 ans. Diabète 2 : 60\% à 15
ans

FR = durée et intensité de l'hyperglycémie

Physiopatho : hyperglycémie \thus \{accumulation sorbitol, glycation, stress
oxydatif\} \thus \{inflammation, activation rénine-angiotensine, modif flux
sanguin rétinien, production VEGF\} puis 
\begin{itemize}
\item occlusion capillaire et néovascularisation
\item \oe{}dème maculaire
\end{itemize}

Cécité possible du jour au lendemain 

\subsubsection{Diagnostic}
\label{sec:org0319dd6}
Photographie du FO = référence et dépistage
\begin{itemize}
\item microanévrisme rétitiens : dilatations punctiformes rouges (postérieure)
\item hémorragie rétiniennes punctiformes
\item nodules cotonneux
\item signes d'ischémie : hémorragies intrarétiniennes "en taches" ou en flammèche, anomalies
microvasculaire intrarétinienne, dilatations veineuses "en chapelet",
néovaisseaux prérétiniens et précapillaires, hémorragie prérétiniennes/intravitréennes
\item \uline{complications} : hémorragie intravitréenne, décollement de la rétine \emph{par
traction}, néovascularisation irienne (\thus glaucome néovasculaire !)
\item signes maculaires : \oe{}dème maculaire (cystoïde), exsudats lipidiques
(souvent en couronne)
\end{itemize}

Complémentaire : OCT pour diagnostic et suivi \oe{}dème maculaire, echo en mode B
pour décollement de rétine par traction

\subsubsection{Dépistage RD}
\label{sec:org39b4ca6}
Diabète 1 : photo FO à la découverte puis surveillance annuelle
\begin{itemize}
\item enfant: pas avant 10 ans
\item grossesse : avant puis tous les 3 mois (tous les mois si RD !)
\end{itemize}
Diabète 2 : dès découverte

Surveillance :
\begin{itemize}
\item \(\emptyset\) RD on non proliférante minime : annuelle
\item sinon tous 4 à 6 mois
\item renforcé si puberté, adolescence, si équilibrage trop rapide de la glycémie,
chir bariatrique, diabète ancien mal équilibré, chir de la cataracte,
\oe{}dème maculaire
\end{itemize}

Classification :
\begin{itemize}
\item \(\emptyset\) RD
\item RD non proliférante : minime, modérée, sévère (si hémorragie rétiniennes dans 4
quadrants ou dilatations veineuses 2 quadrants ou AMIR 1 quadrant)
\item RD proliférante (néovaisseaux): minime, sévère, compliquée (décollement de
rétine par traction, gls:GNV)
\end{itemize}

Progression lente, aggravations rapides possibles. \danger prolifération néovasc
peut donner cécité 

\subsubsection{Traitement}
\label{sec:orga833791}
Médical :
\begin{itemize}
\item équilibre glycémique et hypertension pour 2 diabètes !
\item pas de ttt médicamenteux
\end{itemize}
RD proliférante 
\begin{itemize}
\item photocoagulation panrétinienne : régression dans 90\%. Indication : RD
proliférante ou (RD non proliférante sévère si grossesse, sujet jeune
diabétique 1 avec normalisation rapide glycémie, chir cataracte)
\item injection intravitréenne anti-VEGF (pour néovasc. iridienne, glaucome
néovasculaire)
\item chir si RD proliférante avec hémorragie intravitréenne persistante/décollement
de rétine tractionnel
\end{itemize}
\OE{}dème maculaire 
\begin{itemize}
\item photocoagulation au laser si exsudats lipidiques ou liquides
\item injection anti-VEGF mensuelle. Dexaméthasone retard possible mais cataracte,
risque d'hypertonie oculaire (30\%)
\end{itemize}
\subsubsection{Autres complications}
\label{sec:orgd60793d}
cataracte (+ freq), glaucome néovasculaire (redoutable),
 paralysie oculomotrice (régresse spontanément qq mois)


\section{Glossaire}
\label{sec:orgeccb889}
\printglossaries
\end{document}
