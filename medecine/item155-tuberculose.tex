% Created 2021-04-16 Fri 16:27
% Intended LaTeX compiler: pdflatex
\documentclass[a4paper,11pt,twoside,twocolumn]{article}
\usepackage[utf8]{inputenc}
\usepackage[T1]{fontenc}
\usepackage{graphicx}
\usepackage{grffile}
\usepackage{longtable}
\usepackage{wrapfig}
\usepackage{rotating}
\usepackage[normalem]{ulem}
\usepackage{amsmath}
\usepackage{textcomp}
\usepackage{amssymb}
\usepackage{capt-of}
\usepackage{hyperref}
\usepackage{tabularx}
\usepackage{booktabs}
\usepackage{enumitem}
\usepackage{titlesec}
\usepackage[margin=1cm]{geometry}
\usepackage{adjustbox}
\date{}
\title{Tuberculose}
\hypersetup{
 pdfauthor={Alexis},
 pdftitle={Tuberculose},
 pdfkeywords={},
 pdfsubject={},
 pdfcreator={Emacs 27.2 (Org mode 9.5)}, 
 pdflang={English}}
\begin{document}

\def\dec{$\searrow{}$}
\def\inc{$\nearrow{}$}

\setlist{nolistsep}

\titlespacing*{\subsection}{0pt}{1ex}{1.3ex}
\titlespacing*{\subsubsection}{0pt}{0.5ex}{0.5ex}
\section*{Tuberculose maladie}
\label{sec:orga73a050}
Dissémination bronchogène. Décès sans traitement.
\subsection*{Clinique}
\label{sec:org16afab4}
\begin{itemize}
\item Toux prolongée, expectoration muco-purulente ou hémoptoïque
\item \textpm{} Douleurs thoraciques, dyspnée
\item AEG: amaigrissement, asthénie, fièvre souvent vespérale, sueurs nocturnes
\end{itemize}
\subsection*{Examens}
\label{sec:org8d29c62}
RX et scanner thorax
\begin{itemize}
\item infiltrats des sommets uni/bilatéraux (\textbf{\textbf{excavés}}), caverne(s), nodule isolé (tuberculome)
\end{itemize}

\textbf{\textbf{Diagnostic de certitude = bactériologique}}
\begin{itemize}
\item Prélèvements :
\begin{itemize}
\item bronchiques : expectorations si toux productive (3 j de suite), tubages gastriques le matin à jeun (3 j de suite) sinon , LBA si nécessaire
\item biopsiques
\end{itemize}
\item Techniques :
\begin{itemize}
\item examen microscopique
\item Dépistage rapide par PCR : différencie rapidement Mycobacterium tuberculosis d'une autre mycobactérie, gène de résistance au ttt
\item culture (Lowenstein­Jensen = 3-4 semaines)
\item \textbf{\textbf{antibiogramme}}
\item PCR
\end{itemize}
\end{itemize}
\subsection*{Evolution}
\label{sec:orgb057e45}

\begin{itemize}
\item Non traitée : mortelle (50 \%), guérison spontanée (25\%), chronique (25\%)
\item Traitée : guérison quasi constante
\end{itemize}
Complications : extra-pulmonaire, miliaire

\section*{Tuberculose miliaire}
\label{sec:orgbe6feaf}

Dissémination par voie hématogène + multiples granulomes de la taille d'un grain de mil.
\subsection*{Clinique}
\label{sec:org2ed3d81}
\begin{itemize}
\item Fièvre prolongée, sueurs nocturnes,
\item SDRA,
\item neuro-méningés (nourrissons),
\item péricardite
\end{itemize}
\subsection*{Examens}
\label{sec:org93ccf1f}
\begin{itemize}
\item RX thorax et TDM : images micronodulaires (1 à 2 mm) disséminées
\item Biologie : pancytopénie (infiltration médullaire), cholestase anictérique
\item BK sur culture:
\begin{itemize}
\item Hémocultures sur milieux spéciaux
\item Sécrétions bronchiques
\item LCS
\item Biopsie(s) : hépatique, BOM
\end{itemize}
\end{itemize}

Décès en l'absence de ttt précoce
\section*{Extra-pulmonaire (25\%)}
\label{sec:org047807f}
\textbf{\textbf{histologie}} : granulome épithélioïde et gigantocellulaire avec nécrose caséeuse
\subsection*{Ganglionnaire}
\label{sec:orgaee8db8}
\begin{itemize}
\item basicervicales >> médiastinale
\item adénites souvent volumineuses, fistulisation
\item Biopsie/ponction
\item BAAR au microscope + culture
\item complications: fistule
\end{itemize}
\subsection*{Osseuse}
\label{sec:org0500cde}
\begin{itemize}
\item Spondylodiscite
\item Radiographie osseuse, IRM rachis
\item Ponction-biopsie
\item complications: épidurite, compression médullaire
\end{itemize}
\subsection*{Pleurésie (rare)}
\label{sec:orgf3f21d2}
\begin{itemize}
\item Insidieux, sd pleural (toux, douleur pleurale)
\item RX thorax + ponction : liquide clair exsudatif lymphocytaire
\item Biopsie pleurale
\item complications : fibrose pleurale
\end{itemize}
\subsection*{Péricardite (rare)}
\label{sec:orga2a4bc1}
\begin{itemize}
\item Subaigu : fièvre, douleur tho, tamponnade
\item ECG: anomalies diffuses
\item RX thorax et echo  cardiaque
\item Culture liquide péricardite
\item complication: tamponnade, chronique
\end{itemize}
\subsection*{Neuroméningé}
\label{sec:org77905c4}
\begin{itemize}
\item AEG puis progessif
\item sd méningé, déficit focaux
\item Hyponatrémie( SIADH)
\item PL: méningite lymphocytaire avec hyperprot et hypogly
\item complications : décès, séquelles neruo sévères
\end{itemize}
\subsection*{Urinaire}
\label{sec:orgd1b69f4}
\begin{itemize}
\item \textbf{*leucocyturie aseptique}
\item fréquent, asympto/dysurie douleur les flancs,
\item urines 3j de suite
\item complications : néphrite interstitielle granulomateuse, hydronéphrose, rétraction vésicale
\end{itemize}

\subsection*{Génitale}
\label{sec:orgeaea2f2}
\begin{itemize}
\item homme: prostatite, épididydmite, masse scrotale
\item femme : trouble menstruels, douleur abdo
\item calcifications chez l'homme, culture sur frottis/menstruations chez femme
\end{itemize}
\subsection*{Digestive}
\label{sec:org57bccff}
\begin{itemize}
\item fibro OGD, colonoscopie + biopsie
\end{itemize}
\subsection*{Laryngé (rare)}
\label{sec:org2d2e6e4}
\begin{itemize}
\item ulcération douloureuse, toux, dysphagie
\item prélèvement local
\end{itemize}
\end{document}